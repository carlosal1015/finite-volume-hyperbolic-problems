% arara: clean: {
% arara: --> extensions:
% arara: --> ['aux', 'bbl', 'bcf', 'blg', 'log', 'nav',
% arara: --> 'out', 'pdf', 'run.xml', 'snm', 'toc', 'vrb']
% arara: --> }
% arara: lualatex: {
% arara: --> shell: yes,
% arara: --> draft: yes,
% arara: --> interaction: batchmode
% arara: --> }
% arara: biber
% arara: lualatex: {
% arara: --> shell: yes,
% arara: --> draft: no,
% arara: --> interaction: batchmode
% arara: --> }
% arara: lualatex: {
% arara: --> shell: yes,
% arara: --> draft: no,
% arara: --> interaction: batchmode
% arara: --> }
% arara: clean: {
% arara: --> extensions:
% arara: --> ['aux', 'bbl', 'bcf', 'blg', 'log', 'nav',
% arara: --> 'out', 'run.xml', 'snm', 'toc', 'vrb']
% arara: --> }
\PassOptionsToPackage{svgnames}{xcolor}
\documentclass[
    8pt,
    aspectratio=1610,
    c,
    intlimits,
    leqno,
    professionalfonts,
]{beamer}

\usepackage{diffcoeff}
\usepackage{mathtools}
\usepackage{minted}
\usepackage{newunicodechar}
\usepackage[
	citestyle=numeric,
	style=numeric,
	backend=biber,
]{biblatex}
\addbibresource{beamer.bib}

\usefonttheme[onlymath]{serif}
\setbeamertemplate{navigation symbols}{}
\setbeamercolor{structure}{fg=DarkBlue}
\setbeamertemplate{frametitle}[default][center]
\setbeamertemplate{items}[ball]

\hypersetup{
	pdfencoding=auto,
	linktocpage=true,
	colorlinks=true,
	linkcolor=DarkBlue,
	urlcolor=DarkBlue,
	pdfpagelabels,
}

\renewcommand\theequation{{\color{DarkBlue}\arabic{equation}}}

\newunicodechar{ₒ}{\makebox[\fontcharwd\font`a]{${}_{\text{o}}$}}
\newunicodechar{₁}{\makebox[\fontcharwd\font`a]{${}_{\text{1}}$}}
\newunicodechar{₂}{\makebox[\fontcharwd\font`a]{${}_{\text{2}}$}}
\newunicodechar{₃}{\makebox[\fontcharwd\font`a]{${}_{\text{3}}$}}
\newunicodechar{⋅}{\makebox[\fontcharwd\font`a]{$\cdot$}}
\newunicodechar{ψ}{\makebox[\fontcharwd\font`a]{$\Phi$}}
\newunicodechar{θ}{\makebox[\fontcharwd\font`a]{$\theta$}}
\newunicodechar{β}{\makebox[\fontcharwd\font`a]{$\beta$}}
\newunicodechar{∂}{\makebox[\fontcharwd\font`a]{$\partial$}}
\newunicodechar{ξ}{\makebox[\fontcharwd\font`a]{$\xi$}}
\newunicodechar{─}{\makebox[\fontcharwd\font`a]{$-$}}

\title{Travelling wave solution for a simple in situ combustion model}
\subtitle{Seminario de Investigación I}
\author{Angel Ramírez Gutiérrez \and Fidel Jara Huanca}
\institute{\small Instituto de Matemática y Ciencias Afines}
\date{\today}

\begin{document}

\begin{frame}
	\titlepage
\end{frame}

\section{Mathematical model}

\begin{frame}
	\frametitle{\secname}

	\begin{block}{Two-Phase Flow model with combustion~\cite{Gargar2020}}
		\begin{align}
			C_{\text{m}}
			\diffp{T}{t}+
			\diffp{
				\big[
					c_{\text{o}}
					\rho_{\text{o}}
					S_{\text{o}}
					u_{\text{o}}
					\big(
					T-T_{\text{res}}
					\big)
					\big]
			}{x}                                                          & =
			\rho_{\text{o}}
			Q_{\text{r}}
			Y
			S_{\text{g}}
			S_{\text{o}}
			W_{\text{r}}-
			\alpha
			\left(T-T_{\text{res}}\right).                                    \\
			\varphi
			\diffp{\big(YS_{\text{g}}\rho_{\text{g}}\big)}{t}+
			\diffp{\big(YS_{\text{g}}\rho_{\text{g}}u_{\text{g}}\big)}{x} & =
			-w\varphi\rho_{\text{o}}YS_{\text{g}}S_{\text{o}}W_{\text{r}}.    \\
			\varphi
			\diffp{\big(\rho_{\text{o}}S_{\text{o}}\big)}{t}+
			\diffp{\big(\rho_{\text{o}}S_{\text{o}}u_{\text{o}}\big)}{x}
			                                                              & =
			-\varphi\rho_{\text{o}}YS_{\text{g}}S_{\text{o}}W_{\text{r}}.     \\
			W_{\text{r}}                                                  & =
			k_{p}
			\exp
			\left(
			-\frac{E_{r}}{RT}
			\right).
		\end{align}
		where
		\begin{columns}
			\begin{column}{.3\paperwidth}
				\begin{itemize}
					\item

					      $T$ is the temperature.

					\item

					      $Y$ is the oxygen fraction.

					\item

					      $\rho_{\text{o}}$ is the oil density.

					\item

					      $u_{\text{g}}$ is the gas velocity.

					\item

					      $\varphi$ is the porosity.

					\item

					      $E_{r}$ is the activation energy.

					      % \item

					      %       $\alpha$ is the heat loss coefficient.
				\end{itemize}
			\end{column}
			\begin{column}{.3\paperwidth}
				\begin{itemize}
					\item

					      $S_{\text{o}}$ is the oil saturation.

					\item

					      $C_{\text{m}}$ is the heat capacity.

					\item

					      $\rho_{\text{g}}$ is the gas density.

					\item

					      $T_{\text{res}}$ is the reservoir temperature.

					\item

					      $\omega$ is the stoichiometric coefficient.

					\item

					      $R$ is the universal gas constant.
				\end{itemize}
			\end{column}
			\begin{column}{.3\paperwidth}
				\begin{itemize}
					\item

					      $S_{\text{g}}$ is the gas saturation.

					\item


					      $c_{\text{o}}$ is the oil molar heat capacity.


					\item

					      $u_{\text{o}}$ is the oil velocity.

					\item

					      $Q_{\text{r}}$ is the reaction heat.

					\item

					      $W_{\text{r}}$ is the reaction rate.

					\item

					      $k_{p}$ is pre-exponential coefficient.
				\end{itemize}
			\end{column}
		\end{columns}
	\end{block}
\end{frame}

\section{Dimensionless model}

\begin{frame}
	\frametitle{\secname}
	By change of variables,
	\begin{equation}
		\overline{t}=
		\frac{t}{t^{\ast}},\quad
		\overline{x}=
		\frac{T_{\text{res}}k_{p}x}{\rho_{\text{g}}u},\quad
		\overline{\theta}=
		\frac{T-T_{\text{res}}}{T_{\text{res}}},\quad
		T^{\ast}=T_{\text{res}},\quad
		t^{\ast}=
		\frac{10^{8}}{k_{p}},\quad
		x^{\ast}=
		\frac{\rho_{g}}{\rho^{\text{res}}}
		\frac{u}{k_{p}}.
	\end{equation}
	Now, omitting bars we have.

	\begin{block}{PDE system with initial and boundary
			conditions~\cite{Quispe2020}}

		Let $\theta\left(x,t\right)$ the \alert{temperature},
		$S_{y}\left(x,t\right)$ the \alert{oxygen saturation} and
		$S_{\text{o}}\left(x,t\right)$ the \alert{oil saturation}.
		\begin{align}
			\diffp{\theta}{t}+
			a_{1}
			\diffp{
				\big(
				\theta
				S_{\text{o}}
				\big)
			}{x}                    & =
			b_{1}
			S_{\text{o}}
			S_{y}
			\Phi-
			\beta\theta.
			\label{eq:6}                \\
			\diffp{S_{y}}{t}+
			a_{2}
			\diffp{S_{y}}{x}        & =
			-b_{2}
			S_{\text{o}}
			S_{y}
			\Phi.
			\label{eq:7}                \\
			\diffp{S_{\text{o}}}{t}+
			a_{3}
			\diffp{S_{\text{o}}}{x} & =
			-b_{3}
			S_{\text{o}}
			S_{y}
			\Phi.
			\label{eq:8}                \\
			S_{\text{o}}
			S_{y}
			\Phi                    & =
			0.
			\tag{Initial Condition}
		\end{align}
		where
		\begin{equation}
			a_{1}=
			\frac{
			t^{\ast}
			u_{\text{o}}
			c_{\text{o}}
			\rho_{\text{o}}
			}{C_{\text{m}}x^{\ast}},\quad
			a_{2}=
			\frac{t^{\ast}u_{\text{g}}}{\varphi x^{\ast}},\quad
			a_{3}=
			\frac{t^{\ast}u_{\text{o}}}{\varphi x^{\ast}},\quad
			b_{1}=
			\frac{
				\rho_{\text{o}}
				Q_{\text{r}}
			}{
				C_{\text{m}}
				T_{\text{res}}
			},\quad
			b_{2}=
			\frac{w\rho_{\text{o}}}{\rho_{\text{g}}},\quad
			b_{3}=
			1,\quad
			\beta=
			\frac{\alpha}{k_{p}C_{\text{m}}}
		\end{equation}
		are the dimensionless coefficients.
		\setcounter{equation}{12}
		\begin{equation}
			\big(
			\theta^{L},
			S^{L}_{y},
			S^{L}_{\text{o}}
			\big)=
			\big(
			0,
			S^{L}_{y},
			0
			\big)
			\text{ and }
			\big(
			\theta^{R},
			S^{R}_{y},
			S^{R}_{\text{o}}
			\big)                           =
			\big(
			0,
			0,
			S^{R}_{\text{o}}
			\big).
			\label{eq:13}
		\end{equation}
	\end{block}
\end{frame}

\section{Travelling wave analysis}

\begin{frame}
	\frametitle{\secname}

	Rewrite the equations~\eqref{eq:6},\eqref{eq:7} and~\eqref{eq:8},
	via the \alert{travelling wave substitution}
	\begin{math}
		\left(x,t\right)\leftarrow
		\left(\xi,t\right)
	\end{math},
	$\xi=x-ct$ with $c>0$.
	\setcounter{equation}{14}
	\begin{align}
		-c
		\diff{\theta}{\xi}+
		a_{1}
		\diff{
			\big(
			\theta
			S_{\text{o}}
			\big)
		}{\xi}                   & =
		b_{1}
		S_{\text{o}}
		S_{y}
		\Phi-
		\beta\theta.
		\label{eq:15}                   \\
		-c
		\diff{S_{y}}{\xi}+
		a_{2}
		\diff{S_{y}}{\xi}        & =
		-b_{2}
		S_{\text{o}}
		S_{y}
		\Phi.
		\label{eq:16}                   \\
		-c
		\diff{S_{\text{o}}}{\xi}+
		a_{3}
		\diff{S_{\text{o}}}{\xi} & =
		-b_{3}
		S_{\text{o}}
		S_{y}
		\Phi.
		\label{eq:17}
		\shortintertext{Replace~\eqref{eq:16} in~\eqref{eq:17}.}
		\frac{b_{3}}{b_{2}}
		\left(
		\textcolor{DarkBlue}{
			-c
			\diff{S_{y}}{\xi}+
			a_{2}
			\diff{S_{y}}{\xi}
		}
		\right)                  & =
		-c\diff{S_{\text{o}}}{\xi}+
		a_{3}
		\diff{S_{\text{o}}}{\xi}.\notag \\
		-c
		\frac{b_{3}}{b_{2}}
		\int
		\diff{S_{y}}{\xi}
		\dl\xi+
		a_{2}
		\frac{b_{3}}{b_{2}}
		\int
		\diff{S_{y}}{\xi}
		\dl\xi                   & =
		-c
		\int
		\diff{S_{\text{o}}}{\xi}
		\dl\xi+
		a_{3}
		\int
		\diff{S_{\text{o}}}{\xi}
		\dl\xi.
		\notag                          \\
		-c
		\frac{b_{3}}{b_{2}}
		S_{y}+
		a_{2}
		\frac{b_{3}}{b_{2}}
		S_{y}                    & =
		-c
		S_{\text{o}}+
		a_{3}
		S_{\text{o}}+
		K.\notag
		\shortintertext{Multiply both sides by $b_{2}$.}
		b_{2}
		\left(
		c-a_{3}
		\right)
		S_{\text{o}}-
		b_{3}
		\left(
		c-
		a_{2}
		\right)
		S_{y}                    & =
		K.\label{eq:18}
		\shortintertext{Take the limits $\xi\to\pm\infty$
			in~\eqref{eq:18}.}
		\lim_{\xi\to-\infty}
		\left[
			b_{2}
			\left(
			c-a_{3}
			\right)
			S_{\text{o}}-
			b_{3}
			\left(
			c-
			a_{2}
			\right)
			S_{y}
		\right]                  & =
		\lim_{\xi\to\infty}
		K=
		\lim_{\xi\to\infty}
		\left[
			b_{2}
			\left(
			c-a_{3}
			\right)
			S_{\text{o}}-
			b_{3}
			\left(
			c-
			a_{2}
			\right)
			S_{y}
		\right].\notag                  \\
		b_{2}
		\left(
		c-a_{3}
		\right)
		S^{L}_{\text{o}}-
		b_{3}
		\left(
		c-
		a_{2}
		\right)
		S^{L}_{y}                & =
		K=
		b_{2}
		\left(
		c-a_{3}
		\right)
		S^{R}_{\text{o}}-
		b_{3}
		\left(
		c-
		a_{2}
		\right)
		S^{R}_{y}.\label{eq:19}
	\end{align}
\end{frame}

\begin{frame}
	\frametitle{\secname}

	\begin{align}
		\shortintertext{Substitute the BC $S^{L}_{\text{o}}=0$ and
			$S^{R}_{y}=0$ from~\eqref{eq:13} and solve~\eqref{eq:19}
			respect to $c$ and $S^{L}_{y}$, respectively.}
		c                        & =
		\frac{
			a_{2}
			b_{3}
			S^{L}_{y}+
			a_{3}
			b_{2}
			S^{R}_{\text{o}}
		}{
			b_{2}
			S^{R}_{\text{o}}+
			b_{3}
			S^{L}_{y}
		}.                           \\
		S^{L}_{y}                & =
		-\frac{b_{2}\left(c-a_{3}\right)}{b_{3}\left(c-a_{2}\right)}
		S^{R}_{\text{o}}.
		\shortintertext{If $S^{L}_{\text{o}}=0$, \eqref{eq:19} results
		$\textcolor{DarkBlue}{-b_{3}\left(c-a_{2}\right)S^{L}_{y}}=K$ and
		replace it in~\eqref{eq:18}.}
		b_{2}
		\left(
		c-a_{3}
		\right)
		S_{\text{o}}-
		b_{3}
		\left(
		c-
		a_{2}
		\right)
		S_{y}                    & =
		\textcolor{DarkBlue}{-b_{3}\left(c-a_{2}\right)S^{L}_{y}}.
		\notag
		\shortintertext{Solve respect to $S_{y}$.}
		S_{y}                    & =
		S^{L}_{y}+
		\frac{b_{2}\left(c-a_{3}\right)}{b_{3}\left(c-a_{2}\right)}
		S_{\text{o}}.\label{eq:22}
		\shortintertext{Solve~\eqref{eq:17} respect to
			$\diff{S_{\text{o}}}{\xi}$.}
		\diff{S_{\text{o}}}{\xi} & =
		\frac{
		b_{3}
		S_{\text{o}}
		S_{y}
		\Phi
		}{c-a_{3}}.\label{eq:23}
	\end{align}
\end{frame}

\begin{frame}
	\frametitle{\secname}

	\begin{align}
		\shortintertext{From $\textcolor{DarkBlue}{b_{2}}\times\eqref{eq:15}+\textcolor{DarkBlue}{b_{1}}\times\eqref{eq:16}$.}
		\textcolor{DarkBlue}{b_{2}}
		\left[
			-c
			\diff{\theta}{\xi}+
			a_{1}
			\textcolor{DarkBlue}{\diff{\big(\theta S_{\text{o}}\big)}{\xi}}
			\right]+
		\textcolor{DarkBlue}{b_{1}}
		\left[
			-c
			\diff{S_{y}}{\xi}+
			a_{2}
			\diff{S_{y}}{\xi}
		\right]           & =
		\textcolor{DarkBlue}{b_{2}}
		\left[
			b_{1}
			S_{\text{o}}
			S_{y}
			\Phi-
			\beta\theta
			\right]+
		\textcolor{DarkBlue}{b_{1}}
		\left[
		-b_{2}
		S_{\text{o}}
		S_{y}
		\Phi
		\right].\notag           \\
		-b_{2}c
		\diff{\theta}{\xi}+
		b_{2}a_{1}
		\textcolor{DarkBlue}{
			\left(
			\theta
			\diff{S_{\text{o}}}{\xi}+
			S_{\text{o}}
			\diff{\theta}{\xi}
			\right)
		}-
		b_{1}c
		\diff{S_{y}}{\xi}+
		b_{1}a_{2}
		\diff{S_{y}}{\xi} & =
		-b_{2}\beta\theta.\notag \\
		-b_{2}
		\left(c-a_{1}S_{\text{o}}\right)
		\alert{\diff{\theta}{\xi}}+
		b_{2}a_{1}
		\theta
		\diff{S_{\text{o}}}{\xi}-
		b_{1}
		\left(c-a_{2}\right)
		\diff{S_{y}}{\xi} & =
		-b_{2}
		\beta
		\theta.
		\notag                   \\
		\shortintertext{Solve respect to $\alert{\diff{\theta}{\xi}}$.}
		\setcounter{equation}{25}
		\frac{
		b_{1}\left(c-a_{2}\right)
		\textcolor{DarkBlue}{\diff{S_{y}}{\xi}}-
		b_{2}a_{1}\theta
		\diff{S_{\text{o}}}{\xi}
		-b_{2}\beta\theta
		}{-b_{2}
		\left(c-a_{1}S_{\text{o}}\right)
		}                 & =
		\alert{\diff{\theta}{\xi}}.\label{eq:26}
		\shortintertext{Replace~\eqref{eq:22} in~\eqref{eq:26}.}
		\frac{
		b_{1}\left(c-a_{2}\right)
		\textcolor{DarkBlue}{
			\frac{b_{2}\left(c-a_{3}\right)}{b_{3}\left(c-a_{2}\right)}
			\diff{S_{\text{o}}}{\xi}
		}-
		b_{2}a_{1}\theta
		\diff{S_{\text{o}}}{\xi}
		-b_{2}\beta\theta
		}{
		-b_{2}
		\left(c-a_{1}S_{\text{o}}\right)
		}                 & =
		\diff{\theta}{\xi}.
		\shortintertext{If $c\neq a_{2}$ and cancel $b_{2}$, then}
		\setcounter{equation}{28}
		\frac{
		\left[
			\frac{b_{1}}{b_{3}}
			\left(c-a_{3}\right)
			a_{1}\theta
			\right]
		\textcolor{DarkBlue}{\diff{S_{\text{o}}}{\xi}}
		-\beta\theta
		}{
		a_{1}S_{\text{o}}-c
		}                 & =
		\diff{\theta}{\xi}.
		\label{eq:29}
	\end{align}
\end{frame}

\begin{frame}
	\frametitle{\secname}

	\begin{align}
		\shortintertext{Multiply~\eqref{eq:29} LHS by
			$\frac{b_{3}}{b_{3}}$ and replace~\eqref{eq:23}.}
		\frac{
		\left[
			b_{1}
			\left(c-a_{3}\right)
			a_{1}\theta
			\right]
		\textcolor{DarkBlue}{
		\frac{
		b_{3}
		S_{\text{o}}
		S_{y}
		\Phi
		}{c-a_{3}}
		}
		-b_{3}\beta\theta
		}{
		b_{3}
		\left(
		a_{1}S_{\text{o}}-c
		\right)
		}                       & =
		\diff{\theta}{\xi}.\notag   \\
		\setcounter{equation}{33}
		\Phi\left(\theta\right) & =
		\exp
		\left(
		-\frac{E_{r}}{
			R\left(T^{\ast}\theta+T_{\text{res}}\right)
		}
		\right).\label{eq:34}       \\
		S_{y}                   & =
		S^{L}_{y}+
		\frac{b_{2}\left(c-a_{3}\right)}{b_{3}\left(c-a_{2}\right)}
		S_{\text{o}}=
		\left(
		S^{R}_{\text{o}}-
		S_{\text{o}}
		\right)
		\frac{S^{L}_{y}}{S^{R}_{\text{o}}}.
		\label{eq:35}               \\
		\diff{\theta}{\xi}
		                        & =
		\frac{
			\beta\theta
			\left(c-a_{3}\right)+
			\left[
				a_{1}
				b_{3}
				\theta-
				b_{1}
				\left(c-a_{3}\right)
				\right]
			S_{\text{o}}
			S_{y}
			\Phi}{
			\left(c-a_{1}S_{\text{o}}\right)
			\left(c-a_{3}\right)
		}.\label{eq:36}
	\end{align}
\end{frame}

\section{Relation $f\left(\theta\right)$, $f\left(S_{y}\right)$ and
  $f\left(S_{\text{o}}\right)$}

\begin{frame}
	\frametitle{\secname}

	\begin{block}{$f\left(\theta\right)$,
			$f\left(S_{y}\right)$ and $f\left(S_{\text{o}}\right)$}

		Setting~$\eqref{eq:36}=0$, ~$\eqref{eq:22}=0$
		and~$\eqref{eq:23}=0$.
		\begin{align*}
			\diff{\theta}{\xi}=
			0
			 & =
			\frac{
				\beta\theta
				\left(c-a_{3}\right)+
				\left[
					a_{1}
					b_{3}
					\theta-
					b_{1}
					\left(c-a_{3}\right)
					\right]
				S_{\text{o}}
				S_{y}
				\Phi}{
				\left(c-a_{1}S_{\text{o}}\right)
				\left(c-a_{3}\right)
			}.
			% 	f\left(\theta\right)       & =
			% 	0.                             \\
			% 	f\left(S_{y}\right)        & =
			% 	0.                             \\
			% 	f\left(S_{\text{o}}\right) & =
			% 	0.                             \\
			% 	\jacob{u,v,w}{x,y,z}       & =
			% 	0.
		\end{align*}

		\begin{equation*}
			f\left(\theta\right)\coloneqq
			\frac{
				\beta\theta\left(c-a_{3}\right)}{
				-\left[
					a_{1}
					b_{3}
					\theta-
					b_{1}
					\left(c-a_{3}\right)
					\right]
				\Phi
			}
			\frac{S^{R}_{\text{o}}}{S^{L}_{y}}.
		\end{equation*}

		\begin{align*}
			\Gamma_{1} & =
			\left\{
			\left(S_{\text{o}},\theta\right)\mid
			S_{\text{o}}
			S_{y}=
			f\left(\theta\right),\theta\geq0
			\right\}.      \\
			\Gamma_{1} & =
			\left\{
			\left(S_{\text{o}},\theta\right)\mid
			S_{\text{o}}
			\left(
			S^{R}_{\text{o}}-
			S_{\text{o}}
			\right)=
			f\left(\theta\right),\theta\geq0
			\right\}.
			% \Gamma_{2} & =
			% \left\{
			% \left(S_{\text{o}},\theta\right)\mid
			% \text{expresion }2=
			% f\left(S_{\text{o}}\right),\theta\geq0
			% \right\}.      \\
			% \Gamma_{3} & =
			% \left\{
			% \left(S_{\text{o}},\theta\right)\mid
			% \text{expresion }3=
			% f\left(\theta\right),\theta\geq0
			% \right\}.
		\end{align*}

		\begin{equation*}
			F\left(S_{\text{o}},\theta\right)=
			\begin{bmatrix}
				\frac{b_{3}S_{y}S_{\text{o}}}{c-a_{3}}\Phi \\
				\frac{
					\beta\theta
					\left(c-a_{3}\right)+
					\left[
						a_{1}
						b_{3}
						\theta-
						b_{1}
						\left(c-a_{3}\right)
						\right]
					S_{\text{o}}
					S_{y}
					\Phi}{
					\left(c-a_{1}S_{\text{o}}\right)
					\left(c-a_{3}\right)
				}
			\end{bmatrix}
		\end{equation*}

		\begin{equation*}
			JF\left(S_{\text{o}},\theta\right)=
			\begin{bmatrix}
				\diffp{F_{1}}{S_{\text{o}}} & \diffp{F_{1}}{\theta} \\
				\diffp{F_{2}}{S_{\text{o}}} & \diffp{F_{2}}{\theta}
			\end{bmatrix}
		\end{equation*}

		We define
		\setcounter{equation}{38}
		\begin{equation}
			g\left(\theta\right)\coloneqq
			\left[
				a_{1}
				b_{3}
				\theta-
				b_{1}
				\left(c-a_{3}\right)
				\right]
			\Phi\left(\theta\right).\label{eq:39}
		\end{equation}
	\end{block}
\end{frame}

\begin{frame}
	\begin{align}
		F_{1}\left(S_{\text{o}},\theta\right) & =
		\frac{
		b_{3}
		S_{\text{o}}
		S_{y}
		\Phi\left(\theta\right)
		}{c-a_{3}}.
		\\
		F_{2}\left(S_{\text{o}},\theta\right) & =
		\frac{
			\beta\theta
			\left(c-a_{3}\right)+
			g\left(\theta\right)
			S_{\text{o}}
			S_{y}
			\Phi\left(\theta\right)}{
			\left(c-a_{1}S_{\text{o}}\right)
			\left(c-a_{3}\right)
		}
	\end{align}

	\begin{align*}
		\diffp{F_{1}}{S_{\text{o}}} & =
		\frac{
		b_{3}
		}{c-a_{3}}
		\left(
		\diff{S_{y}}{S_{\text{o}}}
		S_{\text{o}}+
		S_{y}
		\right)
		\Phi\left(\theta\right)
		.                                       \\
		\diffp{F_{1}}{\theta}       & =
		\frac{b_{3}S_{y}S_{\text{o}}}{c-a_{3}}
		\diff{\Phi\left(\theta\right)}{\theta}. \\
		\diffp{F_{2}}{S_{\text{o}}} & =
		\frac{
			g\left(\theta\right)
			\left(
			\diff{S_{y}}{S_{\text{o}}}
			S_{\text{o}}+
			S_{y}
			\right)
			\left(c-a_{1}S_{\text{o}}\right)+
			a_{1}
			\left(
			\beta\theta
			\left(c-a_{3}\right)+
			g\left(\theta\right)
			S_{y}
			S_{\text{o}}
			\right)
		}{
			{\left(c-a_{1}S_{\text{o}}\right)}^{2}
			\left(c-a_{3}\right)
		}                                       \\
		\diffp{F_{2}}{\theta}       & =
		\frac{
			\beta
			\left(c-a_{3}\right)+
			\diff{g\left(\theta\right)}{\theta}
			S_{y}
			S_{\text{o}}}{
			\left(c-a_{1}S_{\text{o}}\right)
			\left(c-a_{3}\right)
		}.
	\end{align*}
	It is easy to see that:
	\begin{align*}
		\diff{\Phi\left(\theta\right)}{\theta} & =
		\Phi\left(\theta\right)
		h\left(\theta\right),
		\text{ where }
		h\left(\theta\right)=
		\frac{E_{\text{r}}T^{\ast}}{R{\left(T^{\ast}\theta+T_{\text{res}}\right)}^{2}}, \\
		\diff{g\left(\theta\right)}{\theta}    & =
		\left[
		a_{1}b_{3}+
		\left(
		a_{1}b_{3}\theta-
		b_{1}\left(c-a_{3}\right)
		\right)
		h\left(\theta\right)
		\right]
		\Phi\left(\theta\right).
	\end{align*}
	and $\Phi\left(\theta\right)$ is given by~\eqref{eq:34}.
\end{frame}

\begin{frame}
	In this paper we are interested in studying the combustion from and
	thus we assume that the reaction does not occur at the boundaries,
	i.e., the reaction rate $\Phi$ in~\eqref{eq:34} vanish when: there
	is no oxygen $(Y=0)$, which we call ``Oxygen Controlled'' case
	(OC); there is no fuel $(S_{\text{o}}=0)$, which we call
	``Fuel Controlled'' case (FC).
	In order to approximate applications, we consider that the left
	injection end of the domain is FC, and the production right end of
	the domain is OC.
	The OC and FC equilibria are given by~\eqref{eq:13}, then:
	\begin{align*}
		\text{OC: }
		\big(S^{L}_{\text{o}},\theta^{L}\big) & =
		\left(0,0\right).                         \\
		\text{FC: }
		\big(S^{R}_{\text{o}},\theta^{R}\big) & =
		\big(S^{R}_{\text{o}},0\big).
	\end{align*}

	From~\eqref{eq:39}, we obtain:
	\setcounter{equation}{47}
	\begin{equation}
		g\left(0\right)=
		-b_{1}\left(c-a_{3}\right)\Phi\left(0\right).
	\end{equation}
	From~\eqref{eq:34}, we obtain:
	\begin{equation}
		\Phi\left(0\right)=
		t^{\ast}
		k_{p}
		\exp\left(-\frac{E_{\text{r}}}{RT_{\text{res}}}\right).
	\end{equation}
	At OC and FC equilibria, the matrix $JF$ has the forms:
	\begin{equation}\label{eq:50}
		JF_{\text{OC}}=
		\begin{bmatrix}
			\frac{b_{3}S^{L}_{y}}{c-a_{3}}\Phi\left(0\right) & 0               \\
			-\frac{b_{1}S^{L}_{y}}{c}\Phi\left(0\right)      & \frac{\beta}{c}
		\end{bmatrix}.
	\end{equation}
	\begin{equation}\label{eq:51}
		JF_{\text{FC}}=
		\begin{bmatrix}
			-\frac{b_{3}S^{L}_{y}}{c-a_{3}}\Phi\left(0\right)                & 0                                     \\[.5\baselineskip]
			\frac{b_{1}S^{L}_{y}}{c-a_{1}S^{R}_{\text{o}}}\Phi\left(0\right) & \frac{\beta}{c-a_{1}S^{R}_{\text{o}}}
		\end{bmatrix}.
	\end{equation}
	because from~\eqref{eq:35} we obtain:
	\begin{equation*}
		S_{y}
		\big(S^{R}_{\text{o}}\big)=0
		\quad\text{ and }\quad
		\diff{S_{y}}{S_{\text{o}}}=
		-\frac{S^{L}_{y}}{S^{R}_{\text{o}}}.
	\end{equation*}
\end{frame}

\begin{frame}
	Defining $e=S^{L}_{y}\Phi\left(0\right)$, then the corresponding
	eigenvalues and eigenvectors of~\eqref{eq:50} are:
	\begin{equation*}
		\begin{vmatrix}
			JF_{\text{OC}}-\lambda I
		\end{vmatrix}=
		\begin{vmatrix}
			\frac{b_{3}e}{c-a_{3}}-\lambda & 0                       \\
			-\frac{b_{1}e}{c}              & \frac{\beta}{c}-\lambda
		\end{vmatrix}=
		\left(\frac{b_{3}e}{c-a_{3}}-\lambda\right)
		\left(\frac{\beta}{c}-\lambda\right)=
		0.
	\end{equation*}
	then:
	\begin{equation*}
		\lambda^{1}_{\text{OC}}=
		\frac{b_{3}e}{c-a_{3}},\qquad
		\lambda^{2}_{\text{OC}}=
		\frac{\beta}{c}.
	\end{equation*}

	For $\lambda^{1}_{\text{OC}}$:
	\begin{equation*}
		\begin{bmatrix}
			0                 & 0                                      \\
			-\frac{b_{1}e}{c} & \frac{\beta}{c}-\frac{b_{3}e}{c-a_{3}}
		\end{bmatrix}
		\begin{bmatrix}
			v_{1} \\
			v_{2}
		\end{bmatrix}=
		\begin{bmatrix}
			0 \\
			0
		\end{bmatrix}\implies
		v_{2}=1\implies
		v=\begin{bmatrix}
			\frac{
			\left(c-a_{3}\right)\beta-b_{3}ec
			}{b_{1}e\left(c-a_{3}\right)} \\
			1
		\end{bmatrix}.
	\end{equation*}

	For $\lambda^{2}_{\text{OC}}$:
	\begin{equation*}
		\begin{bmatrix}
			\frac{b_{3}e}{c-a_{3}}-\frac{\beta}{c} & 0 \\
			-\frac{b_{1}e}{c}                      & 0
		\end{bmatrix}
		\begin{bmatrix}
			v_{1} \\
			v_{2}
		\end{bmatrix}=
		\begin{bmatrix}
			0 \\
			0
		\end{bmatrix}\implies
		v=\begin{bmatrix}
			0 \\
			1
		\end{bmatrix}.
	\end{equation*}

	The corresponding eigenvalues and eigenvectors of~\eqref{eq:51}
	are:
	\begin{equation*}
		\begin{vmatrix}
			JF_{\text{FC}}-\lambda I
		\end{vmatrix}=
		\begin{vmatrix}
			-\frac{b_{3}e}{c-a_{3}}-\lambda        & 0                                             \\[.5\baselineskip]
			\frac{b_{1}e}{c-a_{1}S^{R}_{\text{o}}} & \frac{\beta}{c-a_{1}S^{R}_{\text{o}}}-\lambda
		\end{vmatrix}=
		\left(-\frac{b_{3}e}{c-a_{3}}-\lambda\right)
		\left(\frac{\beta}{c-a_{1}S^{R}_{\text{o}}}-\lambda\right)=
		0.
	\end{equation*}
	then:
	\begin{equation*}
		\lambda^{1}_{\text{FC}}=
		-\frac{b_{3}e}{c-a_{3}},\qquad
		\lambda^{2}_{\text{FC}}=
		\frac{\beta}{c-a_{1}S^{R}_{\text{o}}}.
	\end{equation*}
\end{frame}

\begin{frame}
	For $\lambda^{1}_{\text{FC}}$:
	\begin{equation*}
		\begin{bmatrix}
			0                                      & 0                                                            \\[.5\baselineskip]
			\frac{b_{1}e}{c-a_{1}S^{R}_{\text{o}}} & \frac{\beta}{c-a_{1}S^{R}_{\text{o}}}+\frac{b_{3}e}{c-a_{3}}
		\end{bmatrix}
		\begin{bmatrix}
			v_{1} \\
			v_{2}
		\end{bmatrix}=
		\begin{bmatrix}
			0 \\
			0
		\end{bmatrix}\implies
		v_{1}=0\implies
		v=\begin{bmatrix}
			\frac{
			\left(c-a_{3}\right)\beta-b_{3}ec
			}{b_{1}e\left(c-a_{3}\right)} \\
			1
		\end{bmatrix}.
	\end{equation*}

	For $\lambda^{2}_{\text{FC}}$:
	\begin{equation*}
		\begin{bmatrix}
			-\frac{b_{3}e}{c-a_{3}}-\frac{\beta}{c-a_{1}S^{R}_{\text{o}}} & 0 \\[.5\baselineskip]
			\frac{b_{1}e}{c-a_{1}S^{R}_{\text{o}}}                        & 0
		\end{bmatrix}
		\begin{bmatrix}
			v_{1} \\
			v_{2}
		\end{bmatrix}=
		\begin{bmatrix}
			0 \\
			0
		\end{bmatrix}\implies
		v=\begin{bmatrix}
			0 \\
			1
		\end{bmatrix}.
	\end{equation*}
\end{frame}

\begin{frame}
	Let be
	\begin{equation*}
		\Gamma_{1}=
		\left\{
		\left(S_{\text{o}},\theta\right)\mid
		S_{\text{o}}
		\left(
		S^{R}_{\text{o}}-
		S_{\text{o}}
		\right)=
		f\left(\theta\right),\theta\geq0
		\right\}
	\end{equation*}
	where
	\begin{equation*}
		f\left(\theta\right)=
		\beta
		\frac{S^{R}_{\text{o}}}{S^{L}_{y}}
		\frac{
			\left(c-a_{3}\right)\theta}{
			b_{1}
			\left(c-a_{3}\right)-
			a_{1}
			b_{3}
			\theta
		}
		\exp
		\left(
		\frac{E_{r}}{
			R\left(T^{\ast}\theta+T_{\text{res}}\right)
		}
		\right).
	\end{equation*}

	Then,
	\begin{equation*}
		S_{\text{o}_{\pm}}=
		\frac{S^{R}_{\text{o}}}{2}\pm
		\sqrt{{\left(\frac{S^{R}_{\text{o}}}{2}\right)}^{2}-f\left(\theta\right)}.
	\end{equation*}
	Furthermore,
	\begin{enumerate}
		\item

		      \begin{math}
			      f\left(\alert{0}\right)=
			      \beta
			      \frac{S^{R}_{\text{o}}}{S^{L}_{y}}
			      \frac{
				      \left(c-a_{3}\right)\alert{0}}{
				      b_{1}
				      \left(c-a_{3}\right)-
				      a_{1}
				      b_{3}
				      \alert{0}
			      }
			      \exp
			      \left(
			      \frac{E_{r}}{
				      R\left(T^{\ast}\alert{0}+T_{\text{res}}\right)
			      }
			      \right)=0
		      \end{math}.

		      \

		\item

		      \begin{math}
			      \lim\limits_{\theta\to\infty}
			      f\left(\theta\right)=
			      \lim\limits_{\theta\to\infty}
			      \left[
				      \beta
				      \frac{S^{R}_{\text{o}}}{S^{L}_{y}}
				      \frac{
					      \left(c-a_{3}\right)\theta}{
					      b_{1}
					      \left(c-a_{3}\right)-
					      a_{1}
					      b_{3}
					      \theta
				      }
				      \exp
				      \left(
				      \frac{E_{r}}{
					      R\left(T^{\ast}\theta+T_{\text{res}}\right)
				      }
				      \right)
				      \right]=
			      \left[
				      \lim\limits_{\theta\to\infty}
				      \beta
				      \frac{S^{R}_{\text{o}}}{S^{L}_{y}}
				      \right]
			      \left[
				      \lim\limits_{\theta\to\infty}
				      \frac{
					      \left(c-a_{3}\right)\theta}{
					      b_{1}
					      \left(c-a_{3}\right)-
					      a_{1}
					      b_{3}
					      \theta
				      }
				      \right]
			      \left[
				      \lim\limits_{\theta\to\infty}
				      \exp
				      \left(
				      \frac{E_{r}}{
					      R\left(T^{\ast}\theta+T_{\text{res}}\right)
				      }
				      \right)
				      \right]=
			      \left[
				      \beta
				      \frac{S^{R}_{\text{o}}}{S^{L}_{y}}
				      \right]
			      \left[\frac{c-a_{3}}{a_{1}b_{3}}\right]
			      \left[1\right]=
			      -\beta
			      \frac{S^{R}_{\text{o}}}{S^{L}_{y}}
			      \frac{c-a_{3}}{a_{1}b_{3}}
		      \end{math}.

		      \

		\item

		      If $\theta\geq 0$, $c-a_{3}\geq 0$ and
		      $b_{1}\left(c-a_{3}\right)-a_{1}b_{3}\theta>0$,
		      then $f\left(\theta\right)\geq 0$.
	\end{enumerate}
\end{frame}

\begin{frame}
	\begin{align*}
		f\left(\theta\right) & =
		\beta
		\frac{S^{R}_{\text{o}}}{S^{L}_{y}}
		\frac{1}{t^{\ast}k_{p}}
		\frac{
			\left(c-a_{3}\right)\theta}{
			b_{1}
			\left(c-a_{3}\right)-
			a_{1}
			b_{3}
			\theta
		}
		\exp
		\left(
		\frac{E_{r}}{
			R\left(T^{\ast}\theta+T_{\text{res}}\right)
		}
		\right).                                                      \\
		f\left(\theta\right) & =
		\alpha
		\frac{S^{R}_{\text{o}}}{S^{L}_{y}}
		\frac{1}{k_{p}C_{\text{m}}}
		\frac{
		\left(c-a_{3}\right)\theta}{
		b_{1}
		\left(c-a_{3}\right)
		\left[
		1-\frac{a_{1}b_{3}\theta}{b_{1}\left(c-a_{3}\right)}
		\right]
		}
		\exp
		\left(
		\frac{E_{r}}{
			R\left(T^{\ast}\theta+T_{\text{res}}\right)
		}
		\right).                                                      \\
		f\left(\theta\right) & =
		\frac{m\theta\exp\left(\frac{n}{\theta+p}\right)}{1-q\theta}. \\
		f\left(\theta\right) & =
		m\left[\frac{\theta}{1-q\theta}\right]
		\exp\left(\frac{n}{\theta+p}\right).
	\end{align*}

	\begin{equation*}
		m\coloneqq \alpha\frac{S^{R}_{\text{o}}}{S^{L}_{y}b_{1}}
		\frac{1}{k_{p}C_{\text{m}}},\quad
		n\coloneqq \frac{E_{r}}{RT^{\ast}},\qquad
		p\coloneqq \frac{T_{\text{res}}}{T^{\ast}},\qquad
		q\coloneqq \frac{a_{1}b_{3}}{b_{1}\left(c-a_{3}\right)}.
	\end{equation*}
\end{frame}

\begin{frame}
	\begin{align*}
		\diff{f\left(\theta\right)}{\theta} & =
		m\left[\frac{1}{1-q\theta}\right]
		\exp\left(\frac{n}{p+\theta}\right)+
		m\left[\frac{q\theta}{\left(1-q\theta\right)^{2}}\right]
		\exp\left(\frac{n}{p+\theta}\right)-
		m\left[\frac{n\theta}{{\left(p+\theta\right)}^{2}\left(1-q\theta\right)}\right]
		\exp\left(\frac{n}{p+\theta}\right).    \\
		\diff{f\left(\theta\right)}{\theta} & =
		m\left[
			\frac{
				nq\theta^{2}-
				n\theta+
				p^{2}+
				2p\theta+
				\theta^{2}
			}{
				{\left(p+\theta\right)}^{2}
					{\left(1-q\theta\right)}^{2}
			}
			\right]
		\exp\left(\frac{n}{p+\theta}\right).
	\end{align*}

	\begin{align*}
		\diff{f\left(\theta\right)}{\theta} & =
		0.                                      \\
		nq\theta^{2}-
		n\theta+
		p^{2}+
		2p\theta+
		\theta^{2}                          & =
		0.                                      \\
		\left(1+nq\right)
		\theta^{2}+
		\left(2p-n\right)\theta+
		p^{2}                               & =
		0.                                      \\
		\frac{
		\left(n-2p\right)\pm
		\sqrt{{\left(2p-n\right)}^{2}-4\left(1+nq\right)p^{2}}
		}{
		2\left(1+nq\right)
		}
		                                    & =
		\theta^{C}_{\pm}.                       \\
		\frac{
			\left(n-2p\right)\pm
			\sqrt{n\left(n-4p-4p^{2}q\right)}
		}{
			2\left(1+nq\right)
		}
		                                    & =
		\theta^{C}_{\pm}.
	\end{align*}

	Let be $\Delta\coloneqq {\left(2p-n\right)}^{2}-4\left(1+nq\right)p^{2}=n\left(n-4p-4p^{2}q\right)$.

	\begin{description}
		\item[First case $\Delta<0$,]

		      $f\left(\theta\right)$ has no critical points and $S_{\text{o}^{\pm}}\left(\theta\right)$ are strictly increasing.

		\item[Second case $\Delta=0$,]

		      $f\left(\theta\right)$ has one single critical point.

		\item[Third case $\Delta>0$,]

		      $f\left(\theta\right)$ has two critical points.
	\end{description}
	% TODO: En la gráfica 2a) $\varpsilon$ es como $n$ y el $\theta_{0}$ es como $p$
	% \begin{equation*}
	% 	\fcolorbox{DarkBlue}{yellow}{
	% 		\begin{math}
	% 			\frac{
	% 				-\left(n+2\right)\pm
	% 				\sqrt{{\left(n+2\right)}^{2}-4\left(1-nq\right)}
	% 			}{
	% 				2\left(1+nq\right)
	% 			}
	% 			=
	% 			\theta^{C}_{\pm}.
	% 		\end{math}
	% 	}
	% \end{equation*}
\end{frame}

\begin{frame}
	In other hand, we analyze $S_{\text{o}^{\pm}}$ as function of $\theta$.

	\begin{align*}
		S_{\text{o}^{\pm}}\left(\theta\right) & =
		\frac{S^{R}_{\text{o}}}{2}\pm
		\sqrt{\left(\frac{S^{R}_{\text{o}}}{2}\right)^{2}-f\left(\theta\right)}. \\
		S_{\text{o}^{\pm}}\left(\theta\right) & =
		\frac{S^{R}_{\text{o}}}{2}\pm
		\sqrt{\frac{\left(S^{R}_{\text{o}}\right)^{2}-4f\left(\theta\right)}{4}}.
	\end{align*}

	\begin{align*}
		\diff{S_{\text{o}^{\pm}}\left(\theta\right)}{\theta} & =
		\pm\frac{
			\diff{f\left(\theta\right)}{\theta}
		}{\sqrt{{\left(S^{R}_{\text{o}}\right)}^{2}-4f\left(\theta\right)}}.
	\end{align*}

	Therefore,
	\begin{enumerate}[i.]
		\item

		      The functions $S_{\text{o}^{+}}$ are well-defined in the set
		      \begin{math}
			      \left\{
			      \theta>0:
			      f\left(\theta\right)<\frac{1}{4}{\left(S^{R}_{\text{o}}\right)}^{2}
			      \right\}
		      \end{math}.

		\item

		      the critical points of $S_{\text{o}^{+}}\left(\theta\right)$
		      and those of $f\left(\theta\right)$ are the same;

		\item

		      there exists $\theta_{\text{M}}>0$ such that $4f\left(\theta_{\text{M}}\right)={\left(S^{R}_{\text{o}}\right)}^{2}$ then
		      \begin{math}
			      \lim\limits_{\theta\to\theta_{\text{M}}}
			      \diff{S_{\text{o}^{+}}\left(\theta\right)}{\theta}=
			      \pm\infty
		      \end{math}.

		      \begin{description}
			      \item[First case $\Delta<0$,]

			            $f\left(\theta\right)$ has no critical points and $f\left(\theta\right)$ and $S_{\text{o}^{+}}\left(\theta\right)$ are strictly increasing.



			      \item[Second case $\Delta=0$,]
			            $f\left(\theta\right)$ has one single critical point.

			      \item[Third case $\Delta>0$,]

			            $f\left(\theta\right)$ has two critical points.
		      \end{description}
	\end{enumerate}
\end{frame}

\begin{frame}
	\begin{figure}[ht!]
		\centering
		\includegraphics[width=.75\paperwidth]{fthetaplot}
		\caption{Plot $f\left(\theta\right)$.}
	\end{figure}
\end{frame}

\begin{frame}[fragile]
	\begin{columns}
		\begin{column}{0.51\textwidth}
			\begin{table}[ht!]
				\begin{tabular}{cc}
					\hline
					$\alpha$           & $0.2$            \\
					\hline
					$S^{L}_{y}$        & $0.5$            \\
					\hline
					$S^{R}_{\text{o}}$ & $0.5$            \\
					\hline
					$a_{1}$            & $0.00189$        \\
					\hline
					$a_{2}$            & $27.6$           \\
					\hline
					$a_{3}$            & $0.00252$        \\
					\hline
					$b_{1}$            & $1.1$            \\
					\hline
					$b_{2}$            & $30.4$           \\
					\hline
					$b_{3}$            & $1$              \\
					\hline
					$C_{\text{m}}$     & $2\times 10^{6}$ \\
					\hline
					$k_{p}$            & $24556$          \\
					\hline
					$R$                & $8.314$          \\
					\hline
					$T_{\text{res}}$   & $300$            \\
					\hline
				\end{tabular}
				\caption{Values.}
			\end{table}
		\end{column}
		\begin{column}{0.45\textwidth}
			\inputminted[fontsize=\tiny]{text}{calculations.txt}
		\end{column}
	\end{columns}
\end{frame}

\begin{frame}
	\begin{figure}[ht!]
		\centering
		\includegraphics[width=.75\paperwidth]{phaseportrait}
		\caption{Phase portrait of $\Gamma_{1}$.}
	\end{figure}
\end{frame}

\begin{frame}
	\begin{figure}[ht!]
		\centering
		\includegraphics[width=.75\paperwidth]{phaseportrait2}
		\caption{Phase portrait of $\Gamma_{1}$.}
	\end{figure}
\end{frame}

\begin{frame}[fragile]
	\frametitle{\secname}

	\begin{columns}
		\begin{column}{0.43\textwidth}
			\inputminted[fontsize=\tiny,firstline=5,lastline=35]{python}{main.py}
			\inputminted[fontsize=\tiny,firstline=45,lastline=56]{python}{main.py}
		\end{column}
		\begin{column}{0.53\textwidth}
			\inputminted[fontsize=\tiny,firstline=1,lastline=27]{text}{main.txt}
		\end{column}
	\end{columns}
\end{frame}

\begin{frame}[fragile]
	\frametitle{\secname}

	\begin{columns}
		\begin{column}{0.43\textwidth}
			\inputminted[fontsize=\tiny,firstline=67,lastline=107]{python}{main.py}
		\end{column}
		\begin{column}{0.53\textwidth}
			\inputminted[fontsize=\tiny,firstline=29,lastline=63]{text}{main.txt}
		\end{column}
	\end{columns}
\end{frame}

\begin{frame}[fragile]
	\frametitle{\secname}

	\begin{columns}
		\begin{column}{0.43\textwidth}
			\inputminted[fontsize=\tiny,firstline=109,lastline=120]{python}{main.py}
		\end{column}
		\begin{column}{0.53\textwidth}
			\inputminted[fontsize=\tiny,firstline=64,lastline=75]{text}{main.txt}
		\end{column}
	\end{columns}
\end{frame}

\begin{frame}
	\frametitle{References}

	\nocite{*}
	\printbibliography[heading=none]
\end{frame}

\end{document}
