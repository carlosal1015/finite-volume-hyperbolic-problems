% arara: clean: {
% arara: --> extensions:
% arara: --> ['aux', 'bbl', 'bcf', 'blg', 'log', 'nav',
% arara: --> 'out', 'pdf', 'run.xml', 'snm', 'toc', 'vrb']
% arara: --> }
% arara: lualatex: {
% arara: --> shell: yes,
% arara: --> draft: yes,
% arara: --> interaction: batchmode
% arara: --> }
% arara: biber
% arara: lualatex: {
% arara: --> shell: yes,
% arara: --> draft: no,
% arara: --> interaction: batchmode
% arara: --> }
% arara: lualatex: {
% arara: --> shell: yes,
% arara: --> draft: no,
% arara: --> interaction: batchmode
% arara: --> }
% arara: clean: {
% arara: --> extensions:
% arara: --> ['aux', 'bbl', 'bcf', 'blg', 'log', 'nav',
% arara: --> 'out', 'run.xml', 'snm', 'toc', 'vrb']
% arara: --> }
\PassOptionsToPackage{svgnames}{xcolor}
\documentclass[
    8pt,
    aspectratio=1610,
    c,
    intlimits,
		handout,
    leqno,
    professionalfonts,
]{beamer}

\usepackage{mathtools}
\usepackage{unicode-math}
\usepackage{diffcoeff}
\usepackage{minted}
\usepackage{newunicodechar}
\usepackage[
	citestyle=numeric,
	style=numeric,
	backend=biber,
]{biblatex}
\addbibresource{beamer.bib}

\addtobeamertemplate{theorem begin}{\normalfont}{}
\usefonttheme[onlymath]{serif}
\setbeamertemplate{navigation symbols}{}
\setbeamercolor{structure}{fg=DarkBlue}
\setbeamertemplate{frametitle}[default][center]
\setbeamertemplate{items}[ball]

\DeclareMathAlphabet{\mathbb}{U}{msb}{m}{n}
\DeclareMathAlphabet{\mathcal}{OMS}{cmsy}{m}{n}

\title{Formulation as conservation law for a simple in situ combustion model}
\subtitle{Seminario de Investigación I}
\author{Angel Ramírez Gutiérrez\and Fidel Jara Huanca\and Andres Alcalde Sosa\and Carlos Aznarán Laos}
\institute{\small Instituto de Matemática y Ciencias Afines}
\date{\today}

\begin{document}

\begin{frame}
	\titlepage
\end{frame}

\begin{frame}
	\frametitle{Motivation}
	\begin{block}{PDE system with initial and boundary
			conditions~\cite{Quispe2020}}

		Let $\theta\left(x,t\right)$ be the \alert{temperature},
		$S_{y}\left(x,t\right)$ be the \alert{oxygen saturation} and
		$S_{\text{o}}\left(x,t\right)$ be the \alert{oil saturation}.
		\begin{align}
			\diffp{\theta}{t}+
			a_{1}
			\diffp{
				\big(
				\theta
				S_{\text{o}}
				\big)
			}{x}                    & =
			b_{1}
			S_{\text{o}}
			S_{y}
			\Phi-
			\beta\theta.
			\label{eq:6}                \\
			\diffp{S_{y}}{t}+
			a_{2}
			\diffp{S_{y}}{x}        & =
			-b_{2}
			S_{\text{o}}
			S_{y}
			\Phi.
			\label{eq:7}                \\
			\diffp{S_{\text{o}}}{t}+
			a_{3}
			\diffp{S_{\text{o}}}{x} & =
			-b_{3}
			S_{\text{o}}
			S_{y}
			\Phi.
			\label{eq:8}                \\
			S_{\text{o}}
			S_{y}
			\Phi                    & =
			0.                          \\
			\big(
			\theta^{L},
			S^{L}_{y},
			S^{L}_{\text{o}}
			\big)                   & =
			\big(
			0,
			S^{L}_{y},
			0
			\big).                      \\
			\big(
			\theta^{R},
			S^{R}_{y},
			S^{R}_{\text{o}}
			\big)                   & =
			\big(
			0,
			0,
			S^{R}_{\text{o}}
			\big).
		\end{align}
	\end{block}
	% Existence and Uniqueness.
\end{frame}

\begin{frame}
	\begin{definition}[\cite{Godlewski2021}]
		Let be $\Omega\subset\mathbb{R}^{p}$ an open set.
		A \textcolor{DarkBlue}{\bfseries conservation law system with source term}
		is
		\begin{equation}\label{eq:systemofconservationlaw}
			\diffp{\symbf{u}}{t}+
			\diffp{}{x}
			\symbf{f}\left(\symbf{u}\right)=
			\symbf{s}\left(\symbf{u}\right).
		\end{equation}
		Where $\Omega$ is the set of states,
		\begin{math}
			\symbf{f}\in
			\symbf{C}^{1}
			\left(\Omega,\mathbb{R}^{p}\right)
		\end{math}
		is a flux function,
		\begin{math}
			\symbf{s}\in
			\symbf{C}^{1}
			\left(\Omega,\mathbb{R}^{p}\right)
		\end{math}
		is a source function without derivative of $\symbf{u}$ terms and
		\begin{math}
			\symbf{u}\in
			\symbf{C}^{1}\left(\mathbb{R}\times\left[0,\infty\right[,\Omega\right)
		\end{math} is a solution of~\eqref{eq:systemofconservationlaw}.
		\begin{align*}
			\symbf{f}\colon\Omega
			                               & \longrightarrow\mathbb{R}^{p} &
			\symbf{s}\colon\Omega
			                               & \longrightarrow\mathbb{R}^{p} &
			\symbf{u}\colon\mathbb{R}\times\left[0,\infty\right[
			                               & \longrightarrow\Omega           \\
			\begin{bmatrix}
				u_{1}  \\
				\vdots \\
				u_{p}
			\end{bmatrix}                & \longmapsto
			\begin{bmatrix}
				f_{1}\left(\symbf{u}\right) \\
				\vdots                      \\
				f_{p}\left(\symbf{u}\right)
			\end{bmatrix}, &
			\begin{bmatrix}
				u_{1}  \\
				\vdots \\
				u_{p}
			\end{bmatrix}                & \longmapsto
			\begin{bmatrix}
				s_{1}\left(\symbf{u}\right) \\
				\vdots                      \\
				s_{p}\left(\symbf{u}\right)
			\end{bmatrix}, &
			\left(x,t\right)
			                               & \longmapsto
			\begin{bmatrix}
				u_{1}\left(x,t\right) \\
				\vdots                \\
				u_{p}\left(x,t\right)
			\end{bmatrix}=
			\begin{bmatrix}
				u_{1}  \\
				\vdots \\
				u_{p}
			\end{bmatrix}.
		\end{align*}
	\end{definition}

	\pause

	\begin{definition}
		A conservation law system~\eqref{eq:systemofconservationlaw} is
		\textcolor{DarkBlue}{\bfseries hyperbolic} if and only if
		\begin{equation}
			\forall\symbf{u}\in\Omega\!:
			\forall\omega\in\mathbb{R}\setminus\left\{0\right\}\!:
			\exists\left\{\left(\lambda_{k},\symbf{r}_{k}\right)\right\}^{p}_{k=1}\subset
			\mathbb{R}\times\mathbb{R}^{p}
			\text{ such that }
			\symbf{A}\left(\symbf{u},\omega\right)
			\symbf{r}_{k}=
			\lambda_{k}
			\symbf{r}_{k}.
		\end{equation}
		Where
		\begin{math}
			\symbf{A}\left(\symbf{u},\omega\right)\coloneqq
			\omega
			{
				\begin{bmatrix}
					\diffp{f_{i}}{u_{k}}
					\left(\symbf{u}\right)
				\end{bmatrix}}_{\substack{1\leq i\leq p\\1\leq k\leq p}}
		\end{math}
		is a multiple of jacobian matrix of $\mathbf{f}$.
	\end{definition}
\end{frame}

\begin{frame}
	\begin{definition}
		The \textcolor{DarkBlue}{\bfseries Cauchy problem} associated
		to~\eqref{eq:systemofconservationlaw} is
		\begin{equation}\label{eq:cauchysystemofconservationlaw}
			\begin{cases}
				\diffp{\symbf{u}}{t}+
				\diffp{}{x}
				\symbf{f}\left(\symbf{u}\right)=
				\symbf{s}\left(\symbf{u}\right) &
				\text{in }\mathbb{R}\times\left(0,\infty\right). \\
				\symbf{u}=\symbf{u}_{0}         &
				\text{on }\mathbb{R}\times\left\{t=0\right\}.
			\end{cases}
		\end{equation}
	\end{definition}
	Let be the states $\symbf{u}_{l},\symbf{u}_{r}\in\Omega$.
	A \textcolor{DarkBlue}{\bfseries Riemann problem} is a Cauchy problem where
	\begin{equation*}
		\symbf{u}_{0}\left(x\right)=
		\begin{cases}
			\symbf{u}_{l}, & x<0. \\
			\symbf{u}_{r}, & x>0.
		\end{cases}
	\end{equation*}

	\pause

	\begin{example}[The system of gas dynamics with gravity and friction]
		\begin{equation*}
			\begin{dcases}
				\diffp{\rho}{t}+
				\diffp{}{x}
				\left(\rho u\right)=0                                                            &
				\text{in }\mathbb{R}\times\left(0,\infty\right).                                   \\
				\diffp{}{t}
				\left(\rho u\right)+
				\diffp{}{x}
				\left(\rho u^{2}+p\right)=\rho\left(g-\alpha\varphi\left(u\right)\right)         &
				\text{in }\mathbb{R}\times\left(0,\infty\right).                                   \\
				\diffp{}{t}
				\left(\rho e\right)+
				\diffp{}{x}
				\left(\left(\rho e+p\right)u\right)=\rho\left(gu-\alpha\psi\left(u\right)\right) &
				\text{in }\mathbb{R}\times\left(0,\infty\right).                                   \\
				\rho=\rho_{0}                                                                    &
				\text{on }\mathbb{R}\times\left\{t=0\right\}.                                      \\
				\rho u={\left(\rho u\right)}_{0}                                                 &
				\text{on }\mathbb{R}\times\left\{t=0\right\}.                                      \\
				\rho e={\left(\rho e\right)}_{0}                                                 &
				\text{on }\mathbb{R}\times\left\{t=0\right\}.
			\end{dcases}
		\end{equation*}
		Where $g$ stands for the gravity force and $\alpha>0$ is a friction constant coefficient.
	\end{example}
\end{frame}

\begin{frame}
	\begin{definition}
		Let be $I\subset\mathbb{R}$ a domain.
		The initial boundary value problem for a conservation law with source term is
		\begin{equation}
			\begin{dcases}
				\diffp{\symbf{u}}{t}+
				\diffp{}{x}\symbf{f}\left(\symbf{u}\right)=
				\symbf{s}\left(\symbf{u}\right) & \text{ in }I\times\left(0,T\right].          \\
				\symbf{u}                                                      =
				\symbf{u}_{0}                   & \text{a.e. on }I\times\left\{t=0\right\}.    \\
				\symbf{u}                                                       =
				\symbf{0}                       & \text{ on }\partial I\times\left[0,T\right].
			\end{dcases}
		\end{equation}
	\end{definition}

	\pause

	\begin{block}{Formulation of PDE model}
		Find
		\begin{math}
			\symbf{u}\in
			\symbf{C}^{1}\left(I\times\left[0,T\right],\Omega\right)
		\end{math}
		in the initial boundary value problem~\eqref{eq:complicatedsystem}
		\begin{equation}\label{eq:complicatedsystem}
			\begin{cases}
				\diffp{\symbf{u}}{t}+\diffp{}{x}\symbf{f}\left(\symbf{u}\right)=
				\symbf{s}\left(\symbf{u}\right) & \text{ in }I\times\left(0,T\right].          \\
				\symbf{u}                                                      =
				\symbf{u}_{0}                   & \text{ on }I\times\left\{t=0\right\}.        \\
				\symbf{u}                                                       =
				\symbf{0}                       & \text{ on }\partial I\times\left[0,T\right].
			\end{cases}
		\end{equation}
		Where
		\begin{math}
			\symbf{u}_{0}\colon I\to
			\mathbb{R}^{3}
		\end{math},
		\begin{math}
			\symbf{s}\colon\Omega\to
			\mathbb{R}^{3}
		\end{math}
		are known and
		\begin{math}
			\symbf{f}\colon\Omega\to
			\mathbb{R}^{3}
		\end{math}
		is given by
		\begin{math}
			\symbf{f}\left(\symbf{u}\right)=
			\symbf{a}\odot\symbf{u}+
			\mathcolor{DarkRed}{a_{1}u_{1}\left(u_{3}-1\right)\symbf{e_{1}}}
		\end{math},
		\begin{math}
			\symbf{s}\left(\symbf{u}\right)=
			\begin{bmatrix}
				b_{1}u_{2}u_{3}\Phi\left(u_{1}\right)-\beta u_{1} \\
				-b_{2}u_{2}u_{3}\Phi\left(u_{1}\right)            \\
				-b_{3}u_{2}u_{3}\Phi\left(u_{1}\right)
			\end{bmatrix}
		\end{math}
		being $\symbf{a}\in\mathbb{R}^{3}\setminus\left\{\symbf{0}\right\}$.
	\end{block}
\end{frame}

\begin{frame}
	We have the problem
	\begin{equation*}
		\diffp{\symbf{u}}{t}+
		\diffp{}{x}\symbf{f}\left(\symbf{u}\right)=
		\symbf{s}\left(\symbf{u}\right)
	\end{equation*}
	where
	\begin{equation*}
		\symbf{u}=
		\begin{bmatrix}
			\theta       \\
			S_{\text{y}} \\
			S_{\text{o}}
		\end{bmatrix},\qquad
		\symbf{f}\left(\symbf{u}\right)=
		\begin{bmatrix}
			a_{1}\theta S_{\text{o}} \\
			a_{2}S_{\text{y}}        \\
			a_{3}S_{\text{o}}
		\end{bmatrix},\qquad
		\symbf{s}\left(\symbf{u}\right)=
		\begin{bmatrix}
			b_{1} \Phi S_{\text{y}}S_{\text{o}}-\beta\theta \\
			-b_{2}\Phi S_{\text{y}}S_{\text{o}}             \\
			-b_{3}\Phi S_{\text{y}}S_{\text{o}}
		\end{bmatrix}.
	\end{equation*}
	Let's see if it's hyperbolic~\cite{Toro2024}.
	We find the Jacobian of $f\left(\symbf{u}\right)$:
	\begin{equation*}
		J\symbf{f}\left(\symbf{u}\right)=
		\begin{bmatrix}
			a_{1}S_{\text{o}} & 0     & a_{1} \theta \\
			0                 & a_{2} & 0            \\
			0                 & 0     & a_{3}
		\end{bmatrix}
	\end{equation*}
	and its eigenvalues are
	\begin{equation*}
		\det\left(J\symbf{f}\left(\symbf{u}\right)-\lambda I\right)=
		0\iff
		\left(a_{1}S_{\text{o}}-\lambda\right)
		\left(c_{2}-\lambda\right)
		\left(a_{3}-\lambda\right)=
		0\iff
		\lambda_{1}=
		a_{1}S_{\text{o}}\wedge
		\lambda_{2}=
		a_{2}\wedge
		\lambda_{3}=
		a_{3}.
	\end{equation*}
	and its eigenvectors are
	\begin{equation*}
		\lambda_{1}=
		a_{1}S_{\text{o}}:
		\begin{bmatrix}
			0 & 0                       & a_{1}                   \\
			0 & a_{2}-a_{1}S_{\text{o}} & 0                       \\
			0 & 0                       & a_{3}-a_{1}S_{\text{o}}
		\end{bmatrix}
		\begin{bmatrix}
			x_{1} \\
			x_{2} \\
			x_{3}
		\end{bmatrix}=
		\begin{bmatrix}
			0 \\
			0 \\
			0
		\end{bmatrix}\implies
		v_{1}=
		\begin{bmatrix}
			1 \\
			0 \\
			0
		\end{bmatrix}.
	\end{equation*}
	Where $a_{1}\theta\neq 0$, $\left(a_{3}-a_{1}S_{\text{o}}\right)\neq 0$.
	% $$
	% 	a_{1} \theta x_{3}=0 ; \quad\left(a_{1}-a_{1}S_{\text{o}}\right) x_{2}=0 ; \quad\left(a_{3}-a_{1}S_{\text{o}}\right) x_{3}=0
	% $$
\end{frame}

\begin{frame}
	\begin{equation*}
		\lambda_{2}=
		a_{2}:
		\begin{bmatrix}
			a_{1}S_{\text{o}}-a_{2} & 0 & a_{1} \theta \\
			0                       & 0 & 0            \\
			0                       & 0 & a_{3}-a_{2}
		\end{bmatrix}
		\begin{bmatrix}
			x_{1} \\
			x_{2} \\
			x_{3}
		\end{bmatrix}=
		\begin{bmatrix}
			0 \\
			0 \\
			0
		\end{bmatrix}\implies
		v_{2}=
		\begin{bmatrix}
			0 \\
			1 \\
			0
		\end{bmatrix}.
	\end{equation*}
	Where $a_{1}S_{\text{o}}-a_{2}\neq0$, $a_{1}\theta\neq 0$.
	% $\left(d_{3}-d_{2}\right) \neq 0:\left(d_{3}-d_{2}\right) x_{3}=0 \quad \eta \quad \begin{gathered}=0 \\ x_{3}=0\end{gathered} \Rightarrow v_{2}=\left(\begin{array}{l}0 \\ 1 \\ 0\end{array}\right)$\\
	% $\left(a_{1} S_{0}^{\prime}-d_{2} F x_{1}+a_{10} x_{3}=0\right.$\\
	\begin{equation*}
		\lambda_{3}=
		a_{3}:
		\begin{bmatrix}
			a_{1} s_{0}-a_{3} & 0           & a_{1} \theta \\
			0                 & a_{2}-a_{3} & 0            \\
			0                 & 0           & 0
		\end{bmatrix}
		\begin{bmatrix}
			x_{1} \\
			x_{2} \\
			x_{3}
		\end{bmatrix}=
		\begin{bmatrix}
			0 \\
			0 \\
			0
		\end{bmatrix}\implies
		v_{3}=
		\begin{bmatrix}
			v \\
			0 \\
			1
		\end{bmatrix},v\in\mathbb{R}.
	\end{equation*}
	$\therefore\left\{v_{1},v_{2},v_{3}\right\}$ form a basis of $\mathbb{R}^{3}$.
	% Where
	% $$
	% 	\begin{aligned}
	% 		 & x_{1}=0 \\
	% 		 & x_{3}=0 \\
	% 		 & 1       \\
	% 		 & 2       \\
	% 		 & 3
	% 	\end{aligned} \Rightarrow v_{2}=\left(\begin{array}{l}
	% 			0 \\
	% 			1 \\
	% 			0
	% 		\end{array}\right)=\left(\begin{array}{l}
	% 			0 \\
	% 			0 \\
	% 			0
	% 		\end{array}\right)
	% $$

	% $\left(a_{2}-a_{3}\right) \neq 0:\left(a_{2}-a_{3}\right) x_{2}=0 \rightarrow x_{2}=0$
\end{frame}

\begin{frame}
	\frametitle{References}

	\nocite{*}
	\printbibliography[heading=none]
\end{frame}

\end{document}
