\question

Responda adecuadamente a los siguientes ítems (justificando su
respuesta).

\begin{parts}
	\part Explique la diferencia entre un estadístico, un parámetro, un
	estimador y una estimación.

	\part

	¿En qué consiste un estimador eficiente?

	\part

	¿En qué consiste un estimador asintóticamente insesgado?

	\part

	¿En qué consiste el método de momentos?
\end{parts}

\begin{solutionordottedlines}
	\begin{parts}
		\part

		Un parámetro es una característica fija de la población, mientras
		que un estadístico es una función de los datos de una muestra.
		Un estimador es la regla o fórmula (estadístico) para aproximar
		el parámetro, y la estimación es el valor concreto que produce el
		estimador al aplicarlo a una muestra.

		\part

		Un estimador eficiente es aquel que, siendo insesgado, tiene la
		varianza más pequeña posible entre todos los estimadores
		insesgados para ese parámetro.
		Esto significa que es el más preciso y con menor error de
		muestreo.

		\part

		Un estimador es asintóticamente insesgado cuando su sesgo
		(diferencia entre su valor esperado y el parámetro) tiende a cero
		a medida que el tamaño de la muestra aumenta indefinidamente.
		En muestras pequeñas puede tener sesgo, pero este desaparece
		asintóticamente.

		\part

		El método de momentos consiste en igualar los momentos muestrales
		(por ejemplo, la media muestral) con los correspondientes
		momentos poblacionales, que son funciones del parámetro a
		estimar.
		Al resolver este sistema de ecuaciones, se obtienen las
		estimaciones de los parámetros.
	\end{parts}
\end{solutionordottedlines}

\question

Suponga que una máquina dispensadora de bebidas gaseosas la cantidad
que envasa es una variable aleatoria $X$ que tiene distribución
normal con media $\mu=10$ onzas y desviación estándar de $\sigma=1$.
Y nos proponemos hacer $25$ mediciones del líquido dispensado.

\begin{parts}
	\part

	Exprese el significado de $\overline{X}$.

	\part

	¿Qué distribución tiene $\overline{X}$?

	\part

	¿Cuál es la probabilidad de que $\overline{X}$ sea por lo menos
	$10.3$?
\end{parts}

\begin{solutionordottedlines}
	\begin{parts}
		\part

		$\overline{X}$ es la cantidad promedio de gaseosas dispensada en
		las $25$ mediciones.

		\part

		De acuerdo con el Teorema~\ref{thm:1}, $\overline{X}$ tiene una
		distribución normal con media $\mu_{\overline{X}}=\mu=10$ y
		varianza
		$\sigma^{2}_{\overline{X}}=\frac{\sigma^{2}}{n}=\frac{1}{25}$.
		\begin{math}
			\therefore
			\overline{X}\sim
			\mathcal{N}\left(10,{\left(\frac{1}{5}\right)}^{2}\right)
		\end{math}.
		\begin{theorem}\label{thm:1}
			Si $X_{1},\dotsc,X_{n}$ es una muestra aleatoria proveniente de
			una población con distribución normal de media $\mu$ y varianza
			$\sigma^{2}$.
			Entonces, $\overline{X}=\frac{1}{n}\sum_{i=1}^{n}X_{i}$ tiene
			distribución normal con media $\mu$ y varianza
			$\frac{\sigma^{2}}{\sqrt{n}}$.
			Esto es,
			\begin{math}
				\overline{X}\sim
				\mathcal{N}
				\left(\mu,\frac{\sigma^{2}}{\sqrt{n}}\right)
			\end{math}.
			Es claro que si
			\begin{math}
				\overline{X}\sim
				\mathcal{N}
				\left(\mu,\frac{\sigma^{2}}{\sqrt{n}}\right)
			\end{math},
			entonces la variable
			\begin{math}
				Z=
				\frac{\overline{X}-\mu}{\frac{\sigma}{\sqrt{n}}}=
				\frac{\sqrt{n}\left(\overline{X}-\mu\right)}{\sigma}\sim
				\mathcal{N}\left(0,1\right)
			\end{math}.
		\end{theorem}

		\part

		\vspace*{-\baselineskip}\setlength\belowdisplayshortskip{0pt}
		\begin{equation*}
			\mathbb{P}
			\left[\overline{X}\geq 10.3\right]=
			\mathbb{P}
			\left[
				\frac{\overline{X}-\mu}{\tfrac{1}{5}}\geq\frac{10.3-10}{\tfrac{1}{5}}
				\right]=
			\mathbb{P}\left[Z\geq 1.5\right]=
			1-\mathbb{P}\left[Z\leq 1.5\right]=
			1-0.933\approx
			0.067.
		\end{equation*}

		\centering
		\includegraphics[width=.57\paperwidth]{normal.pdf}
	\end{parts}
\end{solutionordottedlines}

\question

Se toma una muestra aleatoria de tamaño $n$ de una población con
función de cuantía de probabilidad de Poisson con media $\lambda$.

\begin{parts}
	\part

	Encuentre el estimador de máxima verosimilitud para $\lambda$.

	\part

	Encuentre el valor esperado y la varianza del estimador.
	\part

	¿Es un estimador consistente?
\end{parts}

\begin{solutionordottedlines}
	\begin{parts}
		\part

		La función de densidad de probabilidad de una distribución de
		Poisson está dada por
		\begin{equation*}
			\forall x\in\mathbb{N}\cup\left\{0\right\}:
			\mathbb{P}\left(x,\lambda\right)=
			\frac{e^{-\lambda}\lambda^{x}}{x!}.
		\end{equation*}

		\part

		La función de verosimilitud está dada por
		\begin{align*}
			L\left(\lambda\right)                 & =
			\prod_{i=1}^{n}
			\frac{e^{-\lambda}\lambda^{x_{i}}}{x_{i}!}=
			\left(e^{-\lambda}\right)^{n}
			\frac{\lambda^{\sum_{i=1}^{n}x_{i}}}{\prod_{i=1}^{n}x_{i}!}. \\
			\ln\left[L\left(\lambda\right)\right] & =
			-n\lambda+
			\ln\left(\lambda\right)
			\sum_{i=1}^{n}x_{i}-
			\ln\left(\prod_{i=1}^{n}x_{i}!\right).
		\end{align*}
		Derivando e igualando a cero
		\begin{math}
			\diffp*{\ln\left[L\left(\lambda\right)\right]}{\lambda}=
			-n+\frac{1}{\lambda}\sum_{i=1}^{n}x_{i}=
			0
		\end{math},
		encontramos el estimador de máxima
		verosimilitud
		\begin{align*}
			n                 & =
			\sum_{i=1}^{n}\frac{x_{i}}{\lambda}. \\
			\widehat{\lambda} & =
			\frac{1}{n}\sum_{i=1}^{n}x_{i}=
			\overline{X}.
			\therefore\widehat{\lambda}=\overline{X}.
		\end{align*}

		\part

		Dado que $\widehat{\lambda}=\overline{X}$ y
		$\mathbb{E}\left(X_{i}\right)=\lambda$, entonces
		\begin{math}
			\mathbb{E}\big(\widehat{\lambda}\big)=
			\mathbb{E}\left(\overline{X}\right)=
			\lambda
		\end{math}
		y
		\begin{math}
			V\left(\widehat{\lambda}\right)=
			V\left(\overline{X}\right)=
			\frac{\lambda}{n}
		\end{math}.

		\part

		Sí, es consistente.
		Como es insesgado $\mathbb{E}\big(\widehat{\lambda}\big)=\lambda$,
		el error cuadrático medio es igual a su varianza,
		$V\big(\widehat{\lambda}\big)=\frac{\lambda}{n}$.
		Dado que
		\begin{math}
			\lim_{n\to\infty}
			V
			\big(\widehat{\lambda}\big)=
			0
		\end{math}, el estimador es consistente.
	\end{parts}
\end{solutionordottedlines}

\question

Las tensiones de ruptura de los cables fabricados por una empresa
siguen una distribución normal con media desconocida y $\sigma=120$.
A partir de una muestra de $70$ cables se ha obtenido una tensión
media de ruptura de $2100$ kilos.

\begin{parts}
	\part

	Hallar un intervalo de confianza del 95\% para la tensión media de
	ruptura.

	\part

	¿Qué tamaño debe tener la muestra para obtener un intervalo de
	confianza al 99\% con una amplitud igual a la anterior?
\end{parts}

\begin{solutionordottedlines}
	\begin{parts}
		\part

		La media muestral es $\overline{x}=2100$.
		Si $1-\alpha=0.95$, entonces $\frac{\alpha}{2}=0.025$.
		$\therefore Z_{0.025}\approx 1.96$.

		El intervalo de confianza al $95\%$ para la tensión de ruptura
		media $\mu$ de los cables
		$\operatorname{IC}_{0.95}\left(\mu\right)$ es
		\begin{equation*}
			\left(
			\overline{x}-z_{\frac{\alpha}{2}}\frac{\sigma}{\sqrt{n}},
			\overline{x}+z_{\frac{\alpha}{2}}\frac{\sigma}{\sqrt{n}}
			\right)=
			\left(
			2100-1.96\cdot\frac{120}{\sqrt{70}},
			2100+1.96\cdot\frac{120}{\sqrt{70}}
			\right)\approx
			\left(2071.88,2128.11\right).
		\end{equation*}

		\part

		Si para seleccionar una muestra de tamaño $n$, el nivel de
		confianza se exige que sea del $99\%$.

		Si $1-\alpha=0.99$, entonces $\frac{\alpha}{2}=0.005$.
		$\therefore Z_{0.005}=2.575$.
		Así,
		\begin{math}
			E=
			z_{\frac{\alpha}{2}}
			\frac{\sigma}{\sqrt{n}}=
			1.96\cdot
			\frac{120}{\sqrt{70}}\approx
			28.112
		\end{math}.
		Por tanto, $n$ debe cumplir
		\begin{equation*}
			2.575\cdot
			\frac{120}{\sqrt{n}}\leq
			28.112\implies
			n\geq
			{\left(\frac{2.575\cdot 120}{28.112}\right)}^{2}
			\approx 120.82\implies
			\boxed{n=121}.
		\end{equation*}

		Es decir, debe tomarse una muestra de al menos $121$ cables para
		estimar la resistencia media $\mu$ a la ruptura con las
		condiciones propuestas.
	\end{parts}
\end{solutionordottedlines}
