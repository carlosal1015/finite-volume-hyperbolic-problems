\question

Sea $\Omega$ el espacio muestral.
Demostrar que
\begin{equation*}
	\forall A,B\subset\Omega:
	0\leq
	\mathbb{P}\left(A\cap B\right)\leq
	\mathbb{P}\left(A\right)\leq
	\mathbb{P}\left(A\cup B\right)\leq
	\mathbb{P}\left(A\right)+
	\mathbb{P}\left(B\right)\leq
	2.
\end{equation*}

\begin{solutionordottedlines}
	Sean $A,B\subset\Omega$ dos eventos.
	Tenemos que
	\begin{equation*}
		\begin{aligned}
			A                        & \subseteq A\cup B. \\
			\mathbb{P}\left(A\right) & \leq
			\mathbb{P}\left(A\cup B\right).
		\end{aligned}\quad
		\begin{aligned}
			A                        & \subseteq \Omega. \\
			\mathbb{P}\left(A\right) & \leq
			\mathbb{P}\left(\Omega\right)=1.
		\end{aligned}\quad
		\begin{aligned}
			B                        & \subseteq \Omega. \\
			\mathbb{P}\left(B\right) & \leq
			\mathbb{P}\left(\Omega\right)=1.
		\end{aligned}
	\end{equation*}
	\vspace*{-\baselineskip}\setlength\belowdisplayshortskip{0pt}
	\begin{align*}
		\shortintertext{Luego,}
		\emptyset\subseteq A\cap B                                           & \subseteq A. \\
		0=\mathbb{P}\left(\emptyset\right)\leq\mathbb{P}\left(A\cap B\right) & \leq
		\mathbb{P}\left(A\right).                                                           \\
		\shortintertext{Además,}
		\mathbb{P}\left(A\cup B\right)                                       & =
		\mathbb{P}\left(A\right)+
		\mathbb{P}\left(B\right)-
		\mathbb{P}\left(A\cap B\right)                                                      \\
		\mathbb{P}\left(A\cup B\right)                                       & \leq
		\mathbb{P}\left(A\right)+
		\mathbb{P}\left(B\right)\leq \mathbb{P}\left(\Omega\right)+\mathbb{P}\left(\Omega\right)=1+1=2.
	\end{align*}
	Así, de ambas desigualdades obtenemos las desigualdades pedidas.
	\begin{equation*}
		\forall A,B\subset\Omega:
		0\leq
		\mathbb{P}\left(A\cap B\right)\leq
		\mathbb{P}\left(A\right)\leq
		\mathbb{P}\left(A\cup B\right)\leq
		\mathbb{P}\left(A\right)+
		\mathbb{P}\left(B\right)\leq
		2.
	\end{equation*}
\end{solutionordottedlines}

\question

Sean $X$ e $Y$ dos variables aleatorias discretas e independientes
con valores en $\left\{0,1,2,\dotsc\right\}$ tal que sus funciones
generatrices respectivas existen.
Demostrar que
\begin{math}
	G_{X+Y}
	\left(t\right)=
	G_{X}
	\left(t\right)
	G_{X}
	\left(t\right)
\end{math}.

\begin{solutionordottedlines}
	Suponga que las variables aleatorias $X$ e $Y$ son independientes.
	Además,
	$G_{X}\left(t\right)$ y $G_{Y}\left(t\right)$ existen.
	\begin{align*}
		G_{X}
		\left[t\right]
		G_{Y}
		\left[t\right] & \stackrel{\operatorname{def}}{=}
		\mathbb{E}\left[t^{X}\right]
		\mathbb{E}\left[t^{Y}\right].                     \\
		               & =
		\sum_{j=0}^{\infty}
		\left(
		\mathbb{P}\left[X=j\right]t^{j}
		\right)
		\cdot
		\sum_{k=0}^{\infty}
		\left(
		\mathbb{P}\left[Y=k\right]t^{k}
		\right).                                          \\
		% \shortintertext{Por independencia,
		% 	\begin{math}
		% 		\mathbb{E}\left[t^{X}\right]
		% 		\mathbb{E}\left[t^{X}\right]=
		% 		\mathbb{E}\left[t^{X}t^{Y}\right]
		% 	\end{math}.
		% }
		               & =
		\sum_{j=0}^{\infty}
		\left(
		\sum_{k=0}^{\infty}
		\mathbb{P}\left[X=j\right]
		\mathbb{P}\left[Y=k\right]
		t^{j+k}
		\right).
		\shortintertext{Reindexamos la suma $n=j+k$}
		% \shortintertext{Por el teorema de la convolución para variables independientes}
		               & =
		\sum_{n=0}^{\infty}
		\left(
		\sum_{j=0}^{\infty}
		\mathbb{P}\left[X=j\right]
		\mathbb{P}\left[X=n-j\right]
		t^{n}
		\right).
		\shortintertext{La suma interna es la convolución de las distribuciones de $X$ e $Y$.}
		               & =
		\sum_{n=0}^{\infty}
		\left(
		\mathbb{P}\left[\left(X+Y\right)=n\right]
		t^{n}
		\right).                                          \\
		               & =
		\mathbb{E}
		\left[t^{X+Y}\right].                             \\
		               & \stackrel{\operatorname{def}}{=}
		G_{X+Y}
		\left(t\right).
	\end{align*}
\end{solutionordottedlines}

\question

Una aerolínea muestra la siguiente información (distribución de probabilidades)
sobre el número de salidas con retraso en vuelos nacionales por día $X$
y del número de salidas con retraso en vueltos internacionales por día $Y$

\begin{table}[ht!]
	\centering
	\begin{tabular}{|C|C|C|C|C|C|}
		\hline
		X                                 & 1           & 2             & 3            & 4            & 5 \\
		\hline
		f\left(x\right)=P\left[X=x\right] & \frac{1}{4} & \frac{21}{50} & \frac{3}{20} & \frac{1}{10} & a \\
		\hline
	\end{tabular}
\end{table}

\begin{table}[ht!]
	\centering
	\begin{tabular}{|C|C|C|C|C|C|}
		\hline
		Y                                 & 1              & 2              & 3            & 4            & 5 \\
		\hline
		f\left(y\right)=P\left[X=y\right] & \frac{27}{100} & \frac{37}{100} & \frac{9}{50} & \frac{3}{25} & b \\
		\hline
	\end{tabular}
\end{table}

La empresa implementará cambios en aquel tipo de vuelo, nacional o internacional, cuyo número de retrasos
sea más variable.
¿En qué tipo de vuelo se harán los cambios?

\begin{solutionordottedlines}
	\begin{tabular}{|C|C|C|C|C|C|}
		\hline
		X & f\left(x\right) & Xf\left(x\right) & \left(x-\text{promedio}\right)^{2}f\left(x\right) \\
		\hline
		1 & 0.25            & 0.25             & 0.4489                                            \\
		\hline
		2 & 0.42            & 0.84             & 0.048552                                          \\
		\hline
		3 & 0.15            & 0.45             & 0.06534                                           \\
		\hline
		4 & 0.1             & 0.4              & 0.27556                                           \\
		\hline
		5 & 0.08            & 0.4              & 0.566048                                          \\
		\hline
	\end{tabular}
	\begin{tabular}{|C|C|C|C|C|C|}
		\hline
		Y & f\left(y\right) & Yf\left(y\right) & \left(y-\text{promedio}\right)^{2}f\left(y\right) \\
		\hline
		1 & 0.27            & 0.27             & 0.477603                                          \\
		\hline
		2 & 0.37            & 0.74             & 0.040293                                          \\
		\hline
		3 & 0.18            & 0.54             & 0.080802                                          \\
		\hline
		4 & 0.12            & 0.48             & 0.334668                                          \\
		\hline
		5 & 0.06            & 0.3              & 0.427734                                          \\
		\hline
	\end{tabular}

	\begin{itemize}
		\item

		      El promedio en los vuelos nacionales es $2.34$ y la
		      varianza es $1.4044$.

		\item

		      El promedio en los vuelos internacionales es $2.33$ y la
		      varianza es $1.3611$.
	\end{itemize}

	Se realizarán los cambios en los vuelos nacionales, ya que tiene
	mayor variabilidad.
\end{solutionordottedlines}
