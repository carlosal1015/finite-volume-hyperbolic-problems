\question

Suponga que los resultados de un examen siguen una distribución
normal con media $\mu=78$ y varianza $\sigma^{2}=36$.

\begin{parts}
	\part[2]

	¿Cuál es la probabilidad de que un estudiante obtenga en el examen
	una calificación superior a $72$?

	\part[3]

	Si se sabe que la calificación de un estudiante es mayor que $72$.
	¿Cuál es la probabilidad de que su calificación sea, de hecho,
	superior a $84$?
\end{parts}

Sugerencia:
Use la propiedad de simetría de la distribución normal estándar
\begin{math}
	\mathbb{P}
	\left(
	\left\{
	Z>-z
	\right\}
	\right)=
	\mathbb{P}
	\left(
	\left\{
	Z\leq z
	\right\}
	\right)
\end{math}
y que
\begin{math}
	\mathbb{P}
	\left(
	\left\{
	Z\leq 1
	\right\}
	\right)\approx
	0.8413447460685429
\end{math}.

\begin{solutionordottedlines}
	\begin{parts}
		\part[2]

		La probabilidad de que un estudiante obtenga una calificación
		superior a $72$ es
		\begin{align*}
			\mathbb{P}
			\left(
			\left\{
			X>72
			\right\}
			\right) & =
			\mathbb{P}
			\left(
			\left\{
			X-\mu>
			72-\mu
			\right\}
			\right)=
			\mathbb{P}
			\left(
			\left\{
			\mathcolor{green}{\frac{X-\mu}{\sigma}}>
			\frac{72-78}{6}
			\right\}
			\right)=
			\mathbb{P}
			\left(
			\left\{
			\mathcolor{green}{Z}>-1
			\right\}
			\right).               \\
			        & =
			1-
			\mathbb{P}
			\left(
			\left\{
			Z\leq -1
			\right\}
			\right).               \\
			        & \approx
			1-0.15865525393145707. \\
			\Aboxed{
				\mathcolor{blue}{
					\mathbb{P}
					\left(
					\left\{
					X>72
					\right\}
					\right)
			}       & \approx
				\mathcolor{blue}{0.8413447460685429}.
			}
		\end{align*}

		\part[3]

		La probabilidad condicional de que un estudiante obtenga una
		calificación superior a $84$ dado que obtuvo una calificación
		superior a $72$ es
		\begin{align*}
			\mathbb{P}
			\left(
			\left\{
			X>84
			\right\}
			\mid
			\left\{
			X>72
			\right\}
			\right) & =
			\frac{
				\mathbb{P}
				\left(
				\left\{
				X>84
				\right\}\cap
				\left\{
				X>72
				\right\}
				\right)}{
				\mathbb{P}
				\left(
				\left\{
				X>72
				\right\}
				\right)
			}=
			\frac{
				\mathcolor{red}{
					\mathbb{P}
					\left(
					\left\{X>84\right\}
					\right)
				}
			}{
				\mathcolor{blue}{
					\mathbb{P}
					\left(
					\left\{X>72\right\}
					\right)
				}
			}.
			\shortintertext{Pero, }
			\mathbb{P}
			\left(
			\left\{
			X>84
			\right\}
			\right) & =
			\mathbb{P}
			\left(
			\left\{
			X-\mu>84-\mu
			\right\}
			\right)=
			\mathbb{P}
			\left(
			\left\{
			\mathcolor{green}{\frac{X-\mu}{\sigma}}>
			\frac{84-78}{6}
			\right\}
			\right)=
			\mathbb{P}
			\left(
			\left\{\mathcolor{green}{Z}>1\right\}
			\right).                                                                           \\
			        & =
			1-
			\mathbb{P}
			\left(
			\left\{Z\leq 1\right\}
			\right).                                                                           \\
			        & \approx
			1-0.8413447460685429.                                                              \\
			\Aboxed{
				\mathcolor{red}{
					\mathbb{P}
					\left(
					\left\{
					X>84
					\right\}
					\right)
			}       & \approx
				\mathcolor{red}{0.15865525393145707}.
			}                                                                                  \\
			\therefore
			\mathbb{P}
			\left(
			\left\{X>84\right\}\mid
			\left\{X>72\right\}
			\right) & =
			\frac{
				\mathcolor{red}{
					\mathbb{P}
					\left(
					\left\{X>84\right\}
					\right)
				}
			}{
				\mathcolor{blue}{
					\mathbb{P}
					\left(
					\left\{X>72\right\}
					\right)}
			}.                                                                                 \\
			        & \approx
			\frac{\mathcolor{red}{0.15865525393145707}}{\mathcolor{blue}{0.8413447460685429}}. \\
			\Aboxed{
				\mathbb{P}
				\left(
				\left\{X>84\right\}\mid
				\left\{X>72\right\}
			\right) & \approx
				0.18857341734506025.
			}
		\end{align*}
	\end{parts}
\end{solutionordottedlines}

\question

Sea
\begin{math}
	A=
	\left\{
	\left(x,y\right)\in\mathbb{R}^{2}\mid
	0<y<x<1
	\right\}
\end{math}
y la función
\begin{math}
	f_{X,Y}
	\left(x,y\right)=
	k\cdot
	\mathbf{1}_{A}
	\left(x,y\right)=
	\begin{cases}
		k, & \left(x,y\right)\in A.                        \\
		0, & \left(x,y\right)\in\mathbb{R}^{2}\setminus A.
	\end{cases}
\end{math}

\begin{parts}
	\part[2]

	Determine el valor de $k\in\mathbb{R}\setminus\left\{0\right\}$
	para que $f_{X,Y}$ sea una función de densidad de una variable
	aleatoria bidimensional $\left(X,Y\right)$.

	\part[1]

	Hallar las funciones de densidad marginales.
	¿Son $X$ e $Y$ independientes?

	\part[1]

	Hallar las funciones de distribución marginales.

	\part[1]

	Hallar las funciones de densidad condicionadas.
\end{parts}

\begin{solutionordottedlines}
	\begin{parts}
		\part[2]

		Una función de densidad $f_{X,Y}\left(x,y\right)$ verifica
		\begin{align*}
			1            & =
			\iint_{\mathbb{R}^{2}}
			f_{X,Y}\left(x,y\right)
			\dl x\dl y=
			\iint_{\mathbb{R}^{2}}
			k\cdot
			\mathbf{1}_{A}
			\left(x,y\right)
			\dl x\dl y=
			k
			\iint_{A}
			\dl x\dl y=k
			\int_{0}^{1}
			\int_{0}^{x}
			\dl x
			\dl y.           \\
			1            & =
			k
			\int_{0}^{1}
			\mathcolor{red}{
				\left(
				\int_{0}^{x}\dl y
				\right)
			} \dl x=
			k
			\int_{0}^{1}
			\mathcolor{red}{y\big|^{x}_{0}}
			\dl x=
			k
			\int_{0}^{1}
			\mathcolor{red}{x}
			\dl x=
			k
			\frac{x^{2}}{2}\bigg|^{1}_{0}=
			\frac{k}{2}.     \\
			\Aboxed{
			\therefore 2 & =
				k.
			}
		\end{align*}

		\part[1]

		Sean las funciones de densidad marginales
		\begin{align*}
			\forall x\in\left(0,1\right):
			f_{X}\left(x\right) & \stackrel{\operatorname{def}}{=}
			\int_{\mathbb{R}}
			f_{X,Y}\left(x,y\right)
			\dl y=
			\int_{\mathbb{R}}
			k\cdot
			\mathbf{1}_{A}
			\left(x,y\right)\dl y=
			2
			\mathcolor{red}{
				\int_{0}^{x}
				\dl y
			}=
			2\mathcolor{red}{y\big|^{x}_{0}}=
			2x.                     \\
			\forall y\in\left(0,1\right):
			f_{Y}\left(y\right) & \stackrel{\operatorname{def}}{=}
			\int_{\mathbb{R}}
			f_{X,Y}\left(x,y\right)
			\dl x=
			\int_{\mathbb{R}}
			k\cdot
			\mathbf{1}_{A}
			\left(x,y\right)\dl x=
			2
			\mathcolor{red}{
				\int_{y}^{1}
				\dl x
			}=
			2\mathcolor{red}{x\big|^{1}_{y}}=
			2-2y.
		\end{align*}
		$X$ e $Y$ no son variables aleatorias independientes porque
		$\exists\left(x,y\right)\in A$ tal que
		\begin{align*}
			f_{X}
			\left(x,y\right)\cdot
			f_{Y}
			\left(x,y\right)=
			2x\cdot\left(2-2y\right)
			\neq
			2\cdot\mathbf{1}_{A}
			\left(x,y\right)=
			f_{X,Y}\left(x,y\right).
		\end{align*}

		\part[1]

		Sean las funciones de distribución marginales
		\begin{align*}
			\forall x\in\left(0,1\right):
			f_{X}\left(x\right) & \stackrel{\operatorname{def}}{=}
			\int_{-\infty}^{x}
			f_{X}\left(t\right)\dl t=
			\int_{0}^{x}
			f_{X}\left(t\right)\dl t=
			\int_{0}^{x}
			2t\dl t=
			x^{2}.                                                 \\
			\forall y\in\left(0,1\right):
			f_{Y}\left(y\right) & \stackrel{\operatorname{def}}{=}
			\int_{-\infty}^{y}
			f_{Y}\left(t\right)\dl t=
			\int_{0}^{y}
			f_{Y}\left(t\right)\dl t=
			\int_{0}^{y}
			\left(2-2t\right)\dl t=
			2y-y^{2}.
		\end{align*}

		\part[1]

		Sean las funciones de densidad condicionadas
		\begin{align*}
			\forall\left(x,y\right)\in A:
			f_{X\mid Y}\left(x\mid y\right) & \stackrel{\operatorname{def}}{=}
			\frac{f_{X,Y}\left(x,y\right)}{f_{Y}\left(y\right)}=
			\frac{2}{2-2y}=
			\frac{1}{1-y}.                                                     \\
			\forall\left(x,y\right)\in A:
			f_{Y\mid X}\left(y\mid x\right) & \stackrel{\operatorname{def}}{=}
			\frac{f_{X,Y}\left(x,y\right)}{f_{X}\left(x\right)}=
			\frac{2}{2x}=
			\frac{1}{x}.
		\end{align*}
	\end{parts}

	\begin{center}
		\includegraphics[width=.75\paperwidth]{2final}
	\end{center}
\end{solutionordottedlines}
