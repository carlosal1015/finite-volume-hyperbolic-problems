\question[10]

Determine
\begin{math}\displaystyle
    \diff{}{x}
    \left(
    \int_{3x-1}^{0}
    \frac{\dl t}{t+4}
    \right).
\end{math}

% \begin{figure}[ht!]
%     \centering
%     \begin{tikzpicture}[xscale=0.5,yscale=0.5]
%         \draw[help lines] (0,0) grid (32,40);
%     \end{tikzpicture}
% \end{figure}

\begin{solutionordottedlines}

    Defina las siguientes funciones y calcule sus derivadas
    \begin{equation}
        \begin{split}
            f\left(x\right) & \coloneqq
            \int_{0}^{x}
            \frac{\dl t}{t+4}.          \\
            g\left(x\right) & \coloneqq
            3x-1.
        \end{split}
        \xRightarrow{\quad\text{Teorema Fundamental del Cálculo}\quad}
        \begin{split}
            \diff{}{x}f\left(x\right) & =
            \frac{1}{x+4}.                \\
            \diff{}{x}g\left(x\right) & =
            3.
        \end{split}
    \end{equation}
    Note que
    \begin{align*}
        \left(f\circ g\right)\left(x\right) & =
        f\left(3x-1\right)=
        \int_{0}^{3x-1}
        \frac{\dl t}{t+4}.
        \shortintertext{Aplique la regla de la cadena}
        \diff{}{x}
        \left(f\circ g\right)\left(x\right)
                                            & =
        \diff{}{x}f\left(g\left(x\right)\right)
        \diff{}{x}g\left(x\right)=
        \frac{3}{3x-1+4}=
        \frac{1}{x+1}.
        \shortintertext{Así,}
        -\frac{1}{x+1}=
        -\diff{}{x}
        \left(f\circ g\right)\left(x\right)
                                            & =
        -\diff{}{x}
        \left(
        \int_{0}^{3x-1}
        \frac{\dl t}{t+4}
        \right)=
        \diff{}{x}
        \left(
        \int_{3x-1}^{0}
        \frac{\dl t}{t+4}
        \right).
    \end{align*}
\end{solutionordottedlines}