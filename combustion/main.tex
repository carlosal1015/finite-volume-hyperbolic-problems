% arara: clean: {
% arara: --> extensions:
% arara: --> ['aux', 'bbl', 'bcf', 'blg', 'log', 'nav', 'out', 'run.xml', 'snm', 'pdf', 'toc']
% arara: --> }
% arara: lualatex: {
% arara: --> shell: yes,
% arara: --> draft: yes,
% arara: --> interaction: batchmode
% arara: --> }
% arara: biber
% arara: lualatex: {
% arara: --> shell: yes,
% arara: --> draft: no,
% arara: --> interaction: batchmode
% arara: --> }
% arara: lualatex: {
% arara: --> shell: yes,
% arara: --> draft: no,
% arara: --> interaction: batchmode
% arara: --> }
% arara: clean: {
% arara: --> extensions:
% arara: --> ['aux', 'bbl', 'bcf', 'blg', 'log', 'nav', 'out', 'run.xml', 'snm', 'toc']
% arara: --> }
\documentclass[
  aspectratio=1610,
  c,
  handout,
  9pt,
  spanish
]{beamer}
\usepackage[spanish]{babel}
\usepackage{mathtools}
\usepackage{diffcoeff}
\usepackage[
	citestyle=numeric,
	style=numeric,
	backend=biber,
]{biblatex}
\addbibresource{references.bib}

\usefonttheme[onlymath]{serif}
\setbeamertemplate{navigation symbols}{}
\setbeamertemplate{headline}{}
\setbeamertemplate{footline}[frame number]{}
\setbeamertemplate{caption}[numbered]
\setbeamertemplate{frametitle}[default][center]

\title{Solución de ondas viajeras para un modelo de combustión in situ simple}
\subtitle{}
\institute{}
\providecommand{\name}{Nombre}
\author{\scshape\name}
\date{\today}

\begin{document}
\begin{frame}
    \maketitle
\end{frame}

\section{¿Qué es la combustión in situ?~\cite{Storey2022}}

\begin{frame}
    \frametitle{\secname}

    Es una técnica de \alert{recuperación mejorada de petróleo} (EOR)
    que implica inyectar aire u oxígeno en un yacimiento petrolífero
    para encender el petróleo y crear un frente de combustión
    controlada que desplaza el petróleo hacia los pozos de
    producción.
    Las zonas de reacción son
    \begin{description}
        \item[Oxidación a baja temperatura ($25-300^{\circ}$C)]

              Oxidación sin CO$_{2}$ (alcoholes, cetonas).

        \item[Oxidación a temperatura media ($300-350^{\circ}$C)]

              Deposición de coque.

        \item[Oxidación a alta temperatura ($>350^{\circ}$C)]
              Combustión del coque, generando CO$_{2}$ y calor.
    \end{description}

    % https://cdn-ceo-ca.s3.amazonaws.com/1htavbg-geosciences-12-00340-v2.pdf
    \begin{figure}[ht!]
        \centering
        \includegraphics{schematicdiagram}
        \caption{Diagrama esquemático que muestra las zonas de
            reacción en todo el yacimiento entre las zonas de pre y post
            quema.
            Recuperado de~\cite{Storey2022}.}
    \end{figure}
\end{frame}

\section{Modelo de Buckley-Leverett~\cite{booth2008a}}

\begin{frame}
    \frametitle{\secname}

    Modela el flujo inmiscible a través de medios porosos, como el
    desplazamiento del petróleo por el agua.
    Las fracciones de espacio poroso que están ocupadas
    (saturaciones) por petróleo y agua son $S_{\text{oil}}$ y
    $S_{\text{water}}$, respectivamente.
    Los flujos de petróleo y agua se denotan por $u_{\text{oil}}$ y
    $u_{\text{water}}$ respectivamente y la presión en el petróleo y
    el agua se da por $p_{\text{oil}}$ y $p_{\text{water}}$.

    Despreciando el efecto de la gravedad para simplificar, este
    efecto se puede modelar mediante la introducción de
    permeabilidades relativas de modo que en cada fluido
    \begin{align}
        u_{\text{oil}}   & =
        -\frac{k_{\text{rel oil}}}{\mu_{\text{oil}}}
        K\nabla p_{\text{oil}}. \\
        u_{\text{water}} & =
        -\frac{k_{\text{rel water}}}{\mu_{\text{water}}}
        K\nabla p_{\text{water}}.
    \end{align}
    donde las permeabilidades relativas, $k_{\text{rel oil}}$ y
    $k_{\text{rel water}}$ se determinan empíricamente y dependen de
    la fracción de la fase acuosa presente, $S_{\text{water}}$.
    La conservación de la masa aplicada al petróleo y al agua da
    \begin{align}
        \phi
        \diffp{S_{\text{water}}}{t}+
        \nabla\cdot u_{\text{water}} & =0. \\
        \phi
        \diffp{S_{\text{oil}}}{t}+
        \nabla\cdot u_{\text{oil}}   & =0.
    \end{align}
    donde $\phi$ es la porosidad de la roca.
\end{frame}

\begin{frame}
    \frametitle{\secname}

    La diferencia entre la presión en la fase acuosa y la presión en
    la fase oleosa viene dada por la presión capilar.
    \begin{equation}
        p_{c}\left(S_{\text{water}}\right)=
        p_{\text{oil}}-p_{\text{water}}.
    \end{equation}
    La presión capilar es una consecuencia de la tensión superficial
    y, para un poro individual, lleno de agua, la presión capilar
    podría determinarse mediante
    \begin{equation}
        p_{c}=
        \frac{2\gamma\cos\theta}{r\left(S_{\text{water}}\right)},
    \end{equation}
    donde $\gamma$ es el coeficiente de tensión superficial, $\theta$
    es el ángulo de humectación en el punto donde se encuentran el
    petróleo, el agua y la roca, y $r\left(S_{\text{water}}\right)$
    es un radio típico para un poro en el que están presentes tanto
    el agua como el petróleo.
    \begin{equation}
        \diffp{S_{\text{water}}}{t}+
        U\diffp{F\left(S_{\text{water}}\right)}{x}=0,
    \end{equation}
    donde la función de flujo fraccional es
    \begin{equation*}
        F\left(S_{\text{water}}\right)=
        {
        \left(
        1+
        \frac{
            k_{\text{rel oil}}\left(S_{\text{water}}\right)
            \mu_{\text{water}}
        }{
            k_{\text{rel water}}\left(S_{\text{water}}\right)
            \mu_{\text{oil}}
        }
        \right)}^{-1}.
    \end{equation*}
\end{frame}

\section{Modelo de Koval~\cite{booth2008a,Koval1963}}

\begin{frame}
    \frametitle{\secname}

    Se ha observado en experimentos que, en un desplazamiento
    miscible a través de un canal, la evolución de la concentración
    de disolvente es muy diferente a la predicha por el modelo
    unidimensional de Peaceman.
    La causa de esta diferencia es la ``digitación'' del disolvente.
    Koval derivó un modelo unidimensional simple a partir de las
    ecuaciones de Buckley-Leverett y viene dado por
    \begin{equation}
        \diffp{c}{t}+U\diffp{F\left(c\right)}{x}=0,
    \end{equation}
    donde la función de flujo fraccional es
    \begin{equation*}
        F\left(c\right)=
        \frac{M_{e}c}{1+\left(M_{e}-1\right)c}
    \end{equation*}
    y la relación de movilidad efectiva, $M_{e}$, se define como la
    relación entre la viscosidad del petróleo y una mezcla específica
    de solvente y petróleo con una fracción de volumen de solvente $c_{e}$.
    Al comparar con experimentos, Koval determinó que $c_{e}=0.22$.
    % \begin{align}
    %     f_{s}\left(c_{s},T\right) & =
    %     \frac{Kc_{s}}{1+c_{s}\left(K-1\right)}.    \\
    %     K\left(T\right)           & =
    %     \frac{\mu_{\text{oil}}}{\mu_{\text{mix}}}. \\
    %     \mu_{\text{mix}}          & =+.
    % \end{align}
\end{frame}

\section{Modelo de Gagar~\cite{Gargar2020}}

\begin{frame}
    \frametitle{\secname}

    También conocido como modelo de Koval modificado sujeto a un
    desplazamiento miscible para inyección de aire a alta presión.

    Asumiendo que la temperatura de la roca sólida, el líquido y el gas son iguales,
    escribimos la ecuación de balance de calor como
    \begin{equation*}
        \diffp{}{t}
        \left(
        C_{m}\Delta T+
        \varphi C_{0}\Delta T\right)
        \diffp{}{x}
        \left(C_{o} u\Delta T\right)=
        \lambda
        \diffp[2]{T}{x}+
        QR,
    \end{equation*}
    donde $\Delta T=T-T_{\text{ini}}$ es la temperatura relativa al reservorio inicial,
    $\lambda$ es la conductividad térmica, $Q$ es el calor de reacción de combustión y $C_{m}$
    es la matriz de la capacidad calorífica de la roca.
    La tasa de reacción $R$ es
    \begin{equation*}
        R=\varphi A_{r}c_{\text{oil}}c^{n}_{\text{ox}}
        \exp\left(-\frac{T_{\text{ac}}}{T}\right),
    \end{equation*}

    donde $A_{r}$ es el factor de frecuencia para la tasa de oxidación del petróleo y $n$ es
    el orden de la reacción con respecto al oxígeno.
    % https://sci-hub.se/10.1007/s10596-020-09977-y
\end{frame}

\begin{frame}
    \frametitle{\secname}
    \framesubtitle{Recuperado de~\url{https://alexei.impa.br/data/_uploaded/file/notes/lncc_interpore.pdf}}
    \begin{align}
        \varphi
        \diffp{c_{s}}{t}+
        \diffp{}{x}
        \left(uf_{s}\right)         & =
        \left(\nu_{in}-\nu_{ox}\right)R \\
        \varphi
        \diffp{c_{oil}}{t}+
        \diffp{}{x}
        \left(uf_{oil}\right)       & =
        -\nu_{oil}R,                    \\
        \varphi
        \diffp{c_{ox}}{t}+
        \diffp{}{x}
        \left(uf_{ox}\right)        & =
        -\nu_{ox}R,                     \\
        \diffp{}{t}
        \left(C_{m}\Delta T+\varphi C_{o}\Delta T\right)+
        \diffp{}{x}
        \left(C_{o}u\Delta T\right) & =
        \lambda
        \diffp[2]{T}{x}+QR
    \end{align}

    donde $f_{oil}=1-f_{s}$ y $f_{ox}=\frac{c_{ox}}{c_s}f_{s}$.
\end{frame}

\begin{frame}\transblindsvertical
    \frametitle{Referencias}

    \nocite{*}
    \printbibliography[heading=none]
\end{frame}

\end{document}