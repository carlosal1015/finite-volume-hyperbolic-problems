% arara: clean: {
% arara: --> extensions:
% arara: --> ['aux', 'bbl', 'bcf', 'blg', 'log', 'nav', 'out', 'pdf', 'run.xml', 'snm', 'toc']
% arara: --> }
% arara: lualatex: {
% arara: --> shell: yes,
% arara: --> draft: yes,
% arara: --> interaction: batchmode
% arara: --> }
% arara: biber
% arara: lualatex: {
% arara: --> shell: yes,
% arara: --> draft: no,
% arara: --> interaction: batchmode
% arara: --> }
% arara: lualatex: {
% arara: --> shell: yes,
% arara: --> draft: no,
% arara: --> interaction: batchmode
% arara: --> }
% arara: clean: {
% arara: --> extensions:
% arara: --> ['aux', 'bbl', 'bcf', 'blg', 'log', 'nav', 'out', 'run.xml', 'snm', 'toc']
% arara: --> }
%% filename: amsart-template.tex
%% version: 1.1
%% date: 2014/07/24
%%
%% American Mathematical Society
%% Technical Support
%% Publications Technical Group
%% 201 Charles Street
%% Providence, RI 02904
%% USA
%% tel: (401) 455-4080
%%      (800) 321-4267 (USA and Canada only)
%% fax: (401) 331-3842
%% email: tech-support@ams.org
%% 
%% Copyright 2008-2010, 2014 American Mathematical Society.
%% 
%% This work may be distributed and/or modified under the
%% conditions of the LaTeX Project Public License, either version 1.3c
%% of this license or (at your option) any later version.
%% The latest version of this license is in
%%   http://www.latex-project.org/lppl.txt
%% and version 1.3c or later is part of all distributions of LaTeX
%% version 2005/12/01 or later.
%% 
%% This work has the LPPL maintenance status `maintained'.
%% 
%% The Current Maintainer of this work is the American Mathematical
%% Society.
%%
%% ====================================================================

%     AMS-LaTeX v.2 template for use with amsart
%
%     Remove any commented or uncommented macros you do not use.

\documentclass[11pt,oneside,a4paper,final]{amsart}

\usepackage{commath}
\usepackage{mdframed}

\usepackage[backend=biber,style=alphabetic]{biblatex} % backend=bibtex works too
\bibliography{Bibliography}

\usepackage[bookmarksopen=true,bookmarksnumbered=true]{hyperref}
\hypersetup{
  colorlinks   = true, %Colours links instead of ugly boxes
  urlcolor     = blue, %Colour for external hyperlinks
  linkcolor    = blue, %Colour of internal links
  citecolor    = blue %Colour of citations
}

\newtheoremstyle{bold}              	 %Name
  {}                                     %Space above
  {}                                     %Space below
  {\itshape}		                     %Body font
  {}                                     %Indent amount
  {\scshape}                             %Theorem head font
  {.}                                    %Punctuation after theorem head
  { }                                    %Space after theorem head, ' ', 
  										 %	or \newline
  {} 
\theoremstyle{bold}
\newtheorem*{definition*}{Definition}
\newtheorem{definition}{Definition}[section]
\newtheorem*{lemma*}{Lemma}
\newtheorem{lemma}{Lemma}[section]
\newtheorem{Proof}{Proof}[section]
\newtheorem{proposition}{Proposition}[section]
\newtheorem{properties}{Properties}[section]
\newtheorem{corollary}{Corollary}[section]
\newtheorem*{theorem*}{Theorem}
\newtheorem{theorem}{Theorem}[section]
\newtheorem{example}{Example}[section]
\newtheorem*{remark*}{Remark}
\newtheorem{remark}{Remark}[section]

% \newtheorem{theorem}{Theorem}[section]
% \newtheorem{lemma}[theorem]{Lemma}

% \theoremstyle{definition}
% \newtheorem{definition}[theorem]{Definition}
% \newtheorem{example}[theorem]{Example}
% \newtheorem{xca}[theorem]{Exercise}

% \theoremstyle{remark}
% \newtheorem{remark}[theorem]{Remark}

% \numberwithin{equation}{section}

\begin{document}

\title{Classical Fourier Analysis: Interpolation of $L^p$ Spaces}

%    Remove any unused author tags.

%    author one information
\author{Yannis B\"{a}hni}
\address[Yannis B\"{a}hni]{University of Zurich, R\"{a}mistrasse 71, 8006 Zurich}
\curraddr{}
\email[Yannis B\"{a}hni]{\href{mailto:yannis.baehni@uzh.ch}{yannis.baehni@uzh.ch}}
\thanks{}

%    author two information
% \author{}
% \address{}
% \curraddr{}
% \email{}
% \thanks{}

\subjclass[2010]{Primary }

\keywords{}

\date{}

\dedicatory{}

\begin{abstract}
    In this written seminar work I will basically follow the section
    \emph{Interpolation} in the third edition of the book \emph{Classical Fourier Analysis} by Loukas Grafakos.
    I will review three basic but important theorems on interpolation of operators on $L^p$ spaces, starting with
    the \emph{Riesz-Thorin Interpolation Theorem} based on complex analysis, its generalization,
    the \emph{Stein-Weiss Interpolation Theorem of Analytic Families of Operators} and finally a theorem
    based on real methods, the \emph{Marcinkiewicz Interpolation Theorem}.
    We are mainly concerned with the notion of linear operators as well as slight
    generalizations of them.
\end{abstract}

\maketitle

\tableofcontents

\section{Introduction and Basic Definitions}
Suppose $\del{p_0,q_0},\del{ p_1,q_1} \in \intcc{1,\infty} \times \intcc{1,\infty}$ are two pairs of indices and assume that the estimates

\begin{equation*}
    \norm[0]{T(f)}_{L^{q_0}} \leq M_0 \norm[0]{f}_{L^{p_0}} \quad \text{and} \quad \norm[0]{T(f)}_{L^{q_1}} \leq M_1\norm{f}_{L^{p_1}}
\end{equation*}

hold, where $T$ is an appropriately choosen operator. Does this imply that

\begin{equation*}
    \norm[0]{T(f)}_{L^q} \leq M \norm{f}_{L^p} \quad \text{for other pairs $\del{p,q} \in \intcc{1,\infty}$?}
\end{equation*}

Those and similar questions will be answered by a tool called \emph{interpolation}, in our case interpolation of $L^p$ spaces. Using interpolation it is possible to reduce difficult estimates to endpoint estimates and so interpolation can (but not always does) simplify matters. Among the numerous applications of interpolation is by far the shortest proof of \emph{Young's inequality for convolutions} \textup{\cite[22--23]{grafakos:fourier:2014}}. However, there is not \emph{the} interpolation theorem, merely a family of theorems which can be
roughly divided into two main categories: \emph{real} and \emph{complex} interpolation methods. Real methods use so called \emph{cut-off} functions to divide the functions in the domain of the operator $T$ into a bounded and unbounded part and then establish bounds on each of those parts whereas complex interpolation theorems are based upon standard results in complex analysis and are more restrictive on the operator $T$ in question but yield more natural bounds (even continuous estimates) and will therefore be considered in this task. First we need a rigorous idea of what an appropriately choosen operator means in the context of Lebesgue spaces. For simplicity we may assume that any measure will be complete.

\begin{mdframed}
    \begin{definition}
        Let $\del{X,\mu}$ and $\del{Y,\nu}$ be measure spaces. Further let $T$ be an operator defined on a linear space of complex-valued measurable functions on $X$ and taking values in the set of all complex-valued, finite almost everywhere, measurable functions on $Y$. Then $T$ is called \emph{linear} if for all functions $f$ and $g$ in the domain of $T$ and all $z \in \mathbb{C}$

        \begin{equation}
            T(f + g) = T(f) + T(g) \qquad T(zf) = zT(f)
            \label{eq:linear}
        \end{equation}

        \noindent holds and \emph{quasi-linear} if

        \begin{equation}
            \abs[0]{T(f + g)} \leq K \del[0]{\abs[0]{ T(f)} + \abs[0]{ T(g)}} \qquad \abs[0]{ T(zf) } = \abs{z}\abs[0]{ T(f)}
            \label{eq:quasilinear}
        \end{equation}

        \noindent holds for some real constant $K > 0$. If $K = 1$, $T$ is called \emph{sublinear}.
    \end{definition}
\end{mdframed}

\appendix
\begin{appendix}
    \section{Limit superior and limit inferior revisited}
    \begin{definition}
        Let $(X,d)$ a metric space, $E \subseteq X$, $f: E \to \mathbb{R}$ and $a \in X$ be a limit point of $E$. Then we define the \emph{upper limit of  $f$ at $a$} as

        \begin{equation*}
            \limsup_{x \to a}f(x) := \lim_{\varepsilon \searrow 0} \sbr[0]{ \sup\cbr[0]{ f(x) : x \in E \cap \dot{B}_\varepsilon(a)}}
        \end{equation*}

        \noindent and the \emph{lower limit of  $f$ at $a$} as

        \begin{equation*}
            \liminf_{x \to a}f(x) := -\limsup_{x \to a}( -f)(x)
        \end{equation*}
    \end{definition}

    \begin{proposition}
        Let $(X,d)$ a metric space, $E \subseteq X$, $f,g: E \rightarrow \mathbb{R}$, where $f$ is bounded and $a \in X$ be a limit point of $E$. Then

        \begin{equation*}
            \limsup_{x \to a} (fg)(x) = \limsup_{x \to a} f(x) \lim_{x \to a} g(x)
        \end{equation*}

        \noindent whenever both sides exist and $\lim_{x \to a}g(x) \geq 0$.
        \label{prop:limsup}
    \end{proposition}

    \begin{proof}
        Write

        \begin{equation*}
            fg = f\lim_{x\to a}g(x) + f\sbr[2]{ g - \lim_{x \to a}g(x)}.
        \end{equation*}

        By \cite[358]{bourbaki:general_topology:1995} we have

        \begin{gather*}
            \begin{aligned}
                \limsup_{x \to a} (fg)(x) & = \limsup_{x \to a}\del[2]{ f(x)\lim_{x\to a}g(x) + f(x)\sbr[2]{ g(x) - \lim_{x \to a}g(x)}}                          \\
                                          & = \limsup_{x \to a}\del[2]{ f(x)\lim_{x \to a}g(x)} + \lim_{x \to a}\del[2]{ f(x)\sbr[2]{ g(x) - \lim_{x \to a}g(x)}} \\
                                          & = \limsup_{x \to a}\del[2]{ f(x) \lim_{x \to a}g(x)}
            \end{aligned}
        \end{gather*}

        \noindent since $\lim_{x \to a}\sbr[0]{ g(x) - \lim_{x \to a} g(x)} = 0$ and $f$ is bounded. Fix $\varepsilon > 0$. By \cite[357]{bourbaki:general_topology:1995} and $\lim_{x \to a}g(x) \geq 0$ we get

        \begin{equation*}
            \sup\cbr[2]{ f(x)\lim_{x \to a}g(x) : x \in E \cap \dot{B}_\varepsilon(a)} = \sup\cbr[0]{ f(x) : x \in E \cap \dot{B}_\varepsilon(a)}\lim_{x \to a}g(x)
        \end{equation*}

        Hence

        \begin{equation*}
            \begin{aligned}
                \limsup_{x \to a} (fg)(x) & = \limsup_{x \to a}\del[2]{ f(x) \lim_{x \to a}g(x) }                                                                \\
                                          & = \lim_{\varepsilon \searrow 0} \sbr[2]{ \sup\cbr[2]{ f(x)\lim_{x \to a}g(x) : x \in E \cap \dot{B}_\varepsilon(a)}} \\
                                          & = \lim_{\varepsilon \searrow 0} \sbr[0]{ \sup\cbr[0]{ f(x) : x \in E \cap \dot{B}_\varepsilon(a)}}\lim_{x \to a}g(x) \\
                                          & = \limsup_{x \to a}f(x) \lim_{x \to a} g(x)
            \end{aligned}
        \end{equation*}
    \end{proof}

    \section{Measure Theory}
    The following statement can be found in \cite[27]{folland:real_analysis:1999}. We give a proof.

    \begin{lemma}
        Let $(X,\mu)$ be a measure space. If $\mu$ is $\sigma$-finite then it is also semifinite.
    \end{lemma}

    \begin{proof}
        Let $X = \bigcup_{n \in \mathbb{N}} X_n$ where $\mu(X_n) < \infty$ and $E$ is measurable with $\mu(E) = \infty$. By letting $Y_n := \bigcup_{k \leq n} X_k$, $Y_n$ is an increasing sequence. Then $E \cap Y_n$ is measurable and since $E \cap Y_n \subseteq Y_n$, $\mu(E \cap Y_n) < \infty$ for each $n \in \mathbb{N}$. By the continuity from below (see \cite[26]{folland:real_analysis:1999}) we have

        \begin{equation*}
            \infty = \mu(E) = \mu(E \cap X) = \mu\del[4]{E \cap \del[4]{\bigcup_{n \in \mathbb{N}} Y_n}} = \mu\del[4]{\bigcup_{n \in \mathbb{N}} \del[0]{ E \cap Y_n}} = \lim\limits_{n \rightarrow \infty} \mu( E \cap Y_n )
        \end{equation*}

        Hence for every $C > 0$ there exists $N \in \mathbb{N}$, such that $\infty > \mu(E \cap Y_n) > C$ for $n > N$.
    \end{proof}

\end{appendix}

\printbibliography

\end{document}

%-----------------------------------------------------------------------
% End of amsart-template.tex
%-----------------------------------------------------------------------
