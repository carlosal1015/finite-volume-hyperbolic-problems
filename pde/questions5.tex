\question

Sea $S=\left(0,\pi\right)^{2}$.
Resuelva la ecuación de onda 2D
\begin{equation*}
	\begin{cases}
		\difc.A.{}{}u=0        &
		\text{en }S\times\left(0,T\right).          \\
		\difcp{u}{n}=0         &
		\text{en }\partial S\times\left(0,T\right). \\
		u=0                    &
		\text{en }S\times\left\{t=0\right\}.        \\
		\difcp{u}{t}=
		\sen^{2}\left(x\right) &
		\text{en }S\times\left\{t=0\right\}.
	\end{cases}
\end{equation*}

Sea $R=\left(0,a\right)\times\left(0,b\right)$.
Resuelva la ecuación de onda 2D
\begin{equation*}
	\begin{cases}
		\difc.A.{}{}u=0                      &
		\text{en }R\times\left(0,T\right).          \\
		u=0                                  &
		\text{en }\partial R\times\left(0,T\right). \\
		u=xy\left(b-y\right)\left(a-x\right) &
		\text{en }R\times\left\{t=0\right\}.        \\
		\difcp{u}{t}=
		0                                    &
		\text{en }R\times\left\{t=0\right\}.
	\end{cases}
\end{equation*}

\question

Sea $D=\left(0,1\right)^{2}$.
Resuelva el problema de valores propios
\begin{equation*}
	\begin{cases}
		-\difc.L.{}{}u=
		\lambda u       &
		\text{en }D.                                                           \\
		u=0             &
		\text{en }\partial D\setminus\left(0,1\right)\times\left\{y=1\right\}. \\
		\difcp{u}{n}=-u &
		\text{en }\left(0,1\right)\times\left\{y=1\right\}.
	\end{cases}
\end{equation*}

\begin{parts}
	\part

	Muestre que todos los valores propios son positivos.

	\part

	Encuentre una ecuación para los valores propios $\lambda$.
	Demuestre que pueden expresarse en términos de las raíces de la
	ecuación $s+\tan s=0$.

	\part

	Encuentre las soluciones de la última ecuación gráficamente.
	Encuentre una fórmula aproximada para el $\left(m,n\right)$-ésimo
	valor propio para valores grandes de $\left(m,n\right)$.
\end{parts}

\question

Encuentra la dimensión de cada uno de los siguientes espacios
vectoriales.

\begin{parts}
	\part

	El espacio de las soluciones de
	\begin{math}
		\diff[2]{u}{x}+x^{2}u=0
	\end{math}.

	\part

	El espacio propio con valor propio
	\begin{math}
		{\left(
			\frac{2\pi}{l}
			\right)}^{2}
	\end{math}
	del operador
	\begin{math}
		-\diff[2]{}{t}
	\end{math}
	en el intervalo $\left(-l,l\right)$ con las condiciones de contorno
	periódicas.

	\part

	El espacio de funciones armónicas en el disco unitario con la
	condición de frontera Neumann homogénea.

	\part

	El espacio propio con valor propio $\lambda=25\pi^{2}$ de
	$-\difc.L.{}{}$ en el cuadrado unitario $\left(0,1\right)^{2}$ con
	las condiciones de frontera Neumann homogéneas en los cuatro lados.

	\part

	El espacio de las soluciones de
	\begin{math}
		\difc.A.{}{}u=0
	\end{math}
	en
	\begin{math}
		\mathbb{R}\times\left(0,\infty\right)
	\end{math}.
\end{parts}

\question

Sea
\begin{math}
	D=
	\left\{
	\left(r,\theta\right)\mid
	r<a
	\right\}\subset\mathbb{R}^{2}
\end{math}.
Determinar las vibraciones de un parche de tambor circular
\begin{equation*}
	\begin{cases}
		\frac{1}{c^{2}}\difcp[2]{u}{t}=
		\difcp[2]{u}{r}
		+\frac{1}{r}\difcp{u}{r}
		+\frac{1}{r^{2}}\difcp[2]{u}{\theta} &
		\text{en }D\times\left(0,T\right).     \\
		u=0                                  &
		\text{en }\partial D.                  \\
		u=1-\frac{r^{2}}{a^{2}}              &
		\text{en }S\times\left\{t=0\right\}.   \\
		\difcp{u}{t}=
		0                                    &
		\text{en }S\times\left\{t=0\right\}.
	\end{cases}
\end{equation*}

\question

Encuentre las soluciones de la ecuación de onda de la forma
\begin{math}
	u=e^{-i\omega t}
	f\left(r\right)
\end{math}
que sean finitas en el origen, donde
\begin{math}
	r=\sqrt{x^{2}+y^{2}}
\end{math}.

\question

Resuelva la ecuación de difusión en el disco de radio $a$, con
$u=B\in\mathbb{R}$ en la frontera y $u\left(x,y,0\right)=0$.

\question

Resuelva la ecuación de difusión en el anillo
\begin{math}
	\left\{
	a^{2}<
	x^{2}+y^{2}<
	b^{2}
	\right\}
\end{math}
con $u=B\in\mathbb{R}$ en la frontera.

\question

Sea $D$ el semidisco
\begin{math}
	\left\{
	x^{2}+y^{2}<
	b^{2},
	y>0
	\right\}
\end{math}.
Considere la ecuación de difusión en $D$ con las condiciones:
$u=0$ en la frontera de $D$ y
\begin{math}
	u\left(r,\theta,0\right)=
	\varphi\left(r,\theta\right)
\end{math}.
Escriba el desarrollo completo de la solución
$u\left(r,\theta,t\right)$, incluyendo las fórmulas para los
coeficientes.

\question

Calcule las constantes de normalización para los armónicos esféricos
utilizando los datos apropiados sobre las funciones de Legendre.

\question

Resolver la ecuación de onda en la bola $\left\{r<a\right\}$ de radio
$a$, con las condiciones $\difcp{u}{r}=0$ en $\left\{r=a\right\}$,
$u\left(x,y,z,0\right)=z=r\cos\theta$ y
\begin{math}
	\difcp{u}{t}\left(x,y,z,0\right)=0
\end{math}.

\question

Resuelva la ecuación de difusión en la esfera de radio $a$, con
$u=B\in\mathbb{R}$ en la frontera y
\begin{math}
	u\left(x,y,z,0\right)=
	C\in\mathbb{R}
\end{math}.

\question

Consideremos un huevo como una bola homogénea de radio $\pi$
centímetros.
Inicialmente, a $20^{\circ}$C, se coloca en una olla con agua
hirviendo (a $100^{\circ}$C).
¿Cuánto tarda el centro en alcanzar los $50^{\circ}$C?
Supongamos que la constante de difusión es
$k=6\times 10^{-3}$cm$^{2}$/seg.
\textbf{Sugerencia}: La temperatura es función de $r$ y $t$.
Aproxime $u\left(0,t\right)$ mediante el primer término de la
expansión.

\question

\begin{parts}
	\part

	Considere la ecuación de difusión en la esfera de radio $a$, con
	$\difcp{u}{r}=B\in\mathbb{R}$ en la frontera y
	$u\left(x,y,z,0\right)=C\in\mathbb{R}$.
	Encuentre los términos de no decaimiento en el desarrollo de la
	solución.

	\part

	Encuentre los términos de no decaimiento, incluyendo una ecuación
	simple satisfecha por los valores propios.
\end{parts}

\question

\begin{parts}
	\part

	Halle las funciones propias radiales (depende únicamente de
	la distancia $r$ al origen) de $-\difc.L.{}{}$ en la
	bola
	\begin{math}
		B=
		\left\{
		x^{2}+y^{2}+z^{2}<
		a^{2}
		\right\}
	\end{math}
	con la condición de frontera Neumann.
	\textbf{Sugerencia}: Un método simple consiste en hacer
	$v\left(r\right)=ru\left(r\right)$.

	\part

	Halle una fórmula explícita simple para los valores propios.

	\part

	Escriba la solución de $\difcp{u}{t}=k\difc.L.{}{}u$ en $B$,
	$\difcp{u}{r}=0$ en la frontera de $B$,
	\begin{math}
		u\left(x,y,z0\right)=
		\varphi\left(r\right)
	\end{math}
	como una serie infinita, incluyendo las fórmulas para los
	coeficientes.

	\part
	En el inciso anterior,
	¿por qué $u\left(x,y,z,t\right)$ depende únicamente de $r$ y $t$?
\end{parts}

\question

Resuelva la ecuación de difusión en la bola
\begin{math}
	\left\{
	x^{2}+
	y^{2}+
	z^{2}<
	a^{2}
	\right\}
\end{math}
con $u=0$ en la frontera y una condición inicial radial
$u\left(x,y,z,0\right)=\varphi\left(r\right)$, donde
$r^{2}=x^{2}+y^{2}+z^{2}$.
\textbf{Sugerencia}: Un método simple consiste en hacer
$v\left(r\right)=ru\left(r\right)$.

\question

Halle la función armónica en el exterior $\left\{r>a\right\}$ de una
esfera que satisface la condición de contorno
\begin{math}
	\difcp{u}{r}=
	-\cos\theta
\end{math}
en $r=a$ y que está acotada en el infinito.

\question

Halla la función armónica en la semiesfera
\begin{math}
	\left\{
	x^{2}+
	y^{2}+
	z^{2}<
	a^{2},
	z>0
	\right\}
\end{math}
con la condición de frontera $u=f\left(z\right)$ en el hemisferio
\begin{math}
	\left\{
	z=
	\sqrt{a^{2}-x^{2}-y^{2}}
	\right\}
\end{math}
y la condición de frontera $u=0$ en el disco
\begin{math}
	\left\{
	z=0,
	x^{2}+y^{2}<a^{2}
	\right\}
\end{math}.
Incluye las fórmulas para los coeficientes.
\textbf{Sugerencia}: Usa coordenadas esféricas y extiende la solución para
que sea impar en el plano $xy$.

\question

Una sustancia se difunde en el espacio infinito con una concentración
inicial $\varphi\left(r\right)=1$ para $r<a$ y
$\varphi\left(r\right)=0$ para $r>a$.
Halla una fórmula para la concentración en instantes posteriores.
\textbf{Sugerencia}: Es radial. Puedes sustituir $v=ru$ para obtener un
problema en una semirrecta.

\question

Dibuje el conjunto nodal de la función propia
\begin{math}
	v\left(x,y\right)=
	\sen\left(3x\right)
	\sen\left(y\right)+
	\sen\left(x\right)
	\sen\left(3y\right)
\end{math}
en el cuadrado $\left(0,\pi\right)^{2}$.
\textbf{Sugerencia}: Use las fórmulas para
$\sen\left(3x\right)$ y $\sen\left(3y\right)$
junto con la factorización para reescribirlo como
\begin{math}
	v\left(x,y\right)=
	2\sen\left(x\right)
	\sen\left(y\right)
	\left[3-2\sen^{2}\left(x\right)-2\sen^{2}\left(y\right)\right]
\end{math}.

\question

Muestre que
\begin{align*}
	J_{0}\left(z\right) & =
	1-{\left(\frac{z}{2}\right)}^{2}+
	\frac{1}{\left(2!\right)^{2}}
	{\left(\frac{z}{2}\right)}^{4}-
	\frac{1}{{\left(3!\right)}^{2}}
	{\left(\frac{z}{2}\right)}^{6}+\cdots.
	\shortintertext{y}
	J_{1}\left(z\right) & =
	-\diff{J_{0}}{z}=
	\frac{z}{2}-
	\frac{1}{2!}
	{\left(\frac{z}{2}\right)}^{3}+
	\frac{3}{{\left(3!\right)}^{2}}
	{\left(\frac{z}{2}\right)}^{5}+\cdots.
\end{align*}

\question

Escriba fórmulas simples para $J_{\frac{3}{2}}$ y $J_{-\frac{3}{2}}$.

\question

Derivar las relaciones de recurrencia de las funciones de Bessel.

\begin{align*}
	J_{s\pm 1}\left(z\right) & =
	\frac{s}{z}J_{s}\left(z\right)\mp
	\diff{J_{s}}{z}
	\text{y}
	J_{s-1}\left(z\right)+
	J_{s+1}\left(z\right)=
	\frac{2s}{z}J_{s}\left(z\right).
\end{align*}

\question

Convierta la ecuación de Bessel
\begin{equation*}
	\diff[2]{u}{z}+
	\frac{1}{z}\diff{u}{z}+
	\left(1-\frac{s^{2}}{z^{2}}\right)u=
	0
\end{equation*}
en
\begin{equation*}
	\diff{v}{z}+
	\left(1-\frac{s^{2}-\frac{1}{4}}{z^{2}}\right)=
	0
\end{equation*}
al sustituir $u=z^{-\frac{1}{2}}v$.

\question

Muestre que si $u$ satisface la ecuación de Bessel, entonces
$v=z^{\alpha}u\left(\lambda z^{\beta}\right)$ satisface la ecuación
diferencial
\begin{equation*}
	\diff[2]{v}{z}+
	\frac{1-2\alpha}{z}
	\diff{v}{z}+
	\left[
		\left(\lambda\beta z^{\beta-1}\right)^{2}-
		\frac{s^{2}\beta^{2}-\alpha^{2}}{z^{2}}
		\right]v=0.
\end{equation*}

\question

Encuentra las soluciones $u\left(x\right)$ de
\begin{math}
	x\diff[2]{u}{x}-\diff{u}{x}+xu=0
\end{math}.
\textbf{Sugerencia}: Sustituye $u=xv$.

\question

Muestre que
\begin{math}
	\cos\left(x\sen\theta\right)=
	J_{0}\left(x\right)+2
	\sum_{k=1}^{\infty}
	J_{2k}\left(x\right)
	\cos\left(2k\theta\right)
\end{math}.

\question

Resuelve la ecuación
\begin{math}
	-\difcp[2]{u}{x}-
	\difcp[2]{u}{y}+
	k^{2}u=0
\end{math}
en el disco
\begin{math}
	\left\{
	x^{2}+y^{2}<
	a^{2}
	\right\}
\end{math}
con $u=1$ en la frontera del círculo.
Escribe tu respuesta en términos de las funciones de Bessel
$J_{s}\left(iz\right)$ de argumento imaginario.

\question

Resuelve la ecuación
\begin{math}
	-\difcp[2]{u}{x}-
	\difcp[2]{u}{y}+
	k^{2}u=0
\end{math}
en el exterior
\begin{math}
	\left\{
	x^{2}+
	y^{2}>
	a^{2}
	\right\}
\end{math}
del disco con $u=1$ en la frontera del círculo
$u\left(x,y\right)$ acotada en el infinito.
Escribe tu respuesta en términos de las funciones de Hankel
$H_{s}\left(iz\right)$ del argumento imaginario.

\question

Resuelve la ecuación
\begin{math}
	-\difcp[2]{u}{x}-
	\difcp[2]{u}{y}-
	\difcp{u}{z}+
	k^{2}u=0
\end{math}
en la esfera
\begin{math}
	\left\{
	x^{2}+y^{2}+z^{2}<
	a^{2}\right\},
\end{math}
con $u=1$ en la frontera esfera.
Escribe tu respuesta en términos de funciones elementales.

\question

Resuelve la ecuación
\begin{math}
	-\difcp[2]{u}{x}-
	\difcp[2]{u}{y}-
	\difcp[2]{u}{z}+
	k^{2}u=0
\end{math}
en el exterior
\begin{math}
	\left\{
	x^{2}+
	y^{2}+
	z^{2}>
	a^{2}
	\right\}
\end{math}
de la bola con $u=1$ en la frontera de la esfera y
$u\left(x,y,z\right)$ acotada en el infinito.
Escribe tu respuesta en términos de funciones elementales.

\question

Encuentre una ecuación para los valores propios y encuentre las
funciones propias de $-\difc.L.{}{}$ en el disco
\begin{math}
	\left\{
	x^{2}+
	y^{2}<
	a^{2}
	\right\}
\end{math}
con la condición de frontera Robin
\begin{math}
	\difcp{v}{r}+hv=0
\end{math}
en la circunferencia, donde $h\in\mathbb{R}$.

\question

Encuentre una ecuación para los valores propios y encuentre las
funciones propias de $-\difc.L.{}{}$ en el anillo
\begin{math}
	\left\{
	a^{2}<
	x^{2}+
	y^{2}<
	b^{2}
	\right\}
\end{math}
con condición de frontera Dirichlet en ambas circunferencias.

\question

Demuestre que los polinomios de Legendre
\begin{equation*}
	P_{l}\left(z\right)=
	\frac{1}{2^{l}}
	\sum_{j=0}^{m}
	\frac{{\left(-1\right)}^{j}}{j!}
	\frac{\left(2l-2j\right)!}{\left(l-2j\right)!\left(l-j\right)!}z^{l-2j}
\end{equation*}
donde $m=\frac{l}{2}$ si $l$ es par y $m=\frac{l-1}{2}$ si $l$ es impar,
satisfacen la relación de recurrencia
\begin{equation*}
	\left(l+1\right)
	P_{l+1}\left(z\right)-
	\left(2l+1\right)z
	P_{l}\left(z\right)+
	lP_{l-1}\left(z\right)=0.
\end{equation*}

\question

Muestre que
\begin{math}
	P_{2n}\left(0\right)=
		{\left(-1\right)}^{n}
	\frac{\left(2n\right)!}{2^{2n}{\left(n!\right)}^{2}}
\end{math}.

\question

Muestre que $\forall l\geq 3:\int_{-1}^{1}x^{2}P_{l}\left(x\right)\dl x=0$.

\question

Encuentra la función armónica en la bola
\begin{math}
	\left\{
	x^{2}+
	y^{2}+
	z^{2}<
	a^{2}
	\right\}
\end{math}
con $u=\cos^{2}\left(\theta\right)$ en la frontera.

\question

Encuentra la función armónica en la bola
\begin{math}
	\left\{
	x^{2}+
	y^{2}+
	z^{2}<
	a^{2}
	\right\}
\end{math}
con la condición de frontera $u=A\in\mathbb{R}$ en el hemisferio superior
\begin{math}
	\left\{
	x^{2}+
	y^{2}+
	z^{2}=
	a^{2},
	z>0\right\}
\end{math}
y con $u=B\in\mathbb{R}$ en el hemisferio inferior
\begin{math}
	\left\{
	x^{2}+
	y^{2}+
	z^{2}=
	a^{2},
	z<0
	\right\}
\end{math}.

\question

Resuelva la ecuación de difusión en el cono sólido
\begin{math}
	\left\{
	x^{2}+
	y^{2}+
	z^{2}<
	a^{2},
	\theta<\alpha
	\right\}
\end{math}
con $u=0$ en la frontera y con condiciones iniciales generales.
\textbf{Sugerencia}: Separe las variables y escriba la solución como una
serie con términos de la forma separada
\begin{math}
	T\left(t\right)
	R\left(r\right)
	q\left(\phi\right)
	p\left(\cos\theta\right)
\end{math}.
Demuestre que $p\left(s\right)$ satisface la ecuación de Legendre
asociada.
