\section{Series de Fourier}

\begin{definition}[Seccionalmente continua]
    La función
    \begin{math}
        \phi\colon\left(a,b\right)\to\mathbb{R}
    \end{math}
    es \textbf{seccionalmente continua} si se satisfacen:

    \begin{enumerate}[(a)]
        \item

              Es continua en todos los puntos de $\left(a,b\right)$
              \textbf{excepto quizás en un número finito de puntos}.

        \item

              En un punto de discontinuidad $x$, tiene una
              \textbf{discontinuidad de salto}, es decir,
              \begin{align*}
                  \phi\left(x+\right) & \coloneqq
                  \lim_{h\to0^{+}}
                  \phi\left(x+h\right).           \\
                  \phi\left(x-\right) & \coloneqq
                  \lim_{h\to0^{-}}
                  \phi\left(x+h\right).
              \end{align*}

              Ambos \textbf{existen} y son \textbf{finitos}.

        \item

              Los límites $\phi\left(a+\right)$ y
              $\phi\left(b-\right)$ \textbf{existen} y son
              \textbf{finitos}.
    \end{enumerate}
\end{definition}

\begin{definition}[Serie de Fourier seno]
    La \textbf{serie de Fourier seno} de
    $\phi\colon\left(0,l\right)\to\mathbb{R}$ es
    \begin{math}\displaystyle
        \sum_{n=1}^{\infty}
        b_{n}
        \sen
        \left(\frac{n\pi x}{l}\right)
    \end{math},
    donde
    \begin{math}\displaystyle
        \forall n\in
        \mathbb{N}:
        b_{n}\coloneq
        \frac{2}{l}
        \int_{0}^{l}
        \phi\left(x\right)
        \sen
        \left(\frac{n\pi x}{l}\right)
        \dl x
    \end{math}.
\end{definition}

\begin{definition}[Serie de Fourier coseno]
    La \textbf{serie de Fourier coseno} de
    $\phi\colon\left(0,l\right)\to\mathbb{R}$ es
    \begin{math}\displaystyle
        \frac{a_{0}}{2}+
        \sum_{n=1}^{\infty}
        a_{n}
        \cos\left(\frac{n\pi x}{l}\right)
    \end{math},
    donde
    \begin{math}\displaystyle
        \forall n\in
        \mathbb{N}\cup\left\{0\right\}:
        a_{n}=
        \frac{2}{l}
        \int_{0}^{l}
        \phi\left(x\right)
        \cos\left(\frac{n\pi x}{l}\right)
        \dl x
    \end{math}.
\end{definition}

\begin{definition}[Serie de Fourier completa]
    La \textbf{serie de Fourier completa} de
    $\phi\colon\left(-l,l\right)\to\mathbb{R}$ es\linebreak
    \begin{math}\displaystyle
        \frac{a_{0}}{2}+
        \sum_{n=1}^{\infty}
        \left[
            a_{n}
            \cos\left(\frac{n\pi x}{l}\right)+
            b_{n}
            \sen
            \left(\frac{n\pi x}{l}\right)
            \right]
    \end{math},
    donde
    \begin{math}\displaystyle
        \forall n\in
        \mathbb{N}\cup\left\{0\right\}:
        a_{n}=
        \frac{1}{l}
        \int_{-l}^{l}
        \phi\left(x\right)
        \cos\left(\frac{n\pi x}{l}\right)
        \dl x
    \end{math}
    y
    \begin{math}\displaystyle
        \forall n\in
        \mathbb{N}:
        b_{n}=
        \frac{1}{l}
        \int_{-l}^{l}
        \phi\left(x\right)
        \sen
        \left(\frac{n\pi x}{l}\right)
        \dl x
    \end{math}.
\end{definition}

\begin{definition}[Serie de Fourier completa versión compleja]
    La \textbf{serie de Fourier completa versión compleja} de
    $\phi\colon\left(-l,l\right)\to\mathbb{R}$ es
    \begin{math}\displaystyle
        \sum_{n=-\infty}^{\infty}
        c_{n}
        \exp\left(\frac{in\pi x}{l}\right)
    \end{math},
    donde
    \begin{math}\displaystyle
        \forall n\in
        \mathbb{Z}:
        c_{n}=
        \frac{1}{2l}
        \int_{-l}^{l}
        \phi\left(x\right)
        \exp\left(-\frac{in\pi x}{l}\right)
        \dl x
    \end{math}
    e $i=\sqrt{-1}\in\mathbb{C}$.
\end{definition}

% \begin{definition}
%     Una función $f\colon I\subset\mathbb{R}\to\mathbb{R}$ es
%     \textbf{uniformemente continua} si y solamente si
%     \begin{equation*}
%         \forall\varepsilon>0:
%         \exists\delta>0
%         \text{ tal que }
%         \forall x,y\in I:
%         \left|x-y\right|\implies
%         \left|f\left(x\right)-f\left(y\right)\right|<
%         \varepsilon.
%     \end{equation*}
% \end{definition}

% \begin{theorem}
%     Si $f\colon\left[a,b\right]\to\mathbb{R}$ es continua, entonces
%     $f$ es \textbf{uniformemente continua}.
% \end{theorem}

\begin{theorem}[Lema de Riemann-Lebesgue]
    Sean $\left\{a_{n}\right\}_{n\in\mathbb{N}}$ y
    $\left\{b_{n}\right\}_{n\in\mathbb{N}}$ los coeficientes de la
    serie de Fourier completa de $\phi\in L^{2}\left(a,b\right)$.
    Entonces,
    \begin{math}
        \lim\limits_{n\to\infty}
        a_{n}=0
    \end{math}
    y
    \begin{math}
        \lim\limits_{n\to\infty}
        b_{n}=0
    \end{math}.
\end{theorem}

\section{Problema de Sturm-Liouville}

\begin{definition}[Problema de Sturm-Liouville]
    Considere el siguiente problema de valor de frontera para una
    función $X\left(x\right)$ en $\left[0,l\right]$:
    \begin{equation*}
        \left\{
        \begin{aligned}
            -{\left(
            p\left(x\right)
            X^{\prime}\left(x\right)
            \right)}^{\prime}-
            q\left(x\right)
            X\left(x\right)                    & =
            \lambda
            \sigma\left(x\right)
            X\left(x\right).                       \\
            \alpha_{1}X\left(0\right)+
            \alpha_{2}X^{\prime}\left(0\right) & =
            0.                                     \\
            \alpha_{3}X\left(l\right)+
            \alpha_{4}X^{\prime}\left(l\right) & =
            0.
        \end{aligned}
        \right.
    \end{equation*}
    Aquí,
    \begin{itemize}
        \item

              las funciones dadas $p\left(x\right)$,
              $q\left(x\right)$ y $\sigma\left(x\right)$ son no
              negativas en $\left[0,l\right]$ y
              $p\in C^{1}\left(0,l\right)$;

        \item

              las constantes fijas dadas $\alpha_{1}$, $\alpha_{2}$,
              $\alpha_{3}$, $\alpha_{4}$, alguna de ellas es distinta
              de cero.
    \end{itemize}
\end{definition}

\begin{definition}[Eigenfunción y eigenvalor]
    Defina el \textbf{operador de Sturm-Liouville}
    \begin{equation*}
        \forall x\in\left[0,l\right]:
        \mathcal{L}
        X\left(x\right)\coloneqq
        -{\left(
        p\left(x\right)
        X^{\prime}\left(x\right)
        \right)}^{\prime}-
        q\left(x\right).
    \end{equation*}
    $X$ es una \textbf{eigenfunción} de $\mathcal{L}$ con peso
    $\sigma\left(x\right)$ y su correspondiente \textbf{eigenvalor}
    $\lambda$ si y solamente si
    \begin{equation*}
        \mathcal{L}
        X\left(x\right)=
        \lambda
        \sigma\left(x\right)
        X\left(x\right).
    \end{equation*}
\end{definition}

\begin{theorem}[Identidad de Lagrange]
    \begin{equation*}
        \forall X_{1},X_{2}\in C^{2}\left[0,l\right]:
        \int_{0}^{l}
        \left[
        X_{2}\left(x\right)
        \mathcal{L}X_{1}\left(x\right)-
        X_{1}\left(x\right)
        \mathcal{L}X_{2}\left(x\right)
        \right]
        \dl x=
        {
        -p\left(x\right)
        X^{\prime}_{1}\left(x\right)
        X_{2}\left(x\right)
        \Big|}_{0}^{l}+
        {
        p\left(x\right)
        X_{1}\left(x\right)
        X_{2}^{\prime}\left(x\right)
        \Big|}_{0}^{l}.
    \end{equation*}
\end{theorem}

\begin{theorem}
    El operador $\mathcal{L}$ es \textbf{simétrico}.
    Se tiene que para cualquier $X_{1}$ y $X_{2}$
    \begin{align*}
        \left\langle X_{1},\mathcal{L}X_{2}\right\rangle & =
        \left\langle \mathcal{L}X_{1},X_{2}\right\rangle.    \\
        \int_{0}^{l}
        X_{1}\left(x\right)
        \mathcal{L}X_{2}\left(x\right)
        \dl x                                            & =
        \int_{0}^{l}
        \mathcal{L}X_{1}\left(x\right)
        X_{2}\left(x\right)
        \dl x.
    \end{align*}
\end{theorem}

\section{Serie de Fourier asociada a la condición de frontera simétrica}

%Considere $X\in C^{2}\left[a,b\right]$ que satisface cierta condición
%de frontera $X\left(a\right)=X\left(b\right)=0$.
%Defina el operador diferencial, $\mathcal{A}\coloneqq-\diff[2]{}{x}$.
%
%Sea $\lambda\in\mathbb{R}$ y $X\neq 0$ una función que satisface
%las condiciones de frontera tal que $0$.
\begin{definition}[Condición de frontera simétrica]
    $X_{1}\left(x\right)$ es una
    \textbf{condición de frontera simétrica en $\left[a,b\right]$} de
    $X_{2}\left(x\right)$ si y solamente
    \begin{math}
        X_{1}\left(b\right)
        X^{\prime}_{2}\left(b\right)-
        X^{\prime}_{1}\left(b\right)
        X_{2}\left(b\right)=
        X_{1}\left(a\right)
        X^{\prime}_{2}\left(a\right)-
        X^{\prime}_{1}\left(a\right)
        X_{2}\left(a\right).
    \end{math}
\end{definition}

\noindent
Considere el \textbf{problema de eigenvalor}
$\mathcal{A}X=\lambda X$ que satisface alguna condición de frontera
simétrica, donde el operador diferencial
$\mathcal{A}\coloneqq -\diff[2]{}{x}$.
Este operador satisface
\begin{math}
    \left\langle\mathcal{A}X_{1},X_{2}\right\rangle=
    \left\langle X_{1},\mathcal{A}X_{2}\right\rangle
\end{math}.

\begin{definition}[Serie de Fourier general]
    Considere una condición de frontera simétrica y asuma que existe
    una cantidad infinita numerable de eigenvalores
    ${\left\{\lambda_{n}\right\}}_{n\in\mathbb{N}}$ y eigenfunciones
    ${\left\{X_{n}\left(x\right)\right\}}_{n\in\mathbb{N}}$.
    La \textbf{serie de Fourier general} de
    $\phi\in L^{2}\left(a,b\right)$ es
    \begin{math}
        \sum\limits_{n=1}^{\infty}
        A_{n}X_{n}\left(x\right)
    \end{math},
    donde
    \vspace*{-\baselineskip}\setlength\belowdisplayshortskip{0pt}
    \begin{equation*}
        \forall n\in\mathbb{N}:
        A_{n}\coloneqq
        \frac{
        \left\langle
        \phi,X_{n}
        \right\rangle}{
        \left\|
        X_{n}
        \right\|^{2}}=
        \frac{
            \int_{a}^{b}
            \phi\left(x\right)
            X_{n}\left(x\right)
            \dl x}{
            \int_{a}^{b}
            X^{2}_{n}\left(x\right)
            \dl x}
    \end{equation*}
    y $\left\{X_{n}\colon n\in\mathbb{N}\right\}$ es un conjunto
    ortogonal de funciones en $\left(a,b\right)$.
\end{definition}

Para los casos de condiciones de frontera \textbf{Dirichlet}:
$X\left(a\right)=X\left(b\right)=0$, \textbf{Neumann}:
$X^{\prime}\left(a\right)=X^{\prime}\left(b\right)=0$,
\textbf{periódicas}: $X\left(a\right)=X\left(b\right)$ y
$X^{\prime}\left(a\right)=X^{\prime}\left(b\right)$,
la \textbf{serie de Fourier clásica} de $\phi$.
En ese caso, los $X_{n}$ estará formado por senos y cosenos con
frecuencias que aumentan apropiadamente.

\begin{theorem}
    Los eigenvalores del problema de eigenvalor
    $\mathcal{A}X=\lambda X$ sobre las funciones $X\left(X\right)$ en
    $\left[a,b\right]$ que satisface condiciones de frontera
    simétrica, son reales.
\end{theorem}

\section{Convergencia de la serie de Fourier general}

Existen muchas maneras en que una función asociada
$\phi\colon\left(a,b\right)\to\mathbb{R}$ puede converger a otra
función.
Analizamos tres maneras en que una serie de Fourier general
\textbf{puede converger} a $\phi$ en $L^{2}$, puntualmente o
uniformemente.

\begin{definition}
    La sucesión de funciones
    \begin{math}
        \left\{
        f_{n}\colon\left(a,b\right)\to\mathbb{R}
        \right\}_{n\in\mathbb{N}}
    \end{math}
    converge hacia $f$ en $L^{2}\left(a,b\right)$ si y solamente si
    \begin{equation*}
        \int_{a}^{b}
        {\left|
        f_{n}\left(x\right)-
        f\left(x\right)
        \right|}^{2}
        \dl x
        \xrightarrow{n\to\infty}
        0.
    \end{equation*}
\end{definition}

\begin{theorem}[Convergencia de la serie de Fourier general]
    Si $\phi\in L^{2}\left(a,b\right)$, entonces la serie de Fourier
    general asociada con cualquier condición de frontera simétrico
    converge a $\phi$ en el sentido $L^{2}$ sobre el intervalo
    $\left(a,b\right)$.
    Decimos que las eigenfunciones asociadas con la condición de
    frontera simétrica es \textbf{completo} en
    $L^{2}\left(a,b\right)$.
\end{theorem}

\begin{definition}[Convergencia puntual de funciones]
    La sucesión de funciones continuas
    \begin{math}
        \left\{
        f_{n}\colon\left(a,b\right)\to\mathbb{R}
        \right\}_{n\in\mathbb{N}}
    \end{math}
    \textbf{converge puntualmente} a $f$ si y solamente si
    \begin{equation*}
        \forall x\in\left(a,b\right):
        \forall\varepsilon>0:
        \exists n_{0}\in\mathbb{N}
        \text{ tal que }
        \forall n\geq n_{0}:
        \left|
        f\left(x\right)-
        f_{n}\left(x\right)
        \right|<
        \varepsilon.
    \end{equation*}
\end{definition}

\begin{theorem}[Convergencia puntual de la serie de Fourier completa]
    Suponga que $\phi$ y $\phi^{\prime}$ son funciones seccionalmente
    continuas en $\left(-\pi,\pi\right)$ y los límites laterales
    $\phi\left(\left(-\pi\right)+\right)$,
    $\phi^{\prime}\left(\left(-\pi\right)+\right)$
    $\phi\left(\pi-\right)$ y $\phi^{\prime}\left(\pi-\right)$
    existen.
    Extienda $\phi$ periódicamente a todo $\mathbb{R}$.
    Entonces, $\forall x\in\mathbb{R}$, la serie de Fourier clásica
    converge puntualmente a
    $\frac{\phi\left(x+\right)+\phi\left(x-\right)}{2}$.
\end{theorem}

\begin{definition}[Convergencia uniforme de funciones]
    La sucesión de funciones continuas
    \begin{math}
        \left\{
        f_{n}\colon\left(a,b\right)\to\mathbb{R}
        \right\}_{n\in\mathbb{N}}
    \end{math}
    \textbf{converge uniformemente} a $f$ si y solamente si
    \begin{equation*}
        \forall\varepsilon>0:
        \exists n_{0}\in\mathbb{N}
        \text{ tal que }
        \forall n\geq n_{0}:
        \forall x\in\left(a,b\right):
        \left|
        f\left(x\right)-
        f_{n}\left(x\right)
        \right|<
        \varepsilon.
    \end{equation*}
\end{definition}

\begin{theorem}[Convergencia uniforme de la serie de Fourier completa]
    Suponga que $\phi$ es continua y seccionalmente
    $C^{1}\left[-\pi,\pi\right]$ con
    $\phi\left(-\pi\right)=\phi\left(\pi\right)$.
    Entonces, la serie de Fourier completa clásica converge
    absolutamente y uniformemente a $\phi$ en
    $\left[-\pi,\pi\right]$.
    Además, la serie de Fourier completa converge absolutamente y
    uniformemente a la extensión periódica de $\phi$ en $\mathbb{R}$.
\end{theorem}

\begin{theorem}[Diferenciación término a término de la serie de Fourier]
    Suponga que $\phi\in C\left[-\pi,\pi\right]$  satisface
    $\phi\left(-\pi\right)=\phi\left(\pi\right)$ y es
    seccionalmente $C^{1}\left(-\pi,\pi\right)$.
    \begin{enumerate}[(a)]
        \item

              La extensión periódica de $\phi$ en $\mathbb{R}$ es
              continua y seccionalmente
              $C^{1}\left(\mathbb{R}\right)$.
              Considere las series de Fourier
              \begin{equation*}
                  \begin{aligned}
                      \phi\left(x\right)          & =
                      \frac{a_{0}}{2}+
                      \sum_{n=1}^{\infty}
                      \left[
                          a_{n}
                          \cos\left(nx\right)+
                          b_{n}
                          \sen\left(nx\right)
                      \right].                        \\
                      \phi^{\prime}\left(x\right) & =
                      \frac{A_{0}}{2}+
                      \sum_{n=1}^{\infty}
                      \left[
                          A_{n}
                          \cos\left(nx\right)+
                          B_{n}
                          \sen\left(nx\right)
                          \right].
                  \end{aligned}\qquad
                  \begin{aligned}
                      a_{n} & =
                      \frac{1}{\pi}
                      \int_{-\pi}^{\pi}
                      \phi\left(y\right)
                      \cos\left(ny\right)\dl y. \\
                      b_{n} & =
                      \frac{1}{\pi}
                      \int_{-\pi}^{\pi}
                      \phi\left(y\right)
                      \sen\left(ny\right)\dl y.
                  \end{aligned}
              \end{equation*}
              Entonces, $\forall n\in\mathbb{N}$: $A_{0}=a_{0}$,
              $A_{n}=nb_{n}$ y $B_{n}=-na_{n}$.

        \item

              Supongamos que la extensión de $\phi^{\prime}$ es
              seccionalmente $C^{1}$ en cualquier intervalo de
              $\mathbb{R}$ en el que sea continua.
              Entonces,
              \begin{equation*}
                  \phi^{\prime}\left(x\right)=
                  \sum_{n=1}^{\infty}
                  \left[
                      nb_{n}
                      \cos\left(nx\right)-
                      na_{n}
                      \sen\left(nx\right)
                      \right].
              \end{equation*}
              Si $x$ es un punto donde $\phi^{\prime}$ presenta una
              discontinuidad de salto, entonces esta serie de Fourier
              converge a
              \begin{math}
                  \frac{
                      \phi^{\prime}\left(x+\right)+
                      \phi^{\prime}\left(x-\right)
                  }{2}
              \end{math}.
    \end{enumerate}
\end{theorem}

\begin{theorem}[Integración término a término de la serie de Fourier]
    Sea $\phi\left(x\right)$ seccionalmente continua en
    $\left[-\pi,\pi\right]$ y extienda $\phi$ periódicamente a
    $\mathbb{R}$ (con período $2\pi$).
    Defina la siguiente primitiva de $\phi$:
    \begin{equation*}
        \Phi\left(x\right)\coloneqq
        \int_{0}^{x}
        \phi\left(y\right)\dl y.
    \end{equation*}
    Entonces, $\Psi$ es continua, seccionalmente
    $C^{1}\left(0,x\right)$ y en cualquier $x$ donde $\phi$ es
    continua: $\Psi^{\prime}\left(x\right)=\phi\left(x\right)$.
    Además, la serie de Fourier para $\Psi$ es dada por
    \begin{equation*}
        \frac{A_{0}}{2}+
        \sum_{n=1}^{\infty}
        \left[
        \frac{-b_{n}}{n}
        \cos\left(nx\right)+
        \frac{a_{n}}{n}
        \sen\left(x\right)
        \right],\qquad
        \frac{A_{0}}{2}=
        \frac{1}{2\pi}
        \int_{-\pi}^{\pi}
        \Phi\left(y\right)
        \dl y.
    \end{equation*}
\end{theorem}

\begin{theorem}[Desigualdad de Bessel]
    Sean $\phi\in L^{2}\left(a,b\right)$ y
    $\left\{X_{n}\left(x\right)\right\}_{n\in\mathbb{N}}$ una familia
    ortonormal de funciones en $\left(a,b\right)$.
    Defina
    \begin{align*}
        A_{n}\coloneqq
        \frac{
        \left\langle
        \phi,X_{n}
        \right\rangle
        }{
        {\left\|X_{n}\right\|}^{2}
        }     & =
        \frac{
        \int_{a}^{b}
        \phi\left(x\right)
        X_{n}\left(x\right)\dl x
        }{
        \int_{a}^{b}
        {\left|X_{n}\left(x\right)\right|}^{2}\dl x
        }.\notag
        \shortintertext{Entonces,}
        \sum_{n=1}^{\infty}
        A^{2}_{n}
        \int_{a}^{b}
        \left|X_{n}\left(x\right)\right|
        \dl x & \leq
        \int_{a}^{b}
        {\left|\phi\left(x\right)\right|}^{2}
        \dl x.\tag{Desigualdad de Bessel}
    \end{align*}
    Cuando la igualdad se cumple, llamaremos la
    \textbf{identidad de Parseval}.
\end{theorem}
