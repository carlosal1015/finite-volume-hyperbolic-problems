\question

Para cada una de las siguientes EDPs, indique el orden y si es
no lineal, lineal no homogénea o lineal homogénea; proporcione
razones.
\begin{multicols}{3}
	\begin{parts}
		\part

		\begin{math}
			\difcp{u}{t}-
			\difcp[2]{u}{x}+
			1=
			0
		\end{math}.

		\part

		\begin{math}
			\difcp{u}{t}-
			\difcp[2]{u}{x}+
			xu=
			0
		\end{math}.

		\part

		\begin{math}
			\difcp{u}{t}-
			\difcp[2,1]{u}{xt}+
			xu=
			0
		\end{math}.

		\part

		\begin{math}
			\difcp[2]{u}{t}-
			\difcp{u}{x}+
			x^{2}=
			0
		\end{math}.

		\part

		\begin{math}
			i\difcp{u}{t}-
			\difcp[2]{u}{x}+
			\frac{u}{x}=
			0
		\end{math}.

		\part

		\begin{math}
			\frac{\difcp{u}{x}}{\sqrt{1+\difcp[2]{u}{x}}}+
			\frac{\difcp{u}{y}}{\sqrt{1+\difcp[2]{u}{y}}}=
			0
		\end{math}.

		\part

		\begin{math}
			\difcp{u}{x}+
			e^{y}
			\difcp{u}{y}=
			0
		\end{math}.

		\part

		\begin{math}
			\difcp{u}{t}+
			\difcp[4]{u}{x}+
			\sqrt{1+u}=
			0
		\end{math}.

		\part

		\begin{math}
			\sqrt{1+x^{2}}
			\difcp[3]{u}{yxy}-
			\frac{x}{y}u=
			0
		\end{math}.
	\end{parts}
\end{multicols}

\question

Dado $c\in\mathbb{R}$, estudie la linealidad y el orden de la EDP
\begin{math}
	\difcp{u}{x}+
	c\difcp{u}{y}=
	0
\end{math}
en $\Omega=\mathbb{R}^{2}$.
Interprete geométricamente la ecuación y calcule sus soluciones.

\question

Resuelva las EDPs en
$\Omega=\mathbb{R}\times\left(0,\infty\right)$.
\begin{multicols}{2}
	\begin{parts}
		\part

		\begin{math}
			\difcp{u}{x}+
			x\difcp{u}{t}=
			0
		\end{math}.

		\part

		\begin{math}
			\difcp{u}{t}+
			2tx^{2}\difcp{u}{x}=
			0
		\end{math}.

		\part

		\begin{math}
			\left\{
			\begin{aligned}
				\difcp{u}{t}-\difcp{u}{x} & =
				u^{2}.                        \\
				u\left(x,0\right)         & =
				\frac{e^{-x}}{2}.
			\end{aligned}
			\right.
		\end{math}
	\end{parts}
\end{multicols}

\question

Resuelva las EDPs en $\Omega=\mathbb{R}^{2}$.
\begin{multicols}{3}
	\begin{parts}
		\part

		\begin{math}
			\left\{
			\begin{aligned}
				\difcp{u}{x}+
				u\difcp{u}{y}     & =
				2.                    \\
				u\left(0,y\right) & =
				y.
			\end{aligned}
			\right.
		\end{math}

		\part

		\begin{math}
			\left\{
			\begin{aligned}
				\difcp{u}{y}          & =
				x^{2}+
				y^{2}.                    \\
				u\left(x,x^{2}\right) & =
				x+
				x^{2}.
			\end{aligned}
			\right.
		\end{math}

		\part

		\begin{math}
			\left\{
			\begin{aligned}
				\left(\sqrt{1-x^{2}}\right)
				\difcp{u}{x}+
				\difcp{u}{y}      & =
				0.                    \\
				u\left(x,0\right) & =
				y.
			\end{aligned}
			\right.
		\end{math}
	\end{parts}
\end{multicols}

\question

Muestre que
\begin{equation*}
	\forall\left\{\alpha,\beta,\gamma,\delta\right\}\subset
	\mathbb{R}:
	u\left(x,y\right)=
	\alpha\left(3y^{2}+x^{3}\right)+
	\beta\left(y^{3}+x^{3}y\right)+
	\gamma\left(6xy^{2}+x^{4}\right)+
	\delta\left(2xy^{3}+x^{4}y\right)
\end{equation*}
es una solución polinómica de la ecuación de Euler-Tricomi
\begin{math}
	\difcp[2]{u}{x}+
	x\difcp[2]{u}{y}=
	0
\end{math}.

\question

Encuentre la solución integral $u\colon\mathbb{R}^{3}\to\mathbb{R}$
de cada una de las siguientes EDPs.
\begin{multicols}{3}
	\begin{parts}
		\part

		\begin{math}
			\difcp{u}{x}=
			0
		\end{math}.

		\part

		\begin{math}
			\difcp[2]{u}{x,y}=
			0
		\end{math}.

		\part

		\begin{math}
			\difcp[3]{u}{x,y,z}=
			0
		\end{math}.
	\end{parts}
\end{multicols}

\question

Muestre que
\begin{math}
	u\left(x,t\right)=
	f^{\prime}
	\left(\frac{x}{t}\right)
\end{math}
resuelve
\begin{math}
	\difcp{u}{t}+
	\difcp{f\left(u\right)}{x}=
	0
\end{math}
en $\Omega=\mathbb{R}\times\left(0,\infty\right)$,
donde
\begin{math}
	f\in
	C^{1}\left(\mathbb{R}\right)
\end{math}
con $f\left(0\right)=0$.

\question

Muestre que
\begin{math}
	\forall\varepsilon>0:
	u\left(x,y\right)=
	\ln\sqrt{x^{2}+y^{2}}
\end{math}
resuelve
$\difc.L.{u}{}=0$ en
\begin{math}
	\Omega=
	\left\{
	\left(x,y\right)\in\mathbb{R}^{2}\mid
	x^{2}+y^{2}>
	\varepsilon
	\right\}
\end{math}.

\question

Muestre que
\begin{math}
	\forall\varepsilon>0:
	u\left(x_{1},x_{2},x_{3},x_{4}\right)=
	\frac{1}{x^{2}_{1}+x^{2}_{2}+x^{2}_{3}+x^{2}_{4}}
\end{math}
resuelve $\difc.L.{u}{}=0$ en
\begin{math}
	\Omega=
	\left\{
	\symbf{x}\in\mathbb{R}^{4}\mid
	\left\|\symbf{x}\right\|>
	\varepsilon
	\right\}
\end{math}.

\question

Muestre que
\begin{math}
	\forall c>0:
	u\left(x,t\right)=
	\frac{c}{2}
	\operatorname{sech}^{2}
	\left[
		\frac{\sqrt{c}}{2}\left(x-ct\right)
		\right]
\end{math}
es una solución de la ecuación de Korteweg-de Vries (KdV)
\begin{math}
	\difcp{u}{t}+
	\difcp[3]{u}{x}-
	6u\difcp{u}{x}=
	0
\end{math}.
% Aquí $\operatorname{sech}$ denota la función secante hiperbólico,
% es decir, $\operatorname{sech}y=\frac{2}{e^{y}+e^{-y}}$.

Use cualquier software y grafique esta solución para $c=2$ y
$t=0,1,2,5$.
En cualquier momento fijo $t$, $u\left(\cdot,t\right)$ denota la
forma del impulso onda.
Interprete el rol de $c$ tanto en términos de velocidad como de forma
del impulso.
Esta solución es llamado un solitón.
Busque en línea el concepto de solitón y lo que significa en el
contexto de la ecuación KdV y las ondas de agua.

\question

Considere la ecuación de Burgers invíscida
$\difcp{u}{t}+u\difcp{u}{x}=0$ con la condición inicial
$u\left(x,0\right)=x^{3}$.
Utilice el método de las características para escribir la solución
en términos de un $x_{0}$ que está implícitamente relacionado con
$\left(x,t\right)$.
Muestre que
\begin{math}
	\forall\left(x,t\right)\in
	\mathbb{R}\times\left(0,\infty\right):
	\exists!\: x_{0}\in\mathbb{R}.
\end{math}
Dibuje las características.
Utilizando el método que prefiera, dibuje los perfiles
$u\left(x,0\right)$, $u\left(x,1\right)$, $u\left(x,3\right)$.

\question

Considere la ecuación invíscida de Burgers
$\difcp{u}{t}+u\difcp{u}{x}=0$ con la condición inicial
\begin{equation*}
	u\left(x,0\right)=
	\begin{cases}
		1,   & x\leq 0.       \\
		1-x, & 0\leq x\leq 1. \\
		0,   & x\geq 1.
	\end{cases}
\end{equation*}

Dibuje las características en el plano $x$ versus $t$ para
$0\leq t\leq 1$.
Dibuje $u\left(x,0\right)$, $u\left(x,\frac{1}{2}\right)$,
$u\left(x,\frac{3}{4}\right)$, $u\left(x,1\right)$.
Escriba la solución explícita
\begin{math}
	\forall\left(x,t\right)\in
	\mathbb{R}\times\left[0,1\right]:
	u\left(x,t\right)
\end{math}.

\question

Muestre que la EDP
$\difcp{u}{x}+x\difcp{u}{y}=0$ sujeta a la condición inicial
$u\left(x,0\right)=e^{x}$ que sea válida en una vecindad del origen
no tiene solución.
Dibuje las características para ver geométricamente qué salió mal.

\question

Utilice el método de características para resolver
\begin{math}
	y\difcp{u}{x}-
	2xy\difcp{u}{y}=
	2xu
\end{math}
con $u=y^{3}$ cuando $x=0$ e $1\leq y\leq 2$.
\textbf{Sugerencia}: Para resolver las EDO características para $x$ e
$y$, multiplique la primera por $2x$ y luego sume.

\question

Resuelva $a\difcp{u}{x}+b\difcp{u}{y}=f\left(x,y\right)$, donde
$f$ es una función dada.
Si $a\neq0$, escriba la solución en la forma
\begin{equation*}
	u\left(x,y\right)=
	\frac{1}{\sqrt{a^{2}+b^{2}}}
	\int_{L}
	f\dl s+
	g\left(bx-ay\right),
\end{equation*}
donde $g$ es una función arbitraria de una variable, $L$ es el
segmento característico que va del eje $y$ al punto
$\left(x,y\right)$, y la integral es una integral de línea.
\textbf{Sugerencia}: Use el método de cambio de coordenadas.

\question

Utilice el método de cambio de coordenadas para resolver la EDP
\begin{equation*}
	\difcp{u}{x}+
	2\difcp{u}{y}+
	\left(2x-y\right)u=
	2x^{2}+
	3xy-
	2y^{2}.
\end{equation*}

\question

Determine el orden, grado, linealidad, término principal de las EDPs
\begin{multicols}{3}
	\begin{parts}
		\part

		\begin{math}
			\difcp{u}{t}-
			\left(x^{2}+u\right)
			\difcp[2]{u}{x}=
			x-t
		\end{math}.

		\part

		\begin{math}
			u^{2}\difcp[2]{u}{t}-
			\frac{1}{2}{\left(\difcp{u}{x}\right)}^{2}
			=e^{u}
		\end{math}.

		\part

		\begin{math}
			{\left(\difcp[2]{u}{x,y}\right)}^{2}-
			\difcp[2]{u}{x}+
			\difcp{u}{t}=
			0
		\end{math}.
	\end{parts}
\end{multicols}

\question

Considere la EDP
\begin{math}
	\difcp{u}{t}+
	\difcp[2]{u}{x}=0
\end{math}
y verifique que dados $A\in\mathbb{R}$, $T>0$,
\begin{equation*}
	\forall x\in\mathbb{R},
	t<T:
	u\left(x,t\right)=
	\frac{A\sqrt{T}}{\sqrt{T-t}}
	\exp\left(-\frac{x^{2}}{4\left(T-t\right)}\right)
\end{equation*}
es una solución.

\question

Resuelva el problema
\begin{equation*}
	\begin{cases}
		2y\difcp{u}{x}+
		\difcp{u}{y}=
		\left(2y^{2}+x\right)
		\sen\left(2xy\right)  &
		\text{en }\mathbb{R}^{2}. \\
		u\left(x,e^{-2x}\right)=
		\cos^{2}
		\left(xe^{-2x}\right) &
		\text{en }\mathbb{R}.
	\end{cases}
\end{equation*}

\question

Calcule las curvas características planas de la EDP y resuelva
el problema
\begin{equation*}
	\begin{cases}
		-y\difcp{u}{x}+
		x\difcp{u}{y}=4xy &
		\text{en }\mathbb{R}\times\left(0,\infty\right). \\
		u\left(x,0\right)=
		f\left(x\right)   &
		\text{en }\mathbb{R}.
	\end{cases}
\end{equation*}

\question

Identifique las condiciones iniciales y/o de frontera en los
problemas siguientes
indicando si el problema es de Cauchy, de frontera o mixto.
Indique también si alguna de las funciones dadas tiene que satisfacer
las condiciones de compatibilidad.

\begin{parts}
	\part

	\begin{math}
		\begin{cases}
			x\difcp{u}{x}-
			y\difcp{u}{y}=
			x^{2}+y^{2} &
			\text{en }\mathbb{R}^{2}. \\
			u\left(x^{3},x^{5}\right)=
			x^{2}+1     &
			\text{en }\mathbb{R}.
		\end{cases}
	\end{math}

	\part

	\begin{math}
		\begin{cases}
			x\difcp{u}{x}+
			y\difcp{u}{y}=
			0       &
			\text{en }x^{2}+y^{2}<4. \\
			u\left(2\sen x,2\cos x\right)=
			x\sen x &
			\text{en }\left[0,2\pi\right].
		\end{cases}
	\end{math}

	\part

	\begin{math}
		\begin{cases}
			t\difcp[2]{u}{x}+
			2x\difcp{u}{x,t}-
			t\difcp[2]{u}{t}+
			x^{2}\difcp{u}{x}+
			t^{2}\difcp{u}{t}=
			e^{x}\cos\left(t\right) &
			\text{en }\mathbb{R}\times\left(0,\infty\right). \\
			u\left(x,0\right)=
			f\left(x\right)         &
			\text{en }\mathbb{R}.                            \\
			\difcp{u}{t}\left(x,0\right)=
			g\left(x\right)         &
			\text{en }\mathbb{R}.
		\end{cases}
	\end{math}

	\part

	\begin{math}
		\begin{cases}
			y^{2}\difcp[2]{u}{x}+
			x^{2}\difcp[2]{u}{y}-
			u=
			f\left(x,t\right)    &
			\text{en }\left(0,1\right)\times\left(0,\infty\right). \\
			u\left(0,t\right)=
			\alpha\left(t\right),
			u\left(1,t\right)=
			\beta\left(t\right)  &
			\text{en}\left(0,\infty\right).                        \\
			u\left(x,0\right)=
			\gamma\left(x\right) &
			\text{en }\left[0,1\right].
		\end{cases}
	\end{math}
\end{parts}

\question

Indique el orden de las EDPs
\begin{multicols}{2}
	\begin{parts}
		\part

		\begin{math}
			\left(\diffp{u}{x}\right)^{2}+
			\diffp[3]{u}{y}=
			0
		\end{math}.

		\part

		\begin{math}
			u\difcp[2]{u}{x,t}+
			\difcp{u}{x}=
			u^{2}+1
		\end{math}.

		\part

		\begin{math}
			\difcp[2]{u}{x,t}=
			\sen\left(u\right)
		\end{math}.

		\part

		\begin{math}
			x^{3}\difcp{u}{x}-
			u^{3}\difcp{u}{t}+
			\difcp[2]{u}{x}=
			x^{5}+t^{4}
		\end{math}.
	\end{parts}
\end{multicols}

\question

Sea $c\neq0$ y $u_{0}\in C^{1}\left(\mathbb{R}\right)$.
Muestre que $u\left(x,t\right)=u_{0}\left(x-ct\right)$ es una
solución del problema de Cauchy
\begin{equation*}
	\begin{cases}
		\difcp{u}{t}+
		c\difcp{u}{x}=
		0     &
		\text{en }\mathbb{R}\times\left(0,\infty\right). \\
		u=
		u_{0} &
		\text{en }\mathbb{R}\times\left\{t=0\right\}.
	\end{cases}
\end{equation*}

\question

Encuentre una solución de cada problema de Cauchy
\begin{multicols}{2}
	\begin{math}
		\begin{cases}
			\difcp{u}{t}+
			\difcp{u}{x}=
			0                       &
			\text{en }\mathbb{R}\times\left(0,\infty\right). \\
			u=
			\exp\left(-x^{2}\right) &
			\text{en }\mathbb{R}\times\left\{t=0\right\}.
		\end{cases}
	\end{math}

	\begin{math}
		\begin{cases}
			\difcp{v}{t}+
			\difcp{v}{x}=
			0                      &
			\text{en }\mathbb{R}\times\in(0,\infty). \\
			v=
			\sen\left(\pi x\right) &
			\text{en }\mathbb{R}\times\left\{t=0\right\}.
		\end{cases}
	\end{math}
\end{multicols}

Muestre que $w=u+v$ es una solución del problema de Cauchy
\begin{equation*}
	\begin{cases}
		\difcp{w}{t}+
		\difcp{w}{x}=
		0                      &
		\text{en }\mathbb{R}\times\left(0,\infty\right). \\
		w=
		\exp\left(-x^{2}\right)+
		\sen\left(\pi x\right) &
		\text{en }\mathbb{R}\times\left\{t=0\right\}.
	\end{cases}
\end{equation*}

\question

Muestre que
\begin{math}
	u\left(x,t\right)=
	\dfrac{x}{t+1}
\end{math}
y
\begin{math}
	v\left(x,t\right)=
	1
\end{math}
son soluciones de cada problema de Cauchy
\begin{multicols}{2}
	\begin{math}
		\begin{cases}
			\difcp{u}{t}+
			u
			\difcp{u}{x}=
			0 &
			\text{en }\mathbb{R}\times\left(0,\infty\right). \\
			u=
			x &
			\text{en }\mathbb{R}\times\left\{t=0\right\}.
		\end{cases}
	\end{math}

	\begin{math}
		\begin{cases}
			\difcp{v}{t}+
			v
			\difcp{v}{x}=0 &
			\text{en }\mathbb{R}\times\left(0,\infty\right). \\
			v=
			1              &
			\text{en }\mathbb{R}\times\left\{t=0\right\}.
		\end{cases}
	\end{math}
\end{multicols}

Muestre que $w=u+v$ no es una solución del problema de Cauchy
\begin{equation*}
	\begin{cases}
		\difcp{w}{t}+
		w
		\difcp{w}{x}=
		0   &
		\text{en }\mathbb{R}\times\left(0,\infty\right). \\
		w=
		x+1 &
		\text{en }\mathbb{R}\times\left\{t=0\right\}.
	\end{cases}
\end{equation*}
¿Cuál es la principal diferencial entre esta EDP y la EDP de la
pregunta anterior?

\question

Considere el problema de Cauchy
\begin{equation*}
	\begin{cases}
		\difcp[2]{u}{t}+
		\difcp[2]{u}{x,t}-
		2\difcp[2]{u}{x}=
		0                    &
		\text{en }\mathbb{R}\times\left(0,\infty\right). \\
		u=
		\cos\left(x\right)   &
		\text{en }\mathbb{R}\times\left\{t=0\right\}.    \\
		\difcp{u}{t}=
		-4\sen\left(x\right) &
		\text{en }\mathbb{R}\times\left\{t=0\right\}.
	\end{cases}
\end{equation*}
Resuelva por el cambio de coordenadas $\alpha=x+t$ y $\mu=x-2t$.

\question

Pruebe directamente que si $u$ y $v$ son soluciones de una de las
siguientes ecuaciones, entonces toda combinación lineal de $u$ y $v$
también lo es.
\begin{multicols}{3}
	\begin{parts}
		\part

		\begin{math}
			\difcp{u}{x}+xu=
			0
		\end{math}.

		\part

		\begin{math}
			y\difcp[2]{u}{x}=
			0
		\end{math}.

		\part

		\begin{math}
			y\difcp[2]{u}{x}+
			x\difcp{u}{y}=
			xyu
		\end{math}.
	\end{parts}
\end{multicols}

\question

Una EDP de la forma $Pp+Qq=R$, donde $P$, $Q$ y $R$ son funciones de
$x$, $y$, $z$ es conocida como ecuación lineal de Lagrange.
Para resolver esta ecuación, resuelva el sistema EDO de Lagrange
\begin{math}
	\frac{\dl x}{P}=
	\frac{\dl y}{Q}=
	\frac{\dl z}{R}
\end{math}.
\begin{multicols}{3}
	\begin{parts}
		\part

		\begin{math}
			p+q=
			\cos x
		\end{math}.

		\part

		\begin{math}
			6p+7q=8
		\end{math}.

		\part

		\begin{math}
			px+qy=5z
		\end{math}.

		\part

		\begin{math}
			x^{2}p+y^{2}q=z^{2}
		\end{math}

		\part

		\begin{math}
			yzp+zxq=
			xy
		\end{math}

		\part

		\begin{math}
			p-q=\ln\left(x+y\right)
		\end{math}.

		\part

		\begin{math}
			\left(p-q\right)\left(x+y\right)=
			z
		\end{math}.

		\part

		\begin{math}
			xzp+yzq=
			xy
		\end{math}.

		\part

		\begin{math}
			5p-6q=5x^{4}\cos\left(6x+5y\right)
		\end{math}.
	\end{parts}
\end{multicols}

\question

Resuelva por las EDPs por el método de Charpit.
% Considere la compatibilidad de las siguientes EDPs de primer orden
% \begin{align*}
%     F\left(x,y,u,p,q\right) & =0 \\
%     G\left(x,y,u,p,q\right) & =0
% \end{align*}
% donde $p=\difcp{u}{x}$ y $q=\difcp{u}{y}$.
% Calculando las derivadas de $x$ e $y$ nos da
% \begin{align*}
%     \difcp{F}{x}+
%     p\difcp{F}{u}+
%     \difcp[2]{u}{x}\difcp{F}{p}+
%     \difcp[2]{F}{xy}\difcp{F}{q} & =0 \\
%     \difcp{F}{y}+
%     q\difcp{F}{u}+
%     \difcp[2]{u}{xy}\difcp{F}{p}+
%     \difcp[2]{F}{y}\difcp{F}{q}  & =0 \\
%     \difcp{G}{x}+
%     p\difcp{G}{u}+
%     \difcp[2]{u}{x}\difcp{G}{p}+
%     \difcp[2]{F}{xy}\difcp{G}{q} & =0 \\
%     \difcp{G}{y}+
%     q\difcp{G}{u}+
%     \difcp[2]{u}{xy}\difcp{G}{p}+
%     \difcp[2]{u}{y}\difcp{G}{q}  & =0
% \end{align*}
% Resolviendo las primeras tres ecuaciones para $\difcp[2]{u}{x}$, $\difcp[2]{u}{xy}$ y $\difcp[2]{u}{y}$
% resulta
% \begin{align*}
%     \difcp[2]{u}{x}  & =
%     \frac{-\difcp{F}{x}\difcp{G}{q}-p\difcp{F}{u}\difcp{G}{q}+\difcp{F}{q}\difcp{G}{x}+p\difcp{F}{q}\difcp{G}{u}}{\difcp{F}{p}\difcp{G}{q}-\difcp{F}{q}\difcp{G}{p}} \\
%     \difcp[2]{u}{xy} & =
%     \frac{-\difcp{F}{p}\difcp{G}{x}-p\difcp{F}{p}\difcp{G}{u}+\difcp{F}{x}\difcp{G}{p}+p\difcp{F}{u}\difcp{G}{p}}{\difcp{F}{p}\difcp{G}{q}-\difcp{F}{q}\difcp{G}{p}} \\
%     \difcp[2]{u}{y}  & =
%     \frac{}{\left(\difcp{F}{p}\difcp{G}{q}-\difcp{F}{q}\difcp{G}{p}\right)\difcp{F}{q}}
% \end{align*}
\begin{multicols}{2}
	\begin{parts}
		\part

		\begin{math}
			x\difcp{u}{x}-
			\left(\difcp{u}{y}\right)^{2}=
			2u
		\end{math}.

		\part

		\begin{math}
			\difcp{u}{x}
			\difcp{u}{y}=
			1
		\end{math}.

		\part

		\begin{math}
			\left(\difcp{u}{x}\right)^{2}+
			\left(\difcp{u}{y}\right)^{2}=
			u^{2}
		\end{math}.

		\part

		\begin{math}
			\difcp{u}{x}
			\difcp{u}{y}-
			x\difcp{u}{x}-
			y\difcp{u}{y}=
			0
		\end{math}.
	\end{parts}
\end{multicols}
