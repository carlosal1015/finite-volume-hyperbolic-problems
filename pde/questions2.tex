\question

Mediante integración muestre que se cumplen las identidades.
\begin{align*}
	\forall m,n\in\mathbb{Z},m\neq n:
	\int_{0}^{l}
	\sen
	\left(\frac{m\pi x}{l}\right)
	\sen
	\left(\frac{n\pi x}{l}\right)
	\dl x & =
	0.                                \\
	\forall m\in\mathbb{Z}:
	\int_{0}^{l}
	\sen^{2}
	\left(\frac{m\pi x}{l}\right)
	\dl x & =
	\frac{l}{2}.                      \\
	\forall m,n\in\mathbb{Z}:
	\int_{0}^{l}
	\cos
	\left(\frac{m\pi x}{l}\right)
	\cos
	\left(\frac{n\pi x}{l}\right)
	\dl x & =
	\begin{cases}
		0,           & \text{si }n\neq m. \\
		\frac{l}{2}, & \text{si }n=m.
	\end{cases} \\
	\forall m,n\in\mathbb{Z}:
	\int_{-l}^{l}
	\sen\left(\frac{m\pi x}{l}\right)
	\cos\left(\frac{n\pi x}{l}\right)
	\dl x & =
	0.
\end{align*}

\question

Defina
\begin{equation*}
	\phi\left(x\right)=
	\begin{cases}
		1, & 0<x\leq 1. \\
		2, & 1<x\leq 2. \\
		3, & 2<x\leq 3. \\
		4, & 3<x\leq 4.
	\end{cases}\quad
	\vcenter{\hbox{\includegraphics[width=.55\paperwidth]{2}}}
\end{equation*}
Extienda $\phi$ por periodicidad en todo $\mathbb{R}$; es decir,
\begin{math}
	\forall x\in\mathbb{R}:
	\phi\left(x+4\right)=
	\phi\left(x\right)
\end{math}.
Considere la serie de Fourier completa para $\phi$
\begin{equation*}
	\phi\left(x\right)\sim
	\frac{a_{0}}{2}+
	\sum_{n=1}^{\infty}
	\left[
		a_{n}
		\cos\left(\frac{n\pi x}{l}\right)+
		b_{n}
		\sen
		\left(\frac{n\pi x}{l}\right)
		\right].
\end{equation*}

\begin{parts}
	\part

	¿A qué valores convergerá esta serie de Fourier en $x=0$, $x=1$,
	$x=4$, $x=7.4$ y $x=40$?

	\part

	¿La serie de Fourier converge uniformemente a
	$\phi\left(x\right)$?
	Explique.

	\part

	Encuentre $a_{0}$.
\end{parts}

\question

Encuentre la serie de Fourier completa de las funciones definidas en
el intervalo $\left(-1,1\right)$.

\begin{multicols}{2}
	\begin{parts}
		\part

		\begin{math}
			\phi\left(x\right)=
			\cos\left(26\pi x\right)-
			4-
			3\sen\left(\pi x\right)
		\end{math}.

		\part

		\begin{math}
			\phi\left(x\right)=
			x
		\end{math}.

		\part

		\begin{math}
			\phi\left(x\right)=
			\left|x\right|+
			1
		\end{math}.

		\part

		\begin{math}
			\phi\left(x\right)=
			x^{3}+x^{2}
		\end{math}.
	\end{parts}
\end{multicols}

Dibuje las sumas parciales correspondientes a los primeros $3$, $5$ y
$30$ términos.
Utilice cualquier programa informático para superponer sus gráficos
con los de la función original.

\question

\begin{parts}
	\part

	Sean las siguientes funciones y sus respectivos coeficientes de
	Fourier.
	\begin{equation*}
		\begin{aligned}
			f_{1}
			\left(x\right) & =
			\frac{a_{0}}{2}+
			\sum_{n=1}^{\infty}
			a_{n}
			\cos\left(n\pi x\right). \\
			f_{2}
			\left(x\right) & =
			\sum_{n=1}^{\infty}
			b_{n}
			\sen
			\left(n\pi x\right).     \\
			f_{3}
			\left(x\right) & =
			\frac{c_{0}}{2}+
			\sum_{n=1}^{\infty}
			\left[
				c_{n}
				\cos\left(n\pi x\right)+
				d_{n}
				\sen\left(n\pi x\right)
				\right].
		\end{aligned}\qquad
		\begin{aligned}
			a_{n} & =
			2\int_{0}^{1}
			\left(x^{3}+1\right)
			\cos\left(n\pi x\right)\dl x. \\
			b_{n} & =
			2\int_{0}^{1}
			\left(x^{3}+1\right)
			\sen\left(n\pi x\right)\dl x. \\
			c_{n} & =
			\int_{-1}^{1}
			\left(x^{3}+1\right)
			\cos\left(n\pi x\right)\dl x. \\
			d_{n} & =
			\int_{-1}^{1}
			\left(x^{3}+1\right)
			\sen\left(n\pi x\right)\dl x.
		\end{aligned}
	\end{equation*}
	Dibuje las gráficas de $f_{1}$, $f_{2}$ y $f_{3}$ para $-3<x< 3$;
	es decir, dibuje los límites puntuales de las respectivas series
	de Fourier e indique los valores para
	$x\in\left\{-2,-1,0,1,2\right\}$.

	\part

	¿Cuál es el $\lim\limits_{n\to\infty}d_{n}$?
	¿Cuánto es
	\begin{math}
		c^{2}_{0}+
		\sum\limits_{n=1}^{\infty}
		\left(
		c^{2}_{n}+d^{2}_{n}
		\right)
	\end{math}?

	\part

	Sea
	\begin{math}
		f_{4}\left(x\right)=
		\sum\limits_{n=0}^{\infty}
		b_{n}
		\sen
		\left(n\pi x+\frac{\pi x}{2}\right)
	\end{math},
	donde
	\begin{math}
		b_{n}=
		2\int\limits_{0}^{1}
		x^{3}
		\left[
			\sen
			\left(n\pi x+\frac{\pi x}{2}\right)
			\right]
		\dl x
	\end{math}.
	Dibuje las gráficas de $f_{4}$ para $-5<x<5$; es decir,
	dibuje los límites puntuales de las respectivas series de
	Fourier.
\end{parts}

\question

Encuentre la serie de Fourier completa versión compleja de
\begin{math}
	f\left(x\right)=
	\exp\left(x\right)
\end{math}
en $\left(-1,1\right)$.

\question

Utilice la identidad de Parseval para los siguientes problemas.

\begin{parts}
	\part

	Considere $f\left(x\right)=\left|x\right|$ en $-\pi<x<\pi$ y
	muestre que
	\begin{math}
		\sum\limits_{n=0}^{\infty}
		\frac{1}{{\left(2n+1\right)}^{2}}=
		\frac{\pi^{2}}{8}
	\end{math}.

	\part

	Considere $f\left(x\right)=x$ en $-\pi<x<\pi$
	y muestre que
	\begin{math}
		\sum\limits_{n=1}^{\infty}
		\frac{1}{n^{2}}=
		\frac{\pi^{2}}{6}
	\end{math}
	y
	\begin{math}
		\sum\limits_{n=1}^{\infty}
		\frac{{\left(-1\right)}^{n+1}}{n^{2}}=
		\frac{\pi^{2}}{12}
	\end{math}.

	\part

	Considere $f\left(x\right)=x^{2}$ en $-\pi<x<\pi$ y muestre que
	\begin{math}
		\sum\limits_{n=1}^{\infty}
		\frac{1}{n^{4}}=
		\frac{\pi^{4}}{90}
	\end{math}.
\end{parts}

\question

Sean $f,g\colon\left[-\pi,\pi\right)\to\mathbb{R}$ dos funciones
integrables.
Extienda $f$ y $g$ por periodicidad a $\mathbb{R}$.
Defina la función de convolución $f\ast g$ como
\begin{equation*}
	\left(f\ast g\right)
	\left(x\right)\coloneqq
	\frac{1}{2\pi}
	\int_{-\pi}^{\pi}
	f\left(x-y\right)
	g\left(y\right)
	\dl y.
\end{equation*}

\begin{parts}
	\part

	Muestre que $f\ast g$ es una función periódica con período $2\pi$.

	\part

	Muestre que $f\ast g=g\ast f$.

	\part

	Sean los pares de coeficientes la serie de Fourier completa
	$\left(a^{f}_{n},b^{f}_{n}\right)$,
	$\left(a^{g}_{n},b^{g}_{n}\right)$,
	$\left(a^{f\ast g}_{n},b^{f\ast g}_{n}\right)$
	de las funciones, respectivamente, $f$, $g$ y $f\ast g$:

	Muestre que
	\vspace*{-\baselineskip}\setlength\belowdisplayshortskip{0pt}
	\begin{align*}
		\forall n\in
		\mathbb{N}\cup\left\{0\right\}:
		a^{f\ast g}_{n} & =
		a^{f}_{n}
		a^{g}_{n}.          \\
		\forall n\in\mathbb{N}:
		b^{f\ast g}_{n} & =
		b^{f}_{n}
		b^{g}_{n}.
	\end{align*}

	\part

	Muestre que $f\ast g$ es integrable en $\left(-\pi,\pi\right)$
	con
	\begin{equation*}
		\int_{-\pi}^{\pi}
		\left|
		\left(f\ast g\right)\left(x\right)
		\right|\dl x
		\leq
		\left(
		\int_{-\pi}^{\pi}
		\left|
		f\left(x\right)
		\right|\dl x
		\right)
		\left(
		\int_{-\pi}^{\pi}
		\left|
		g\left(x\right)
		\right|\dl x
		\right).
	\end{equation*}

	\part

	Si $f,g\in L^{2}\left(a,b\right)$, entonces
	\begin{math}
		\left|\left\langle f,g\right\rangle\right|\leq
		\left\|f\right\|
		\left\|g\right\|
	\end{math}.

	\textbf{Sugerencia}:
	Minimice la función
	\begin{math}
		G\left(t\right)\coloneqq
		{\left\|f+tg\right\|}^{2}
	\end{math}
	para cualquier $t\in\mathbb{R}$.
\end{parts}

\question

Muestre que la serie de Fourier completa de
\begin{math}
	\phi\left(x\right)=
	\begin{cases}
		-1, & \text{si }-\pi<x<0.    \\
		1,  & \text{si }0\leq x<\pi.
	\end{cases}
\end{math}
es
\begin{math}
	\dfrac{4}{\pi}
	\sum\limits_{n\text{ impar}}
	\dfrac{\sen\left(nx\right)}{n}
\end{math}.

\question

Muestre que el conjunto de eigenfunciones
\begin{math}
	\left\{
	\cos\left(\frac{n\pi x}{L}\right)
	\right\}_{n=0}^{\infty}
\end{math}
del problema
\begin{math}
	\left\{
		\begin{aligned}
			X^{\prime\prime}+
			\lambda X                & =
			0.                           \\
			X^{\prime}\left(0\right)=
			X^{\prime}\left(L\right) & =
			0.
		\end{aligned}
		\right.
	\end{math}\linebreak
es ortogonal.

\question

\begin{parts}
	\part

	Encuentre los eigenvalores y eigenfunciones del problema
	\begin{math}
		\left\{
		\begin{aligned}
			X^{\prime\prime}+
			X^{\prime}+
			\lambda X       & =
			0.                    \\
			X\left(0\right)=
			X\left(2\right) & =0. \\
		\end{aligned}
		\right.
	\end{math}

	\part

	Escriba la ecuación diferencial en la forma autoadjunta.

	\part

	Dé una relación de ortogonalidad.
\end{parts}

\question

Considere el problema de Sturm-Liouville
\begin{equation*}
	\left\{
	\begin{aligned}
		y^{\prime\prime}+
		\lambda y         & =
		0.                    \\
		y\left(0\right)=
		y\left(\pi\right) & =
		0.
	\end{aligned}
	\right.\qquad\qquad
	\vcenter{\hbox{\includegraphics[width=.25\paperwidth]{sturmapplication}}}
\end{equation*}
que tiene soluciones no triviales
\begin{math}
	\forall m\in\mathbb{N}:
	y_{m}\left(x\right)=
	\sen\left(nx\right)
\end{math}
y $\lambda_{m}=m^{2}$.

\begin{description}
	\item[Paso 1]

	      Resolvamos este problema numéricamente.
	      Introduciendo la diferenciación finita centrada
	      \begin{equation*}
		      \forall n=1,\dotsc,N:
		      y^{\prime\prime}_{n}\approx
		      \frac{y_{n+1}-2y_{n}+y_{n-1}}{\left(\Delta x\right)^{2}}
	      \end{equation*}
	      donde $y_{n}=y\left(x_{n}\right)$, $x_{n}=n\Delta x$ y
	      $\Delta x=\frac{\pi}{N+1}$.

	      Muestre que la forma de diferencia finita del problema de
	      valor de frontera dado
	      \begin{equation*}
		      -h^{2}y_{n+1}+
		      2h^{2}y_{n}-
		      h^{2}y_{n-1}=
		      \lambda y_{n}
	      \end{equation*}
	      con $y_{0}=y_{N+1}=0$ y $h=\frac{1}{\Delta x}$.

	\item[Paso 2]

	      Resuelva la ecuación anterior como un problema de
	      eigenvalor algebraico usando $N=1,\dotsc$.
	      Muestre que se puede escribir de la forma matricial
	      \begin{equation*}
		      h^{2}
		      \begin{bmatrix}
			      2      & -1     & 0      & \cdots & 0      & 0      & 0      \\
			      -1     & 2      & -1     & \cdots & 0      & 0      & 0      \\
			      0      & -1     & 2      & \cdots & 0      & 0      & 0      \\
			      \vdots & \vdots & \vdots & \ddots & \vdots & \vdots & \vdots \\
			      0      & 0      & 0      & \cdots & -1     & 2      & -1     \\
			      0      & 0      & 0      & \cdots & 0      & -1     & 2
		      \end{bmatrix}
		      \begin{bmatrix}
			      y_{1}   \\
			      y_{2}   \\
			      y_{3}   \\
			      \vdots  \\
			      y_{N-1} \\
			      y_{N}
		      \end{bmatrix}=
		      \lambda
		      \begin{bmatrix}
			      y_{1}   \\
			      y_{2}   \\
			      y_{3}   \\
			      \vdots  \\
			      y_{N-1} \\
			      y_{N}
		      \end{bmatrix}.
	      \end{equation*}
	      Note que la matriz de coeficientes es simétrica.

	\item[Paso 3]

	      Escriba un programa que $\forall N\geq 2$ calcule los
	      eigenvalores.
	      Se proporciona la tabla para que pueda comprobar su
	      programa.
	      Con su programa, responda las siguientes preguntas:
	      ¿Cómo se comparan los autovalores calculados con los
	      valores propios dados por el problema de Sturm-Liouville?
	      ¿Qué sucede al aumentar $N$?
	      ¿Qué autovalores calculados concuerdan mejor con los dados
	      por el problema de Sturm-Liouville?
	      ¿Cuáles presentan peores diferencias en la comparación?

	      \begin{table}[ht!]
		      \centering
		      \begin{tabular}{CCCCCCCC}
			      \hline
			      N  & \lambda_{1} & \lambda_{2} & \lambda_{3} & \lambda_{4} & \lambda_{5} & \lambda_{6} & \lambda_{7} \\
			      \hline
			      1  & 0.81057     &             &             &             &             &             &             \\
			      2  & 0.91189     & 2.73567     &             &             &             &             &             \\
			      3  & 0.94964     & 3.24228     & 5.53491     &             &             &             &             \\
			      4  & 0.96753     & 3.50056     & 6.63156     & 9.16459     &             &             &             \\
			      5  & 0.97736     & 3.64756     & 7.29513     & 10.94269    & 13.61289    &             &             \\
			      6  & 0.98333     & 3.73855     & 7.71996     & 12.13899    & 16.12040    & 18.87563    &             \\
			      7  & 0.98721     & 3.79857     & 8.00605     & 12.96911    & 17.93217    & 22.13966    & 24.95100    \\
			      20 & 0.99813     & 3.97023     & 8.84993     & 15.52822    & 23.85591    & 33.64694    & 44.68265    \\
			      50 & 0.99972     & 3.99498     & 8.97438     & 15.91922    & 24.80297    & 35.59203    & 48.24538    \\
			      \hline
		      \end{tabular}
	      \end{table}

	\item[Paso 4]

	      Represente gráficamente el $m$-ésimo eigenvector
	      $Cy^{\left(m\right)}_{n}$ en función de $x_{n}$, donde
	      $n,m=1,\dotsc,N$ y $C$ es tal que
	      \begin{math}
		      C^{2}\Delta x
		      \sum\limits_{n=1}^{N}\left[y^{\left(m\right)}_{n}\right]^{2}=
		      1
	      \end{math}.
	      En el mismo marco, grafique
	      $y_{m}=\sqrt{\frac{2}{\pi}}\sen\left(mx\right)$.
	      ¿Por qué elegimos $C$ como lo hicimos?
	      ¿Cuáles eigenvectores y eigenfunciones concuerdan mejor?
	      ¿Qué eigenvectores y eigenfunciones concuerdan peor?
	      ¿Por qué?
	      ¿Por qué hay $N$ eigenvectores y un número infinito de eigenfunciones?
\end{description}

\question

Encuentre los eigenvalores y eigenfunciones de los problemas de
eigenvalor en $\left[0,l\right]$ con condiciones de frontera:

\begin{multicols}{3}
	\begin{parts}
		\part

		Neumann
		\begin{equation*}
			\left\{
			\begin{aligned}
				X^{\prime\prime}+\lambda X                        & =
				0.                                                    \\
				X^{\prime}\left(0\right)=X^{\prime}\left(l\right) & =
				0.
			\end{aligned}
			\right.\qquad
		\end{equation*}

		\part

		Mixta
		\begin{equation*}
			\left\{
			\begin{aligned}
				X^{\prime\prime}+\lambda X               & =
				0.                                           \\
				X\left(0\right)=X^{\prime}\left(l\right) & =
				0.
			\end{aligned}
			\right.\qquad
		\end{equation*}

		\part

		Periódica
		\begin{equation*}
			\left\{
			\begin{aligned}
				X^{\prime\prime}+\lambda X & =0.                        \\
				X\left(0\right)            & =X\left(l\right).          \\
				X^{\prime}\left(0\right)   & =X^{\prime}\left(l\right).
			\end{aligned}
			\right.\qquad
		\end{equation*}
	\end{parts}
\end{multicols}

\question

Encuentre los eigenvalores positivos y las eigenfunciones
correspondientes para el problema de eigenvalor
$\mathcal{B}X=\lambda X$, donde $\mathcal{B}=-\diff[4]{}{x}$
con las condiciones de contorno
\begin{math}
	X\left(0\right)=
	X\left(l\right)=
	X^{\prime\prime}\left(0\right)=
	X^{\prime\prime}\left(l\right)=
	0
\end{math}.

\question

Verifique la condición de ortogonalidad mediante integración en cada
problema de Sturm-Liouville.

\begin{multicols}{2}

	\begin{parts}
		\part

		\begin{math}
			X_{n}\left(x\right)=
			\sen\left(\frac{n\pi x}{l}\right)
		\end{math}
		es solución de
		\begin{equation*}
			\left\{
			\begin{aligned}
				X^{\prime\prime}+\lambda X      & =0. \\
				X\left(0\right)=X\left(l\right) & =0.
			\end{aligned}
			\right.
		\end{equation*}

		\part

		\begin{math}
			X_{n}\left(x\right)=
			\cos\left(\frac{n\pi x}{l}\right)
		\end{math}
		es solución de
		\begin{equation*}
			\left\{
			\begin{aligned}
				X^{\prime\prime}+\lambda X                        & =0. \\
				X^{\prime}\left(0\right)=X^{\prime}\left(l\right) & =0.
			\end{aligned}
			\right.
		\end{equation*}

		\part

		\begin{math}
			X_{n}\left(x\right)=
			\sen\left(\frac{2n-1}{2l}\pi x\right)
		\end{math}
		es solución de
		\begin{equation*}
			\left\{
			\begin{aligned}
				X^{\prime\prime}+\lambda X               & =0. \\
				X\left(0\right)=X^{\prime}\left(l\right) & =0.
			\end{aligned}
			\right.
		\end{equation*}

		\part

		\begin{math}
			X_{n}\left(x\right)=
			\cos\left(\frac{2n-1}{2l}\pi x\right)
		\end{math}
		es solución de
		\begin{equation*}
			\left\{
			\begin{aligned}
				X^{\prime\prime}+\lambda X               & =0. \\
				X^{\prime}\left(0\right)=X\left(l\right) & =0.
			\end{aligned}
			\right.
		\end{equation*}
	\end{parts}
\end{multicols}

\question

Resuelva el problema de Sturm-Liouville dada por
\begin{equation*}
	\left\{
	\begin{aligned}
		{\left(e^{-6x}y^{\prime}\right)}^{\prime}+\left(1+\lambda\right)e^{-6x}y & =0. \\
		y\left(0\right)=y\left(8\right)                                          & =0.
	\end{aligned}
	\right.
\end{equation*}
