\question

Sea $h\colon\mathbb{R}\times\left(0,\infty\right)\to\mathbb{R}$ dado.
Utilice la transformada de Fourier para resolver el problema
\begin{equation*}
	\begin{cases}
		\difcp[2]{u}{t}-
		c^{2}
		\difcp[2]{u}{x}=
		h              &
		\text{en }\mathbb{R}\times\left(0,\infty\right). \\
		u=0            &
		\text{en }\mathbb{R}\times\left\{t=0\right\}.    \\
		\difcp{u}{t}=0 &
		\text{en }\mathbb{R}\times\left\{t=0\right\}.
	\end{cases}
\end{equation*}

\question

Utilice la transformada de Fourier para resolver el problema
\begin{equation*}
	\begin{cases}
		\difcp[2]{u}{t}-\difc.L.{u}{}=0 &
		\text{en }\mathbb{R}^{n}\times\left(0,\infty\right). \\
		u=g                             &
		\text{en }\mathbb{R}^{n}\times\left\{t=0\right\}.    \\
		\difcp{u}{t}=0                  &
		\text{en }\mathbb{R}^{n}\times\left\{t=0\right\}.
	\end{cases}
\end{equation*}
\begin{equation*}
	\text{donde }
	g\in\mathcal{S}\coloneqq
	\left\{
	u\in C^{\infty}\left(\mathbb{R}^{n}\right)\,\middle|\,
	\forall j\in\mathbb{N}_{0}:
	\forall\alpha\in\mathbb{N}^{n}_{0}:
	\sup_{x\in\mathbb{R}^{n}}
	\left\{
	\left[1+\left|x\right|^{j}\right]
	\left|
	D^{\alpha}
	u\left(x\right)
	\right|
	\right\}<\infty
	\right\}.
\end{equation*}

\textbf{Sugerencia}: Use la transformada Fourier de $u$ en la variable $x$,
la solución de la EDO resultante son de la forma $\beta e^{t\gamma}$,
donde $\left\{\beta,\gamma\right\}\subset\mathbb{C}$.

\question

Use el cambio de variables $\xi=x+ct$, $\eta=x-ct$ y transforme la
ecuación de onda 1D $\difcp[2]{u}{t}-c^{2}\difcp[2]{u}{x}=0$ en
$\difcp{u}{\xi,\eta}=0$ y halle la fórmula de D'Alembert.

\question

Sean $g\in C^2\left(\mathbb{R}\right)$,
$h\in C^{1}\left(\mathbb{R}\right)$ dados.
Encuentre una fórmula explícita para el problema
\begin{equation*}
	\begin{cases}
		\difcp[2]{u}{t}-\difcp[2]{u}{x}=0 &
		\text{en }\left(0,\infty\right)\times\left(0,\infty\right). \\
		u=0                               &
		\text{en }\left\{x=0\right\}\times\left(0,\infty\right).    \\
		u=g                               &
		\text{en }\left(0,\infty\right)\times\left\{t=0\right\}.    \\
		\difcp{u}{t}=
		h                                 &
		\text{en }\left(0,\infty\right)\times\left\{t=0\right\}.
	\end{cases}
\end{equation*}

\question

Sean $g\in C^3\left(\mathbb{R}^{3}\right)$,
$h\in C^{2}\left(\mathbb{R}^{3}\right)$ dados.
Pruebe que $u\in C^{2}\left(\mathbb{R}^{3}\times[0,\infty)\right)$
satisface el problema de valor inicial para la ecuación de onda 3D
\begin{equation*}
	\begin{cases}
		\difcp[2]{u}{t}-
		\difc.L.{u}{}=0 &
		\text{en }\mathbb{R}^{3}\times(0,\infty).         \\
		u=g             &
		\text{en }\mathbb{R}^{3}\times\left\{t=0\right\}. \\
		\difcp{u}{t}=
		h               &
		\text{en }\mathbb{R}^{3}\times\left\{t=0\right\}.
	\end{cases}
\end{equation*}

\question

Sea $u\in C^{2}\left(\mathbb{R}\times\left[0,\infty\right)\right)$
una solución de la ecuación de onda 1D
\begin{equation*}
	\begin{cases}
		\difcp[2]{u}{t}-
		\difcp[2]{u}{x}=0 &
		\text{en }\mathbb{R}\times\left(0,\infty\right). \\
		u=g               &
		\text{en }\mathbb{R}\times\left\{t=0\right\}.    \\
		\difcp{u}{t}=h    &
		\text{en }\mathbb{R}\times\left\{t=0\right\}.
	\end{cases}
\end{equation*}
Supongamos que $g\in C^{2}_{c}\left(\mathbb{R}\right)$ y
$h\in C^{1}_{c}\left(\mathbb{R}\right)$.
Se definen la energía cinética $k\left(t\right)$ y la energía
potencial $p\left(t\right)$ en el tiempo $t$ como
\begin{equation*}
	k\left(t\right)\coloneqq
	\frac{1}{2}
	\int_{-\infty}^{\infty}
	{\left(\difcp{u}{t}\right)}^{2}\dl x,\quad
	p\left(t\right)\coloneqq
	\frac{1}{2}
	\int_{-\infty}^{\infty}
	{\left(\difcp{u}{x}\right)}^{2}
	\dl x.
\end{equation*}
Pruebe que
\begin{multicols}{2}
	\begin{parts}
		\part


		\begin{math}
			\forall t\geq0: k\left(t\right)+p\left(t\right)
		\end{math}
		es constante.

		\part

		\begin{math}
			\lim\limits_{t\to\infty}
			k\left(t\right)=
			\lim\limits_{t\to\infty}
			p\left(t\right)
		\end{math}.
	\end{parts}
\end{multicols}
\textbf{Sugerencia}: Use que $u$ viene dado por
$u\left(x,t\right)=F\left(x+t\right)+G\left(x-t\right)$.

\question

Sea $u$ la solución de la ecuación de onda
\begin{equation*}
	\begin{cases}
		\difcp{u}{t}-
		\difc.L.{u}{}=0 &
		\text{en }\mathbb{R}^{3}\times(0,\infty).         \\
		u=g.            &
		\text{en }\mathbb{R}^{3}\times\left\{t=0\right\}. \\
		\difcp{u}{t}=h  &
		\text{en }\mathbb{R}^{3}\times\left\{t=0\right\}.
	\end{cases}
\end{equation*}
dada por la fórmula de Kirchhoff, donde $g,h$ son suaves y tienen
soporte compacto.
Muestre que $\exists C\in\mathbb{R}$ tal que
\begin{math}
	\forall\left(x,t\right)\in
	\mathbb{R}^{3}\times\left(0,\infty\right):
	\left|u\left(x,t\right)\right|\leq
	\frac{C}{t}
\end{math}.

\question

Se define una solución débil de la ecuación de onda 1D a una función
$u$ tal que
\begin{equation*}
	\forall\phi\in
	C^{2}_{c}\left(\mathbb{R}^{2}\right):
	\iint_{\mathbb{R}^{2}}
	u\left(x,t\right)
	\left[
		\difcp[2]{\phi}{t}\left(x,t\right)-
		c^{2}
		\difcp[2]{\phi}{x}\left(x,t\right)
		\right]
	\dl x\dl t=
	0.
\end{equation*}

\begin{parts}
	\part

	Muestre que toda solución clásica de la ecuación de onda 1D es una
	solución débil y que toda solución débil regular de la ecuación de
	onda es solución clásica.

	\part

	Muestre que las funciones discontinuas
	\begin{math}
		u_{1}\left(x,t\right)=
		H\left(x-t\right)
	\end{math}
	y
	\begin{math}
		u_{2}\left(x,t\right)=
		H\left(x+t\right)
	\end{math}
	son soluciones débiles de la ecuación de onda 1D,
	donde $H$ es la función de Heaviside
	\begin{equation*}
		H\left(x\right)=
		\begin{cases}
			0, & x<0.     \\
			1, & x\geq 0.
		\end{cases}.
	\end{equation*}
\end{parts}

\question

Sea $f\in C_{c}\left(\mathbb{R}\right)$.
Pruebe que $u\left(x,t\right)\equiv f\left(x-t\right)$ es solución
débil de la ecuación de onda 1D en el sentido del ejercicio anterior.

\question

Sea $\lambda\in\mathbb{R}$.
Encuentre una solución de
\begin{math}
	\difcp[2]{u}{t}-
	\difcp[2]{u}{x}=
	\lambda^{2}u
\end{math} de la forma $u=f\left(x^{2}-t^{2}\right)=f\left(s\right)$,
donde $f\left(0\right)=1$, en forma de serie de potencias en $s$.

\question

Hallar la solución de
\begin{equation*}
	\begin{cases}
		\difcp[2]{u}{t}-
		\difcp[2]{u}{x}=0 &
		\text{en }\left(0,\infty\right)\times\left(0,\infty\right). \\
		u=f               &
		\text{en }\left\{x=0\right\}\times\left(0,\infty\right).    \\
		u=g               &
		\text{en }\left(0,\infty\right)\times\left\{t=0\right\}.    \\
		\difcp{u}{t}=h    &
		\text{en }\left(0,\infty\right)\times\left\{t=0\right\}.
	\end{cases}
\end{equation*}
con $f,g\in C^{2}\left(\left(0,\infty\right)\right)$,
$h\in C^{1}\left(\left(0,\infty\right)\right)$ que satisfacen
$f\left(0\right)=g\left(0\right)$,
$f^{\prime}\left(0\right)=h\left(0\right)$
y $f^{\prime\prime}\left(0\right)=g^{\prime\prime}\left(0\right)$.
Verificar que la solución obtenida tiene derivadas de segundo orden
continuas aún sobre la característica $x=t$.

\question

Pruebe que si
\begin{math}
	u\left(\symbf{x},t\right)=
	w\left(\left|\symbf{x}\right|,t\right)
\end{math}
es una solución clásica radial de la ecuación de onda 3D, se tiene
que existen $F$ y $G$ tales que
\begin{math}
	u\left(x,t\right)=
	\dfrac{
		F\left(\left|\symbf{x}\right|-t\right)+
		G\left(\left|\symbf{x}\right|+t\right)
	}{\left|\symbf{x}\right|}
\end{math}.

\question

Sean $\symbf{B}\left(x,y,z,t\right)$, $\symbf{E}\left(x,y,z,t\right)$
el campo magnético y eléctrico, respectivamente.
Su evolución se rige por las leyes de Maxwell, que para el vacío, se
expresa como
\begin{align*}
	\nabla\times\symbf{E}                & =
	-\diffp{\symbf{B}}{t}.
	                                     &
	\nabla\times\symbf{B}=
	\frac{1}{c^{2}}\diffp{\symbf{E}}{t}. &   \\
	\nabla\cdot\symbf{E}                 & =
	0.                                   &
	\nabla\cdot\symbf{B}=0.              &
\end{align*}
Muestre que si
\begin{math}
	\symbf{B}\left(x,y,z,t\right)=
	\left(0,B\left(x,t\right),0\right)
\end{math}
y
\begin{math}
	\symbf{E}\left(x,y,z,t\right)=
	\left(0,0,E\left(x,t\right)\right)
\end{math},
entonces $B\left(x,t\right)$ y $E\left(x,t\right)$ satisfacen la
ecuación de onda
\begin{math}
	\difcp[2]{u}{t}-c^{2}\difcp[2]{u}{x}=
	0.
\end{math}

\question

Utilice la fórmula de Kirchhoff para escribir la solución explícita
de las ecuaciones de Maxwell.
Escribe la respuesta en forma vectorial.

\question

Resuelva el problema de Cauchy
\begin{equation*}
	\begin{cases}
		\difcp[2]{u}{t}-
		\difcp[2]{u}{x}=0                    &
		\text{en }\mathbb{R}\times(0,\infty).  \\
		u\left(x,0\right)=\sen\left(x\right) &
		\text{en }\mathbb{R}.                  \\
		\difcp{u}{t}\left(x,0\right)=x       &
		\text{en }\mathbb{R}.
	\end{cases}
\end{equation*}

\question

Sea $t^{\ast}>0$ un tiempo fijo, pero arbitrario.
Sean $u_{1}$ y $u_{2}$ las soluciones de los problemas de Cauchy
\begin{equation*}
	\begin{cases}
		\difcp[2]{u_{1}}{t}-
		\difcp[2]{u_{1}}{x}=0  &
		\text{en }\mathbb{R}\times(0,\infty). \\
		u_{1}\left(x,0\right)=
		\phi_{1}\left(x\right) &
		\text{en }\mathbb{R}.                 \\
		\difcp{u_{1}}{t}\left(x,0\right)=
		\psi_{1}\left(x\right) &
		\text{en }\mathbb{R}.
	\end{cases}\qquad
	\begin{cases}
		\difcp[2]{u_{2}}{t}-
		\difcp[2]{u_{2}}{x}=0  &
		\text{en }\mathbb{R}\times(0,\infty). \\
		u_{2}\left(x,0\right)=
		\phi_{2}\left(x\right) &
		\text{en }\mathbb{R}.                 \\
		\difcp{u_{2}}{t}\left(x,0\right)=
		\psi_{2}\left(x\right) &
		\text{en }\mathbb{R}.
	\end{cases}
\end{equation*}
Asuma que
\begin{math}
	\phi_{1},
	\phi_{2},
	\psi_{1},
	\psi_{2}\colon\mathbb{R}\to\mathbb{R}
\end{math}
son funciones acotadas.
Considere la siguiente proposición:
\begin{equation*}
	\forall\varepsilon>0:
	\exists\delta>0
	\text{ tal que }
	{
		\left\|
		\phi_{1}\left(x\right)-
		\phi_{2}\left(x\right)
		\right\|
	}_{\infty}<
	\delta,
	{
			\left\|
			\psi_{1}\left(x\right)-
			\psi_{2}\left(x\right)
			\right\|}_{\infty}<
	\delta
	\implies
	{
		\left\|
		u_{1}\left(x,t^{\ast}\right)-
		u_{2}\left(x,t^{\ast}\right)
		\right\|
	}_{\infty}<
	\varepsilon.
\end{equation*}

\begin{parts}
	\part

	Explique con palabras qué significa esta afirmación en
	términos de resolver el problema de valor inicial para la
	ecuación de onda.

	\part

	Prueba la proposición.
\end{parts}

\question

Resuelva el problema de valor inicial
\begin{equation*}
	\begin{cases}
		\difcp[2]{u}{x}-
		\difcp{u}{x,t}-
		4\difcp[2]{u}{t}=0                 &
		\text{en }\mathbb{R}\times(0,\infty). \\
		u\left(x,0\right)=
		x^{2}                              &
		\text{en }\mathbb{R}.                 \\
		\difcp{u}{t}\left(x,0\right)=e^{x} &
		\text{en }\mathbb{R}.
	\end{cases}
\end{equation*}
\textbf{Sugerencia}: ``Factorice'' la ecuación.

\question

Considere la ecuación de onda 1D para una cuerda de piano infinita
que está sujeta tanto a un ``punteo'' como a un ``golpe de martillo''
en $t=0$.
Para modelar esto, supongamos que la velocidad es $1$ y sea
\begin{equation*}
	\phi\left(x\right)=
	\begin{cases}
		1-\left|x-1\right|, & \left|x-1\right|<1.     \\
		0,                  & \left|x-1\right|\geq 1.
	\end{cases}\qquad
	\psi\left(x\right)=
	\begin{cases}
		1, & 3<x<4.                 \\
		0, & \text{caso contrario}.
	\end{cases}
\end{equation*}

\begin{parts}
	\part

	Use cualquier software para graficar los perfiles
	$u\left(x,t_{\text{fijo}}\right)$ como funciones de $x$ para los
	tiempos $t_{\text{fijo}}=0,\frac{1}{2},1,2,4,8$.
	Haz que el software genere una película de los primeros $20$
	segundos.

	\part

	Repita esta vez para la cuerda semiinfinita con un punto final fijo
	en $x=0$.
\end{parts}

\question

Considere la EDP
$\difcp[2]{u}{t}+\nu\difcp{u}{t}=c^{2}\difcp[2]{u}{x}$,
donde $\nu$ es el coeficiente de fricción.
Pruebe que la energía total (cinética más potencial) no se conserva,
pero decrece con el tiempo.

\question

Sea $u\colon\mathbb{R}\times(0,\infty)\to\mathbb{R}$
la solución del problema de valor inicial
\begin{equation*}
	\begin{cases}
		\difcp[2]{u}{t}-
		\difcp[2]{u}{x}=0  &
		\text{en }\mathbb{R}\times(0,\infty). \\
		u\left(x,0\right)=
		\phi\left(x\right) &
		\text{en }\mathbb{R}.                 \\
		\difcp{u}{t}\left(x,0\right)=
		\psi\left(x\right) &
		\text{en }\mathbb{R}.
	\end{cases}
\end{equation*}
Sean $t_{1}$ y $t_{2}$ dos instantes de tiempos fijos
con $0<t_{1}<t_{2}$.
Muestre que la solución del problema de valor inicial
\begin{equation*}
	\begin{cases}
		\difcp[2]{u}{t}-
		\difcp[2]{u}{x}=0 &
		\text{en }\mathbb{R}\times(t_{1},\infty). \\
		u\left(x,t_{1}\right)=
		\widetilde{\phi}
		\left(x\right)    &
		\text{en }\mathbb{R}.                     \\
		\difcp{u}{t}\left(x,t_{1}\right)=
		\widetilde{\psi}
		\left(x\right)    &
		\text{en }\mathbb{R}.
	\end{cases}
\end{equation*}
es
\begin{math}\displaystyle
	u\left(x,t_{2}\right)=
	\frac{1}{2}
	\left[
		u\left(x+c\left(t_{2}-t_{1}\right),t_{1}\right)+
		u\left(x-c\left(t_{2}-t_{1}\right),t_{1}\right)
		\right]+
	\frac{1}{2c}
	\int_{x-c\left(t_{2}-t_{1}\right)}^{x+c\left(t_{2}-t_{1}\right)}
	\difcp{u}{t}\left(s,t_{1}\right)\dl s
\end{math}.

Decimos que $u$ posee la propiedad de semigrupo.

\question

Discretice la ecuación de onda 1D por diferencias finitas sobre el
espacio $\left[-L,L\right]$ y el tiempo $\left[0, T\right]$ por
tamaños de paso $\Delta x>0$ y $\Delta t>0$.
Considere los puntos malla
\begin{align*}
	\forall j=-J,\dotsc,J:
	x_{j} & =j\Delta x,\quad
	J\Delta x=L.             \\
	\forall n=0,\dotsc,K:
	t_{n} & =n\Delta t,\quad
	K\Delta t=T.
\end{align*}
Sea $U^{n}_{j}\coloneqq u\left(j\Delta x,n\Delta t\right)$
la solución en los puntos malla.
Aproximemos las derivadas por
\begin{align*}
	\difcp[2]{u}{x} & \approx
	\frac{
		u\left(j\Delta x +\Delta x,n\Delta t\right)-
		2u\left(j\Delta x,n\Delta t\right)+
		u\left(j\Delta x-\Delta x, n\Delta t\right)
	}{
		{\left(\Delta x\right)}^{2}
	}=
	\frac{
	U^{n}_{j+1}-2U^{n}_{j}+U^{n}_{j-1}
	}{{\left(\Delta x\right)}^{2}}. \\
	\difcp[2]{u}{t} & \approx
	\frac{
		u\left(j\Delta x, n\Delta t+\Delta t\right)-
		2u\left(j\Delta x,n\Delta t\right)+
		u\left(j\Delta x,n\Delta t-\Delta t\right)}{
		{\left(\Delta t\right)}^{2}
	}=
	\frac{
	U^{n+1}_{j}-2U^{n}_{j}+U^{n-1}_{j}
	}{
	{\left(\Delta t\right)}^{2}
	}.                              \\
\end{align*}
Reemplace las aproximaciones de las derivadas de segundo orden en la
ecuación de onda
\begin{align*}
	\difcp[2]{u}{t} & =
	c^{2}\difcp[2]{u}{x}.     \\
	\frac{
	U^{n+1}_{j}-2U^{n}_{j}+U^{n-1}_{j}
	}{
	{\left(\Delta t\right)}^{2}
	}               & \approx
	c^{2}
	\frac{
	U^{n}_{j+1}-2U^{n}_{j}+U^{n}_{j-1}
	}{
	{\left(\Delta x\right)}^{2}
	}.
	\shortintertext{Sea
	$r={\left(\frac{c\Delta t}{\Delta x}\right)}^{2}$.
	El esquema Leapfrog viene dado por}
	\Aboxed{
	\forall j=-J,\dotsc,J:
	\forall n=0,\dotsc,K:
	U^{n+1}_{j}     & =
	r\left(U^{n}_{j+1}+U^{n}_{j-1}\right)+
	2\left(1-r\right)U^{n}_{j}-
	U^{n-1}_{j}.
	}
\end{align*}

Calcule la solución de la ecuación de onda con $c=1$ y datos
iniciales
\begin{equation*}
	\phi\left(j\Delta x\right)=
	\begin{cases}
		2, & j=0.                   \\
		1, & j=-1,1.                \\
		0, & \text{caso contrario}.
	\end{cases}\qquad
	\forall j=-J,\dotsc,J:
	\psi\left(j\Delta x\right)=
	0.
\end{equation*}

\begin{parts}
	\part

	Sea $\Delta x=0.1$ y calcule la solución después de $10$ pasos de
	tiempo usando $r=1,\frac{1}{2},2$.
	Grafique la solución en cada caso.

	\part

	Sea $\Delta x=0.1$ y $r=1$ calcule la solución aproximada en
	$t=10$.

	\part

	Repita los dos pasos anteriores para los datos discretos
	\begin{equation*}
		\forall j=-J,\dotsc,J:
		\phi\left(j\Delta x\right)=
		0.\qquad
		\psi\left(j\Delta x\right)=
		\begin{cases}
			2, & j=0.                   \\
			1, & j=-1,1.                \\
			0, & \text{caso contrario}.
		\end{cases}
	\end{equation*}
\end{parts}

\question

La solución de la ecuación de onda dada por la fórmula de D'Alembert
satisface el principio del máximo
\begin{equation*}
	\forall\left(x,t\right)\in\mathbb{R}\times\left(0,\infty\right):
	\left|u\left(x,t\right)\right|\leq
	\max_{y\in\mathbb{R}}
	\left|\phi\left(y\right)\right|.
\end{equation*}
Muestre que el esquema Leapfrog con $r=1$ satisface el principio del
máximo discreto
\begin{equation*}
	\forall j\in\mathbb{Z}:
	\forall n\in\mathbb{N}:
	\left|U^{n}_{j}\right|\leq
	\max_{i\in\mathbb{Z}}
	\left|\phi\left(i\Delta x\right)\right|.
\end{equation*}

\begin{parts}
	\part

	Muestre que
	\begin{math}
		U^{1}_{j}=
		\frac{1}{2}
		\left(U^{0}_{j+1}+U^{0}_{j-1}\right)
	\end{math}
	y concluya que
	\begin{math}
		\left|U^{1}_{j}\right|\leq
		\max_{i\in\mathbb{Z}}
		\left|\phi\left(i\Delta x\right)\right|
	\end{math}.

	\part

	Ahora, pruebe que
	\begin{math}
		U^{2}_{j}=
		\frac{1}{2}
		\left(U^{0}_{j+2}+U^{0}_{j-2}\right)
	\end{math}
	y concluya que
	\begin{math}
		\left|U^{2}_{j}\right|\leq
		\max_{i\in\mathbb{Z}}
		\left|\phi\left(i\Delta x\right)\right|
	\end{math}.

	\part

	Pruebe el principio del máximo discreto por inducción matemática.

	\part

	Para mostrar que el esquema Leapfrog preserva el área.
	Primero muestre que si $\phi\left(x\right)>0$ con
	\begin{math}\displaystyle
		\int_{-\infty}^{\infty}
		\phi\left(x\right)\dl x<
		\infty
	\end{math},
	entonces
	\begin{math}\displaystyle
		\forall t>0:
		\int_{-\infty}^{\infty}
		u\left(x,t\right)\dl x=
		\int_{-\infty}^{\infty}
		\phi\left(x\right)\dl x
	\end{math}.

	Luego, muestre que
	\begin{math}
		\forall n\in\mathbb{N}:
		\sum_{j\in\mathbb{Z}}
		U^{n}_{j}=
		\sum_{j\in\mathbb{Z}}
		\phi\left(j\Delta x\right)
	\end{math}.
\end{parts}

\question

Resuelva la ecuación de onda 3D con datos iniciales
$\phi\left(x,y,z,t\right)\equiv 0$ y
$\psi\left(x,y,z,t\right)=y$
y rapidez $c=1$.

\question

Considere el problema de valor inicial de la ecuación de onda 3D
\begin{equation*}
	\begin{cases}
		\difcp[2]{u}{t}=
		c^{2}\difc.L.{u}{}                      &
		\text{en }\mathbb{R}^{3}\times\left(0,\infty\right). \\
		u\left(\symbf{x},0\right)=
		\phi\left(\left|\symbf{x}\right|\right) &
		\text{en }\mathbb{R}^{3}.                            \\
		\difcp{u}{t}\left(\symbf{x},0\right)=
		\psi\left(\left|\symbf{x}\right|\right) &
		\text{en }\mathbb{R}^{3}.
	\end{cases}
\end{equation*}
Sean $\widetilde{\phi}$ y $\widetilde{\psi}$ es la extensión par de
$\phi$ y $\psi$ a todo $\mathbb{R}$, respectivamente.
Muestre que la solución es
\begin{equation*}
	u\left(\symbf{x},t\right)=
	\frac{1}{2\left|\symbf{x}\right|}
	\left(\left|\symbf{x}\right|+ct\right)
	\widetilde{\phi}\left(\left|\symbf{x}\right|+ct\right)+
	\left(\left|\symbf{x}\right|-ct\right)
	\widetilde{\phi}\left(\left|\symbf{x}\right|-ct\right)+
	\frac{1}{2\left|\symbf{x}\right|}
	\int_{\left|\symbf{x}\right|-ct}^{\left|\symbf{x}\right|+ct}
	s\widetilde{\psi}
	\left(s\right)
	\dl s.
\end{equation*}
Sugerencia: No intente demostrar con la fórmula de Kirchhoff,
sino busque una solución
\begin{math}
	u\left(\symbf{x},t\right)=
	u\left(r,t\right)
\end{math},
donde $r=\left|\symbf{x}\right|$ es la norma vectorial de
$\symbf{x}\in\mathbb{R}^{3}$, trabajando en coordenadas esféricas y
notando que
\begin{math}
	v\left(r,t\right)\coloneqq
	ru\left(r,t\right)
\end{math}
resuelve la ecuación de onda 1D.
Use la siguiente identidad, donde
\begin{math}
	f\left(\left|\symbf{x}\right|\right)=
	g\left(r\right)
\end{math}.
\begin{equation*}
	\iiint_{B\left(\symbf{0},a\right)}
	f\left(\symbf{x}\right)\dl{\symbf{x}}=
	\int_{0}^{\pi}
	\int_{0}^{2\pi}
	\int_{0}^{a}
	r^{2}
	g\left(r\right)
	\sen\phi
	\dl r
	\dl\theta
	\dl\phi=
	\int_{0}^{a}
	r^{2}
	g\left(r\right)
	\dl r
	\left(
	\int_{0}^{\pi}
	\int_{0}^{2\pi}
	\sen\phi
	\dl\theta
	\dl\phi
	\right)=
	4\pi
	\int_{0}^{a}
	r^{2}
	g\left(r\right).
\end{equation*}
% \iiint_{\partial B\left(\symbf{0},a\right)}
% f\left(\symbf{x}\right)\dl S
%                                       & =
% \int_{0}^{\pi}
% \int_{0}^{2\pi}
% g\left(a\right)
% a^{2}
% \sen\phi
% \dl\theta
% \dl\phi=
% 4\pi a^{2}
% g\left(a\right).

\question

Aplique el método de separación de variables para resolver
\begin{equation*}
	\begin{cases}
		\difcp[2]{u}{t}=
		c^{2}\difc.L.{u}{}         &
		\text{en }B\left(\symbf{0},1\right)\times
		\left(0,\infty\right).               \\
		u\left(\symbf{x},t\right)=
		0                          &
		\text{en }\partial B\left(\symbf{0},1\right)\times
		\left(0,\infty\right).               \\
		u\left(\symbf{x},0\right)=
		\phi\left(\symbf{x}\right) &
		\text{en }B\left(\symbf{0},1\right). \\
		\difcp{u}{t}\left(\symbf{x},0\right)=
		\psi\left(\symbf{x}\right) &
		\text{en }B\left(\symbf{0},1\right).
	\end{cases}
\end{equation*}
donde $B\left(\symbf{0},1\right)\subset\mathbb{R}^{3}$ es la bola
unitaria.
Busque soluciones separadas en coordenadas esféricas de la forma
\begin{math}
	u\left(r,\phi,\theta,t\right)=
	U\left(r,\phi,\theta\right)T\left(t\right)=
	R\left(r\right)
	\Phi\left(\phi\right)
	\Theta\left(\theta\right)
	T\left(t\right)
\end{math}
y obtenga
\begin{math}
	U\left(r,\phi,\theta\right)
	T^{\prime\prime}\left(t\right)=
	c^{2}\difc.L.{U}{}
	\left(r,\phi,\theta\right)
	T\left(t\right)
\end{math}.

\question

Determine una fórmula explícita de la solución del problema de valor
inicial de la ecuación de onda 3D no homogénea
\begin{equation*}
	\begin{cases}
		\difcp[2]{u}{t}-c^{2}\difc.L.{u}{}=
		f\left(\symbf{x},t\right)  &
		\text{en }\mathbb{R}^{3}\times\left(0,\infty\right). \\
		u\left(\symbf{x},t\right)=
		\phi\left(\symbf{x}\right) &
		\text{en }\mathbb{R}^{3}.                            \\
		\difcp{u}{t}\left(\symbf{x},0\right)=
		\psi\left(\symbf{x}\right) &
		\text{en }\mathbb{R}^{3}.
	\end{cases}
\end{equation*}
en dos etapas.
\vspace*{-\baselineskip}\setlength\belowdisplayshortskip{0pt}
\begin{enumerate}
	\item

	      Primero, muestre que
	      \begin{math}\displaystyle
		      u\left(\symbf{x},t\right)=
		      \int_{0}^{t}
		      w\left(\symbf{x},t;s\right)
		      \dl s
	      \end{math}
	      es la solución del problema de valor inicial
	      \begin{equation*}
		      \begin{cases}
			      \difcp[2]{u}{t}-c^{2}\difc.L.{u}{}=
			      f\left(\symbf{x},t\right) &
			      \text{en }\mathbb{R}^{3}\times\left(0,\infty\right). \\
			      u\left(\symbf{x},t\right)=
			      0                         &
			      \text{en }\mathbb{R}^{3}.                            \\
			      \difcp{u}{t}\left(\symbf{x},0\right)=
			      0                         &
			      \text{en }\mathbb{R}^{3}.
		      \end{cases}
	      \end{equation*}
	      donde para cada $s<t$ fijo,
	      $w\left(\symbf{x},t;s\right)$ es la solución indexada por $s$
	      del problema
	      \begin{equation*}
		      \begin{cases}
			      \difcp[2]{w}{t}\left(\symbf{x},t;s\right)=
			      c^{2}\difc.L.{w}{}\left(\symbf{x},t;s\right) &
			      \text{en }\mathbb{R}^{3}\times\left(s,\infty\right). \\
			      w\left(\symbf{x},s;s\right)=
			      0                                            &
			      \text{en }\mathbb{R}^{3}.                            \\
			      \difcp{w}{t}\left(\symbf{x},s;s\right)=
			      f\left(\symbf{x},s\right)                    &
			      \text{en }\mathbb{R}^{3}.
		      \end{cases}
	      \end{equation*}
	      Este resultado se conoce como el Principio de Duhamel.

	\item

	      Finalmente, use la fórmula de Kirchhoff para encontrar la
	      solución explícita del problema original.
\end{enumerate}

\question

Sea $u\left(\symbf{x},t\right)$ la solución del problema de valor
inicial de la ecuación de onda 3D
\begin{equation*}
	\begin{cases}
		\difcp[2]{u}{t}=
		c^{2}\difc.L.{u}{} &
		\text{en }\mathbb{R}^{3}\times\left(0,\infty\right). \\
		u=
		\phi               &
		\text{en }\mathbb{R}^{3}\times\left\{t=0\right\}.    \\
		\difcp{u}{t}=\psi  &
		\text{en }\mathbb{R}^{3}\times\left\{t=0\right\}.
	\end{cases}
\end{equation*}
Donde $\phi$ y $\psi$ son funciones suaves de soporte compacto.

\begin{parts}
	\part

	Muestre que la energía total (cinética más potencial)
	\begin{equation*}
		\forall t\geq0:
		E\left(t\right)=
		\frac{1}{2}
		\iint_{\mathbb{R}^{3}}
		\left(
		\difcp[2]{u}{t}
		\left(\symbf{x},t\right)+
		c^{2}
		{\left|\nabla u\left(\symbf{x},t\right)\right|}^{2}
		\right)
		\dl{\symbf{x}}
	\end{equation*}
	es conservado en el tiempo, es decir, $E^{\prime}\left(t\right)=0$.

	\part

	A partir del ítem anterior, pruebe que existe al menos una solución
	del problema de valor inicial de la ecuación de onda 3D.
\end{parts}
