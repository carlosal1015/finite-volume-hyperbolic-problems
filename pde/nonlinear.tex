\section{EDPs no lineales}

El mundo que nos rodea es modelado por EDPs no lineales.
A continuación, se muestra una breve lista de lugares en los que se
pueden encontrar estas ecuaciones.

\begin{description}
	\item[Difusión no lineal]
	      es una ecuación diferencial parcial no lineal que modela la
	      transferencia de calor en un medio donde la conductividad
	      térmica puede depender de la temperatura.
	      \begin{equation*}
		      \difcp{u}{t}=
		      \difcp{\left[D\left(u\right)\difcp{u}{x}\right]}{x}.
	      \end{equation*}

	\item[Onda no lineal]
	      modela la propagación de ondas y aparece en aplicaciones
	      que involucran gases unidimensionales, ondas en aguas poco
	      profundas, líneas de filamento longitudinales, cuerdas no
	      lineales finitas, materiales elastoplásticos y líneas de
	      transmisión, por nombrar solo algunas.
	      \begin{equation*}
		      \difcp[2]{u}{t}=
		      \difcp{\left[c\left(u\right)^{2}\difcp{u}{x}\right]}{x}.
	      \end{equation*}

	\item[Burgers]
	      es una ecuación diferencial parcial que incorpora tanto la
	      no linealidad como la difusión.
	      Se introdujo inicialmente como un modelo simplificado de
	      turbulencia y aparece en diversas áreas de las matemáticas
	      aplicadas, como el flujo suelo-agua, la acústica no lineal
	      y el flujo de tráfico.
	      \begin{equation*}
		      \difcp{u}{t}+u\difcp{u}{x}=
		      \nu\difcp[2]{u}{x}.
	      \end{equation*}

	\item[Fisher]
	      es un modelo propuesto para la onda de avance de genes
	      ventajosos y tiene aplicaciones en la agricultura temprana,
	      la propagación de ondas químicas, los reactores nucleares,
	      la cinética química y en la teoría de la combustión.
	      \begin{equation*}
		      \difcp{u}{t}=
		      \difcp[2]{u}{x}+
		      u\left(1-u\right).
	      \end{equation*}

	\item[Fitzhugh-Nagumo]
	      modela la transmisión de impulsos nerviosos y surge en
	      modelos de genética de poblaciones.
	      \begin{equation*}
		      \difcp{u}{t}=
		      \difcp[2]{u}{x}+
		      u\left(1-u\right)
		      \left(u+\lambda\right).
	      \end{equation*}

	\item[Korteweg de Vries]
	      describe la evolución de largas ondas de agua a lo largo de
	      un canal de sección transversal rectangular.
	      También se ha demostrado que modela ondas longitudinales
	      que se propagan en una red unidimensional, ondas
	      iónico-acústicas en un plasma frío, ondas en varillas
	      elásticas y se utiliza para describir la componente axial
	      de la velocidad en un flujo de fluido giratorio a lo largo de
	      un tubo.
	      \begin{equation*}
		      \difcp{u}{t}+
		      6u\difcp{u}{x}+
		      \difcp[3]{u}{x}=
		      0.
	      \end{equation*}

	\item[Boussinesq]
	      fue introducido por Boussinesq en $1871$ para modelar ondas
	      en aguas poco profundas en canales largos.
	      También surge en otras aplicaciones, como en ondas
	      reticulares no lineales unidimensionales, vibraciones en una
	      cuerda no lineal y ondas sonoras iónicas en un plasma.
	      \begin{equation*}
		      \difcp[2]{u}{t}+
		      \difcp[2]{u}{x}+
		      \difcp{\left(2u\difcp{u}{x}\right)}{x}
		      \frac{1}{3}\difcp[4]{u}{x}=
		      0.
	      \end{equation*}

	\item[Eikonal]
	      aparece en la óptica de rayos.
	      \begin{equation*}
		      \left|\nabla u\right|=
		      F\left(\symbf{x}\right).
	      \end{equation*}

	\item[Gross-Pitáyevski]
	      es un modelo para la función de onda de una sola partícula
	      en un condensado de Bose-Einstein.
	      \begin{equation*}
		      i\difcp{\psi}{t}=
		      -\nabla^{2}\psi+
		      \left[
		      V\left(x\right)+
		      {\left|\psi\right|}^{2}
		      \right]
		      \psi.
	      \end{equation*}

	\item[Plateau]
	      surge en el estudio de superficies mínimas.
	      \begin{equation*}
		      \left[
		      1+
		      {\left(\difcp{u}{y}\right)}^{2}
		      \right]
		      \difcp[2]{u}{x}-
		      2\difcp{u}{x}
		      \difcp{u}{y}
		      \difcp[2]{u}{x,y}+
		      \left[
		      1+{\left(\difcp{u}{x}\right)}^{2}
		      \right]
		      \difcp[2]{u}{y}=
		      0.
	      \end{equation*}

	\item[Sine-Gordon]
	      surge en el estudio de superficies de curvatura negativa
	      constante y en el estudio de dislocaciones cristalinas.
	      \begin{equation*}
		      \difcp[2]{u}{x,y}=
		      \sen\left(u\right).
	      \end{equation*}

	\item[Equilibrio]
	      surgen en elasticidad.
	      Aquí, $\sigma_{xx}$, $\sigma_{xy}$ y $\sigma_{yy}$ son
	      tensiones normales y cortantes, y $F_{x}$ y $F_{y}$ son
	      fuerzas del cuerpo.
	      Estas han sido utilizadas para modelar materiales granulares
	      con alta fricción.
	      \begin{align*}
		      \diffp{\sigma_{xx}}{x}+
		      \diffp{\sigma_{xy}}{y}+
		      F_{x} & =
		      0.        \\
		      \diffp{\sigma_{xy}}{x}+
		      \diffp{\sigma_{yy}}{y}+
		      F_{y} & =
		      0.
	      \end{align*}

	\item[Navier-Stokes]
	      describe el campo de velocidad y presión de fluidos
	      incompresibles.
	      Aquí $\nu$ es la viscosidad cinemática, $\symbf{u}$ es la
	      velocidad de la parcela de fluido, $P$ es la presión y $\rho$
	      es la densidad del fluido.
	      \begin{align*}
		      \nabla\cdot\symbf{u}          & =
		      0.                                \\
		      \difcp{\symbf{u}}{t}+
		      \symbf{u}\cdot\nabla\symbf{u} & =
		      -\frac{\nabla P}{\rho}+
		      \nu\nabla^{2}\symbf{u}.
	      \end{align*}
\end{description}
