\section{Ecuación de onda}

\begin{theorem}[Solución general de la ecuación de onda 1D]
	La solución general de
	\begin{math}
		\difcp[2]{u}{t}-
		c^{2}\difcp[2]{u}{x}=
		0
	\end{math}
	es
	\begin{math}
		u\left(x,t\right)=
		f\left(x+ct\right)+
		g\left(x-ct\right)
	\end{math}
	para funciones arbitrarias
	\begin{math}
		f,g\in
		C^{2}\left(\mathbb{R}\right)
	\end{math}.
\end{theorem}

\begin{theorem}[Fórmula de D'Alembert]
	Sean
	\begin{math}
		\phi\in
		C^{2}\left(\mathbb{R}\right),
		\psi\in
		C^{1}\left(\mathbb{R}\right)
	\end{math}
	dados.
	La solución del problema de valor inicial de la ecuación de onda
	\begin{equation*}
		\begin{cases}
			\difcp[2]{u}{t}=
			c^{2}\difcp[2]{u}{x} &
			\text{en }\mathbb{R}\times\left(0,\infty\right). \\
			u=\phi               &
			\text{en }\mathbb{R}\times\left\{t=0\right\}.    \\
			\difcp{u}{t}=\psi    &
			\text{en }\mathbb{R}\times\left\{t=0\right\}.
		\end{cases}
	\end{equation*}
	es
	\begin{math}\displaystyle
		u\left(x,t\right)=
		\frac{1}{2}
		\left[
			\phi\left(x+ct\right)+
			\phi\left(x-ct\right)
			\right]+
		\frac{1}{2c}
		\int_{x-ct}^{x+ct}
		\psi\left(s\right)\dl s
	\end{math}.
\end{theorem}

\begin{theorem}
	Sean
	\begin{math}
		\phi\in
		C^{2}\left(\mathbb{R}\right),
		\psi\in
		C^{1}\left(\mathbb{R}\right)
	\end{math}
	dados y $D$ es el dominio de dependencia asociado con
	$\left(x,t\right)$.
	La solución del problema de valor inicial de la ecuación de onda
	\begin{equation*}
		\begin{cases}
			\difcp[2]{u}{t}-c^{2}\difcp[2]{u}{x}=
			f    &
			\text{en }\mathbb{R}\times\left(0,\infty\right). \\
			u=
			\phi &
			\text{en }\mathbb{R}\times\left\{t=0\right\}.    \\
			\difcp{u}{t}=
			\psi &
			\text{en }\mathbb{R}\times\left\{t=0\right\}.
		\end{cases}
	\end{equation*}
	es
	\begin{math}\displaystyle
		u\left(x,t\right)=
		\frac{1}{2}
		\left[
			\phi\left(x+ct\right)+
			\phi\left(x-ct\right)
			\right]+
		\frac{1}{2c}
		\int_{x-ct}^{x+ct}
		\psi\left(s\right)\dl s+
		\frac{1}{2c}
		\iint_{D}
		f\left(y,\tau\right)\dl y\dl\tau
	\end{math}.
\end{theorem}

\begin{theorem}
	Sean
	\begin{math}
		\phi\in
		C^{2}\left(\mathbb{R}\right),
		\psi\in
		C^{1}\left(\mathbb{R}\right)
	\end{math}
	dados.
	La solución del problema de valor inicial / frontera de la ecuación
	de onda
	\begin{equation*}
		\begin{cases}
			\difcp[2]{v}{t}=
			c^{2}\difcp[2]{v}{x} &
			\text{en }\left[0,\infty\right)\times\left[0,\infty\right). \\
			v=
			0                    &
			\text{en }\left\{x=0\right\}\times\left[0,\infty\right).    \\
			v=
			\phi                 &
			\text{en }\left[0,\infty\right)\times\left\{t=0\right\}.    \\
			\difcp{v}{t}=
			\psi                 &
			\text{en }\left[0,\infty\right)\times\left\{t=0\right\}.
		\end{cases}
	\end{equation*}
	es
	\begin{math}
		v\left(x,t\right)=
		\begin{cases}
			\displaystyle
			\frac{1}{2}
			\left[
				\phi\left(x+ct\right)+
				\phi\left(x-ct\right)
				\right]+
			\frac{1}{2c}
			\int_{x-ct}^{x+ct}
			\psi\left(s\right)\dl s, & x\geq ct.   \\
			\displaystyle
			\frac{1}{2}
			\left[
				\phi\left(x+ct\right)+
				\phi\left(ct-x\right)
				\right]+
			\frac{1}{2c}
			\int_{ct-x}^{x+ct}
			\psi\left(s\right)\dl s, & 0<x\leq ct.
		\end{cases}
	\end{math}
\end{theorem}

\begin{theorem}
	Sean
	\begin{math}
		\phi\in
		C^{2}\left(\mathbb{R}\right),
		\psi\in
		C^{1}\left(\mathbb{R}\right)
	\end{math}
	dados.
	La solución del problema de valor inicial / frontera de la ecuación
	de onda
	\begin{equation*}
		\begin{cases}
			\difcp[2]{u}{t}=
			c^{2}\difcp[2]{u}{x} &
			\text{en }\left(0,l\right)\times\left(0,\infty\right).     \\
			u=0                  &
			\text{en }\left\{x=0,l\right\}\times\left[0,\infty\right). \\
			u=\phi               &
			\text{en }\left(0,l\right)\times\left\{t=0\right\}.        \\
			\difcp{u}{t}=\psi    &
			\text{en }\left(0,l\right)\times\left\{t=0\right\}.
		\end{cases}
	\end{equation*}
	es
	\begin{math}\displaystyle
		u\left(x,t\right)=
		\sum_{n=1}^{\infty}
		\left[
			c_{n}
			\cos\left(\frac{n\pi ct}{l}\right)+
			d_{n}
			\sen
			\left(\frac{n\pi ct}{l}\right)
			\right]
		\sen
		\left(\frac{n\pi x}{l}\right)
	\end{math},
	donde los coeficientes de la serie son
	\begin{equation*}
		\forall n\in\mathbb{N}:
		c_{n}=
		\frac{2}{l}
		\int_{0}^{l}
		\phi\left(x\right)
		\sen
		\left(\frac{n\pi x}{l}\right)\dl x,\qquad
		d_{n}=
		\frac{2}{n\pi c}
		\int_{0}^{l}
		\psi\left(x\right)
		\sen
		\left(\frac{n\pi x}{l}\right)\dl x.
	\end{equation*}
\end{theorem}

\begin{theorem}
	Sean
	\begin{math}
		\phi\in
		C^{2}\left(\mathbb{R}\right),
		\psi\in
		C^{1}\left(\mathbb{R}\right)
	\end{math}
	dados.
	La solución del problema de valor inicial / frontera de la ecuación
	de onda
	\begin{equation*}
		\begin{cases}
			\difcp[2]{u}{t}=
			c^{2}\difcp[2]{u}{x} &
			\text{en }\left(0,l\right)\times\left(0,\infty\right).     \\
			\difcp{u}{x}=0       &
			\text{en }\left\{x=0,l\right\}\times\left[0,\infty\right). \\
			u=\phi               &
			\text{en }\left(0,l\right)\times\left\{t=0\right\}.        \\
			\difcp{u}{t}=\psi    &
			\text{en }\left(0,l\right)\times\left\{t=0\right\}.
		\end{cases}
	\end{equation*}
	es
	\begin{math}\displaystyle
		u\left(x,t\right)=
		\frac{c_{0}}{2}+
		\frac{d_{0}}{2}t+
		\sum_{n=1}^{\infty}
		\left[
			c_{n}
			\cos\left(\frac{n\pi ct}{l}\right)+
			d_{n}
			\sen
			\left(\frac{n\pi ct}{l}\right)
			\right]
		\cos
		\left(\frac{n\pi x}{l}\right)
	\end{math},
	donde los coeficientes son
	\begin{equation*}
		\forall n\in\mathbb{N}\cup\left\{0\right\}:
		c_{n}=
		\frac{2}{l}
		\int_{0}^{l}
		\phi\left(x\right)
		\cos
		\left(\frac{n\pi x}{l}\right)\dl x,\qquad
		d_{n}=
		\frac{2}{n\pi c}
		\int_{0}^{l}
		\psi\left(x\right)
		\cos
		\left(\frac{n\pi x}{l}\right)\dl x.
	\end{equation*}
\end{theorem}

\begin{theorem}
	Sean
	\begin{math}
		\phi\in
		C^{3}\left(\mathbb{R}^{2}\right),
		\psi\in
		C^{2}\left(\mathbb{R}^{2}\right)
	\end{math}
	dados.
	La solución del problema de valor inicial de la ecuación de onda
	\begin{equation*}
		\begin{cases}
			\difcp[2]{u}{t}=
			c^{2}\difc.L.{u}{} &
			\text{en }\mathbb{R}^{2}\times\left(0,\infty\right). \\
			u=\phi             &
			\text{en }\mathbb{R}^{2}\times\left\{t=0\right\}.    \\
			\difcp{u}{t}=\psi  &
			\text{en }\mathbb{R}^{2}\times\left\{t=0\right\}.
		\end{cases}
	\end{equation*}
	es
	\begin{math}\displaystyle
		u\left(x,t\right)=
		\frac{2}{4\pi ct}
		\iint_{B\left(\symbf{x},ct\right)}
		\frac{
		\phi\left(\symbf{y}\right)+
		\nabla\phi\left(\symbf{y}\right)\cdot
		\left(\symbf{y}-\symbf{x}\right)+
		t\psi\left(\symbf{y}\right)
		}{
		\sqrt{c^{2}t^{2}-{\left|y-x\right|}^{2}}
		}
		\dl{\symbf{y}}
	\end{math}.
\end{theorem}

\begin{theorem}[Fórmula de Kirchoff]
	Sean
	\begin{math}
		\phi\in
		C^{3}\left(\mathbb{R}^{3}\right),
		\psi\in
		C^{2}\left(\mathbb{R}^{3}\right)
	\end{math}
	dados, $\partial B\left(\symbf{x},ct\right)$ es una esfera centrada
	en $\symbf{x}$ con radio $ct$ y $\dl{S_{\symbf{y}}}$ es el
	diferencial de superficie sobre la esfera
	$\partial B\left(\symbf{x},ct\right)$ parametrizada por
	$\symbf{y}$.
	La solución del problema de valor inicial de la ecuación de onda
	\begin{equation*}
		\begin{cases}
			\difcp[2]{u}{t}=
			c^{2}\difc.L.{u}{} &
			\text{en }\mathbb{R}^{3}\times\left(0,\infty\right). \\
			u=\phi             &
			\text{en }\mathbb{R}^{3}\times\left\{t=0\right\}.    \\
			\difcp{u}{t}=\psi  &
			\text{en }\mathbb{R}^{3}\times\left\{t=0\right\}.
		\end{cases}
	\end{equation*}
	es
	\begin{math}\displaystyle
		u\left(x,t\right)=
		\frac{1}{4\pi c^{2}t^{2}}
		\iint_{\partial B\left(\symbf{x},ct\right)}
		\left[
			\phi\left(\symbf{y}\right)+
			\nabla\phi\left(\symbf{y}\right)\cdot
			\left(\symbf{y}-\symbf{x}\right)+
			t\psi\left(\symbf{y}\right)
			\right]
		\dl{S_{\symbf{y}}}
	\end{math}.
\end{theorem}

\begin{theorem}[Principio de Duhamel]
	La solución del problema de valor inicial de la ecuación de onda 1D no homogénea
	\begin{equation*}
		\begin{cases}
			\difcp[2]{u}{t}-
			c^{2}\difcp[2]{u}{x}=
			f\left(x,t\right) &
			\text{en }\mathbb{R}\times\left(0,\infty\right). \\
			u\left(x,t\right)=
			0                 &
			\text{en }\mathbb{R}.                            \\
			\difcp{u}{t}\left(x,0\right)=
			0                 &
			\text{en }\mathbb{R}.
		\end{cases}
	\end{equation*}
	es
	\begin{math}\displaystyle
		u\left(x,t\right)=
		\int_{0}^{t}
		w\left(x,t;s\right)
		\dl s
	\end{math},
	donde para cada $s<t$ fijo, $w\left(x,t;s\right)$ es la solución
	indexada por $s$ del problema
	\begin{equation*}
		\begin{cases}
			\difcp[2]{w}{t}\left(x,t;s\right)=
			c^{2}\difcp[2]{w}{x}\left(x,t;s\right) &
			\text{en }\mathbb{R}\times\left(s,\infty\right). \\
			w\left(x,s;s\right)=
			0                                      &
			\text{en }\mathbb{R}.                            \\
			\difcp{w}{t}\left(x,s;s\right)=
			f\left(x,s\right)                      &
			\text{en }\mathbb{R}.
		\end{cases}
	\end{equation*}
\end{theorem}
