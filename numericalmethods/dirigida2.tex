% arara: clean: {
% arara: --> extensions:
% arara: --> ['aux', 'log', 'out', 'pdf']
% arara: --> }
%! arara: lualatex: {
%! arara: --> shell: yes,
%! arara: --> draft: yes,
%! arara: --> interaction: batchmode
%! arara: --> }
%! arara: biber
% arara: lualatex: {
% arara: --> shell: yes,
% arara: --> draft: no,
% arara: --> interaction: batchmode
% arara: --> }
% arara: lualatex: {
% arara: --> shell: yes,
% arara: --> draft: no,
% arara: --> interaction: batchmode
% arara: --> }
% arara: clean: {
% arara: --> extensions:
% arara: --> ['aux', 'log', 'out']
% arara: --> }
\documentclass{scrletter}
\usepackage{graphicx}
\usepackage[shortlabels]{enumitem}
\usepackage{mathtools}
% \usepackage{plantuml}
\usepackage{hyperref}

\usepackage[
	citestyle=numeric,
	style=numeric,
	backend=biber,
]{biblatex}

\addbibresource{references.bib}

\newcommand{\MVAt}{{\usefont{U}{mvs}{m}{n}\symbol{`@}}}
\renewcommand{\thesection}{\Roman{section}}
\renewcommand{\thesubsection}{\thesection.\Roman{subsection}}

\begin{document}

\providecommand{\faculty}{Facultad}
\noindent\parbox[c]{.18\textwidth}{\includegraphics[width=2.8cm]{logouni}}\hfill
\parbox[c]{1\textwidth}{\raggedright%
    {\large\textbf{UNIVERSIDAD NACIONAL DE INGENIERÍA} \par\smallskip}
    {\large\textbf{\faculty} \par\smallskip}
    {\large\textbf{DEPARTAMENTO ACADÉMICO DE CIENCIAS BÁSICAS} \par\smallskip}
}

\begin{center}\bfseries\large
    Práctica Dirigida 2 de Métodos Numéricos (BMA-18)
    Parte $2$
\end{center}

\vspace{-0.5cm}

\hrulefill
\vspace{-2.5mm}

\rule{16.5cm}{0.8mm}

\section{Semana 2}
\subsection{Interpolación polinomial}

Se dice que una función interpola un conjunto de puntos de datos
si pasa por esos puntos.
Supongamos que se ha recopilado un conjunto de puntos de datos
$\left(x,y\right)$, como $\left(0,1\right)$, $\left(2,2\right)$ y
$\left(3,4\right)$.
Hay una parábola que pasa por los tres puntos.
Los polinomios se utilizan muy a menudo para la interpolación
debido a sus sencillas propiedades matemáticas.
Existe una teoría sencilla sobre cuándo existe un polinomio de
interpolación de un grado determinado para un conjunto dado de
puntos.
Más importante aún, en un sentido real, los polinomios son las
funciones más fundamentales para las computadoras digitales.
Las unidades centrales de procesamiento suelen tener métodos
rápidos en hardware para sumar y multiplicar números de punto
flotante, que son las únicas operaciones necesarias para evaluar
un polinomio.
Las funciones complicadas se pueden aproximar interpolando
polinomios para que sean computables con estas dos operaciones de
hardware.

\begin{questions}
    \question\label{q:lagrangeinterpolation}

    Utilice la interpolación de Lagrange para encontrar un polinomio
    que pase por los puntos.

    \begin{multicols}{3}

        \begin{parts}
            \part

            \begin{math}
                \left(0,1\right),
                \left(2,3\right),
                \left(3,0\right)
            \end{math}.

            \part

            \begin{math}
                \left(-1,0\right),
                \left(2,1\right),
                \left(3,1\right),
                \left(5,2\right)
            \end{math}.

            \part

            \begin{math}
                \left(0,-2\right),
                \left(2,1\right),
                \left(4,4\right)
            \end{math}.
        \end{parts}
    \end{multicols}

    \question

    Utilice las diferencias divididas de Newton, vea el
    Algoritmo~\ref{algo:updateL}, para encontrar los polinomios de
    interpolación de los puntos del
    Ejercicio~\ref{q:lagrangeinterpolation} y verificar la
    concordancia con el polinomio de interpolación de Lagrange.

    \begin{listing}[ht!]
        \tiny
        \centering
        \inputminted[linenos,highlightlines={7-25}]{octave}{newtdd.m}
        \caption{Método de las diferencias divididas de Newton.}
    \end{listing}

    \begin{algorithm}[H]
        \caption{Diferencias divididas de Newton}\label{algo:updateL}
        Dada $x=\left[x_{1},\dotsc,x_{n}\right]$ e $y=\left[y_{1},\dotsc,y_{n}\right]$\;
        \For{$j=1,\dotsc,n$}{
            $f\left[x_{j}\right]\longleftarrow y_{j}$\;
        }
        \For{$i=2,\dotsc,n$}{
            \For{$j=1,\dotsc,n+1-i$}{
                \begin{math}
                    f\left[x_{j},x_{j+i-1}\right]\longleftarrow
                    \frac{f\left[x_{j+1}\dotsc\, x_{j+i-1}\right]-f\left[x_{j}\dotsc\, x_{j+i-2}\right]}{x_{j+i-1}-x_{j}}
                \end{math}\;
            }
        }
    \end{algorithm}
    El polinomio de interpolación es
    \begin{equation*}
        P\left(x\right)=
        \sum_{i=1}^{n}
        f\left[x_{1}\dotsc\, x_{i}\right]
        \left(x-x_{1}\right)\cdots\left(x-x_{i-1}\right)
    \end{equation*}

    \question

    Halla un polinomio $P\left(x\right)$ de grado $3$ o menor
    cuyo gráfico pase por los puntos
    \begin{math}
        \left(0,0\right),
        \left(1,1\right),
        \left(2,2\right),
        \left(3,7\right)
    \end{math}.

    \question Halla otros dos polinomios (de cualquier grado) que
    pasen por estos cuatro puntos.

    \question
    Decide si existe un polinomio $P\left(x\right)$ de grado $3$
    o menor cuyo gráfico pase por los puntos
    \begin{math}
        \left(0,0\right),
        \left(1,1\right),
        \left(2,2\right),
        \left(3,7\right)
    \end{math}
    y
    \begin{math}
        \left(4,2\right)
    \end{math}.

    \question

    ¿Puede un polinomio de grado $3$ intersecar a un polinomio de
    grado $4$ en exactamente cinco puntos?
    Explique.

    \question

    La concentración atmosférica media estimada de dióxido de carbono
    en la atmósfera terrestre se da en la tabla que sigue, en partes
    por millón por volumen.
    Encuentre el polinomio de interpolación de grado $3$ de los datos
    y utilícelo para estimar la concentración de CO$_{2}$ en $1950$ y $2050$.
    (La concentración real en $1950$ fue de $310$ ppm).

    \begin{table}[ht!]
        \centering
        \begin{tabular}{|c|c|}
            \hline
            Año  & CO$_{2}$ (ppm)
            \\
            \hline
            1800 & 280
            \\
            \hline
            1850 & 283
            \\
            \hline
            1900 & 291
            \\
            \hline
            2000 & 370            \\
            \hline
        \end{tabular}
    \end{table}

    \question

    La vida útil esperada de un ventilador industrial cuando funciona
    a la temperatura indicada se muestra en la tabla que sigue.
    Calcule la vida útil a $70^{\circ}$C utilizando

    \begin{multicols}{2}
        \begin{parts}
            \part
            la parábola de los últimos tres puntos de datos.
            \part

            la curva de grado 3 utilizando los cuatro puntos.
        \end{parts}
    \end{multicols}

    \begin{table}[ht!]
        \centering
        \begin{tabular}{|c|c|}
            \hline
            temperatura ($^{\circ}$C) & horas ($\times 1000$)
            \\
            \hline
            25                        & 95
            \\
            \hline
            40                        & 75
            \\
            \hline
            50                        & 63
            \\
            \hline
            60                        & 54                    \\
            \hline
        \end{tabular}
    \end{table}

    \question

    Sea $P_{5}\left(x\right)$ el polinomio de grado $5$ que
    interpola $f\left(x\right)=2x$ en los seis nodos
    $\left\{0, 0.2, 0.4, 0.6, 0.8, 1.0\right\}$ en el intervalo $\left[0,1\right]$.
    Halle los mejores cotas superiores posibles para el error de interpolación
    $\left|2x-P_{5}\left(x\right)\right|$ en $x=0.5$ y $x=0.9$.

    \question

    Suponga que el polinomio $P_{9}\left(x\right)$ interpola la función $f\left(x\right)=e^{-2x}$
    en los $10$ puntos espaciados uniformemente $\left\{0,\frac{1}{9},\frac{2}{9},\dotsc,\frac{8}{9},1\right\}$.
    \begin{parts}
        \part
        Encuentre un límite superior para el error
        \begin{math}
            \left|f\left(\frac{1}{2}\right)-P_{9}\left(\frac{1}{2}\right)\right|.
        \end{math}

        \part

        ¿Cuántos decimales puede garantizar que sean correctos si se usa $P_{9}\left(\frac{1}{2}\right)$ para aproximar $e^{-1}$?
    \end{parts}

    \question

    Utilice el método de diferencias divididas para encontrar el polinomio de interpolación de grado $4$, $P_{4}\left(x\right)$,
    para los datos $\left(0.6, 1.433329\right)$, $\left(0.7,1.632316\right)$,
    $\left(0.8, 1.896481\right)$, $\left(0.9, 2.247908\right)$, y $\left(1.0, 2.718282\right)$.
    Calcule $P_{4}(0.82)$ and $P_{4}\left(0.98\right)$.
    Los datos fueron obtenidos de la función $f\left(x\right)=e^{x^2}$.
    Use la fórmula del error de interpolación para encontrar cotas superiores para el error en $x=0.82$
    y $x=0.98$, y compare las cotas con el error actual.

    \question

    Pruebe que la fórmula de segundo orden para la primera derivada

    \begin{equation*}
        f^{\prime}\left(x\right)=
        \frac{-f\left(x+2h\right)+4f\left(x+h\right)-3f\left(x\right)}{2h}+
        \mathcal{O}\left(h^{2}\right).
    \end{equation*}

    \question

    Encuentre el término de error para la fórmula de aproximación

    \begin{equation*}
        f^{\prime}\left(x\right)=
        \frac{4f\left(x+h\right)-3f\left(x\right)-f\left(x-2h\right)}{6h}.
    \end{equation*}

    \question

    Desarrolle un método de segundo orden para aproximar $f\left(x\right)$
    que use solamente los datos $f\left(x-h\right)$, $f\left(x\right)$, y $f\left(x+3h\right)$.

    \question

    Crea una tabla del error de la fórmula de diferencia centrada de
    tres puntos para $f\left(0\right)$, donde
    $f\left(x\right)=\operatorname{sen}\left(x\right)-\cos\left(x\right)$,
    con $h=10^{-1},\dotsc,10^{-12}$.
    Dibuje un gráfico de los resultados.

    \question

    Aplique la regla de trapezio compuesta con $m=1,2,4$ para aproximar la integral.
    Calcule el error comparando con el valor exacto del cálculo integral.

    \begin{multicols}{3}

        \begin{parts}
            \part
            \begin{equation*}
                \int_{0}^{1}
                x^{2}\dl x.
            \end{equation*}

            \part
            \begin{equation*}
                \int_{0}^{\frac{\pi}{2}}
                \cos x\dl x.
            \end{equation*}

            \part
            \begin{equation*}
                \int_{0}^{1}
                e^{x}\dl x.
            \end{equation*}
        \end{parts}
    \end{multicols}

    \question

    Aplique la regla de Simpson compuesta con $m=1,2,4$ para aproximar la integral
    y reporte los errores.

    \begin{multicols}{3}

        \begin{parts}
            \part
            \begin{equation*}
                \int_{0}^{1}
                xe^{x}\dl x.
            \end{equation*}

            \part
            \begin{equation*}
                \int_{0}^{1}
                \frac{\dl x}{1+x^{2}}
            \end{equation*}

            \part
            \begin{equation*}
                \int_{0}^{\pi}
                x\cos\left(x\right)\dl x
            \end{equation*}
        \end{parts}
    \end{multicols}

    \question

    Aplique la regla del punto medio compuesta con $m=1,2,4$ para aproximar la integral
    y reporte los errores.

    \begin{multicols}{3}

        \begin{parts}
            \part
            \begin{equation*}
                \int_{0}^{1}
                \frac{\dl x}{\sqrt{x}}.
            \end{equation*}

            \part
            \begin{equation*}
                \int_{0}^{1}
                x^{-\frac{1}{3}}\dl x.
            \end{equation*}

            \part
            \begin{equation*}
                \int_{0}^{2}
                \frac{\dl x}{\sqrt{2-x}}.
            \end{equation*}
        \end{parts}
    \end{multicols}

    \question


    Encuentre $c_{1}$, $c_{2}$ y $c_{3}$ tal que la regla
    \begin{equation*}
        \int_{0}^{1}
        f\left(x\right)\dl x\approx
        c_{1}f\left(0\right)+
        c_{2}f\left(\frac{1}{2}\right)+
        c_{3}f\left(1\right).
    \end{equation*}
    tenga grado de precisión mayor que uno.
    Sugerencia: Sustituya $f\left(x\right)=1,x,x^{2}$.

    \question

    Encuentre el grado de precisión de la siguiente aproximación
    \begin{math}
        \int_{-1}^{1}f\left(x\right)\dl x
    \end{math}:

    \begin{multicols}{3}
        \begin{parts}
            \part

            \begin{math}
                f\left(1\right)+
                f\left(-1\right)
            \end{math}.

            \part

            \begin{math}
                \frac{2}{3}
                \left[f\left(-1\right)+f\left(0\right)+f\left(1\right)\right]
            \end{math}.

            \part

            \begin{math}
                f\left(-\frac{1}{3}\right)+
                f\left(\frac{1}{3}\right)
            \end{math}.
        \end{parts}
    \end{multicols}

\end{questions}

\providecommand{\name}{Nombre}
\vfill{Nombre}\footnote{Hecho en \LaTeX}
\hfill{UNI, 25 de enero de 2025}

\end{document}