% arara: clean: {
% arara: --> extensions:
% arara: --> ['aux', 'log', 'out', 'pdf']
% arara: --> }
%! arara: lualatex: {
%! arara: --> shell: yes,
%! arara: --> draft: yes,
%! arara: --> interaction: batchmode
%! arara: --> }
%! arara: biber
% arara: lualatex: {
% arara: --> shell: yes,
% arara: --> draft: no,
% arara: --> interaction: batchmode
% arara: --> }
% arara: lualatex: {
% arara: --> shell: yes,
% arara: --> draft: no,
% arara: --> interaction: batchmode
% arara: --> }
% arara: clean: {
% arara: --> extensions:
% arara: --> ['aux', 'log', 'out']
% arara: --> }
\documentclass{scrletter}
\usepackage{graphicx}
\usepackage[shortlabels]{enumitem}
\usepackage{mathtools}
% \usepackage{plantuml}
\usepackage{hyperref}

\usepackage[
	citestyle=numeric,
	style=numeric,
	backend=biber,
]{biblatex}

\addbibresource{references.bib}

\newcommand{\MVAt}{{\usefont{U}{mvs}{m}{n}\symbol{`@}}}
\renewcommand{\thesection}{\Roman{section}}
\renewcommand{\thesubsection}{\thesection.\Roman{subsection}}

\begin{document}

\noindent\parbox[c]{.18\textwidth}{\includegraphics[width=2.8cm]{logouni}}\hfill
\parbox[c]{1\textwidth}{\raggedright%
    {\large\textbf{UNIVERSIDAD NACIONAL DE INGENIERÍA} \par\smallskip}
    {\large\textbf{FACULTAD DE INGENIERÍA ELÉCTRICA Y ELECTRÓNICA} \par\smallskip}
    {\large\textbf{DEPARTAMENTO ACADÉMICO DE CIENCIAS BÁSICAS} \par\smallskip}
}

\begin{center}\bfseries\large
    Práctica Dirigida 2 de Métodos Numéricos (BMA-18)
\end{center}

\vspace{-0.5cm}

\hrulefill
\vspace{-2.5mm}

\rule{16.5cm}{0.8mm}

\section{Semana 2}
\subsection{Interpolación polinomial}

Se dice que una función interpola un conjunto de puntos de datos
si pasa por esos puntos.
Supongamos que se ha recopilado un conjunto de puntos de datos
$\left(x,y\right)$, como $\left(0,1\right)$, $\left(2,2\right)$ y
$\left(3,4\right)$.
Hay una parábola que pasa por los tres puntos.
Los polinomios se utilizan muy a menudo para la interpolación
debido a sus sencillas propiedades matemáticas.
Existe una teoría sencilla sobre cuándo existe un polinomio de
interpolación de un grado determinado para un conjunto dado de
puntos.
Más importante aún, en un sentido real, los polinomios son las
funciones más fundamentales para las computadoras digitales.
Las unidades centrales de procesamiento suelen tener métodos
rápidos en hardware para sumar y multiplicar números de punto
flotante, que son las únicas operaciones necesarias para evaluar
un polinomio.
Las funciones complicadas se pueden aproximar interpolando
polinomios para que sean computables con estas dos operaciones de
hardware.

\begin{questions}
    \question


    Utilice la interpolación de Lagrange para encontrar un polinomio que pase por los puntos.

    \begin{listing}[ht!]
        \tiny
        \centering
        \inputminted[linenos,highlightlines={7-25}]{octave}{newtdd.m}
        \caption{Método de las diferencias divididas de Newton.}
    \end{listing}
    % \subsection{Estudio del error}

    Algoritmo~\ref{algo:updateL}.

    \begin{algorithm}[H]
        \caption{Diferencias divididas de Newton}\label{algo:updateL}
        Dada $x=\left[x_{1},\dotsc,x_{n}\right]$, $y=\left[y_{1},\dotsc,y_{n}\right]$\;
        \For{$j=1,\dotsc,n$}{
            $f\left[x_{j}\right]\longleftarrow y_{j}$\;
        }
    \end{algorithm}

    \question

    Utilice las diferencias divididas de Newton para encontrar los
    polinomios de interpolación de los puntos del Ejercicio 1 y verificar la concordancia con el polinomio de interpolación de Lagrange.

    \question

    La intersección con el eje $x$ de la recta que pasa por los
    puntos $\left(x_{1},y_{1}\right)$ y $\left(x_{2},y_{2}\right)$ se
    puede calcular utilizando cualquiera de las siguientes fórmulas
    \begin{align*}
        x & =
        \frac{x_{1}y_{2}-x_{2}y_{1}}{y_{2}-y_{1}}. \\
        x & =
        x_{1}-
        \frac{\left(x_{2}-x_{1}\right)y_{1}}{y_{2}-y_{1}}.
    \end{align*}
    con el supuesto de que $y_{1}\neq y_{2}$.

    \begin{parts}
        \part\label{part:acartesian}

        Demuestre que las fórmulas son equivalentes entre sí.

        \part\label{part:bcartesian}

        Calcule la intersección con el eje $x$ utilizando cada
        fórmula cuando
        \begin{math}
            \left(x_{1},y_{1}\right)=
            \left(1.02,3.32\right)
        \end{math}
        y
        \begin{math}
            \left(x_{2},y_{2}\right)=
            \left(1.31,4.31\right)
        \end{math}.
        Use la aritmética de redondeo de tres dígitos.

        \part\label{part:ccartesian}

        Utilice su lenguaje de programación favorito (o una
        calculadora) para calcular la intersección con el eje $x$
        utilizando la precisión total del dispositivo (puede
        utilizar cualquiera de las fórmulas).
        Con este resultado, calcule los errores relativos y absolutos
        de las respuestas que dio en la parte~\eqref{part:bcartesian}.
        Analice qué fórmula es mejor y por qué.
    \end{parts}

    % \subsection{Fuentes de error}

    \question

    Ejecuta el siguiente programa y obtenga el valor numérico del
    épsilon de la máquina.

    \begin{listing}[ht!]
        \tiny
        \centering
        \inputminted[linenos,highlightlines={3-5}]{octave}{epsilonmachine.m}
        \caption{Cálculo del épsilon de máquina $\epsilon_{M}$.}
    \end{listing}

    % \subsection{Principales teoremas del análisis matemático usados en los métodos Numéricos}

    \question

    Vea un extracto del vídeo titulado \emph{Tricks and Tips in Numerical Computing} por Nick Higham.
    En el minuto 6:24 de \url{https://youtu.be/Q9OLOqEhc64?t=1645}, muestra que una aproximación de
    la derivada de la función es
    \begin{equation*}
        f^{\prime}\left(x\right)\approx
        \operatorname{Im}
        \left[
            \frac{f\left(x+ih\right)}{h}
            \right],
    \end{equation*}
    donde $i=\sqrt{-1}$.
    El valor exacto de la derivada de la función
    \begin{math}
        f\left(x\right)=
        \dfrac{\arctan\left(x\right)}{1+e^{-x^{2}}}
    \end{math}
    en $x=2$ es $0.274623728154858$.
    Verifique que una aproximación cuando $h=1\times 10^{-100}$ es
    más precisa que con respecto a la regla de aproximación
    \begin{equation*}
        f^{\prime}\left(x\right)\approx
        \frac{f\left(x+\frac{h}{2}\right)-f\left(x-\frac{h}{2}\right)}{h}.
    \end{equation*}.

    % \subsection{Aritmética de punto flotante}

    \question

    Encuentre la representación de punto flotante de $10.375$.

    \question

    Considere el siguiente modelo para una representación de punto
    flotante normalizada en base $2$:
    \begin{equation*}
        x=\left(-1\right)^{s}
        \left(1.a_{2}a_{3}\right)
        \times 2^{e}
    \end{equation*}
    donde $-1\leq e\leq 1$.
    Encuentre todos los números de máquina positivos (hay $12$ de
    ellos) que se pueden representar en este modelo.
    Convierta los números a base $10$, y luego gráfiquelos
    cuidadosamente en la recta numérica a mano, y comente cómo están
    espaciados los números.

    \question

    Encuentre valores de corte y redondeo de 5 dígitos
    $\left(k=5\right)$ de los números siguientes:

    \begin{multicols}{2}

        \begin{parts}
            \part

            \begin{math}
                \pi=0.314159265\dotsc\times10^{1}
            \end{math}.

            \part

            \begin{math}
                0.0001234567
            \end{math}.
        \end{parts}
    \end{multicols}

    % \subsection{Solución de $f\left(x\right)=0$ mediante el método de la bisección}

    \question

    La ley de los gases ideales para un gas a baja temperatura y
    presión es $PV=nRT$, donde $P$ es la presión (en atm), $V$ es el
    volumen (en L), $T$ es la temperatura (en $K$), $n$ es el número
    de moles del gas y $R=0.0820578$ es la constante molar del gas.
    La ecuación de van der Waals
    \begin{equation*}
        \left(P+\frac{n^{2}a}{V^{2}}\right)
        \left(v-nb\right)
        =nRT
    \end{equation*}
    cubre el caso no ideal en el que estas suposiciones no se cumplen.
    Utilice la ley de los gases ideales para calcular una estimación
    inicial, seguida del método de Newton aplicado a la ecuación de
    van der Waals para encontrar el volumen de un mol de oxígeno a
    $320$K y una presión de $15$ atm.
    Para el oxígeno,
    $a=1.36\operatorname{L}^{2}-\operatorname{atm}/\operatorname{mole}^{2}$
    y $b=0.003183\operatorname{L}/\operatorname{mole}$.
    Indique su estimación inicial y la solución con tres dígitos
    significativos.

    \begin{listing}[ht!]
        \tiny
        \centering
        \inputminted[linenos,highlightlines={7-35}]{octave}{bisect.m}
        \caption{Método de la bisección.}
    \end{listing}

    % \subsection{Falsa posición}

    \question

    Halla la altura que alcanza $1$ metro cúbico de agua almacenada
    en un tanque esférico de radio $1$ metro.
    Da tu respuesta $\pm1$mm usando el método de la falsa posición.
    El volumen del tanque esférico a $H$ metros de profundidad de un
    hemisferio de radio $R$ es $\pi H^{2}\left(R-\frac{H}{3}\right)$.

    % \subsection{Método de la secante}

    % \subsection{Método de Newton}

    \question

    Una cantidad crucial en el diseño de tuberías es la caída de
    presión debido a la fricción bajo flujo turbulento.
    La caída de presión por unidad de longitud se describe mediante
    el número de Darcy $f$, una cantidad adimensional que satisface
    la ecuación empírica de Colebrook
    \begin{equation*}
        \frac{1}{\sqrt{f}}=
        -2\log_{10}\left[\frac{\epsilon}{3.7D}+
            \frac{2.51}{R\sqrt{f}}\right]
    \end{equation*}
    donde $D$ es el diámetro interior de la tubería, $\epsilon$ es la
    altura de rugosidad del interior de la tubería y $R$ es el número
    de Reynolds del flujo.

    \begin{parts}
        \part\label{part:adarcy}

        Para $D=0.3m$, $\epsilon=0.0002$m y $R=105$, utilice el
        método de Newton para calcular el número de Darcy $f$.

        \part\label{part:bdarcy}
        Fije $D$ y como en~\eqref{part:adarcy}, y calcule el número
        de Darcy para varios números de Reynolds $R$ entre $104$ y
        $108$.
        Haga un gráfico del número de Darcy en función del número de
        Reynolds, utilizando un eje logarítmico para este último.
    \end{parts}

    % \subsection{Método de punto fijo}

    \question

    ¿Cuál de las siguientes tres iteraciones de punto fijo converge a
    la raíz cúbica de $4$?
    Clasifique las que convergen de más rápidas a más lentas.

    \begin{multicols}{3}
        \begin{parts}
            \part

            \begin{math}
                g\left(x\right)=
                \dfrac{2}{\sqrt{x}}
            \end{math}.

            \part

            \begin{math}
                g\left(x\right)=
                \dfrac{3x}{4}+\dfrac{1}{x^{2}}
            \end{math}.

            \part

            \begin{math}
                g\left(x\right)=
                \dfrac{2x}{3}+\dfrac{4}{3x^{2}}
            \end{math}.
        \end{parts}
    \end{multicols}

    \question

    Encuentre el conjunto de todas las estimaciones iniciales para
    las cuales la iteración de punto fijo
    $x\rightarrow\dfrac{4}{9}-x^{2}$ converge a un punto fijo.

    \begin{listing}[ht!]
        \tiny
        \centering
        \inputminted[linenos,highlightlines={7-15}]{octave}{fpi.m}
        \caption{Método de punto fijo.}
    \end{listing}

    % \subsection{Solución a sistemas de ecuaciones no lineales}

    \question

    Utilice el método de Newton con la estimación inicial
    $\left(1,2\right)$ para encontrar una solución del sistema no
    lineal.
    \begin{equation*}
        \left\{
        \begin{aligned}
            v-u^{3}       & =0. \\
            u^{2}+v^{2}-1 & =0.
        \end{aligned}
        \right.
    \end{equation*}

    \question

    Aplicar el método de Newton para encontrar ambas soluciones del
    sistema de tres ecuaciones.

    \begin{equation*}
        \left\{
        \begin{aligned}
            2u^{2}-4u+v^{2}+3w^{2}+6w+2 & =0. \\
            u^{2}+v^{2}-2v+2w^{2}-5     & =0. \\
            3u^{2}-12u+v^{2}+3w^{2}+8   & =0.
        \end{aligned}
        \right.
    \end{equation*}

    \question

    Aunque una intersección genérica de tres esferas en el espacio
    tridimensional son dos puntos, puede ser un único punto.
    Aplique el método de Newton multivariable para encontrar el único
    punto de intersección de las esferas con centro $\left(1,0,1\right)$
    y radio $8$, centro $\left(0,2,2\right)$ y radio $\sqrt{2}$,
    y centro $\left(0,3,3\right)$ y radio $\sqrt{2}$.
    ¿La iteración sigue convergiendo cuadráticamente? Explique.
\end{questions}

\vfill{Fidel Jara Huanca}\footnote{Hecho en \LaTeX}
\hfill{UNI, 21 de enero de 2025}

\end{document}