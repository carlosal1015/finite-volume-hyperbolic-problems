% arara: clean: {
% arara: --> extensions:
% arara: --> ['aux', 'log', 'out', 'pdf']
% arara: --> }
%! arara: lualatex: {
%! arara: --> shell: yes,
%! arara: --> draft: yes,
%! arara: --> interaction: batchmode
%! arara: --> }
%! arara: biber
% arara: lualatex: {
% arara: --> shell: yes,
% arara: --> draft: no,
% arara: --> interaction: batchmode
% arara: --> }
% arara: lualatex: {
% arara: --> shell: yes,
% arara: --> draft: no,
% arara: --> interaction: batchmode
% arara: --> }
% arara: clean: {
% arara: --> extensions:
% arara: --> ['aux', 'log', 'out']
% arara: --> }
\documentclass{scrletter}
\usepackage{graphicx}
\usepackage[shortlabels]{enumitem}
\usepackage{mathtools}
% \usepackage{plantuml}
\usepackage{hyperref}

\usepackage[
	citestyle=numeric,
	style=numeric,
	backend=biber,
]{biblatex}

\addbibresource{references.bib}

\newcommand{\MVAt}{{\usefont{U}{mvs}{m}{n}\symbol{`@}}}
\renewcommand{\thesection}{\Roman{section}}
\renewcommand{\thesubsection}{\thesection.\Roman{subsection}}

\begin{document}

\noindent\parbox[c]{.18\textwidth}{\includegraphics[width=2.8cm]{logouni}}\hfill
\parbox[c]{1\textwidth}{\raggedright%
    {\large\textbf{UNIVERSIDAD NACIONAL DE INGENIERÍA} \par\smallskip}
    {\large\textbf{FACULTAD DE INGENIERÍA ELÉCTRICA Y ELECTRÓNICA} \par\smallskip}
    {\large\textbf{DEPARTAMENTO ACADÉMICO DE CIENCIAS BÁSICAS} \par\smallskip}
}

\begin{center}\bfseries\large
    Práctica Dirigida 1 de Métodos Numéricos (BMA-18)
\end{center}

\vspace{-0.5cm}

\hrulefill
\vspace{-2.5mm}

\rule{16.5cm}{0.8mm}

% \section{Semana 1}
% \subsection{Conceptos de métodos numéricos}

\begin{questions}
    \question

    Un algoritmo se describe como robusto si, para cualquier entrada
    de datos válidos que sea razonablemente representable, se
    completa con éxito.
    Consideremos la ecuación cuadrática $ax^{2}+bx+c=0$, $a\neq 0$.
    Es importante que se implemente de manera robusta, lo que
    significa que no fallará y dará respuestas
    razonablemente precisas para cualquier coeficiente $a$, $b$ y
    $c$ que no esté demasiado cerca del desbordamiento o
    subdesbordamiento.
    La forma estándar para los dos raíces es
    \begin{equation*}
        x=
        \frac{-b\pm\sqrt{b^{2}-4ac}}{2a}.
    \end{equation*}
    Surge un problema si $b^{2}$ es mucho mayor que
    $\left|4ac\right|$.
    En el peor de los casos, la diferencia entre las magnitudes de
    $b^{2}$ y $\left|4ac\right|$ es tan grande que $b^{2}-4ac$ evalúa
    a $b^{2}$ y la raíz cuadrada evalúa a $b$, y una de las raíces
    calculados da cero.
    Sin pérdida de generalidad, asumamos que $b>0$ y la menor raíz es
    dada por
    \begin{equation*}
        x=
        \frac{-b+\sqrt{b^{2}-4ac}}{2a}.
    \end{equation*}
    Observamos que hay una pérdida de significancia en el numerador.
    Como hemos visto antes, esto puede llevar a un gran error
    relativo en el resultado en comparación con el error relativo en
    la entrada.
    El problema se puede evitar manipulando la fórmula de la
    siguiente manera:
    \begin{align*}
        x & =
        \frac{-b+\sqrt{b^{2}-4ac}}{2a}\times
        \frac{-b-\sqrt{b^{2}-4ac}}{-b-\sqrt{b^{2}-4ac}}.                         \\
          & =
        \frac{b^{2}-\left(b^{2}-4ac\right)}{2a\left(-b-\sqrt{b^{2}-4ac}\right)}. \\
          & =
        \frac{-2c}{b+\sqrt{b^{2}-4ac}}.
    \end{align*}
    Tome $a=1$, $b=100000$ y $c=1$ y como precisión
    $2\times 10^{-10}$, obtenga el valor de la menor raíz obtenida
    con ambas fórmulas.

    \question

    Los polinomios se pueden evaluar en una forma anidada (también
    llamado método de Horner), que tiene dos ventajas:
    la forma anidada requiere muchos menos cálculos y puede reducir
    el error de redondeo.
    Para
    \begin{equation*}
        p\left(x\right)=
        a_{0}+
        a_{1}x+
        a_{2}x^{2}+
        \cdots+
        a_{n-1}x^{n-1}+
        a_{n}x^{n}
    \end{equation*}
    en su forma anidada
    \begin{equation*}
        p\left(x\right)=
        a_{0}+
        x
        \left(
        a_{1}+
        x
        \left(
        a_{2}+
        \cdots+x
        \left(
        a_{n-1}+
        xa_{n}
        \right)
        \right)
        \right).
    \end{equation*}

    \begin{listing}[ht!]
        \tiny
        \centering
        \inputminted[linenos,highlightlines={9-18}]{octave}{nest.m}
        \caption{Multiplicación anidada.}
    \end{listing}

    Considere el polinomio $p\left(x\right)=x^{2}+1.1x-2.8$.

    \begin{parts}
        \part\label{part:ahorner}

        Calcule $p\left(3.5\right)$ utilizando el redondeo de tres
        dígitos y la aritmética de corte de tres dígitos.
        ¿Cuáles son los errores absolutos?
        El valor exacto de $p\left(3.5\right)$ es $13.3$.

        \part\label{part:bhorner}

        Escribe $x^{2}+ 1.1x-2.8$ en forma anidada mediante estos
        sencillos pasos:
        \begin{equation*}
            x^{2}+1.1x-2.8=
            \left(x^{2}+1.1x\right)-2.8=
            \left(x+1.1\right)x-2.8.
        \end{equation*}

        Luego, calcula $p\left(3.5\right)$ usando el redondeo de tres
        dígitos y el recorte usando la forma anidada.
        ¿Cuáles son los errores absolutos?
        Compara los errores con los que encontraste en~\eqref{part:ahorner}.
    \end{parts}
    % \subsection{Estudio del error}

    \question

    Sean $x=1.123456$, $y=1.123447$.
    Calcule $x-y$ y el error de redondeo resultante utilizando la
    aritmética de $6$ dígitos.

    \question

    Considere el cálculo de $\frac{1-\cos x}{\operatorname{sen}x}$
    cuando $x$ está cerca de cero.
    Este es un problema en el que tenemos tanto la resta de
    cantidades casi iguales, que ocurre en el numerador, como la
    división por un número pequeño, cuando $x$ está cerca de cero.

    Sea $x=0.1$. Continúe con el redondeo de $5$ dígitos, el
    resultado exacto a $8$ dígitos es $0.050041708$, calcule el error
    relativo de este cálculo.
    Ahora usando la siguiente identidad:
    \begin{math}
        \frac{1-\cos x}{\operatorname{sen}x}=
        \frac{\operatorname{sen}x}{1+\cos x}
    \end{math}
    que remueve las dificultades encontradas antes: no hay
    cancelación de dígitos significativos y división por un número
    pequeño.
    Calcule el nuevo error relativo y compárelo con el anterior.
    ¿En cuánto se ha reducido el factor del error?

    \question

    Hay dos fórmulas estándar en los libros de texto para calcular la
    varianza muestral $s^{2}$ de los números $x_{1},\dotsc,x_{n}$:

    \begin{parts}
        \part

        \begin{math}
            s^{2}=
            \frac{1}{n-1}
            \left[
                \sum_{i=1}^{n}
                x^{2}_{i}
                -\frac{1}{n}
                \left(\sum_{i=1}^{n}x_{i}\right)^{2}
                \right].
        \end{math}

        \part

        Primero calcule,
        \begin{math}
            \overline{x}=
            \frac{1}{n}
            \sum_{i=1}^{n}
            x_{i}
        \end{math},
        y luego
        \begin{math}
            s^{2}=
            \frac{1}{n-1}
            \sum_{i=1}^{n}
            {\left(x_{i}-\overline{x}\right)}^{2}
        \end{math}.
    \end{parts}

    Ambas fórmulas pueden sufrir errores de redondeo debido a la
    adición de grandes listas de números si $n$ es grande.
    Sin embargo, la primera fórmula también es propensa a errores
    debido a la cancelación de los dígitos iniciales.
    Realice un experimento con un vector de números aleatorios de
    tamaño $1000$
    cuyos valores están entre $0$ y $100$ y responda cuál de los
    métodos es mejor.

    \question

    La intersección con el eje $x$ de la recta que pasa por los
    puntos $\left(x_{1},y_{1}\right)$ y $\left(x_{2},y_{2}\right)$ se
    puede calcular utilizando cualquiera de las siguientes fórmulas
    \begin{align*}
        x & =
        \frac{x_{1}y_{2}-x_{2}y_{1}}{y_{2}-y_{1}}. \\
        x & =
        x_{1}-
        \frac{\left(x_{2}-x_{1}\right)y_{1}}{y_{2}-y_{1}}.
    \end{align*}
    con el supuesto de que $y_{1}\neq y_{2}$.

    \begin{parts}
        \part\label{part:acartesian}

        Demuestre que las fórmulas son equivalentes entre sí.

        \part\label{part:bcartesian}

        Calcule la intersección con el eje $x$ utilizando cada
        fórmula cuando
        \begin{math}
            \left(x_{1},y_{1}\right)=
            \left(1.02,3.32\right)
        \end{math}
        y
        \begin{math}
            \left(x_{2},y_{2}\right)=
            \left(1.31,4.31\right)
        \end{math}.
        Use la aritmética de redondeo de tres dígitos.

        \part\label{part:ccartesian}

        Utilice su lenguaje de programación favorito (o una
        calculadora) para calcular la intersección con el eje $x$
        utilizando la precisión total del dispositivo (puede
        utilizar cualquiera de las fórmulas).
        Con este resultado, calcule los errores relativos y absolutos
        de las respuestas que dio en la parte~\eqref{part:bcartesian}.
        Analice qué fórmula es mejor y por qué.
    \end{parts}

    % \subsection{Fuentes de error}

    \question

    Ejecuta el siguiente programa y obtenga el valor numérico del
    épsilon de la máquina.

    \begin{listing}[ht!]
        \tiny
        \centering
        \inputminted[linenos,highlightlines={3-5}]{octave}{epsilonmachine.m}
        \caption{Cálculo del épsilon de máquina $\epsilon_{M}$.}
    \end{listing}

    \question
    % \url{https://youtu.be/kOMrRn2tdCs?t=30}
    % \emph{Wilkinson, Numerical Analysis, and Me} por Nick Trefethen.
    Un ejemplo famoso con raíces simples que son difíciles de
    determinar numéricamente es el polinomio de Wilkinson
    \begin{equation*}
        W\left(x\right)=
        \prod_{k=1}^{20}
        \left(x-k\right).
    \end{equation*}
    Las raíces son los números enteros del $1$ al $20$.
    Sin embargo, cuando $W\left(x\right)$ se define de acuerdo con su
    forma expandida, su evaluación sufre la cancelación de números
    grandes casi iguales.
    Para ver el efecto en la búsqueda de raíces, considere el
    siguiente programa
    \begin{listing}[ht!]
        \tiny
        \centering
        \inputminted[linenos,highlightlines={1-13}]{octave}{wilkpoly.m}
        \caption{Polinomio de Wilkinson.}
    \end{listing}
    Un problema se considera sensible si pequeños errores en la
    entrada, en este caso la ecuación a resolver, dan lugar a grandes
    errores en la salida o solución.
    Para entender qué causa esta magnificación del error,
    estableceremos una fórmula que prediga cuánto se mueve una raíz
    cuando se cambia la ecuación.
    Supongamos que el problema es encontrar una raíz $r$ de
    $f\left(x\right)=0$, pero que se realiza un pequeño cambio
    $g\left(x\right)$ en la entrada, donde $\epsilon$ es pequeño.
    Sea $\Delta r$ el cambio correspondiente en la raíz, de modo que
    \begin{equation*}
        f\left(r+\Delta r\right)+\epsilon
        g\left(g+\Delta r\right)=
        0.
    \end{equation*}
    La expansión de $f$ y $g$ en polinomios de Taylor de grado uno
    implica que
    \begin{equation*}
        f\left(r\right)+
        \Delta r
        f^{\prime}\left(r\right)+
        \epsilon
        g\left(r\right)+
        \epsilon
        \Delta r
        g^{\prime}\left(r\right)+
        \mathcal{O}\left(\Delta r^{2}\right)=0.
    \end{equation*}

    donde usamos la notación ``O mayúscula''
    $\mathcal{O}\left(\Delta r^{2}\right)$ para representar términos
    que involucran $\Delta r^{2}$ y potencias mayores de $r$.
    Para $r$ pequeñas, los términos
    $\mathcal{O}\left(\Delta r^{2}\right)$ pueden ignorarse para
    obtener
    \begin{equation*}
        \Delta r
        \left(
        f^{\prime}\left(r\right)+
        \epsilon g^{\prime}\left(r\right)
        \right)
        \approx
        -f\left(r\right)-\epsilon g\left(r\right)=
        -\epsilon g\left(r\right)
    \end{equation*}
    o
    \begin{equation*}
        \Delta r\approx
        \frac{-\epsilon g\left(r\right)}{
            f^{\prime}\left(r\right)+
            \epsilon g^{\prime}\left(r\right)
        }\approx
        -\epsilon\frac{g\left(r\right)}{f^{\prime}\left(r\right)}.
    \end{equation*}
    suponiendo que $\epsilon$ es pequeño comparado con
    $f^{\prime}\left(r\right)$, y en particular, que
    $f^{\prime}\left(r\right)=0$.

    Use cualquier método visto en clase para determinar la solución
    en una vecindad de la solución y use la fórmula
    de sensibilidad de las raíces que expresa los error de entrada
    del tamaño relativo en $W\left(x\right)$ se magnifican por un
    factor en la raíz de salida.

    % \subsection{Principales teoremas del análisis matemático usados en los métodos Numéricos}

    \question

    Considere la función $\operatorname{sen}x$, que tiene la
    expansión en serie de Taylor
    \begin{equation*}
        \operatorname{sen}x=
        x-\frac{x^{3}}{3!}+\frac{x^{5}}{5!}-\cdots
    \end{equation*}
    que converge, para cualquier $x$, al valor en el rango
    $-1\leq\operatorname{sen}x\leq 1$.
    Escriba una rutina en su lenguaje de programación favorito que
    examine el comportamiento al sumar esta serie de términos con
    signos alternos, ya que representa diferentes valores iniciales
    de $x$.
    Decida una condición de convergencia para la suma.
    Para diferentes $x$, ¿cuántos términos se necesitan para lograr
    la convergencia y qué tan grande es el error relativo?


    \question

    Vea un extracto del vídeo titulado
    \emph{Tricks and Tips in Numerical Computing} por Nick Higham.
    En el minuto 6:24 de \url{https://youtu.be/Q9OLOqEhc64?t=1645},
    muestra que una aproximación de la derivada de la función es
    \begin{equation*}
        f^{\prime}\left(x\right)\approx
        \operatorname{Im}
        \left[
            \frac{f\left(x+ih\right)}{h}
            \right],
    \end{equation*}
    donde $i=\sqrt{-1}$.
    El valor exacto de la derivada de la función
    \begin{math}
        f\left(x\right)=
        \dfrac{\arctan\left(x\right)}{1+e^{-x^{2}}}
    \end{math}
    en $x=2$ es $0.274623728154858$.
    Verifique que una aproximación cuando $h=1\times 10^{-100}$ es
    más precisa que con respecto a la regla de aproximación
    \begin{equation*}
        f^{\prime}\left(x\right)\approx
        \frac{
            f\left(x+\frac{h}{2}\right)-
            f\left(x-\frac{h}{2}\right)
        }{h}.
    \end{equation*}.
    % \subsection{Aritmética de punto flotante}

    \question

    Encuentre la representación de punto flotante de $10.375$.

    \question

    Considere el siguiente modelo para una representación de punto
    flotante normalizada en base $2$:
    \begin{equation*}
        x=\left(-1\right)^{s}
        \left(1.a_{2}a_{3}\right)
        \times 2^{e}
    \end{equation*}
    donde $-1\leq e\leq 1$.
    Encuentre todos los números de máquina positivos (hay $12$ de
    ellos) que se pueden representar en este modelo.
    Convierta los números a base $10$, y luego gráfiquelos
    cuidadosamente en la recta numérica a mano, y comente cómo están
    espaciados los números.

    \question

    Encuentre valores de corte y redondeo de $5$ dígitos
    $\left(k=5\right)$ de los números siguientes:

    \begin{multicols}{2}

        \begin{parts}
            \part

            \begin{math}
                \pi=0.314159265\dotsc\times10^{1}
            \end{math}.

            \part

            \begin{math}
                0.0001234567
            \end{math}.
        \end{parts}
    \end{multicols}
    % \subsection{Solución de $f\left(x\right)=0$ mediante el método de la bisección}

    \question

    Use el método de la bisección y calcule la solución de
    $\cos x=\operatorname{sen}x$ en el intervalo $\left[0,1\right]$
    dentro de seis decimales correctos.

    \question

    Utilice el método de bisección para encontrar los dos números
    reales $x$, dentro de seis decimales correctos, que forman el
    determinante de la matriz
    \begin{equation*}
        A=\begin{bmatrix}
            1 & 2  & 3  & x  \\
            4 & 5  & x  & 6  \\
            7 & x  & 8  & 9  \\
            x & 10 & 11 & 12
        \end{bmatrix}
    \end{equation*}
    igual a $1000$.
    Para cada solución que encuentre, verifique calculando el
    determinante correspondiente e informando cuántos decimales
    correctos (después del punto decimal) tiene el determinante
    cuando se utiliza su solución $x$.

    \question

    La matriz de Hilbert es la matriz $n\times n$ cuya entrada es
    \begin{math}
        \frac{1}{i+j-1}
    \end{math}.
    Sea $A$ la matriz de Hilbert $5\times 5$.
    Su valor propio más grande es aproxidamente $1.567$.
    Utilice el método de bisección para decidir cómo cambiar la
    $A_{11}$ para hacer que el valor propio más grande de $A$ sea
    igual a $\pi$.
    Determine $A_{11}$ dentro de seis decimales correctas.

    \question

    Un planeta que orbita alrededor del Sol recorre una elipse.
    La excentricidad $e$ de la elipse es la distancia entre el centro
    de la elipse y cualquiera de sus focos dividida por la longitud
    del semieje mayor.
    El perihelio es el punto más cercano de la órbita al Sol.
    La ecuación de Kepler $M=E-e\operatorname{sen}E$ relaciona la
    anomalía excéntrica $E$, la distancia angular verdadera (en
    radianes) desde el perihelio, con la anomalía media $M$, la
    distancia angular ficticia desde el perihelio si estuviera en una
    órbita circular con el mismo período que la elipse.

    \begin{parts}
        \part\label{part:aelliptic}

        Suponga que $e=0.1$.
        Utilice el método de bisección para encontrar las anomalías
        excéntricas E cuando $M=\frac{\pi}{6}$ y $M=\frac{\pi}{2}$.

        \part\label{part:belliptic}

        Comience por encontrar un intervalo de inicio y explique por
        qué funciona.
        ¿Cómo cambian las respuestas a~\eqref{part:aelliptic} si la
        excentricidad se modifica a $e=0.2$?
    \end{parts}

    \question

    La ley de los gases ideales para un gas a baja temperatura y
    presión es $PV=nRT$, donde $P$ es la presión (en atm), $V$ es el
    volumen (en L), $T$ es la temperatura (en $K$), $n$ es el número
    de moles del gas y $R=0.0820578$ es la constante molar del gas.
    La ecuación de van der Waals
    \begin{equation*}
        \left(P+\frac{n^{2}a}{V^{2}}\right)
        \left(v-nb\right)
        =nRT
    \end{equation*}
    cubre el caso no ideal en el que estas suposiciones no se
    cumplen.
    Utilice la ley de los gases ideales para calcular una estimación
    inicial, seguida del método de Newton aplicado a la ecuación de
    van der Waals para encontrar el volumen de un mol de oxígeno a
    $320$K y una presión de $15$ atm.
    Para el oxígeno,
    \begin{math}
        a=
        1.36\operatorname{L}^{2}-
        \operatorname{atm}/\operatorname{mole}^{2}
    \end{math}
    y
    \begin{math}
        b=
        0.003183\operatorname{L}/\operatorname{mole}
    \end{math}.
    Indique su estimación inicial y la solución con tres dígitos
    significativos.

    \begin{listing}[ht!]
        \tiny
        \centering
        \inputminted[linenos,highlightlines={7-35}]{octave}{bisect.m}
        \caption{Método de la bisección.}
    \end{listing}

    % \subsection{Falsa posición}

    \question

    Halla la altura que alcanza $1$ metro cúbico de agua almacenada
    en un tanque esférico de radio $1$ metro.
    Da tu respuesta $\pm1$mm usando el método de la falsa posición.
    El volumen del tanque esférico a $H$ metros de profundidad de un
    hemisferio de radio $R$ es $\pi H^{2}\left(R-\frac{H}{3}\right)$.

    % \subsection{Método de la secante}

    \question

    Un pescador comercial quiere colocar la red a una profundidad de
    agua donde la temperatura es de $10^{\circ}$.
    Al dejar caer un sedal con un termómetro adjunto, descubre que
    la temperatura es de $8^{\circ}$ grados a una profundidad de $9$
    metros y de $15^{\circ}$ a una profundidad de 5 metros.
    Utilice el método secante para determinar una mejor estimación de
    la profundidad a la que la temperatura es de $10^{\circ}$.

    % \subsection{Método de Newton}

    \question

    Aplique el método de Newton para hallar ambas raíces de la
    función
    \begin{math}
        f\left(x\right)=
        14xe^{x-2}-
        12e^{x-2}-
        7x^{3}+
        20x^2-
        26x+
        12
    \end{math}
    en el intervalo $\left[0,3\right]$.
    Para cada raíz, imprima la secuencia de iteraciones, los errores
    $e_{i}$ y la razón de error relevante $\frac{e_{i+1}}{e^{2}_{i}}$
    o $\frac{e_{i+1}}{e_{i}}$ que converge a un límite distinto de
    cero.

    \question

    Una cantidad crucial en el diseño de tuberías es la caída de
    presión debido a la fricción bajo flujo turbulento.
    La caída de presión por unidad de longitud se describe mediante
    el número de Darcy $f$, una cantidad adimensional que satisface
    la ecuación empírica de Colebrook
    \begin{equation*}
        \frac{1}{\sqrt{f}}=
        -2\log_{10}\left[\frac{\epsilon}{3.7D}+
            \frac{2.51}{R\sqrt{f}}\right]
    \end{equation*}
    donde $D$ es el diámetro interior de la tubería, $\epsilon$ es la
    altura de rugosidad del interior de la tubería y $R$ es el número
    de Reynolds del flujo.

    \begin{parts}
        \part\label{part:adarcy}

        Para $D=0.3$m, $\epsilon=0.0002$m y $R=105$, utilice el
        método de Newton para calcular el número de Darcy $f$.

        \part\label{part:bdarcy}
        Fije $D$ y como en~\eqref{part:adarcy}, y calcule el número
        de Darcy para varios números de Reynolds $R$ entre $104$ y
        $108$.
        Haga un gráfico del número de Darcy en función del número de
        Reynolds, utilizando un eje logarítmico para este último.
    \end{parts}

    \question

    El movimiento de un péndulo se describe mediante la siguiente EDO
    \begin{equation*}
        \diff{\theta}{t}+
        \frac{g}{l}
        \operatorname{sen}
        \theta=
        0
    \end{equation*}
    con la condición inicial
    \begin{math}
        \theta\left(0\right)=
        \theta_{0}
    \end{math}
    y
    \begin{math}
        \diff{\theta}{t}
        \left(0\right)=
        \dot{\theta}_{0}.
    \end{math}

    \begin{figure}[ht!]
        \centering
        \includegraphics[width=.2\paperwidth]{pendulum.png}
    \end{figure}
    La energía del sistema viene dado por
    \begin{equation*}
        E\left(\theta\right)=
        \frac{1}{2}
        \left[
            \dot{\theta}^{2}-\dot{\theta}_{0}^{2}
            \right]+
        \cos\theta_{0}-
        \cos\theta.
    \end{equation*}

    Calcule numéricamente los valores en el intervalo
    $\left[-\pi,\pi\right]$ para el cual la energía del sistema es
    $0$.
    ¿Qué es lo que observa?

    % \subsection{Método de punto fijo}

    \question

    Discuta la convergencia del método de punto fijo para la función
    $f\left(x\right)=ae^{x}$, $a\in\mathbb{R}$ para condiciones
    iniciales en el intervalo $\left[0,1\right]$.

    \question

    ¿Cuál de las siguientes tres iteraciones de punto fijo converge a
    la raíz cúbica de $4$?
    Clasifique las que convergen de más rápidas a más lentas.

    \begin{multicols}{3}
        \begin{parts}
            \part

            \begin{math}
                g\left(x\right)=
                \dfrac{2}{\sqrt{x}}
            \end{math}.

            \part

            \begin{math}
                g\left(x\right)=
                \dfrac{3x}{4}+\dfrac{1}{x^{2}}
            \end{math}.

            \part

            \begin{math}
                g\left(x\right)=
                \dfrac{2x}{3}+\dfrac{4}{3x^{2}}
            \end{math}.
        \end{parts}
    \end{multicols}

    \question

    Encuentre el conjunto de todas las estimaciones iniciales para
    las cuales la iteración de punto fijo
    $x\rightarrow\dfrac{4}{9}-x^{2}$ converge a un punto fijo.

    \begin{listing}[ht!]
        \tiny
        \centering
        \inputminted[linenos,highlightlines={7-15}]{octave}{fpi.m}
        \caption{Método de punto fijo.}
    \end{listing}

    % \subsection{Solución a sistemas de ecuaciones no lineales}

    \question

    Utilice el método de Newton con la estimación inicial
    $\left(1,2\right)$ para encontrar una solución del sistema no
    lineal.
    \begin{equation*}
        \left\{
        \begin{aligned}
            v-u^{3}       & =0. \\
            u^{2}+v^{2}-1 & =0.
        \end{aligned}
        \right.
    \end{equation*}

    \question

    Aplicar el método de Newton para encontrar ambas soluciones del
    sistema de tres ecuaciones.
    \begin{equation*}
        \left\{
        \begin{aligned}
            2u^{2}-4u+v^{2}+3w^{2}+6w+2 & =0. \\
            u^{2}+v^{2}-2v+2w^{2}-5     & =0. \\
            3u^{2}-12u+v^{2}+3w^{2}+8   & =0.
        \end{aligned}
        \right.
    \end{equation*}

    \question

    Aunque una intersección genérica de tres esferas en el espacio
    tridimensional son dos puntos, puede ser un único punto.
    Aplique el método de Newton multivariable para encontrar el único
    punto de intersección de las esferas con centro
    $\left(1,0,1\right)$ y radio $8$, centro $\left(0,2,2\right)$ y
    radio $\sqrt{2}$, y centro $\left(0,3,3\right)$ y radio
    $\sqrt{2}$.
    ¿La iteración sigue convergiendo cuadráticamente?
    Explique.
\end{questions}

\vfill{\noindent Fidel Jara Huanca\\Carlos Aznarán
}
\hfill{UNI, 14 de enero de 2025\footnote{Hecho en \LaTeX}}

\end{document}

- series de Taylor: sen, cos, log.
- aritmética de números punto flotante.

% firstline=1,lastline=18

\begin{listing}[ht!]
    \tiny
    \centering
    \inputminted[linenos,highlightlines={5-16}]{octave}{broyden2.m}
    \caption{Método de Broyden.}
\end{listing}

\begin{listing}[ht!]
    \tiny
    \centering
    \inputminted[linenos,highlightlines={5-15}]{octave}{jacobi.m}
    \caption{Método de jacobi.}
\end{listing}

\begin{listing}[ht!]
    \tiny
    \centering
    \inputminted[linenos,highlightlines={4-20}]{octave}{sparsesetup.m}
    \caption{Ensamblar la matriz dispersa.}
\end{listing}
