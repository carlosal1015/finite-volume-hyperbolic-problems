% arara: clean: {
% arara: --> extensions:
% arara: --> ['aux', 'log', 'out', 'pdf']
% arara: --> }
%! arara: lualatex: {
%! arara: --> shell: yes,
%! arara: --> draft: yes,
%! arara: --> interaction: batchmode
%! arara: --> }
%! arara: biber
% arara: lualatex: {
% arara: --> shell: yes,
% arara: --> draft: no,
% arara: --> interaction: batchmode
% arara: --> }
% arara: lualatex: {
% arara: --> shell: yes,
% arara: --> draft: no,
% arara: --> interaction: batchmode
% arara: --> }
% arara: clean: {
% arara: --> extensions:
% arara: --> ['aux', 'log', 'out']
% arara: --> }
\documentclass{scrletter}
\usepackage{graphicx}
\usepackage[shortlabels]{enumitem}
\usepackage{mathtools}
% \usepackage{plantuml}
\usepackage{hyperref}

\usepackage[
	citestyle=numeric,
	style=numeric,
	backend=biber,
]{biblatex}

\addbibresource{references.bib}

\newcommand{\MVAt}{{\usefont{U}{mvs}{m}{n}\symbol{`@}}}
\renewcommand{\thesection}{\Roman{section}}
\renewcommand{\thesubsection}{\thesection.\Roman{subsection}}

\begin{document}

\providecommand{\faculty}{Facultad}
\noindent\parbox[c]{.18\textwidth}{\includegraphics[width=2.8cm]{logouni}}\hfill
\parbox[c]{1\textwidth}{\raggedright%
    {\large\textbf{UNIVERSIDAD NACIONAL DE INGENIERÍA} \par\smallskip}
    {\large\textbf{\faculty} \par\smallskip}
    {\large\textbf{DEPARTAMENTO ACADÉMICO DE CIENCIAS BÁSICAS} \par\smallskip}
}

\begin{center}\bfseries\large
    Práctica Dirigida 4 de Métodos Numéricos (BMA-18)
\end{center}

\vspace{-0.5cm}

\hrulefill
\vspace{-2.5mm}

\rule{16.5cm}{0.8mm}

\section{Semana 6}

\subsection{Sistema de ecuaciones lineales}

Consideramos el problema de resolver varias ecuaciones lineales
simultáneas cuando el caso en el que el número de ecuaciones y el
número de variables desconocidas son iguales.
La eliminación gaussiana es la técnica más utilizada para sistemas de
ecuaciones lineales de tamaño razonable.
Los métodos iterativos son necesarios para sistemas muy grandes.

\subsubsection{Eliminación gaussiana ingenua}

Es tan simple que no se garantiza que se complete, y mucho menos que
se encuentre una solución precisa.
Se pueden aplicar tres operaciones útiles a un sistema lineal de
ecuaciones que dan como resultado un sistema equivalente, es decir,
uno que tenga las mismas soluciones.
Estas operaciones son las siguientes:

\begin{enumerate}
    \item

          Intercambiar una ecuación por otra.

    \item

          Sumar o restar un múltiplo de una ecuación de otra.

    \item

          Multiplicar una ecuación por una constante distinta de
          cero.
\end{enumerate}

\subsubsection{Conteo de operaciones}

Realizamos un recuento aproximado de las operaciones para las dos
partes de la eliminación gaussiana:
el paso de eliminación y el paso de sustitución hacia atrás.
Para ello, será útil escribir para el caso general las operaciones
que se llevaron a cabo en los dos ejemplos anteriores.
Para comenzar, recordemos dos hechos sobre las sumas de números
enteros.
Para cualquier $n\in\mathbb{N}$, se cumple

\begin{multicols}{2}
    \begin{enumerate}[a)]
        \item
              \begin{math}
                  1+2+3+4+5+\cdots+n=
                  \dfrac{n\left(n+1\right)}{2}
              \end{math},

        \item
              \begin{math}
                  1^{2}+2^{2}+3^{2}+4^{2}+5^{2}+\cdots+n^{2}=
                  \dfrac{n\left(n+1\right)\left(2n+1\right)}{6}
              \end{math}.
    \end{enumerate}
\end{multicols}

La forma general de la tabla para $n$ ecuaciones con $n$ incógnitas
es
\begin{equation*}
    \begin{bNiceArray}{cccc|c}
        a_{11} & a_{12} & \cdots & a_{1n} & b_{1}  \\
        a_{21} & a_{22} & \cdots & a_{2n} & b_{2}  \\
        \vdots & \vdots & \ddots & \vdots & \vdots \\
        a_{n1} & a_{n2} & \cdots & a_{nn} & b_{n}
    \end{bNiceArray}.
\end{equation*}

Para realizar el paso de eliminación, necesitamos poner ceros en el
triángulo inferior, utilizando las operaciones de fila permitidas.
Podemos escribir el paso de eliminación como el bucle donde, por
``eliminar la columna $j$'', queremos decir ``usar operaciones de
fila para poner un cero en cada ubicación debajo de la diagonal
principal, que son las ubicaciones
\begin{math}
    a_{j+1j},a_{j+2j},\dotsc,a_{nj}
\end{math}''.

\begin{listing}[ht!]
    \small
    \centering
    \inputminted[firstline=50,lastline=52]{octave}{dirigida4.m}
\end{listing}
Por ejemplo, para realizar la eliminación en la columna $1$,
necesitamos poner ceros en $a_{21},\dotsc,a_{n1}$.
Esto se puede escribir como el siguiente bucle dentro del bucle
anterior:
\begin{listing}[ht!]
    \small
    \centering
    \inputminted[firstline=54,lastline=60]{octave}{dirigida4.m}
\end{listing}
Queda por completar el paso interior del doble bucle, para aplicar
una operación de fila que establece la entrada $a_{ij}$ en cero.
Por ejemplo, la primera entrada que se eliminará es la entrada
$a_{21}$.
Para lograr esto, restamos $\frac{a_{21}}{a_{11}}$ multiplicado por
la fila $1$ de la fila $2$, suponiendo que $a_{11}\neq 0$.
Es decir, las dos primeras filas cambian de
\begin{equation*}
    \begin{NiceArray}{cccc|c}
        a_{11} & a_{12} & \cdots & a_{1n} & b_{1} \\
        a_{21} & a_{22} & \cdots & a_{2n} & b_{2}
    \end{NiceArray}.
\end{equation*}
a
\begin{equation*}
    \begin{NiceArray}{llll|l}
        a_{11} & a_{12}                             & \cdots & a_{1n}                             & b_{1}                            \\
        0      & a_{22}-\frac{a_{21}}{a_{11}}a_{12} & \cdots & a_{2n}-\frac{a_{21}}{a_{11}}a_{1n} & b_{2}-\frac{a_{21}}{a_{11}}b_{1}
    \end{NiceArray}.
\end{equation*}
Teniendo en cuenta las operaciones, se requiere una división (para
hallar el multiplicador $\frac{a_{21}}{a_{11}}$) más $n$
multiplicaciones y $n$ adiciones.
La operación de fila utilizada para eliminar la entrada $a_{i1}$ de
la primera columna, es decir,
\begin{equation*}
    \begin{NiceArray}{llll|l}
        a_{11} & a_{12}                             & \cdots & a_{1n}                             & b_{1}                            \\
        \vdots & \vdots                             & \ddots & \vdots                             & \vdots                           \\
        0      & a_{i2}-\frac{a_{i1}}{a_{11}}a_{12} & \cdots & a_{in}-\frac{a_{i1}}{a_{11}}a_{1n} & b_{i}-\frac{a_{i1}}{a_{11}}b_{1}
    \end{NiceArray}
\end{equation*}
requiere operaciones similares.
El procedimiento que se acaba de describir funciona siempre que el
número $a_{11}\neq 0$.
Este número y los demás números $a_{ii}$ que son eventualmente
divisores en la eliminación gaussiana se denominan \textbf{pivotes}.
Un pivote cero hará que el algoritmo se detenga.
Observe que eliminar cada entrada $a_{i1}$ en la primera columna
utiliza una división, $n$ multiplicaciones y $n$ summas / restas,
o $2n+1$ operaciones cuando se cuentan juntas.
Poner ceros en la primera columna requiere repetir estas $2n+1$
operaciones un total de $n-1$ veces.
Después de eliminar la primera columna, se utiliza el pivote $a_{22}$
para eliminar la segunda columna de la misma manera y las columnas
restantes después de eso.
Por ejemplo, la operación de fila utilizada para eliminar la entrada
$a_{ij}$ es
\begin{equation*}
    \begin{NiceArray}{lllll|l}
        0      & 0      & a_{jj+1}                               & \cdots & a_{jn}                             & b_{j}                            \\
        \vdots & \vdots & \vdots                                 & \ddots & \vdots                             & \vdots                           \\
        0      & 0      & a_{ij+1}-\frac{a_{ij}}{a_{jj}}a_{jj+1} & \cdots & a_{in}-\frac{a_{ij}}{a_{jj}}a_{jn} & b_{i}-\frac{a_{ij}}{a_{jj}}b_{j}
    \end{NiceArray}.
\end{equation*}
En nuestra notación, $a_{22}$, por ejemplo, se refiere al número
revisado en esa posición despuésde de la eliminación de la columna
$1$, que no es el $a_{22}$ original.
La operación de fila para eliminar $a_{ij}$ requiere una división,
$n-j+1$ multiplicaciones, y $n-j+1$ sumas / restas.
Inserte este paso en el mismo bucle da como resultado
\begin{listing}[ht!]
    \small
    \centering
    \inputminted[firstline=1,lastline=17,highlightlines={8,11,14}]{octave}{dirigida4.m}
\end{listing}
En primer lugar, pedirle al índice $k$ que se mueva de $j$ a $n$.
colocará un cero en la ubicación $a_{ij}$; sin embargo, moverse de
$j+1$ a $n$ es la codificación más eficiente.
Esto último no colocará un cero en la entrada $a_{ij}$, que era la
entrada que estamos tratando de eliminar.
Aunque esto parece ser un error, tenga en cuenta que nunca volveremos
a esta entrada en el resto del proceso de eliminación gaussiana o
sustitución hacia atrás, por lo que en realidad poner un cero allí
representa paso desperdiciado desde el punto de vista de la
eficiencia.
En segundo lugar, le pedimos que termine el programa utilizando el
comando de error si encuentra que un pivote es cero.
Podemos hacer un recuento total de operacionespara el paso de la
eliminación gaussiana.
La eliminación de cada $a_{ij}$ requiere la siguiente cantidad de
operaciones, incluidas divisiones, multiplicaciones y sumas / restas.
\begin{equation*}
    \begin{bNiceArray}{ccccccc}
        0 &  &  &  &  & &\\
        2n+1 & 0 & & & & & \\
        2n+1 & 2\left(n-1\right)+1 & 0 &&&  &  \\
        2n+1 & 2\left(n-1\right)+1 & 2\left(n-2\right)+1 & 0 &&& \\
        \vdots & \vdots & \vdots & \ddots&\ddots&  & \\
        \vdots & \vdots & \vdots & &&  & \\
        2n+1 & 2\left(n-1\right)+1 & 2\left(n-2\right)+1 &\cdots & 2\left(3\right)+1 & 0 &   \\
        2n+1 & 2\left(n-1\right)+1 & 2\left(n-2\right)+1 &\cdots & 2\left(3\right)+1 & 2\left(2\right)+1 & 0
    \end{bNiceArray}.
\end{equation*}
Es conveniente sumar las operaciones en orden inverso a como se
aplican.
Empezando por la derecha, totalizamos las operaciones como
\begin{align*}
    \sum_{j=1}^{n-1}
    \sum_{i=1}^{j}
    2\left(j+1\right)+1 & =
    \sum_{j=1}^{n-1}
    2j\left(j+1\right)+j.                         \\
                        & =
    2\sum_{j=1}^{n-1}j^{2}+
    3\sum_{j=1}^{n-1}j=
    2\frac{\left(n-1\right)n\left(2n-1\right)}{6}+
    3\frac{\left(n-1\right)n}{2}.                 \\
                        & =
    \left(n-1\right)n
    \left[\frac{2n-1}{3}+\frac{3}{2}\right]=
    \frac{n\left(n-1\right)\left(4n+7\right)}{6}. \\
                        & =
    \frac{2}{3}n^{3}+\frac{1}{2}n^{2}-\frac{7}{6}n.
\end{align*}
Por lo tanto, el paso de eliminación para un sistema de $n$
ecuaciones en $n$ variables se puede completar en
$\frac{2}{3}n^{3}+\frac{1}{2}n^{2}-\frac{7}{6}n$ operaciones.
Normalmente, el recuento exacto de operaciones es menos importante
que las estimaciones de orden de magnitud, ya que los detalles de
implementación en los distintos procesadores de computadora difieren.
El punto principal es que el número de operaciones es aproximadamente
proporcional al tiempo de ejecución del algoritmo.
Comúnmente haremos la aproximación de $\frac{2}{3}n^{3}$ operaciones
para la eliminación, que es una aproximación razonablemente precisa
cuando $n$ es grande.

Una vez realizada la eliminación, la tabla queda en la forma
triangular superior:
\begin{equation*}
    \begin{bNiceArray}{cccc|c}
        a_{11} & a_{12} & \cdots & a_{1n} & b_{1}  \\
        0 & a_{22} & \cdots & a_{2n} & b_{2}  \\
        \vdots & \vdots & \ddots & \vdots & \vdots \\
        0 & 0 & \cdots & a_{nn} & b_{n}
    \end{bNiceArray}.
\end{equation*}
En forma de ecuación,
\begin{align*}
    a_{11}x_{1} + a_{12}x_{2} + \cdots + a_{1n}x_{n} & = b_{1}     \\
    a_{22}x_{2} + \cdots + a_{2n}x_{2}               & = b_{2}     \\
                                                     & \;\; \vdots \\
    a_{nn}x_{n}                                      & = b_{n},
\end{align*}
donde, nuevamente, $a_{ij}$ se refieren a las entradas revisadas, no
a las originales.
Para completar el cálculo de la solución $x$, debemos realizar el
paso de sustitución hacia atrás.
\begin{align*}
    x_{1} & = \frac{b_{1}-a_{12}x_{2}-\cdots-a_{1n}x_{n}}{a_{11}} \\
    x_{2} & = \frac{b_{2}-a_{23}x_{3}-\cdots-a_{2n}x_{n}}{a_{22}} \\
          & \;\; \vdots                                           \\
    x_{n} & = \frac{b_{n}}{a_{nn}}.
\end{align*}
Debido a la forma triangular de los coeficientes distintos de cero de
las ecuaciones, comenzamos desde abajo y avanzamos hasta la ecuación
superior.
De esta manera, se conocen los $x_{i}$ requeridos cuando se los
necesita para calcular el siguiente.
Contando operaciones se obtiene
\begin{equation*}
    1+3+5+\cdots+\left(2n-1\right)=
    \sum_{i=1}^{n}
    \left(2i-1\right)=
    2\sum_{i=1}^{n}i-
    \sum_{i=1}^{n}1=
    2\frac{n\left(n+1\right)}{2}-n=
    n^{2}.
\end{equation*}
El paso de sustitución hacia atrás es
\begin{listing}[ht!]
    \small
    \centering
    \inputminted[firstline=19,lastline=26]{octave}{dirigida4.m}
\end{listing}

El recuento de operaciones muestra que la solución directa de $n$
ecuaciones con $n$ incógnitas por eliminación gaussiana es un
proceso $\mathcal{O}\left(n^{3}\right)$.
Este es un hecho útil para estimar el tiempo necesario para resolver
sistemas grandes.
Por ejemplo, para estimar el tiempo necesario para resolver un
sistema de $n=500$ ecuaciones en una computadora en particular,
podríamos obtener una estimación justa resolviendo un sistema de
$n=50$ ecuaciones y luego escalando el tiempo transcurrido por
$10^{3}=1000$.

\begin{questions}
    \question

    Utilice la eliminación gaussiana para resolver los sistemas:

    \begin{multicols}{3}
        \begin{parts}
            \part

            \begin{math}
                \systeme[xyz]{2x-2y-z=-2,4x+y-2z=1,-2x+y-z=-3}
            \end{math}.

            \part

            \begin{math}
                \systeme[xyz]{x+2y-z=2,3y+z=4,2x-y+z=2}
            \end{math}.

            \part

            \begin{math}
                \systeme[xyz]{2x+y-4z=-7,x-y+z=-2,-x+3y-2z=6}
            \end{math}.
        \end{parts}
    \end{multicols}

    \question

    Resolver la forma de tabla

    \begin{multicols}{2}
        \begin{parts}
            \part

            \begin{math}
                \begin{bNiceArray}{ccc|c}
                    3 & -4 & -2 & 3  \\
                    6 & -6 & 1 & 2   \\
                    -3 & 8 & 2 & -1
                \end{bNiceArray}
            \end{math}.

            \part

            \begin{math}
                \begin{bNiceArray}{ccc|c}
                    2 & 1 & -1 & 2  \\
                    6 & 2 & -2 & 8   \\
                    4 & 6 & -3 & 5
                \end{bNiceArray}
            \end{math}.
        \end{parts}
    \end{multicols}

    \question

    Resolver por sustitución hacia atrás:

    \begin{multicols}{2}
        \begin{parts}
            \part

            \begin{math}
                \systeme[xyz]{3x-4y+5z=2,3y-4z=-1,5z=5}
            \end{math}.

            \part

            \begin{math}
                \systeme[xyz]{x-2y+z=2,4y-3z=1,-3z=3}
            \end{math}.
        \end{parts}
    \end{multicols}

    \question

    Si un sistema de $3000$ ecuaciones con $3000$ incógnitas se puede
    resolver mediante eliminación gaussiana en $5$ segundos en una
    computadora determinada, ¿cuántas sustituciones hacia atrás del
    mismo tamaño se pueden realizar por segundo?

    \question

    Sea $H$ la matriz de Hilbert,\footnote{Consulte \url{https://docs.octave.org/latest/Famous-Matrices.html\#index-hilb}.}
    cuya entrada $\left(i,j\right)$ es $\dfrac{1}{i+j-1}$.
    Utilice un programa para resolver $Hx=b$, donde $b$ es el vector
    de todos los unos, para
    \begin{multicols}{4}
        \begin{parts}
            \part

            \begin{math}
                n=2
            \end{math}.

            \part

            \begin{math}
                n=5
            \end{math}.

            \part

            \begin{math}
                n=10
            \end{math}.

            \part

            \begin{math}
                n=20
            \end{math}.
        \end{parts}
    \end{multicols}
\end{questions}

\subsubsection{Factorización $LU$}

Una vez que se conocen $L$ y $U$, el problema $Ax=b$ puede escribirse
como $LUx=b$.
Defina un nuevo vector ``auxiliar'' $c=Ux$.
Luego, la sustitución inversa es un procedimiento de dos pasos:
\begin{enumerate}[(a)]
    \item

          Resuelva $Lc=b$ para $c$.

    \item

          Resuelva $Ux=c$ para $x$.
\end{enumerate}
Ambos pasos son sencillos, ya que $L$ y $U$ son matrices
triangulares.

\begin{algorithm}[H]
    \caption{Sustitución hacia adelante orientada a filas}
    \% Si $L\in\mathbb{R}^{n\times n}$ es una matriz triangular
    inferior y $b\in\mathbb{R}^{n}$, entonces este algoritmo
    sobrescribe $b$ con la solución $Lx=b$. $L$ es no singular.

    $b\left[1\right]=\frac{b\left[1\right]}{L\left[1,1\right]}$

    \For{$i=2,\dotsc,n$}{
        \begin{math}
            b\left[i\right]\longleftarrow
            \frac{b\left[i\right]-L\left[i,1\dotsc,i-1\right]\cdot
                b\left[1\dotsc,i-1\right]}{L\left[i,i\right]}
        \end{math}\;
    }
\end{algorithm}

Este algoritmo requiere $\mathcal{O}\left(n^{2}\right)$ flops y $L$
es accedido por filas.

\begin{algorithm}[H]
    \caption{Sustitución hacia atrás orientada a filas}
    \% Si $U\in\mathbb{R}^{n\times n}$ es una matriz triangular
    superior y $b\in\mathbb{R}^{n}$, entonces este algoritmo
    sobrescribe $b$ con la solución $Ux=b$.
    $U$ es no singular.

    $b\left[n\right]=\frac{b\left[n\right]}{U\left[n,n\right]}$

    \For{$k=n-1,\dotsc,1$}{
        \begin{math}
            b\left[i\right]\longleftarrow
            \frac{b\left[i\right]-U\left[i,i+1\dotsc,n\right]\cdot
                b\left[i+1\dotsc,n\right]}{U\left[i,i\right]}
        \end{math}\;
    }
\end{algorithm}

Este algoritmo requiere $\mathcal{O}\left(n^{2}\right)$ flops y $U$
es accedido por filas.
La eliminación gaussiana clásica implica tanto a $A$ como a $b$ en el
paso de eliminación del cálculo.
Esta es, por lejos, la parte más costosa del proceso, como hemos
visto.
Ahora, supongamos que necesitamos resolver varios problemas
diferentes con el mismo $A$ y diferentes $b$.
Es decir, se nos presenta el conjunto de problemas con varios
vectores del lado derecho $b_{i}$.
La eliminación gaussiana clásica requerirá aproximadamente
$\frac{2k^{3}}{3}$ operaciones, donde $A$ es una matriz $n\times n$,
ya que debemos comenzar desde el principio para cada problema.
Con el enfoque $LU$, por otro lado, el $b$ del lado derecho no entra
en los cálculos hasta que finaliza la eliminación (la factorización
$A=LU$).
Al aislar los cálculos que involucran a $A$ de $b$, podemos resolver
el conjunto anterior de ecuaciones con una sola eliminación, seguida
de dos sustituciones hacia atrás ($Lc=b$, $Ux=c$) para cada nuevo
$b$.
El número aproximado de operaciones con el enfoque $LU$ es, por lo
tanto, $\frac{2n^3}{3}+2kn^{2}$.
Cuando $n^{2}$ es pequeño en comparación con $n^{3}$ (es decir,
cuando $n$ es grande), esta es una diferencia significativa.

\begin{theorem}
    Si $A\in\mathbb{R}^{n\times n}$ y
    \begin{math}
        \forall k\in\left\{1,\dotsc,n-1\right\}:
        \det\left(A\left(1:k,1:k\right)\right)\neq 0
    \end{math},
    entonces existe una única matriz triangular inferior de unos en
    su diagonal $L\in\mathbb{R}^{n\times n}$ y una matriz triangular
    superior $U\in\mathbb{R}^{n\times n}$ tal que $A=LU$.
    Si este es el caso y $A$ es no singular, entonces la
    factorización $LU$ es única y el
    $\det\left(A\right)=u_{11}\cdots u_{nn}$
\end{theorem}

\begin{algorithm}[H]
    \caption{Row-Oriented Generalized $AX$ Plus $Y$ $LU$}
    \% Si $A\in\mathbb{R}^{n\times n}$ tiene la propiedad que
    $A\left[1,\dotsc,k,1,\dotsc,k\right]$ es no singular para
    $k\in\left\{1,\dotsc,n-1\right\}$.
    Este algoritmo calcula la factorización $LU$ donde $L$ es una
    matriz triangular inferior de unos en su diagonal y $U$ es una
    matriz triangular superior.

    Inicialice la matriz $L$ como la matriz identidad y $U$ como la
    matriz de ceros.

    \For{$j=1,\dotsc,n$}{
        \For{$k=1,\dotsc,r$}{
            $C\left[:,j\right]\longleftarrow C\left[:,j\right]+A\left[:,k\right]B\left[k,j\right]$\;
            \eIf{$j=1$}{
                \begin{math}
                    v=A\left[:,1\right]
                \end{math}
            }{
                \begin{math}
                    \widetilde{a}=
                    A\left[:,j\right]
                \end{math}
                \text{Resolver }$L\left[1:j-1,1:j-1\right]\cdot z=\widetilde{a}\left[1:j-1\right]$\text{ para }$z\in\mathbb{R}^{j-1}$
                \begin{math}
                    U\left[1:j-1,j\right]=z
                \end{math}
                \begin{math}
                    v\left[j:n\right]=\widetilde{a}\left[j:n\right]-L\left[j:n,1:j-1\right]\cdot z
                \end{math}
            }
            $U\left[j,j\right]\longleftarrow v\left[j\right]$

            $L\left[j+1:n,j\right]\longleftarrow \frac{v\left[j+1:n\right]}{v\left[j\right]}$
        }
    }
\end{algorithm}

\begin{theorem}
    Si $A$ es no singular y diagonalmente dominante por columna,
    entonces este tiene factorización $LU$ y las entradas de
    \begin{math}
        L=
        \begin{bNiceArray}{c}
            \ell_{ij}
        \end{bNiceArray}
    \end{math}
    satisface
    \begin{math}
        \left|\ell_{ij}\right|\leq 1
    \end{math}.
\end{theorem}

\begin{theorem}
    Si $A\in\mathbb{R}^{n\times n}$ es una matriz simétrica y para
    todo $k\in\left\{1,\dotsc,n-1\right\}$ la submatriz principal
    $A\left(1:k,1:k\right)$ es no singular, entonces existe una
    matriz triangular inferior de unos en su diagonal y una matriz
    diagonal
    \begin{math}
        D=
        \operatorname{diag}
        \left(d_{1},\dotsc,d_{n}\right)
    \end{math}
    tal que $A=LDL^{T}$.
    La factorización es única.
\end{theorem}

\begin{algorithm}[H]
    \caption{$LDL^{T}$}
    \% Si $A\in\mathbb{R}^{n\times n}$ es una matriz simétrica y
    tiene una factorización $LU$, este algoritmo calcula la matriz
    triangular inferior de unos en su diagonal $L$ y una matriz
    diagonal $D=\operatorname{diag}\left(d_{1},\dotsc,d_{n}\right)$
    tal que $A=LDL^{T}$.
    La entrada $a_{ij}$ es sobreescrita con $\ell_{ij}$ si $i>j$ y
    $d_{i}$ si $i=j$.

    \For{$j=1,\dotsc,n$}{
        \For{$i=1,\dotsc,j-1$}{
            \begin{math}
                v\left[i\right]\longleftarrow
                A\left[j,i\right]
                A\left[i,i\right]
            \end{math}\;
        }
        \begin{math}
            A\left[j,j\right]\longleftarrow
            A\left[j,j\right]-
            A\left[j,1,\dotsc,j-1\right]\cdot
            v\left[1,\dotsc,j-1\right]
        \end{math}\;

        \begin{math}
            A\left[j+1,\dotsc,n,j\right]\longleftarrow
            \frac{A\left[j+1,\dotsc,n,j\right]-A\left[j+1,\dotsc n,1,\dotsc,j-1\right]\cdot v\left[1,\dotsc,j-1\right]}{A\left[j,j\right]}
        \end{math}\;
    }
\end{algorithm}

\begin{definition}
    Una matriz $A\in\mathbb{R}^{n\times n}$ es definida positiva (semidefinida positiva) si y solamente si
    $\forall x\in\mathbb{R}^{n}\setminus\left\{0\right\}$ se cumple que $x^{T}Ax>0$. ($x^{T}Ax\geq 0$)
    Decimos que $A$ es indefinida si y solamente si $\exists x,y\in\mathbb{R}^{n}$ tales que
    \begin{math}
        \left(x^{T}Ax\right)
        \left(y^{T}Ay\right)<0
    \end{math}.
\end{definition}

\begin{theorem}
    La matriz $A\in\mathbb{R}^{n\times n}$ es definida positiva si y solamente si la matriz simétrica
    $\frac{A+A^{T}}{2}$ tiene autovalores positivos.
\end{theorem}

\begin{theorem}
    Si $A\in\mathbb{R}^{n\times n}$ es una matriz simétrica definida positiva,
    entonces existe una única matriz triangular inferior $G\in\mathbb{R}^{n\times n}$
    con entradas en su diagonal positivas tal que $A=GG^{T}$.
\end{theorem}

\begin{questions}
    \question

    Encuentra la factorización $LU$ de las matrices dadas.
    Compruébalo mediante la multiplicación de matrices.

    \begin{multicols}{3}
        \begin{parts}
            \part

            \begin{math}
                \begin{bNiceArray}{cccc}
                    1 & -1& 1 & 2 \\
                    0 & 2 & 1 & 0 \\
                    1 & 3 & 4 & 4 \\
                    0 & 2 & 1 & -1
                \end{bNiceArray}
            \end{math}.

            \part

            \begin{math}
                \begin{bNiceArray}{ccc}
                    3 & 1 & 2 \\
                    6 & 3 & 4 \\
                    3 & 1 & 5
                \end{bNiceArray}
            \end{math}.

            \part

            \begin{math}
                \begin{bNiceArray}{ccc}
                    4 & 2 & 0 \\
                    4 & 4 & 2 \\
                    2 & 2 & 3
                \end{bNiceArray}
            \end{math}.
        \end{parts}
    \end{multicols}

    \question

    Resuelva el sistema encontrando la factorización $LU$ y luego
    realizando la sustitución hacia atrás en dos pasos.

    \begin{multicols}{2}
        \begin{parts}
            \part

            \begin{math}
                \begin{bNiceArray}{ccc}
                    3 & 1 & 2 \\
                    6 & 3 & 4 \\
                    3 & 1 & 5
                \end{bNiceArray}
                \begin{bNiceArray}{c}
                    x_{1} \\
                    x_{2} \\
                    x_{3}
                \end{bNiceArray}=
                \begin{bNiceArray}{c}
                    0 \\
                    1 \\
                    3
                \end{bNiceArray}
            \end{math}.

            \part

            \begin{math}
                \begin{bNiceArray}{ccc}
                    4 & 2 & 0 \\
                    4 & 4 & 2 \\
                    2 & 2 & 3
                \end{bNiceArray}
                \begin{bNiceArray}{c}
                    x_{1} \\
                    x_{2} \\
                    x_{3}
                \end{bNiceArray}=
                \begin{bNiceArray}{c}
                    2 \\
                    4 \\
                    6
                \end{bNiceArray}
            \end{math}.
        \end{parts}
    \end{multicols}

    \question

    Resuelva la ecuación $Ax=b$, donde
    \begin{equation*}
        A=
        \begin{bNiceArray}{cccc}
            1 & 0 & 0 & 0\\
            0 & 1 & 0 & 0 \\
            1 & 3 & 1 & 0\\
            4 & 1 & 2 & 1
        \end{bNiceArray}
        \begin{bNiceArray}{cccc}
            2 & 1 & 0 & 0\\
            0 & 1 & 2 & 0 \\
            0 & 0 & -1 & 1\\
            0 & 0 & 0 & 1
        \end{bNiceArray}
        \text{  y  }
        b=
        \begin{bNiceArray}{c}
            1\\
            1 \\
            2\\
            0
        \end{bNiceArray}.
    \end{equation*}
\end{questions}

\subsection{Métodos Iterativos}

\begin{listing}[ht!]
    \small
    \centering
    \inputminted[linenos]{octave}{sparsesetup.m}
\end{listing}

\begin{listing}[ht!]
    \small
    \centering
    \inputminted[linenos]{octave}{jacobi.m}
\end{listing}


\begin{questions}
    \question

    Reordene las ecuaciones $-x+4y+z=2$, $x-y+3z=8$, $2x-z=4$ para que sean estrictamente
    diagonalmente dominantes y aplique dos pasos del método de Gauss-Seidel con un valor
    inicial $[0,0,0]$ para aproximar la solución.

    \question

    Calcule los dos primeros pasos de los métodos de Jacobi y de Gauss-Seidel con el vector inicial $\left[0,\dotsc,0\right]$.

    \begin{multicols}{3}
        \begin{parts}
            \part

            \begin{math}
                \begin{bNiceArray}{cc}
                    3 & -1 \\
                    -1 & 2
                \end{bNiceArray}
                \begin{bNiceArray}{c}
                    u \\
                    v
                \end{bNiceArray}=
                \begin{bNiceArray}{c}
                    5 \\
                    4
                \end{bNiceArray}
            \end{math}

            \part

            \begin{math}
                \begin{bNiceArray}{ccc}
                    2 & -1 & 0 \\
                    -1 & 2 & -1 \\
                    0 & -1 & 2
                \end{bNiceArray}
                \begin{bNiceArray}{c}
                    u \\
                    v \\
                    w
                \end{bNiceArray}=
                \begin{bNiceArray}{c}
                    0 \\
                    2 \\
                    0
                \end{bNiceArray}
            \end{math}


            \part

            \begin{math}
                \begin{bNiceArray}{ccc}
                    3 & 1 & 1 \\
                    1 & 3 & 1 \\
                    1 & 1 & 3
                \end{bNiceArray}
                \begin{bNiceArray}{c}
                    u \\
                    v \\
                    w
                \end{bNiceArray}=
                \begin{bNiceArray}{c}
                    6 \\
                    3 \\
                    5
                \end{bNiceArray}
            \end{math}
        \end{parts}
    \end{multicols}

    \question

    Reordenar las ecuaciones para formar un sistema estrictamente
    diagonalmente dominante.
    Aplicar dos pasos de los métodos de Jacobi y Gauss-Seidel a
    partir del vector inicial $\left[0,\dotsc,0\right]$.

    \question

    Utilice el método de Jacobi para resolver el sistema disperso
    dentro de seis decimales correctas (error hacia adelante en la norma de infinito)
    para $n=100$ y $n=100000$.
    La solución correcta es $\left[1,\dotsc,1\right]$.
    Informe la cantidad de pasos necesarios y el error hacia atrás.
    El sistema es
    \begin{equation*}
        \begin{bNiceArray}{ccccc}
            3 & -1 & 0 & \cdots & 0 \\
            -1 & 3 &-1 & 0 \cdots & 0 \\
            0 & \ddots & \ddots & \ddots & 0\\
            0 & 0 & -1 & 3 & -1 \\
            0 & 0 & 0 &-1 & 3
        \end{bNiceArray}
        \begin{bNiceArray}{c}
            x_{1} \\
            \vdots\\
            \vdots \\
            \vdots \\
            x_{n}
        \end{bNiceArray}=
        \begin{bNiceArray}{c}
            2 \\
            1 \\
            \vdots\\
            1 \\
            2
        \end{bNiceArray}.
    \end{equation*}

    \question

    Demuestre que las siguientes matrices son simétricas definidas
    positivas expresando $x^{T}Ax$ como una suma de cuadrados.
    \begin{multicols}{3}
        \begin{parts}
            \part

            \begin{math}
                \begin{bNiceArray}{ccc}
                    1 & 0 & 0 \\
                    0 & 2 & 0 \\
                    0 & 0 & 3
                \end{bNiceArray}
            \end{math}

            \part

            \begin{math}
                \begin{bNiceArray}{cc}
                    1 & 3  \\
                    3 & 10
                \end{bNiceArray}
            \end{math}

            \part

            \begin{math}
                \begin{bNiceArray}{cc}
                    1 & 0  \\
                    0 & 3
                \end{bNiceArray}
            \end{math}
        \end{parts}
    \end{multicols}

    \question

    Demuestre que si $d>4$, la matriz
    $A=\begin{bNiceArray}{cc}
            1 & 2 \\
            2& d
        \end{bNiceArray}$
    es definida positiva.
\end{questions}

\providecommand{\name}{Nombre}
\vfill{Nombre}\footnote{Hecho en \LaTeX}
\hfill{UNI, 17 de febrero de 2025}

\clearpage

\appendix

\section{Algoritmos}

Presentamos una lista de algoritmos elementales de la multiplicación
de matrices requeridos en la resolución de sistemas de ecuaciones
lineales.

\begin{algorithm}[H]
    \caption{Producto interno}
    \% Si $x,y\in\mathbb{R}^{n}$, entonces el algoritmo calcula su
    producto interno $c=x^{T}y$.

    $c\longleftarrow 0$\;
    \For{$j=1,\dotsc,n$}{
        $c\longleftarrow c+x\left[i\right]y\left[i\right]$\;
    }
\end{algorithm}

La operación del producto escalar es una operación
$\mathcal{O}\left(n\right)$, lo que significa que la cantidad de
trabajo escala linealmente con la dimensión.
El cálculo de Single-Precision $AX$ Plus $Y$ también es
$\mathcal{O}\left(n\right)$.

\begin{algorithm}[H]
    \caption{Single-Precision $AX$ Plus $Y$}
    \% Si $x,y\in\mathbb{R}^{n}$, entonces este algoritmo sobrescribe
    $y$ con $y+ax$.

    \For{$j=1,\dotsc,n$}{
        $y\left[i\right]\longleftarrow y\left[i\right]+ax\left[i\right]$\;
    }
\end{algorithm}

\begin{algorithm}[H]
    \caption{Row-Oriented Generalized $AX$ Plus $Y$}
    \% Si $A\in\mathbb{R}^{m\times n}$, $x\in\mathbb{R}^{n}$ e
    $y\in\mathbb{R}^{m}$, entonces este algoritmo sobrescribe $y$ con
    $Ax+y$.

    \For{$i=1,\dotsc,m$}{
        \For{$j=1,\dotsc,n$}{
            $y\left[i\right]\longleftarrow y\left[i\right]+A\left[i,j\right]x\left[i\right]$\;
        }
    }
\end{algorithm}

Tenga en cuenta que esto implica $\mathcal{O}\left(mn\right)$
trabajo.

\begin{algorithm}[H]
    \caption{Column-Oriented Generalized $AX$ Plus $Y$}
    \% Si $A\in\mathbb{R}^{m\times n}$, $x\in\mathbb{R}^{n}$ e
    $y\in\mathbb{R}^{m}$, entonces este algoritmo sobrescribe $y$ con
    $Ax+y$.

    \For{$j=1,\dotsc,m$}{
        \For{$i=1,\dotsc,n$}{
            $y\left[i\right]\longleftarrow y\left[i\right]+A\left[i,j\right]x\left[i\right]$\;
        }
    }
\end{algorithm}

\begin{algorithm}[H]
    \caption{Actualización producto exterior}
    \% Si $A\in\mathbb{R}^{m\times n}$, $x\in\mathbb{R}^{m}$ e
    $y\in\mathbb{R}^{n}$, entonces este algoritmo sobrescribe $y$ con
    $Ax+y$.

    \For{$i=1,\dotsc,m$}{
        \For{$j=1,\dotsc,n$}{
            $A\left[i,j\right]\longleftarrow A\left[i,j\right]+x\left[i\right]y\left[j\right]$\;
        }
    }
\end{algorithm}

Esto implica $\mathcal{O}\left(mn\right)$ operaciones aritméticas.

\begin{algorithm}[H]
    \caption{Multiplicación de matrices $ijk$}
    \% Si $A\in\mathbb{R}^{m\times r}$, $B\in\mathbb{R}^{r\times n}$
    y $C\in\mathbb{R}^{m\times n}$, entonces este algoritmo
    sobrescribe $C$ con $C+AB$.

    \For{$i=1,\dotsc,m$}{
        \For{$j=1,\dotsc,n$}{
            \For{$k=1,\dotsc,r$}{
                $C\left[i,j\right]\longleftarrow C\left[i,j\right]+A\left[i,k\right]B\left[k,j\right]$\;
            }
        }
    }
\end{algorithm}

Este cálculo implica aritmética $\mathcal{O}\left(mnr\right)$.

\begin{algorithm}[H]
    \caption{Multiplicación de matrices producto interno}
    \% Si $A\in\mathbb{R}^{m\times r}$, $B\in\mathbb{R}^{r\times n}$
    y $C\in\mathbb{R}^{m\times n}$, entonces este algoritmo
    sobrescribe $C$ con $C+AB$.

    \For{$i=1,\dotsc,m$}{
        \For{$j=1,\dotsc,n$}{
            $C\left[i,j\right]\longleftarrow C\left[i,j\right]+A\left[i,:\right]B\left[:,j\right]$\;
        }
    }
\end{algorithm}

\begin{algorithm}[H]
    \caption{Multiplicación de matrices Saxpy}
    \% Si $A\in\mathbb{R}^{m\times r}$, $B\in\mathbb{R}^{r\times n}$
    y $C\in\mathbb{R}^{m\times n}$, entonces este algoritmo
    sobrescribe $C$ con $C+AB$.

    \For{$j=1,\dotsc,m$}{
        \For{$k=1,\dotsc,r$}{
            $C\left[:,j\right]\longleftarrow C\left[:,j\right]+A\left[:,k\right]B\left[k,j\right]$\;
        }
    }
\end{algorithm}

\begin{algorithm}[H]
    \caption{Producto exterior $ijk$}
    \% Si $A\in\mathbb{R}^{m\times r}$, $B\in\mathbb{R}^{r\times n}$
    y $C\in\mathbb{R}^{m\times n}$, entonces este algoritmo
    sobrescribe $C$ con $C+AB$.

    \For{$k=1,\dotsc,r$}{
        \For{$j=1,\dotsc,n$}{
            \For{$i=1,\dotsc,m$}{
                $C\left[i,j\right]\longleftarrow C\left[i,j\right]+A\left[i,k\right]B\left[k,j\right]$\;
            }
        }
    }
\end{algorithm}

\begin{algorithm}[H]
    \caption{Multiplicación de matrices triangulares}
    \% Si $A,B,C\in\mathbb{R}^{n\times n}$ son matrices triangulares,
    entonces este algoritmo sobrescribe $C$ con $C+AB$.

    \For{$k=1,\dotsc,n$}{
        \For{$j=1,\dotsc,n$}{
            \For{$k=i,\dotsc,j$}{
                $C\left[i,j\right]\longleftarrow C\left[i,j\right]+A\left[i,k\right]B\left[k,j\right]$\;
            }
        }
    }
\end{algorithm}

\end{document}