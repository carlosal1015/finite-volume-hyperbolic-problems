% arara: clean: {
% arara: --> extensions:
% arara: --> ['aux', 'log', 'pdf']
% arara: --> }
% arara: lualatex: {
% arara: --> shell: yes,
% arara: --> draft: no,
% arara: --> interaction: batchmode
% arara: --> }
% arara: lualatex: {
% arara: --> shell: yes,
% arara: --> draft: no,
% arara: --> interaction: batchmode
% arara: --> }
% arara: clean: {
% arara: --> extensions:
% arara: --> ['aux', 'log']
% arara: --> }
\documentclass[a4paper,10pt]{scrartcl}
\usepackage[left=1.5cm,right=1.5cm]{geometry}
\usepackage[spanish]{babel}
\decimalpoint
\usepackage{array}
\usepackage{graphicx}
\usepackage{caption}
\usepackage{mathtools}
\usepackage{amsthm}

\theoremstyle{definition}
\newtheorem{definition}{Definición}
\newtheorem{example}{Ejemplo}

\title{
    Métodos numéricos:
    breve introducción a los números de punto flotante\thanks{
    Adaptación de
    \emph{Numerical Methods: brief introduction to floating point numbers}
    por Elena Celledoni.}
}

\author{
    Fidel Jara Huanca\\
    Carlos Aznarán Laos}

\date{\today}

\begin{document}

\maketitle

\section{Representación de números en un ordenador: modelo de punto flotante}

Las computadoras tienen una memoria finita, por lo que no todos los
números se pueden representar exactamente en una computadora.
Por ejemplo, $\sqrt{2}$ y $\pi$ tienen una cantidad infinita de
dígitos, por lo que es imposible representarlos exactamente en una
computadora.
Para que quepan en una computadora, los números reales se aproximan
mediante el \textbf{modelo de punto flotante}:

\begin{itemize}
    \item

          Se utiliza el sistema binario:
          \begin{math}
              r=\pm
              \left(
              \alpha_{k}2^{k}+
              \alpha_{k-1}2^{k-1}+
              \cdots+
              \alpha_{1}2^{1}+
              \alpha_{0}2^{0}
              \right)
          \end{math}
          donde
          \begin{math}
              \alpha_{0},\dotsc,\alpha_{k}\in
              \left\{0,1\right\},
              \alpha_{k}\neq 0.
          \end{math}

    \item

          Se asigna una cantidad \textbf{fija} de memoria para
          representar cada número:
          \begin{align*}
              r
               & =\pm
              0.
              \alpha_{k}\alpha_{k-1}\dotsc
              \alpha_{k-m-1}\alpha_{k-m}\dotsc
              \alpha_{0}\times 2^{k+1}. \\
              \operatorname{fl}
              \left(r\right)
               & =\pm
              0.
              \alpha_{k}\alpha_{k-1}\dotsc
              \alpha_{k-m-1}\widetilde{\alpha}_{k-m}\times
              2^{E}.
          \end{align*}
\end{itemize}

\section{Números de punto flotante}

Un \emph{bit} es una unidad básica de información en la teoría de la
información, el nombre bit proviene del \emph{dígito binario}.
En el sistema de números de doble precisión IEEE 745, $1$ bit está
destinado para el signo, $52$ bits para los dígitos significativos y
$11$ bits para el exponente.

\begin{minipage}{0.35\paperwidth}
    \centering
    \textbf{Precisión doble IEEE 745}\\[.3\baselineskip]
    \begin{tabular}{|c|c|c|}
        \hline
        signo & dígitos significativos & exponente \\
        \hline
        1 bit & 52 bits                & 11 bits   \\
        \hline
    \end{tabular}

    \

    \

    \begin{flushleft}
        \textbf{\bfseries William Kahan, premio Turing en 1989, fue
            el arquitecto principal detrás de la norma IEEE 754-1985.}
    \end{flushleft}
\end{minipage}
\hfill
\begin{minipage}{0.35\paperwidth}
    \centering
    \includegraphics[width=.25\paperwidth]{kahan}
\end{minipage}

Algunos conceptos y definiciones:

\begin{itemize}
    \item

          \textbf{Redondeo} es un procedimiento para determinar la
          contraparte en punto flotante
          \begin{math}
              \operatorname{fl}
              \left(r\right)
          \end{math}
          de un número real $r$.
          Un enfoque alternativo se denomina ``corte''.

    \item

          \textbf{Error por redondeo} es
          \begin{math}
              r-
              \operatorname{fl}
              \left(r\right)
          \end{math}.

    \item

          \textbf{Épsilon de la máquina} $\epsilon_{M}$ es el número
          de punto flotante más pequeño tal que
          \begin{math}
              1+\epsilon_{M}\neq
              1
          \end{math}
          en la computadora.

    \item

          \textbf{Pérdida de dígitos significativos} es la pérdida de
          precisión debido a la resta de números de punto flotante
          muy próximos entre sí.
\end{itemize}

\subsection{Redondeo}

El redondeo es el método más utilizado para aproximar números reales
en una computadora.
Supongamos que $b=10$.
Dado
\begin{equation*}
    0.d_{1}\dotsc d_{p}d_{p+1}\dotsc
    d_{p+k}\times b^{E}
\end{equation*}
redondeando a $p$ dígitos da
\begin{equation*}
    0.d_{1}\dotsc
    d_{p}d_{p+1}\dotsc
    d_{p+k}\times
    b^{E}=
    \begin{cases}
        0.d_{1}\dotsc
        d_{p}\times
        b^{E}, &
        \text{si }
        0.d_{p+1}\dotsc
        d_{p+k}<
        \frac{1}{2}. \\
        0.d_{1}\dotsc
        \widetilde{d}_{p}\times
        b^{E}, &
        \text{si }
        0.d_{p+1}\dotsc
        d_{p+k}=
        \frac{1}{2}. \\
        0.d_{1}\dotsc
        \widehat{d}_{p}\times
        b^{E}, &
        \text{si }
        0.d_{p+1}\dotsc
        d_{p+k}>
        \frac{1}{2}.
    \end{cases}
\end{equation*}
Donde $\widetilde{d}_{p}$ es el dígito par más cercano a
$d_{p}.d_{p+1}\dotsc d_{p+k}$ y $\widehat{d}_{p}=d_{p}+1$.

\section{Pérdida de dígitos significativos}

\begin{align*}
    \shortintertext{Dados los dos números reales}
    x
     & =0.3721478693.  \\
    y
     & =0.37202300572.
    \shortintertext{Su diferencia es}
    x-y
     & = 0.0001248121.
    \shortintertext{Realizamos \textbf{redondeo} a $5$ dígitos, esto
        nos da}
    \operatorname{fl}\left(x\right)
     & =0.37215.       \\
    \operatorname{fl}\left(y\right)
     & =0.37202.
    \shortintertext{Ahora la diferencia de los dos números de punto
        flotante es}
    \operatorname{fl}\left(x\right)-\operatorname{fl}\left(y\right)
     & =
    0.00013.
\end{align*}
En la memoria podemos almacenar $5$ dígitos para
\begin{math}
    \operatorname{fl}\left(x\right)-
    \operatorname{fl}\left(y\right)
\end{math}
pero realmente solo conocemos dos de ellos, los demás se pierden.

\section{Evite la propagación del error de redondeo}

\textbf{Estabilidad}: estudiar cómo se propaga el error debido a
perturbaciones en los datos iniciales.

\

\begin{minipage}{0.6\paperwidth}
    \begin{itemize}
        \item

              Estabilidad del problema.
    \end{itemize}
\end{minipage}
\hfill
\begin{minipage}{0.3\paperwidth}
    \begin{itemize}
        \item

              Estabilidad del algoritmo.
    \end{itemize}
\end{minipage}

\begin{example}
    Encuentra $x$ tal que $ax+b=c$ donde $a$, $b$, $c$ son números
    dados y $a\neq 0$.

    \

    \begin{minipage}{0.4\paperwidth}
        \begin{description}
            \item[Algoritmo 1]\leavevmode
                  \begin{enumerate}
                      \item

                            Divida por $a$: $x+\dfrac{b}{a}=\dfrac{c}{a}$.

                      \item

                            Reste $\dfrac{b}{a}$: $x=\dfrac{c}{a}-\dfrac{b}{a}$.
                  \end{enumerate}
        \end{description}
    \end{minipage}
    \hfill
    \begin{minipage}{0.4\paperwidth}
        \begin{description}

            \item[Algoritmo 2]\leavevmode

                  \begin{enumerate}
                      \item

                            Reste $b$: $ax=c-b$.

                      \item

                            Divida por $a$: $x=\dfrac{c-b}{a}$.
                  \end{enumerate}
        \end{description}
    \end{minipage}
\end{example}

\begin{description}
    \item[Estabilidad del problema] responde a la pregunta:
          ¿Qué sucede con la solución de $ax+b$ si $a\to a\left(1+\delta_{a}\right)$,
          $b\to b\left(1+\delta_{b}\right)$, $c\to c\left(1+\delta_{c}\right)$?

    \item[Estabilidad del algoritmo] responde a la pregunta:
          ¿Qué sucede con la salida del algoritmo si $a\to a\left(1+\delta_{a}\right)$,
          $b\to b\left(1+\delta_{b}\right)$, $c\to c\left(1+\delta_{c}\right)$?
\end{description}

\section{Números de estabilidad y condición}

Un problema es estable cuando el error relativo en la solución de
salida es del mismo tamaño que el error relativo en los datos de
entrada.
Dado un problema estable, solo si elegimos un algoritmo estable para
resolverlo obtenemos errores en la salida que son del mismo tamaño
que los errores en la entrada.

\begin{definition}
    Los números de condición son constantes utilizadas para limitar
    la amplificación del error relativo en la salida por medio del
    error relativo en la entrada.
\end{definition}

Los números de condición son útiles para cuantificar la estabilidad
de un problema así como la estabilidad de un algoritmo.

\begin{example}[Estabilidad de la operación aritmética ``$+$'']
    Sean los números reales $x>0$ e $y>0$.
    Sean
    \begin{math}
        \operatorname{fl}
        \left(x\right)=
        x\left(1+\delta_{x}\right)
    \end{math},
    \begin{math}
        \operatorname{fl}
        \left(y\right)=
        y\left(1+\delta_{y}\right)
    \end{math}
    con
    \begin{math}
        \left|\delta_{x}\right|\leq
        \epsilon
    \end{math}
    y
    \begin{math}
        \left|\delta_{y}\right|\leq
        \epsilon
    \end{math}.
    Mire el error relativo:
    \begin{align*}
        \left|
        \frac{
            x+y-
            \left(
            \operatorname{fl}
            \left(x\right)+
            \operatorname{fl}
            \left(y\right)
            \right)
        }{x+y}
        \right| & =
        \left|
        \frac{
            x+y-
            \left(
            x+x\delta_{x}+y+y\delta_{y}
            \right)
        }{x+y}
        \right|     \\
                & =
        \left|
        -\frac{x}{x+y}\delta_{x}-
        \frac{y}{x+y}\delta_{y}
        \right|\leq
        C\overline{\delta}.
    \end{align*}
    donde
    \begin{math}
        \overline{\delta}=
        \max
        \left\{
        \frac{x}{x+y},\frac{y}{x+y}
        \right\}
    \end{math}
    y
    \begin{math}
        \overline{\delta}=
        2\max
        \left\{
        \left|\delta_{x}\right|,
        \left|\delta_{y}\right|
        \right\}\leq
        2\epsilon
    \end{math}
    La suma de números positivos es una operación estable.
    $C$ es el número de condición.
\end{example}
\end{document}