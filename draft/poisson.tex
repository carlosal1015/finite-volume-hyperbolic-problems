\section{La ecuación de Poisson unidimensional}

% Página 29. Elliptic Differential Equations. Wolfgang Hackbusch
\begin{frame}
	\frametitle{\secname}

	Sea $\Omega=\left(0,1\right)\subset\mathbb{R}$.
	Encuentre una solución
	\begin{math}
		u\in
		C^{2}\left(\Omega\right)\cap
		C^{0}\left(\overline{\Omega}\right)
	\end{math}
	que satisfaga el problema elíptico con condiciones de frontera
	Dirichlet homogénea~\eqref{eq:PoissonBVP}, donde
	\begin{math}
		f\in
		C^{0}\left(\overline{\Omega}\right)
	\end{math}.
	\begin{equation}\label{eq:PoissonBVP}
		\begin{cases}
			-\difc.L.{}{}u=f &
			\text{en }\Omega.  \\
			u=0              &
			\text{en }\partial\Omega.
		\end{cases}
	\end{equation}
	Si integra la EDP~\eqref{eq:PoissonBVP} en $\Omega_{j}$ y emplea el
	Teorema Fundamental del Cálculo, obtenga
	\begin{align}
		\forall j=1,\dotsc,N:
		-\diff{u}{x}{\bigg|}_{x_{j+\tfrac{1}{2}}}+
		\diff{u}{x}{\bigg|}_{x_{j-\tfrac{1}{2}}} & =
		-\int_{\Omega_{j}}
		\diff[2]{u\left(x\right)}{x}\dl x=
		\int_{\Omega_{j}}
		f\left(x\right)\dl x.\label{eq:PoissonBVPIntegral}
		\shortintertext{Aproxime las derivadas por}
		\diff{u}{x}{\bigg|}_{x_{\tfrac{1}{2}}}   & \approx
		\frac{1}{h_{\tfrac{1}{2}}}
		\left[
			\frac{1}{h_{1}}
			\int_{\Omega_{1}}
			u\left(x\right)\dl x
		\right].                                           \\
		\forall j=1,\dotsc,N-1:
		\diff{u}{x}{\bigg|}_{x_{j+\tfrac{1}{2}}} & \approx
		\frac{1}{h_{j+\tfrac{1}{2}}}
		\left[
			\frac{1}{h_{j+1}}
			\int_{\Omega_{j+1}}
			u\left(x\right)\dl x-
			\frac{1}{h_{j}}
			\int_{\Omega_{j}}
			u\left(x\right)\dl x
		\right].                                           \\
		\diff{u}{x}{\bigg|}_{x_{N+\tfrac{1}{2}}} & \approx
		-\frac{1}{h_{N+\tfrac{1}{2}}}
		\left[
			\frac{1}{h_{N}}
			\int_{\Omega_{N}}
			u\left(x\right)\dl x
			\right].
	\end{align}
\end{frame}

\begin{frame}
	\frametitle{\secname}

	Si $x_{0}=-x_{1}$ y $X_{N+1}=2-X_{N}$, entonces $h_{0}=h_{1}$ y $h_{N+1}=h_{N}$.
	Sea $h=\max\limits_{0\leq j\leq N}h_{j}$ y dado que
	$u\in C^{2}\left(\Omega\right)$, por el teorema de Taylor se tiene
	\begin{align*}
		\forall j=0,\dotsc,N:
		u\big(x_{j+\tfrac{1}{2}}\big)+
		\big(x-x_{j+\tfrac{1}{2}}\big)
		\diff{u}{x}{\bigg|}_{x_{j+\frac{1}{2}}}+
		\mathcal{O}\left(h^{2}\right) & =
		u\left(x\right).
		\shortintertext{Para $j=0$, la condición de frontera Dirichlet
			$u\big(x_{\tfrac{1}{2}}\big)=0$.}
		u\big(x_{\tfrac{1}{2}}\big)+
		\frac{h_{1}}{2}
		\diff{u}{x}{\bigg|}_{x_{\tfrac{1}{2}}}+
		\mathcal{O}\left(h^{2}\right) & =
		\frac{1}{h_{1}}
		\int_{\Omega_{1}}
		u\left(x\right)\dl x.
		\shortintertext{Por lo tanto,}
		\left|
		\frac{1}{h_{\tfrac{1}{2}}}
		\left[
			\frac{1}{h_{1}}
			\int_{\Omega_{1}}
			u\left(x\right)\dl x
			\right]-
		\diff{u}{x}{\bigg|}_{x_{\tfrac{1}{2}}}
		\right|                       & \leq
		Ch.
		\shortintertext{Para $j=1,\dotsc,N-1$, en las celdas $\Omega_{j}$ y
			$\Omega_{j+1}$.}
		u\big(x_{j+\tfrac{1}{2}}\big)-
		\frac{h_{j}}{2}
		\diff{u}{x}{\bigg|}_{x_{j+\tfrac{1}{2}}}+
		\mathcal{O}\left(h^{2}\right) & =
		\frac{1}{h_{j}}
		\int_{\Omega_{j}}
		u\left(x\right)\dl x.                \\
		u\big(x_{j+\tfrac{1}{2}}\big)+
		\frac{h_{j+1}}{2}
		\diff{u}{x}{\bigg|}_{x_{j+\tfrac{1}{2}}}+
		\mathcal{O}\left(h^{2}\right) & =
		\frac{1}{h_{j+1}}
		\int_{\Omega_{j+1}}
		u\left(x\right)\dl x.
		\shortintertext{Por lo tanto,}
		\left|
		\frac{1}{h_{j+\tfrac{1}{2}}}
		\left[
			\frac{1}{h_{j+1}}
			\int_{\Omega_{j+1}}
			u\left(x\right)\dl x-
			\frac{1}{h_{j}}
			\int_{\Omega_{j}}
			u\left(x\right)\dl x
			\right]
		-\diff{u}{x}{\bigg|}_{x_{j+\tfrac{1}{2}}}
		\right|                       & \leq
		Ch.
		\shortintertext{Para $j=N$, la condición de frontera Dirichlet
			$u\big(x_{N+\tfrac{1}{2}}\big)=0$.}
		u\big(x_{N+\tfrac{1}{2}}\big)+
		\frac{h_{N}}{2}
		\diff{u}{x}{\bigg|}_{x_{N+\tfrac{1}{2}}}+
		\mathcal{O}\left(h^{2}\right) & =
		\frac{1}{h_{N}}
		\int_{\Omega_{N}}
		u\left(x\right)\dl x.
		\shortintertext{Por lo tanto,}
		\left|
		\frac{1}{h_{N+\tfrac{1}{2}}}
		\left[
			\frac{1}{h_{N}}
			\int_{\Omega_{N}}
			u\left(x\right)\dl x
			\right]-
		\diff{u}{x}{\bigg|}_{x_{N+\tfrac{1}{2}}}
		\right|                       & \leq
		Ch.
	\end{align*}
\end{frame}

\begin{frame}
	\frametitle{\secname}

	Para $j=1,\dotsc,N$, reemplace en~\eqref{eq:PoissonBVPIntegral} y
	defina
	\begin{align*}
		F_{\frac{1}{2}}   & \coloneqq
		-\frac{u_{1}}{h_{\frac{1}{2}}}.           \\
		\forall j=2,\dotsc,N-1:
		F_{j+\frac{1}{2}} & \coloneqq
		-\frac{u_{j+1}-u_{j}}{h_{j+\frac{1}{2}}}. \\
		F_{N+\frac{1}{2}} & \coloneqq
		-\frac{u_{N}}{h_{N+\frac{1}{2}}}.         \\
		\forall j=1,\dotsc,N:
		f_{j}             & \coloneqq
		\int_{\Omega_{j}}
		f\left(x\right)\dl x.
	\end{align*}
	Las cantidades $F_{j+\frac{1}{2}}$ son los flujos numéricos que
	aproximan los flujos $-\diff{u}{x}{\big|}_{x_{j+\frac{1}{2}}}$.
	De este modo, obtenga el esquema de volúmenes finitos
	\begin{equation}
		\forall j=1,\dotsc, N:
		F_{j+\frac{1}{2}}-
		F_{j-\frac{1}{2}}=
		f_{j}.
	\end{equation}
	Reescriba línea por línea el esquema de volúmenes finitos,
	\begin{align*}
		\left(
		\frac{1}{h_{\frac{1}{2}}}+\frac{1}{h_{\frac{3}{2}}}
		\right)u_{1}-
		\frac{1}{h_{\frac{3}{2}}}u_{2}     & =
		f_{1}.                                 \\
		\forall j=2,\dotsc,N-1:
		-\frac{1}{h_{j-\frac{1}{2}}}u_{j-1}+
		\left(
		\frac{1}{h_{j-\frac{1}{2}}}+
		\frac{1}{h_{j+\frac{1}{2}}}
		\right)u_{j}-
		\frac{1}{h_{j+\frac{1}{2}}}u_{j+1} & =
		f_{j}.                                 \\
		-\frac{1}{h_{N-\frac{1}{2}}}u_{N-1}+
		\left(
		\frac{1}{h_{N-\frac{1}{2}}}+
		\frac{1}{h_{N+\frac{1}{2}}}
		\right)u_{N}                       & =
		f_{N}.
	\end{align*}
\end{frame}
% Note que
% \begin{align*}
% 	\begin{bmatrix}
% 		\frac{1}{h_{\frac{1}{2}}}+
% 		\frac{1}{h_{\frac{3}{2}}}    &
% 		-\frac{1}{h_{\frac{3}{2}}}   & \cdots                       & 0      & 0               \\
% 		-\frac{1}{h_{\frac{3}{2}}}   &
% 		\frac{1}{h_{\frac{3}{2}}}+
% 		\frac{1}{h_{\frac{5}{2}}}    &
% 		-\frac{1}{h_{\frac{5}{2}}}   & \cdots                       & 0                        \\
% 		\vdots                       & \ddots                       & \ddots & \ddots & \vdots \\
% 		0                            & \cdots                       & \cdots &
% 		-\frac{1}{h_{N-\frac{1}{2}}} & \frac{1}{h_{N-\frac{1}{2}}}+
% 		\frac{1}{h_{N+\frac{1}{2}}}
% 	\end{bmatrix}
% 	\begin{bmatrix}
% 		u_{1}  \\
% 		\vdots \\
% 		\vdots \\
% 		\vdots \\
% 		u_{N}
% 	\end{bmatrix} & =
% 	\begin{bmatrix}
% 		f_{1}  \\
% 		\vdots \\
% 		\vdots \\
% 		\vdots \\
% 		f_{N}
% 	\end{bmatrix}.      \\
% 	A^{h}U^{h}      & =
% 	F^{h}.
% \end{align*}
% El error de truncamiento se encuentra
% \begin{align*}
% 	h_{j}
% 	\varepsilon_{h}
% 	{\left(u\right)}_{j} & =
% 	-\frac{1}{h_{j-\frac{1}{2}}}
% 	u\left(x_{j-1}\right)+
% 	\left(
% 	\frac{1}{h_{j-\frac{1}{2}}}+\frac{1}{h_{j+\frac{1}{2}}}
% 	\right)
% 	u\left(x_{j}\right)-
% 	\frac{1}{h_{j+\frac{1}{2}}}
% 	u\left(x_{j+1}\right)-
% 	\int_{\Omega_{j}}f\left(x\right)\dl x. \\
% 	\varepsilon_{h}
% 	{\left(u\right)}_{j} & =
% 	\left[
% 		1-
% 		\frac{1}{2h_{j}}
% 		\left(
% 		h_{j-\frac{1}{2}}+
% 		h_{j+\frac{1}{2}}
% 		\right)
% 		\right]
% 	\diff[2]{u}{x}{\bigg|}_{x_{j}}+
% 	\mathcal{O}\left(h\right).
% \end{align*}

% La matriz $A_{h}$ es invertible porque es definida positiva.
% \begin{equation*}
% 	\forall x\in\mathbb{R}^{N}\setminus\left\{0\right\}:
% 	x^{T}A_{h}x=
% 	\frac{x^{2}_{1}}{h_{\frac{1}{2}}}+
% 	\sum_{j=1}^{N-1}
% 	\frac{{\left(x_{j+1}-x_{j}\right)}^{2}}{h_{j+\frac{1}{2}}}+
% 	\frac{x^{2}_{N}}{h_{N+\frac{1}{2}}}
% 	>0.
% \end{equation*}

% Si $f\in C^{0}\left(\overline{\Omega}\right)$, entonces $\exists C>0$
% que no depende de $h$ tal que
% \begin{align*}
% 	\max_{1\leq j\leq N}
% 	\left|
% 	u_{j}-
% 	\frac{1}{h_{j}}
% 	\int_{\Omega_{j}}
% 	u\left(x\right)\dl x
% 	\right| & \leq
% 	Ch.            \\
% 	\max_{1\leq j\leq N}
% 	\left|
% 	u_{j}-
% 	u\left(x_{j}\right)
% 	\right| & \leq
% 	Ch.
% \end{align*}

% \section{La ecuación de transporte unidimensional}

% Sea $a\in\mathbb{R}\setminus\left\{0\right\}$.
% \begin{equation}\label{eq:TransportIVP}
% 	\begin{cases}
% 		\diffp{u}{t}+
% 		a\diffp{u}{x}=0                       &
% 		\text{en }\mathbb{R}\times\left(0,\infty\right). \\
% 		u\left(x,0\right)=u_{0}\left(x\right) &
% 		\text{en }\mathbb{R}\times\left\{0\right\}.
% 	\end{cases}
% \end{equation}
% Cubra el espacio $\mathbb{R}$ por celdas
% \begin{math}
% 	\Omega_{j}=
% 	\big[
% 		x_{j-\frac{1}{2}},
% 		x_{j+\frac{1}{2}}
% 		\big]
% \end{math},
% $j\in\mathbb{Z}$ y buscamos una aproximación constante por partes de
% la solución $u$, que sea constante sobre cada celda $\Omega_{N}$, en
% algunos instantes discretizados.
% Defina el valor medio de $u\left(x,t\right)$ en $\Omega_{j}$ como
% \begin{equation*}
% 	\overline{u}_{j}\left(t\right)\coloneqq
% 	\frac{1}{h_{j}}
% 	\int_{\Omega_{j}}
% 	u\left(x,t\right)\dl x.
% \end{equation*}
% Para cada $t\in\left(0,\infty\right)$ y $j\in\mathbb{Z}$,
% $u_{j}\left(t\right)\in\mathbb{R}$ es la aproximación de
% $\overline{u}_{j}\left(t\right)$.
% % Sea $q\left(x,t\right)=au\left(x,t\right)$.
% Primero, integre~\eqref{eq:TransportIVP} en $\Omega_{j}$
% \begin{align*}
% 	\int_{\Omega_{j}}
% 	\left(
% 	\diffp{u}{t}+
% 	a\diffp{u}{x}
% 	\right)
% 	\left(x,t\right)\dl x           & =
% 	\int_{\Omega_{j}}
% 	0\dl x.                             \\
% 	\int_{\Omega_{j}}
% 	\diffp{u}{t}
% 	\left(x,t\right)
% 	\dl x+
% 	a
% 	\int_{\Omega_{j}}
% 	\diffp{u}{x}
% 	\left(x,t\right)\dl x           & =
% 	0.
% 	\shortintertext{Derive el primer término bajo el signo de la
% 		integral y emplee el Teorema Fundamental del Cálculo en el
% 		segundo término.}
% 	\diff{}{t}
% 	\left[
% 		\int_{\Omega_{j}}
% 		u\left(x,t\right)\dl x
% 		\right]+
% 	a\left[
% 		u\big(x_{j+\frac{1}{2}},t\big)-
% 		u\big(x_{j-\frac{1}{2}},t\big)
% 	\right]                         & =
% 	0.                                  \\
% 	h_{j}
% 	\diff{\overline{u}_{j}\left(t\right)}{t}+
% 	au\big(x_{j+\frac{1}{2}},t\big)-
% 	au\big(x_{j-\frac{1}{2}},t\big) & =
% 	0.
% \end{align*}
% Discretizando
% \begin{align*}
% 	h_{j}
% 	\frac{u^{n+1}_{j}-u^{n}_{j}}{k}+
% 	g\left(u^{n}_{j},u^{n}_{j+1}\right)-
% 	g\left(u^{n}_{j-1},u^{n}_{j}\right) & =
% 	0.                                      \\
% 	\frac{1}{h_{j}}
% 	\int_{\Omega_{j}}
% 	u_{0}\left(x\right)\dl x
% 	\overline{u}_{j}\left(0\right)      & =
% 	u^{0}_{j}.
% \end{align*}
% %Busque una aproximación de
% % La condición inicial es dada por

% Si $u$ es solución de~\eqref{eq:TransportIVP}

% \begin{equation*}
% 	\sum_{l=-1}^{1}
% 	c_{l}v_{l}.
% \end{equation*}

% \subsection{Ejemplos de esquemas lineales}

% Sea $h_{j}=h$, $\lambda=\frac{k}{h}\in\mathbb{R}$ y
% \begin{math}
% 	u^{0}_{j}=
% 	\frac{1}{h}
% 	\int_{\Omega_{j}}
% 	u_{0}\left(x\right)\dl x
% \end{math}.

% \begin{equation*}
% 	u^{n+1}_{j}=
% 	u^{n}_{j}-
% 	\lambda
% 	\left[
% 		a^{-}\left(u^{n}_{j+1}-u^{n}_{j}\right)+
% 		a^{+}\left(u^{n}_{j}-u^{n}_{j-1}\right)
% 		\right].
% \end{equation*}
% Donde $a^{-}=\min\left\{a,0\right\}$ si $a<0$ y $a^{+}=\max\left\{a,0\right\}$ si $a>0$.

% \begin{equation*}
% 	g\left(u,v\right)=
% 	a^{+}u+a^{-}v.
% \end{equation*}

% \begin{equation*}
% 	u^{n+1}_{j}=
% 	u^{n}_{j}-
% 	\frac{\lambda a}{2}
% 	\left(u^{n}_{j+1}-u^{n}_{j-1}\right)+
% 	\frac{\lambda\left|a\right|}{2}
% 	\left(u^{n}_{j+1}-2u^{n}_{j}+u^{n}_{j-1}\right).
% \end{equation*}

% \begin{equation*}
% 	u^{n+1}_{j}=
% 	u^{n}_{j}-
% 	\frac{\lambda a}{2}
% 	\left(u^{n}_{j+1}-u^{n}_{j-1}\right)
% \end{equation*}

% \begin{equation*}
% 	g\left(u,v\right)=
% 	\frac{a}{2}\left(u+v\right).
% \end{equation*}

% \begin{equation*}
% 	u^{n+1}_{j}=
% 	\frac{u^{n}_{j+1}+u^{n}_{j-1}}{2}-
% 	\frac{\lambda a}{2}
% 	\left(u^{n}_{j+1}-u^{n}_{j-1}\right).
% \end{equation*}

% \begin{equation*}
% 	g\left(u,v\right)=
% 	\frac{a}{2}\left(u+v\right)-
% 	\frac{1}{2\lambda}\left(v-u\right).
% \end{equation*}

% \begin{equation*}
% 	u^{n+1}_{j}=
% 	u^{n}_{j}-
% 	\frac{\lambda a}{2}
% 	\left(u^{n}_{j+1}-u^{n}_{j-1}\right)+
% 	\frac{\lambda^{2}a^{2}}{2}
% 	\left(u^{n}_{j+1}-2u^{n}_{j}+u^{n}_{j-1}\right)
% \end{equation*}

% \begin{equation*}
% 	g\left(u,v\right)=
% 	\frac{a}{2}\left(u+v\right)-
% 	\lambda\frac{a^{2}}{2}\left(v-u\right).
% \end{equation*}

% \section{Ley de conservación unidimensional}

% \begin{equation*}
% 	\diffp{u}{t}+
% 	\diffp{}{x}
% 	\left[f\left(u\right)\right]
% \end{equation*}

% \section{Ecuación elíptica bidimensional}

% Sea $\Omega\subset\mathbb{R}^{2}$.

% \begin{equation*}
% 	\begin{cases}
% 		-\nabla\cdot\nabla u+
% 		\nabla\cdot\left(bu\right)+
% 		cu=f & \text{en }\Omega. \\
% 		u\big|_{\partial\Omega}=0.
% 	\end{cases}
% \end{equation*}

% $f\in L^{2}\left(\Omega\right)$, $b\in C^{1}\left(\Omega\right)$.
% $\forall\left(x,y\right)\in\Omega:c\left(x,y\right)\geq 0$.
% $\forall\left(x,y\right)\in\Omega:\nabla\cdot b\left(x,y\right)\geq 0$.

% \appendix

% \section{Cálculo}

% Teorema de Taylor unidimensional

% Sea $n\in\mathbb{N}\cup\left\{0\right\}$.
% Si
% \begin{math}
% 	u\in
% 	C^{n+1}\left(\left[a,b\right],\mathbb{R}\right)
% \end{math}
% y $\forall \left\{x,x_{0}\right\}\subset\left[a,b\right]$,
% entonces
% \begin{equation*}
% 	u\left(x\right)=
% 	\sum_{k=0}^{n}
% 	\left[
% 	\frac{{\left(x-x_{0}\right)}^{k}}{k!}
% 	\diff[k]{u}{x}{\bigg|}_{x_{0}}
% 	\right]+
% 	\int_{x_{0}}^{x}
% 	\frac{{\left(x-t\right)}^{n}}{n!}
% 	\diff[n+1]{u}{x}{\bigg|}_{t}\dl t.
% \end{equation*}

% Forma de Lagrange del resto

% $\exists\xi\in\left(x,x_{0}\right)$

% \begin{equation*}
% 	u\left(x\right)=
% 	\sum_{k=0}^{n}
% 	\frac{{\left(x-x_{0}\right)}^{k}}{k!}
% 	\diff[k]{u}{x}{\bigg|}_{x_{0}}+
% 	\frac{{\left(x-x_{0}\right)}^{n+1}}{\left(n+1\right)!}
% 	\diff[n+1]{u}{x}{\bigg|}_{\xi}.
% \end{equation*}

% \nocite{*}
% \printbibliography[title={Referencias},heading=bibintoc]
