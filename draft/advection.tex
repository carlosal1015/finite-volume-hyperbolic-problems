\section{La ecuación de transporte unidimensional}

\begin{frame}
	\frametitle{\secname}

	Sea
	\begin{math}
		a\in
		\mathbb{R}\setminus\left\{0\right\}
	\end{math}.
	\begin{equation}\label{eq:TransportIVP}
		\begin{cases}
			\diffp{u}{t}+
			a\diffp{u}{x}=0     &
			\text{en }\mathbb{R}\times
			\left(0,\infty\right). \\
			u\left(x,0\right)=
			u_{0}\left(x\right) &
			\text{en }\mathbb{R}\times
			\left\{0\right\}.
		\end{cases}
	\end{equation}
	Cubra el espacio $\mathbb{R}$ por celdas
	\begin{math}
		\Omega_{j}=
		\big[
			x_{j-\frac{1}{2}},
			x_{j+\frac{1}{2}}
			\big]
	\end{math},
	$j\in\mathbb{Z}$ y buscamos una aproximación constante por partes
	de la solución $u$, que sea constante sobre cada celda
	$\Omega_{N}$, en algunos instantes discretizados.
	Defina el valor medio de $u\left(x,t\right)$ en $\Omega_{j}$ como
	\begin{equation*}
		\overline{u}_{j}\left(t\right)\coloneqq
		\frac{1}{h_{j}}
		\int_{\Omega_{j}}
		u\left(x,t\right)\dl x.
	\end{equation*}
	Para cada $t\in\left(0,\infty\right)$ y $j\in\mathbb{Z}$,
	$u_{j}\left(t\right)\in\mathbb{R}$ es la aproximación de
	$\overline{u}_{j}\left(t\right)$.
	% Sea $q\left(x,t\right)=au\left(x,t\right)$.
	Primero, integre~\eqref{eq:TransportIVP} en $\Omega_{j}$
	\begin{align*}
		\int_{\Omega_{j}}
		\left[
			\diffp{u\left(x,t\right)}{t}+
			a\diffp{u\left(x,t\right)}{x}
			\right]
		\dl x                           & =
		\int_{\Omega_{j}}
		0\dl x.                             \\
		\int_{\Omega_{j}}
		\diffp{u\left(x,t\right)}{t}
		\dl x+
		a
		\int_{\Omega_{j}}
		\diffp{u\left(x,t\right)}{x}
		\dl x                           & =
		0.
		\shortintertext{Derive el primer término bajo el signo de la
			integral y emplee el Teorema Fundamental del Cálculo en el
			segundo término.}
		\diff{}{t}
		\left[
			\int_{\Omega_{j}}
			u\left(x,t\right)\dl x
			\right]+
		a\left[
			u\big(x_{j+\frac{1}{2}},t\big)-
			u\big(x_{j-\frac{1}{2}},t\big)
		\right]                         & =
		0.                                  \\
		h_{j}
		\diff{\overline{u}_{j}\left(t\right)}{t}+
		au\big(x_{j+\frac{1}{2}},t\big)-
		au\big(x_{j-\frac{1}{2}},t\big) & =
		0.
	\end{align*}
\end{frame}


\begin{frame}
	\frametitle{\secname}

	Discretizando
	\begin{align*}
		h_{j}
		\frac{u^{n+1}_{j}-u^{n}_{j}}{k}+
		g\left(u^{n}_{j},u^{n}_{j+1}\right)-
		g\left(u^{n}_{j-1},u^{n}_{j}\right) & =
		0.                                      \\
		\frac{1}{h_{j}}
		\int_{\Omega_{j}}
		u_{0}\left(x\right)\dl x
		\overline{u}_{j}\left(0\right)      & =
		u^{0}_{j}.
	\end{align*}
	%Busque una aproximación de
	% La condición inicial es dada por
	Si $u$ es solución de~\eqref{eq:TransportIVP}
	\begin{equation*}
		\sum_{l=-1}^{1}
		c_{l}v_{l}.
	\end{equation*}
\end{frame}


\subsection{Ejemplos de esquemas lineales}

\begin{frame}

	Sea $h_{j}=h$, $\lambda=\frac{k}{h}\in\mathbb{R}$ y
	\begin{math}
		u^{0}_{j}=
		\frac{1}{h}
		\int_{\Omega_{j}}
		u_{0}\left(x\right)\dl x
	\end{math}.

	\begin{equation*}
		u^{n+1}_{j}=
		u^{n}_{j}-
		\lambda
		\left[
			a^{-}\left(u^{n}_{j+1}-u^{n}_{j}\right)+
			a^{+}\left(u^{n}_{j}-u^{n}_{j-1}\right)
			\right].
	\end{equation*}
	Donde $a^{-}=\min\left\{a,0\right\}$ si $a<0$ y $a^{+}=\max\left\{a,0\right\}$ si $a>0$.

	\begin{equation*}
		g\left(u,v\right)=
		a^{+}u+a^{-}v.
	\end{equation*}

	\begin{equation*}
		u^{n+1}_{j}=
		u^{n}_{j}-
		\frac{\lambda a}{2}
		\left(u^{n}_{j+1}-u^{n}_{j-1}\right)+
		\frac{\lambda\left|a\right|}{2}
		\left(u^{n}_{j+1}-2u^{n}_{j}+u^{n}_{j-1}\right).
	\end{equation*}

	\begin{equation*}
		u^{n+1}_{j}=
		u^{n}_{j}-
		\frac{\lambda a}{2}
		\left(u^{n}_{j+1}-u^{n}_{j-1}\right)
	\end{equation*}

	\begin{equation*}
		g\left(u,v\right)=
		\frac{a}{2}\left(u+v\right).
	\end{equation*}

	\begin{equation*}
		u^{n+1}_{j}=
		\frac{u^{n}_{j+1}+u^{n}_{j-1}}{2}-
		\frac{\lambda a}{2}
		\left(u^{n}_{j+1}-u^{n}_{j-1}\right).
	\end{equation*}

	\begin{equation*}
		g\left(u,v\right)=
		\frac{a}{2}\left(u+v\right)-
		\frac{1}{2\lambda}\left(v-u\right).
	\end{equation*}

	\begin{equation*}
		u^{n+1}_{j}=
		u^{n}_{j}-
		\frac{\lambda a}{2}
		\left(u^{n}_{j+1}-u^{n}_{j-1}\right)+
		\frac{\lambda^{2}a^{2}}{2}
		\left(u^{n}_{j+1}-2u^{n}_{j}+u^{n}_{j-1}\right)
	\end{equation*}

	\begin{equation*}
		g\left(u,v\right)=
		\frac{a}{2}\left(u+v\right)-
		\lambda\frac{a^{2}}{2}\left(v-u\right).
	\end{equation*}
\end{frame}
