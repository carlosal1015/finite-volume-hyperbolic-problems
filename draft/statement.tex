\section{Planteamiento del problema}
% Objetivo

\begin{frame}
	\frametitle{\secname}

	\begin{block}{Objeto de estudio}
		El objetivo general es resolver~\eqref{eq:conservationlaw}
		sujeto a una condición inicial.
		Comparamos métodos de volúmenes basados en flujos limitadores
		que satisfacen la propiedad \alert{Variación Total Decreciente}.
	\end{block}

	\begin{definition}[Variación Total Decreciente]
		\begin{equation*}
			\operatorname{TV}\left(U^{n+1}\right)\leq
			\operatorname{TV}\left(U^{n}\right).
		\end{equation*}
		\begin{align*}
			\operatorname{TV}\left(U\right) & =
			\sum_{j=-\infty}^{\infty}
			\left|
			U_{j}-U_{j-1}
			\right|.                            \\
			\operatorname{TV}\left(u\right) & =
			\sup\sum_{j=1}^{N}
			\left|
			u\left(\xi_{j}\right)-
			u\left(\xi_{j-1}\right)
			\right|.
		\end{align*}
	\end{definition}

	\begin{definition}[Flujo limitador]
		Es toda aquella función que satisface la propiedad de simetría.
		\begin{equation}
			\forall\theta\in
			\left(0,\theta_{\max}\right):
			\frac{\phi\left(\theta\right)}{\theta}=
			\phi
			\left(\frac{1}{\theta}\right).\tag{Simetría}
		\end{equation}
	\end{definition}

	\begin{examples}[Flujo limitadoras]
		\begin{columns}
			\begin{column}{.48\paperwidth}
				\begin{itemize}
					\item

					      \begin{math}
						      \phi_{\text{vl}}
						      \left(\theta\right)=
						      \frac{\theta+\left|\theta\right|}{1+\left|\theta\right|}
					      \end{math}.

					      \

					\item

					      \begin{math}
						      \phi_{\text{va1}}
						      \left(\theta\right)=
						      \frac{\theta^{2}+\theta}{\theta^{2}+1}
					      \end{math}.

					      \

					\item

					      \begin{math}
						      \phi_{\text{va2}}
						      \left(\theta\right)=
						      \frac{2\theta}{\theta^{2}+1}
					      \end{math}.

					      \

					\item

					      \begin{math}
						      \phi_{\text{sw}}
						      \left(\theta\right)=
						      \max
						      \left\{
						      0,
						      \min
						      \left\{1,2\theta\right\},
						      \min
						      \left\{2,\theta\right\}
						      \right\}
					      \end{math}.
				\end{itemize}
			\end{column}
			\begin{column}{.48\paperwidth}
				\begin{itemize}
					\item

					      \begin{math}
						      \phi_{\text{\text{sb}}}
						      \left(\theta\right)=
						      \max
						      \left(
						      0,
						      \min\left(1,2\theta\right),
						      \min\left(2,\theta\right)
						      \right)
					      \end{math}.

					      \

					\item

					      \begin{math}
						      \phi_{\text{\text{mc}}}
						      \left(\theta\right)=
						      \max
						      \left\{
						      0,
						      \min
						      \left\{
						      \frac{1+\theta}{2},2\theta,2
						      \right\}
						      \right\}
					      \end{math}.

					      \

					\item

					      \begin{math}
						      \phi_{\text{\text{mm}}}
						      \left(\theta\right)=
						      \operatorname{minmod}
						      \left(1,\theta\right)=
						      \max
						      \left(
						      0,
						      \min
						      \left(
							      1,\theta
							      \right)
						      \right)
					      \end{math}.

					      \

					\item

					      \begin{math}
						      \phi_{\text{\text{kn}}}
						      \left(\theta\right)=
						      \max
						      \left\{
						      0,
						      \min
						      \left\{
						      2\theta,
						      \min
						      \left\{
						      2,
						      \frac{1+2\theta}{3}
						      \right\}
						      \right\}
						      \right\}
					      \end{math}.
				\end{itemize}
			\end{column}
		\end{columns}
	\end{examples}
\end{frame}
