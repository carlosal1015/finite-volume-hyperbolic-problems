\section{Conservative Methods for Nonlinear Problems}

\begin{frame}
    \frametitle{\secname}

    \begin{example}[Burgers equation in conservative form]
        \begin{equation}\tag{12.1}\label{eq:12.1}
            u_{t}+
            {\bigg(\frac{u^{2}}{2}\bigg)}_{x}=
            0.
        \end{equation}
    \end{example}

    \begin{example}
        If we write Burgers equation~\eqref{eq:12.1} in the
        quasilinear form
        \begin{equation}\tag{12.3}\label{eq:12.3}
            u_{t}+
            uu_{x}=
            0
        \end{equation}
        then a natural finite difference method, obtained by a minor
        modification of the upwind method for
        \begin{math}
            u_{t}+
            au_{x}=
            0
        \end{math}
        (and assuming $U^{n}_{j}\geq 0$ for all $j$, $n$) is

        \begin{equation}\tag{12.4}\label{eq:12.4}
            U^{n+1}_{j}=
            U^{n}_{j}-
            \frac{k}{h}
            U^{n}_{j}
            \big(
            U^{n}_{j}-
            U^{n}_{j-1}
            \big).
        \end{equation}
    \end{example}

    \begin{definition}
        The method be in \textbf{conservative form} has the form
        \begin{equation}\tag{12.6}\label{eq:12.6}
            U^{n+1}_{j}=
            U^{n}_{j}-
            \frac{k}{h}
            \left[
                F
                \big(
                U^{n}_{j-p},
                U^{n}_{j-p+1},
                \dotsc,
                U^{n}_{j+q},
                \big)-
                F
                \big(
                U^{n}_{j-p-1},
                U^{n}_{j-p},
                \dotsc,
                U^{n}_{j+q-1}
                \big)
                \right]
        \end{equation}

        for some function $F$ of $p+q+1$ arguments.
        $F$ is called the \textbf{numerical flux function}.
    \end{definition}
\end{frame}

\begin{frame}
    \frametitle{\secname}

    \begin{alertblock}{Remark}
        In the simplest case, $p=0$ and $q=1$ so that $F$ is a
        function of only two variables and~\eqref{eq:12.6} becomes
        \begin{equation}\tag{12.7}\label{eq:12.7}
            U^{n+1}_{j}=
            U^{n}_{j}-
            \frac{k}{h}
            \left[
                F
                \big(
                U^{n}_{j},
                U^{n}_{j+1}
                \big)-
                F
                \big(
                U^{n}_{j-1},
                U^{n}_{j}
                \big)
                \right].
        \end{equation}
    \end{alertblock}

    One way to derive numerical methods in conservation form is to
    use standard finite difference discretizations but to start with
    the conservative form of the PDE rather than the quasilinear
    form~\eqref{eq:12.1} rather than~\eqref{eq:12.3}, we obtain
    \begin{equation}\tag{12.11}\label{eq:12.11}
        U^{n+1}_{j}=
        U^{n}_{j}-
        \frac{k}{h}
        \left[
            \frac{1}{2}{\big(U^{n}_{j}\big)}^{2}-
            \frac{1}{2}{\big(U^{n}_{j-1}\big)}^{2}
            \right].
    \end{equation}

    This is of the form~\eqref{eq:12.7} with
    \begin{equation}\tag{12.12}\label{eq:12.12}
        F\left(v,w\right)=
        \frac{v^{2}}{2}.
    \end{equation}

    Here we again assume that $U^{n}_{j}\geq 0$ for all $j$, $n$, so
    that the ``upwind'' direction is always to the left.
    More generally, for a nonlinear system
    \begin{math}
        u_{t}+
        f\left(u\right)_{x}=
        0
    \end{math}
    for which the Jacobian matrix
    \begin{math}
        f^{\prime}
        \big(U^{n}_{j}\big)
    \end{math}
    has nonnegative eigenvalues
    for all $U^{n}_{j}$, the upwind method is of the
    form~\eqref{eq:12.7} with
    \begin{equation}\tag{12.13}\label{eq:12.13}
        F
        \left(v,w\right)=
        f
        \left(v\right).
    \end{equation}

    For now we simply note that if
    \begin{math}
        f^{\prime}
        \big(U^{n}_{j}\big)
    \end{math}
    has only nonpositive eigenvalues for all $U^{n}_{j}$, then the
    upwind method always uses the point to the right, and the flux
    becomes
    \begin{equation}\tag{12.14}\label{eq:12.14}
        F\left(v,w\right)=
        f\left(w\right).
    \end{equation}
\end{frame}

\section{The entropy condition}

\begin{frame}
    \frametitle{\secname}

    \begin{example}
        Consider Burgers equation with data
        \begin{equation}\tag{12.49}
            u_{0}
            \left(x\right)=
            \begin{cases}
                -1, & x<0, \\
                +1, & x>0.
            \end{cases}
        \end{equation}
        which we might naturally discretize by setting
        \begin{equation}\tag{12.50}
            U^{0}_{j}=
            \begin{cases}
                -1, & j\leq 0, \\
                +1, & j>0.
            \end{cases}
        \end{equation}
        The entropy satisfying weak solution consists of a
        rarefaction wave, but the stationary discontinuity
        \begin{math}
            u
            \left(x,t\right)=
            u_{0}
            \left(x\right)
        \end{math}
        (for all $x$ and $t$) is also a weak solution.
        The Rankine-Hugoniot condition with $s=0$ is satisfied since
        \begin{math}
            f\left(-1\right)=
            f\left(1\right)
        \end{math}
        for Burgers equation.
        There are very natural conservative methods that converge to
        this latter solution rather than to the physically correct
        rarefaction wave.
    \end{example}

    One example is a natural generalization of the upwind
    methods~\eqref{eq:12.13} and~\eqref{eq:12.14} given by
    \begin{equation}\tag{12.51}\label{eq:12.51}
        F\left(v,w\right)=
        \begin{cases}
            f\left(v\right), &
            \text{if }\dfrac{f\left(v\right)-f\left(w\right)}{v-w}\geq 0, \\
            f\left(w\right), &
            \text{if }\dfrac{f\left(v\right)-f\left(w\right)}{v-w}< 0.
        \end{cases}
    \end{equation}
\end{frame}