\section{Lax-Wendroff and MacCormack methods}

\begin{frame}
  \frametitle{\secname}

  Recall the Lax-Wendroff method for a constant coefficient linear
  hyperbolic system
  \begin{math}
    u_{t}+
    Au_{x}=
    0
  \end{math}
  has the form
  \begin{equation}\tag{12.24}\label{eq:12.24}
    U^{n+1}_{j}=
    U_{j}^{n}-
    \frac{k}{2h}
    A
    \big(
    U^{n}_{j+1}-
    U^{n}_{j-1}
    \big)+
    \frac{k^{2}}{2h^{2}}
    A^{2}
    \big(
    U^{n}_{j+1}-
    2U^{n}_{j}+
    U^{n}_{j-1}
    \big).
  \end{equation}

  There are various ways that this can be extended to give a second
  order method for nonlinear conservation laws.
  If we let
  \begin{math}
    A\left(u\right)=
    f^{\prime}
    \left(u\right)
  \end{math}
  be the Jacobian matrix, then a conservative form of Lax-Wendroff is
  \begin{equation}\tag{12.25}\label{eq:12.25}
    \begin{split}
      U^{n+1}_{j}=
      U^{n}_{j}-
      \frac{k}{2h}
      \left[
        f
        \big(U^{n}_{j+1}\big)-
        f
        \big(U^{n}_{j-1}\big)
      \right]+ \\
      \frac{k^{2}}{2h^{2}}
      \left[
        A_{j+\frac{1}{2}}
        \big(
        f
        \big(U^{n}_{j+1}\big)-
        f
        \big(U^{n}_{j}\big)
        \big)-
        A_{j-\frac{1}{2}}
        \big(
        f
        \big(U^{n}_{j}\big)-
        f
        \big(U^{n}_{j-1}\big)
        \big)
        \right],
    \end{split}
  \end{equation}

  where $A_{j\pm\frac{1}{2}}$ is the Jacobian matrix evaluated at
  \begin{math}
    \frac{1}{2}
    \big(
    U_{j}^{n}+
    U_{j\pm 1}^{n}
    \big)
  \end{math}.
  The difficulty with this form is that it requires evaluating the
  Jacobian matrix, and is more expensive to use than other forms that
  only use the function $f\left(u\right)$.

  One way to avoid using $A$ is to use a two-step procedure.
  This was first proposed by Richtmyer, and the
  \textbf{Richtmyer two-step Lax-Wendroff method} is

  \begin{align}\tag{12.26}\label{eq:12.26}
    \textcolor{DarkRed}{
      U_{j+\frac{1}{2}}^{n+\frac{1}{2}}
    }           & =
    \frac{1}{2}
    \big(
    U^{n}_{j}+
    U^{n}_{j+1}
    \big)-
    \frac{k}{2h}
    \left[
      f
      \big(U^{n}_{j+1}\big)-
      f
      \big(U^{n}_{j}\big)
    \right].        \\
    \notag
    U^{n+1}_{j} & =
    U_{j}^{n}-
    \frac{k}{h}
    \left[
      f
      \Big(
      \textcolor{DarkRed}{U_{j+\frac{1}{2}}^{n+\frac{1}{2}}}
      \Big)-
      f
      \Big(
      U_{j-\frac{1}{2}}^{n+\frac{1}{2}}
      \Big)
      \right].
  \end{align}
\end{frame}

\begin{frame}
  \frametitle{\secname}

  Another method of this same type was proposed by
  MacCormack\footnote{\url{https://sci-hub.se/10.2514/6.1969-354}}.
  MacCormack's method uses first forward differencing and then
  backward differencing to achieve second order accuracy:

  \begin{align}\tag{12.27}\label{eq:12.27}
    \textcolor{DarkRed}{U_{j}^{\ast}} & =
    U_{j}^{n}-
    \frac{k}{h}
    \left[
      f\big(U^{n}_{j+1}\big)-
      f\big(U^{n}_{j}\big)
    \right].                              \\
    \notag
    U^{n+1}_{j}                       & =
    \frac{1}{2}
    \big(
    U^{n}_{j}+
    \textcolor{DarkRed}{U_{j}^{\ast}}
    \big)-
    \frac{k}{2h}
    \left[
      f
      \big(\textcolor{DarkRed}{U_{j}^{\ast}}\big)-
      f
      \big(U^{\ast}_{j-1}\big)
      \right].
  \end{align}

  Alternatively, we could use backward differencing in the first step
  and then forward differencing in the second step.
\end{frame}