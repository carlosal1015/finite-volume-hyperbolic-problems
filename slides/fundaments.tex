\section{Fundamentos básicos}

\begin{frame}
    \frametitle{\secname}

    \begin{definition}[Ley de conservación~\cite{Vázquez2015}]
        Sean
        \begin{math}
            \Omega\subset
            \mathbb{R}
        \end{math}
        y
        \begin{math}
            f\colon\Omega\to
            \mathbb{R}
        \end{math}
        una función de clase
        \begin{math}
            C^{1}
            \left(\Omega\right)
        \end{math}.
        Una \textbf{ley de conservación unidimensional} es

        \begin{equation}\label{eq:differentialconservationlaw}
            u_{t}
            \left(x,t\right)+
            f_{x}
            \left(
            u\left(x,t\right)
            \right)=
            0,
        \end{equation}

        donde $u$ es la \textbf{variable conservativa} o variable de
        estado, es la función
        \begin{align*}
            u\colon\mathbb{R}\times
            \left[0,+\infty\right) &
            \longrightarrow
            \Omega                   \\
            \left(x,t\right)       &
            \longmapsto
            u\left(x,t\right).
        \end{align*}
        El conjunto $\Omega$ es llamado el \textbf{conjunto de los estados}
        y la función $f$ es el \textbf{flujo convectivo}.
    \end{definition}
\end{frame}

\begin{frame}
    \frametitle{\secname}

    \begin{definition}
        El \textbf{problema de Cauchy} consiste en encontrar una función

        \begin{equation*}
            u\colon\mathbb{R}\times
            \left[0,+\infty\right)\longrightarrow
            \Omega
        \end{equation*}

        que sea solución de la
        ecuación~\eqref{eq:differentialconservationlaw} y que verifique la
        condición inicial

        \begin{equation}\label{eq:initialcondition}
            u\left(x,0\right)=
            u_{0}\left(x\right),\quad
            x\in\mathbb{R},
        \end{equation}

        donde
        \begin{math}
            u_{0}\colon\mathbb{R}\to\Omega
        \end{math}
        es una función dada.
    \end{definition}
\end{frame}

\begin{frame}
    \frametitle{\secname}


    \begin{columns}
        \begin{column}{0.60\textwidth}
            \begin{example}
                Consideremos el siguiente problema de Cauchy de la
                \textbf{ecuación de transporte}.
                \begin{equation*}
                    \left\{
                    \begin{aligned}
                        u_{t}
                        \left(x,t\right)+
                        c
                        u_{x}
                        \left(x,t\right)
                                             & =
                        0,                   &
                        \left(x,t\right)     & \in
                        \mathbb{R}\times
                        \left(0,+\infty\right),               \\
                        u\left(x,0\right)    & =
                        u_{0}\left(x\right), &
                        x                    & \in\mathbb{R}.
                    \end{aligned}
                    \right.
                \end{equation*}

                Asumimos que la velocidad de propagación es positiva, $c>0$.
                La solución viene dada por la expresión
                \begin{equation*}
                    u\left(x,t\right)=u_{0}\left(x-ct\right).
                \end{equation*}
                Entonces, la solución $u$ simplemente se transporta con velocidad
                constante $c$ con la evolución del tiempo, como se muestra en
                la Figura~\ref{fig:1}.
            \end{example}
        \end{column}
        \begin{column}{0.36\textwidth}
            \begin{figure}[ht!]
                \centering
                \includegraphics[width=.35\paperwidth]{figure1}
                \caption{
                    Evolución de la solución de la ecuación de transporte con el
                    tiempo.
                }
                \label{fig:1}
            \end{figure}
        \end{column}
    \end{columns}
\end{frame}

\begin{frame}
    \frametitle{\secname}

    \begin{definition}
        El \textbf{problema de Riemann} consiste en encontrar una función

        \begin{equation*}
            u\colon\mathbb{R}\times
            \left[0,+\infty\right)\longrightarrow
            \Omega
        \end{equation*}

        solución de la ecuación~\eqref{eq:differentialconservationlaw} que
        verifica la condición inicial

        \begin{equation}\label{eq:initialconditionriemann}
            u_{0}\left(x\right)=
            \begin{cases}
                u_{l}, &
                \text{si } x<x_{0}, \\
                u_{r}, &
                \text{si } x>x_{0},
            \end{cases}
        \end{equation}

        donde $u_{l},u_{r}\in\mathbb{R}$ vienen~dados.
        A este problema se le denota por
        $\operatorname{PR}\left(u_{l},u_{r}\right)$ centrado en $x_{0}$.
    \end{definition}

    \begin{example}
        Para una ecuación unidimensional, $u_{t}+cu_{x}=0$, la solución al
        problema de Riemann centrado en el $x_{0}$ y con datos iniciales
        $\left(u_{l},u_{r}\right)$ viene dada por
        $u\left(x,t\right)=u_{0}\left(x-ct\right)$, es decir,
        \begin{equation}\label{eq:solutionriemann}
            u
            \left(x,t\right)=
            \begin{cases}
                u_{l}, &
                \text{si } x-ct<x_{0}, \\
                u_{r}, &
                \text{si } x-ct>x_{0}.
            \end{cases}
        \end{equation}
    \end{example}
\end{frame}