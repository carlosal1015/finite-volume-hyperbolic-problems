\section{Método de Volúmenes Finitos}

\begin{frame}
    \frametitle{\secname}

    Comenzamos definiendo una partición del espacio
    $\Omega\in\mathbb{R}$, es decir, se define un conjunto de celdas o
    volúmenes finitos

    \begin{equation*}
        c_{i}=
        \big(
        x_{i-\frac{1}{2}},
        x_{i+\frac{1}{2}}
        \big),\quad
        i=1,\dotsc,N_{x}
    \end{equation*}

    cuyos puntos extremos serían

    \begin{equation*}
        x_{\frac{1}{2}}<
        x_{1+\frac{1}{2}}<
        \cdots<
        x_{N_{x}-\frac{1}{2}}<
        x_{N_{x}+\frac{1}{2}},
    \end{equation*}

    y pertenecen al intervalo $\left(a,b\right)$.
    Por simplicidad, asumimos que la longitud de las celdas es constante,
    es decir,
    \begin{math}
        \Delta x=
        \left|c_{i}\right|=
        x_{i+\frac{1}{2}}-
        x_{i-\frac{1}{2}}
    \end{math}.
    En la Figura~\ref{fig:cellsunidimensional}, se muestran dos ejemplos
    de celdas.

    \begin{figure}[ht!]
        \centering
        \includegraphics[width=.5\paperwidth]{figure3}
        \caption{Celdas $c_{i}$ y $c_{i+1}$ en una dimensión espacial.}
        \label{fig:cellsunidimensional}
    \end{figure}
\end{frame}

\begin{frame}
    \frametitle{\secname}

    Denotamos por $u^{n}_{i}$ a la aproximación del
    valor medio de $u$ sobre la celda $c_{i}$ en el tiempo $t_{n}$, es
    decir:

    \begin{equation*}
        u^{n}_{i}\approx
        \frac{1}{\Delta x}
        \int_{c_{i}}
        u
        \left(x,t_{n}\right)
        \dl x=
        \frac{1}{\Delta x}
        \int_{x_{i-\frac{1}{2}}}^{x_{i+\frac{1}{2}}}
        u
        \left(x,t_{n}\right)
        \dl x.
    \end{equation*}

    Ahora podemos expresar la forma integral de la ley de conservación en
    la celda $c_{i}$ como

    \begin{equation}\label{eq:integralformcell}
        \diffp{}{t}
        \int_{c_{i}}
        u\left(x,t\right)
        \dl x=
        f
        \big(
        u\big(x_{i-\frac{1}{2}},t\big)
        \big)-
        f
        \big(
        u\big(x_{i+\frac{1}{2}},t\big)
        \big).
    \end{equation}

    Una vez que se ha discretizado el dominio espacial, procedemos a
    hacerlo con el temporal.
    Consideremos una partición uniforme en el tiempo,

    \begin{equation*}
        0=
        t_{0}<
        t_{1}<
        \cdots<
        t_{N_{t}}=
        T,
    \end{equation*}

    siendo $\Delta t=t_{n+1}-t_{n}$.
    El paso en tiempo lo consideraremos constante.
    Ahora, integramos la expresión~\eqref{eq:integralformcell} en cada
    intervalo de tiempo $\left[t_{n},t_{n+1}\right]$:

    \begin{equation}\label{eq:integrationeachinterval}
        \int_{c_{i}}
        u
        \left(x,t_{n+1}\right)
        \dl x-
        \int_{c_{i}}
        u
        \left(x,t_{n}\right)
        \dl x=
        \int_{t_{n}}^{t_{n+1}}
        \left[
            f
            \big(
            u\big(x_{i-\frac{1}{2}},t\big)
            \big)-
            f
            \big(
            u\big(x_{i+\frac{1}{2}},t\big)
            \big)
            \right]
        \dl t.
    \end{equation}
\end{frame}

\begin{frame}
    \frametitle{\secname}

    Dividiendo la expresión~\eqref{eq:integrationeachinterval} entre
    $\Delta x$ y reordenando los términos, se obtiene

    \begin{gather}\label{eq:expressiondividedoverdeltax}
        \begin{split}
            \frac{1}{\Delta x}
            \int_{c_{i}}
            u
            \left(x,t_{n+1}\right)
            \dl x
             & =
            \frac{1}{\Delta x}
            \int_{c_{i}}
            u
            \left(x,t_{n}\right)
            \dl x          \\
             & \phantom{=}
            -\frac{1}{\Delta x}
            \int_{t_{n}}^{t_{n+1}}
            \left[
                f
                \big(
                u\big(x_{i-\frac{1}{2}}\big)
                \big)-
                f
                \big(
                u\big(x_{i+\frac{1}{2}}\big)
                \big)
                \right]
            \dl t.
        \end{split}
    \end{gather}

    Debido a que, en general, no podemos evaluar exactamente las
    integrales del segundo miembro de la
    expresión~\eqref{eq:expressiondividedoverdeltax}, ya que
    \begin{math}
        u
        \big(x_{i+\frac{1}{2}},t\big)
    \end{math}
    y
    \begin{math}
        u
        \big(x_{i-\frac{1}{2}},t\big)
    \end{math}
    varían con el
    tiempo a lo largo de cada lado de la celda, y a que no conocemos la
    solución exacta, vamos a estudiar métodos numéricos de la forma
    \begin{equation}\label{eq:godunovscheme}
        u^{n+1}_{i}=
        u^{n}_{i}-
        \frac{\Delta t}{\Delta x}
        \big(
        F^{n}_{i+\frac{1}{2}}-
        F^{n}_{i-\frac{1}{2}}
        \big),
    \end{equation}

    donde $F^{n}_{i+\frac{1}{2}}$ es una aproximación del valor medio del
    flujo en el plano $xt$ a lo largo de la recta $x=x_{i}+\frac{1}{2}$
    con $t$ variando entre $t_{n}$ y $t_{n+1}$, es decir,

    \begin{equation}\label{eq:numericalflux}
        F^{n}_{i+\frac{1}{2}}\approx
        \frac{1}{\Delta t}
        \int_{t_{n}}^{t_{n+1}}
        f
        \big(
        u
        \big(x_{i+\frac{1}{2}},t\big)
        \big)
        \dl t.
    \end{equation}
\end{frame}

\subsection{Métodos descentrado para la ecuación de transporte}

\begin{frame}
    \frametitle{\subsecname}

    La ecuación de transporte viene dada por $u_{t}+cu_{x}=0$.
    En primer lugar, supongamos que la información se propaga con
    velocidad positiva, $c>0$.
    Como se muestra en la figura 3.2, el flujo a través del extremo
    izquierdo de la celda $c_{i}$ está completamente determinado por el
    valor $u^{n}_{i-1}$.
    Por tanto, el flujo numérico a lo largo de la recta
    $x=x_{i}-\frac{1}{2}$ se define como

    \begin{equation*}
        F^{n}_{i-\frac{1}{2}}=
        cu^{n}_{i-1}.
    \end{equation*}

    En cambio, el flujo a través del extremo derecho de la celda $c_{i}$
    está completamente determinado por el valor uni en esta misma celda.
    Luego, el flujo numérico a lo largo de la recta $x=x_{i+\frac{1}{2}}$
    se define como

    \begin{equation*}
        F^{n}_{i+\frac{1}{2}}=
        cu^{n}_{i}.
    \end{equation*}

    Sustituyendo en la ecuación~\eqref{eq:godunovscheme}, el método
    descentrado para la ecuación de transporte con velocidad de
    propagación positiva, $c>0$, viene dado por

    \begin{equation}\label{eq:upwind}
        u^{n+1}_{i}=
        u^{n}_{i}-
        \frac{c\Delta t}{\Delta x}
        \left(
        u^{n}_{i}-
        u^{n}_{i-1}
        \right).
    \end{equation}

    denominado \textbf{método descentrado upwind}.
\end{frame}

\begin{frame}
    \frametitle{\subsecname}

    Por otro lado, en el caso de que la información se propague con
    velocidad negativa $c<0$, %como se muestra en la figura 3.3,
    siguiendo el mismo razonamiento que para el caso anterior, resulta
    que el método descentrado viene dado por el siguiente esquema
    numérico

    \begin{equation}\label{eq:downwind}
        u^{n+1}_{i}=
        u^{n}_{i}-
        \frac{c\Delta t}{\Delta x}
        \left(
        u^{n}_{i+1}-
        u^{n}_{i}
        \right),
    \end{equation}

    donde
    \begin{math}
        F^{n}_{i+\frac{1}{2}}=
        cu^{n}_{i+1}
    \end{math}
    y
    \begin{math}
        F^{n}_{i-\frac{1}{2}}=
        cu^{n}_{i}
    \end{math}
    denominado \textbf{método descentrado downwind}.

    \begin{alertblock}{Observación}
        A medida que el tiempo evoluciona, como la solución es constante a lo
        largo de las rectas características, se tiene que

        \begin{equation*}
            u^{n+1}_{i}\approx
            u
            \left(x_{i},t_{n+1}\right)=
            u
            \left(
            x_{i}-
            c\left(t_{n+1}-t_{n}\right),
            t_{n}
            \right)=
            u
            \left(
            x_{i}-
            c\Delta t,t_{n}
            \right).
        \end{equation*}
    \end{alertblock}
\end{frame}
