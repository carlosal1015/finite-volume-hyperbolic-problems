\section{Convergencia, precisión y estabilidad}

\begin{frame}
    \frametitle{\secname}

    Denotaremos por

    \begin{equation*}
        \widehat{u}^{n}_{i}\coloneqq
        \frac{1}{\Delta x}
        \int_{x_{i-\frac{1}{2}}}^{x_{i+\frac{1}{2}}}
        u\left(x,t_{n}\right)\dl x,
    \end{equation*}

    a la media de la solución exacta de la ecuación en la celda $c_{i}$ en el tiempo $t_{n}$.

    \begin{definition}
        El \textbf{error global} en el instante $t_{n}$ de un método
        numérico viene dado por la diferencia

        \begin{equation*}
            E^{n}=
            u^{n}-
            \widehat{u}^{n}.
        \end{equation*}

        Para cuantificar el error en un instante de tiempo fijo, debemos
        elegir alguna norma.
    \end{definition}

    Para funciones discretas, como el error $E^{n}$, introducimos la
    norma definida por
    \begin{equation}\label{eq:l2normdiscrete}
        {\left\|E^{n}\right\|}_{2}=
            {
                \left(
                \Delta x
                \sum_{i=-\infty}^{\infty}
                {\left|E^{n}_{i}\right|}^{2}
                \right)
            }^{\frac{1}{2}},
    \end{equation}

    denominada la norma discreta $L^{2}$.
\end{frame}

\begin{frame}
    \frametitle{\secname}

    \begin{definition}
        Se dice que el método numérico es \textbf{convergente} en el
        instante de tiempo $t^{n}$ para la norma
        ${\left\|\cdot\right\|}_{2}$ si
        \begin{equation*}
            \lim_{\Delta t\to0}
            {\left\|E^{n}\right\|}_{2}=
            0.
        \end{equation*}

        Se dirá que tiene \textbf{orden de convergencia} $s$ si
        \begin{equation*}
            \left\|
            E^{n}
            \right\|_{2}=
            \mathcal{O}
            \left({\Delta t}^{s}\right),
            \text{ cuando }
            \Delta t\to 0.
        \end{equation*}
    \end{definition}

    \begin{exampleblock}{Recordar}
        \begin{itemize}
            \item

                  El método numérico debe ser \textbf{consistente} con la
                  ecuación diferencial.
                  En cada paso de tiempo, se estudia que existe una buena
                  aproximación local.

            \item

                  El método numérico debe ser \textbf{estable}: los errores
                  locales que se producen en cada paso de tiempo no deben
                  crecer demasiado rápido en los pasos de tiempos posteriores.
        \end{itemize}
    \end{exampleblock}
\end{frame}

\begin{frame}
    \frametitle{\secname}

    \begin{example}[Esquemas numéricos de un paso $u^{n+1}=\mathcal{N}\left(u^{n}\right)$]
        \begin{align}
            u^{n+1}_{i} & =
            u^{n}_{i}-
            c\frac{\Delta t}{\Delta x}
            \left(
            u^{n}_{i}-
            u^{n}_{i-1}
            \right)\label{eq:upwindscheme}        \\
            u^{n+1}_{i} & =
            u^{n}_{i}-
            c\frac{\Delta t}{\Delta x}
            \left(
            u^{n}_{i+1}-
            u^{n}_{i}
            \right)\label{eq:downwindscheme}      \\
            u^{n+1}_{i} & =
            u^{n}_{i}-
            c\frac{\Delta t}{2\Delta x}
            \left(
            u^{n}_{i+1}-u^{n}_{i-1}
            \right)\label{eq:ftcsscheme}          \\
            u^{n+1}_{i} & =
            \frac{1}{2}
            \left(
            u^{n}_{i+1}+
            u^{n}_{i-1}
            \right)-
            c\frac{\Delta t}{2\Delta x}
            \left(
            u^{n}_{i+1}-
            u^{n}_{i-1}
            \right)\label{eq:laxfriedrichsscheme} \\
            u^{n+1}_{i} & =
            u^{n}_{i}-
            c\frac{\Delta t}{2\Delta x}
            \left(
            u^{n}_{i+1}-
            u^{n}_{i-1}
            \right)+
            c^{2}
            \frac{{\Delta t}^{2}}{2{\Delta x}^{2}}
            \left(
            u^{n}_{i+1}-
            2u^{n}_{i}+
            u^{n}_{i-1}
            \right)\label{eq:laxwendroffscheme}
        \end{align}

        Los esquemas~\eqref{eq:upwindscheme} y~\eqref{eq:downwindscheme} son
        métodos descentrados, upwind y downwind, respectivamente.
        El esquema numérico~\eqref{eq:ftcsscheme} veremos que es inestable y
        es a partir del cual se introdujo el
        esquema~\eqref{eq:laxfriedrichsscheme}, denominado esquema de
        Lax-Friedrichs.
        Finalmente, el esquema numérico~\eqref{eq:laxwendroffscheme} es el
        conocido por Lax-Wendroff.
    \end{example}
\end{frame}

\begin{frame}
    \frametitle{\secname}

    \begin{definition}
        El \textbf{error de truncamiento local o de consistencia} en el
        paso $n$-ésimo de tiempo, $n=0,\dotsc,N-1$, es el número $\tau^{n}$
        dado por
        \begin{equation*}
            \tau^{n}=
            \frac{1}{\Delta t}
            \left(
            \mathcal{N}\left(u^{n}\right)-
            \widehat{u}^{n+1}
            \right).
        \end{equation*}
    \end{definition}

    \begin{definition}
        Se dice que el método numérico es \textbf{consistente} si
        \begin{equation*}
            \lim_{\Delta t\to0}
            \max_{0\leq n\leq N-1}
            \left|\tau^{n}\right|
            =0.
        \end{equation*}

        Se dirá que tiene \textbf{orden de consistencia} $s$ si
        \begin{equation*}
            \tau^{n}=
            \mathcal{O}
            \left({\Delta t}^{s}\right).
        \end{equation*}
    \end{definition}
\end{frame}

\begin{frame}
    \frametitle{\secname}

    \begin{example}[Error de consistencia del esquema descentrado upwind]
        Dicho esquema viene dado por la expresión~\eqref{eq:upwind}.
        Teniendo en cuenta los desarrollos de Taylor correspondientes,
        cancelando términos comunes y sabiendo que $u$ es la solución
        exacta de la ecuación de transporte, es decir,
        $u_{t}+cu_{x}=0$, el error de consistencia viene dado por
        \begin{equation}\label{eq:truncationerrorupwind}
            \tau^{n}=
            \frac{1}{2}
            c\Delta x
            u_{xx}
            \left(x_{i},t_{n}\right)-
            \frac{1}{2}\Delta t
            u_{tt}
            \left(x_{i},t_{n}\right)+
            \mathcal{O}
            \left(\Delta t^{2}\right).
        \end{equation}

        Dado que $u_{t}+cu_{x}=0$, entonces $u_{t}=-cu_{x}$ y derivando
        dicha igualdad con respecto a $t$ y a $x$, obtenemos que
        $u_{tt}=-cu_{xt}$ y $u_{tx}=-cu_{xx}$, respectivamente.
        De aquí, $u_{tt}=c^{2}u_{xx}$ y sustituyendo en el error de
        consistencia~\eqref{eq:truncationerrorupwind}, se tiene que
        \begin{equation*}
            \tau^{n}=
            \frac{1}{2}
            c\Delta x
            \left(
            1-c\frac{\Delta t}{\Delta x}
            \right)
            u_{xx}
            \left(x_{i},t_{n}\right)+
            \mathcal{O}
            \left(\Delta t^{2}\right).
        \end{equation*}

        Asumiremos que
        \begin{math}
            u\in
            C^{2}
            \left(
            \mathbb{R}\times
            \mathbb{R}^{+}
            \right)
        \end{math}
        y que $\diffp[2]{u}{x}$ está acotada.
        Además, $\frac{\Delta t}{\Delta x}$ permanece constante.
        Entonces tenemos que $\lim_{\Delta t\to0}\tau^{n}=0$, y así queda
        probado que el esquema descentrado upwind para la
        ecuación de transporte es consistente, siendo el orden de
        convergencia $s=1$, pues el término de menor orden es $\Delta x$.
    \end{example}
\end{frame}

\begin{frame}
    \frametitle{\secname}

    \begin{definition}
        Un esquema numérico de un paso es estable si existe una constante $K$ tal que

        \begin{equation}\label{eq:stabilityonestepscheme}
            \sum_{i=-\infty}^{+\infty}
            {\left|u^{n}_{i}\right|}^{2}\leq
            K
            \sum_{i=-\infty}^{+\infty}
            {\left|u^{0}_{i}\right|}^{2}.
        \end{equation}
    \end{definition}

    Multiplicando la desigualdad~\eqref{eq:stabilityonestepscheme} por
    $\Delta x$ y aplicando la norma discreta $L^{2}$ dado por la
    expresión~\eqref{eq:l2normdiscrete}, tenemos que la definición
    anterior es equivalente a

    \begin{equation}\label{eq:stabilityonestepschemeequivalence}
        {\left\|u^{n}\right\|}^{2}_{2}\leq
        K
        {\left\|u^{0}\right\|}^{2}_{2},
    \end{equation}

    para alguna constante $K$.
\end{frame}

\begin{frame}
    \frametitle{\secname}
    \begin{example}
        Consideramos el esquema numérico
        downwind~\eqref{eq:downwindscheme}, esto es,

        \begin{equation}\label{eq:downwindscheme2}
            u^{n+1}_{i}=
            \big(
            1+
            c\frac{\Delta t}{\Delta x}
            \big)
            u^{n}_{i}-
            c\frac{\Delta t}{\Delta x}
            u^{n}_{i+1}.
        \end{equation}

        Denotemos
        \begin{math}
            \alpha=
            1+
            c\frac{\Delta t}{\Delta x}
        \end{math}
        y
        \begin{math}
            \beta=
            -c\frac{\Delta t}{\Delta x}
        \end{math},
        sustituyendo en la expresión~\eqref{eq:downwindscheme2} obtenemos
        que el esquema numérico viene dado por

        \begin{equation*}
            u^{n+1}_{i}=
            \alpha u^{n}_{i}+
            \beta u^{n}_{i+1}.
        \end{equation*}

        A continuación, vamos a demostrar que el esquema numérico es
        estable si
        \begin{math}
            \left|\alpha\right|+
            \left|\beta\right|\leq
            1
        \end{math}.
    \end{example}
\end{frame}

\begin{frame}
    \frametitle{\secname}

    \begin{proofs}
        \begin{align*}
            \sum_{i=-\infty}^{+\infty}
            {\left|u^{n+1}_{i}\right|}^{2} & =
            \sum_{i=-\infty}^{+\infty}
            {
            \left|\alpha u^{n}_{i}+
            \beta u^{n}_{i+1}\right|
            }^{2}.                                \\
                                           & =
            \sum_{i=-\infty}^{+\infty}
            \left|
            \alpha^{2}
            {\left(u^{n}_{i}\right)}^{2}+
            2\alpha\beta
            u^{n}_{i}
            u^{n}_{i+1}+
            \beta^{2}
            {\left(u^{n}_{i+1}\right)}^{2}
            \right|.                              \\
                                           & \leq
            \sum_{i=-\infty}^{+\infty}
            \left(
            \alpha^{2}
            {\left|u^{n}_{i}\right|}^{2}+
            2\left|\alpha\right|
            \left|\beta\right|
            \left|u^{n}_{i}\right|
            \left|u^{n}_{i+1}\right|+
            \beta^{2}
            {\left|u^{n}_{i+1}\right|}^{2}
            \right).
        \end{align*}

        A continuación, vamos a aplicar la desigualdad
        \begin{math}
            2
            \left|u^{n}_{i}\right|
            \left|u^{n}_{i+1}\right|\leq
            {\left|u^{n}_{i}\right|}^{2}+
            {\left|u^{n}_{i+1}\right|}^{2}
        \end{math}
        para separar, por un lado, los sumandos con aproximación de la
        solución en la celda $c_{i}$ y por otro lado, los sumandos con
        aproximación de la solución en la celda $c_{i+1}$.
    \end{proofs}
\end{frame}

\begin{frame}
    \frametitle{\secname}

    \begin{proofs}[\proofname\ (Cont.)]
        Finalmente, tomamos en todos los sumandos, la aproximación de la
        solución en la celda $c_{i}$:

        \begin{align*}
            \sum_{i=-\infty}^{+\infty}
            {\left|u^{n+1}_{i}\right|}^{2} & \leq
            \sum_{i=-\infty}^{+\infty}
            \left(
            \alpha^{2}
            {\left|u^{n}_{i}\right|}^{2}+
            \left|\alpha\right|
            \left|\beta\right|
            \left(
            \left|u^{n}_{i}\right|^{2}+
            \left|u^{n}_{i+1}\right|^{2}
            \right)+
            \beta^{2}
            {\left|u^{n}_{i+1}\right|}^{2}
            \right)                               \\
                                           & \leq
            \sum_{i=-\infty}^{+\infty}
            \left(
            \alpha^{2}
            \left|u^{n}_{i}\right|^{2}+
            \left|\alpha\right|
            \left|\beta\right|
            {\left|u^{n}_{i}\right|}^{2}
            \right)+
            \sum_{i=-\infty}^{+\infty}
            \left(
            \left|\alpha\right|
            \left|\beta\right|
            {\left|u^{n}_{i+1}\right|}^{2}+
            \beta^{2}
            {\left|u^{n}_{i+1}\right|}^{2}
            \right)                               \\
                                           & =
            \sum_{i=-\infty}^{+\infty}
            \left(
            \alpha^{2}
            {\left|u^{n}_{i}\right|}^{2}+
            \left|\alpha\right|
            \left|\beta\right|
            \left|u^{n}_{i}\right|^{2}
            \right)+
            \sum_{i=-\infty}^{+\infty}
            \left(
            \left|\alpha\right|
            \left|\beta\right|
            \left|u^{n}_{i}\right|^{2}+
            \beta^{2}
            {\left|u^{n}_{i}\right|}^{2}
            \right)                               \\
                                           & =
            \sum_{i=-\infty}^{+\infty}
            \left(
            \alpha^{2}+
            2\left|\alpha\right|
            \left|\beta\right|+
            \beta^{2}
            \right)
            \left|u^{n}_{i}\right|^{2}=
                {\left(\left|\alpha\right|+\left|\beta\right|\right)}^{2}
            \sum_{i=-\infty}^{+\infty}
            {\left|u^{n}_{i}\right|}^{2}.
        \end{align*}
    \end{proofs}
\end{frame}

\begin{frame}
    \frametitle{\secname}
    \begin{proof}[\proofname\ (Cont.)]
        Entonces, hemos obtenido que

        \begin{equation*}
            \sum_{i=-\infty}^{+\infty}
            {\left|u^{n+1}_{i}\right|}^{2}\leq
            {
                \left(\left|\alpha\right|+
                \left|\beta\right|\right)
            }^{2}
            \sum_{i=-\infty}^{+\infty}
            {\left|u^{n}_{i}\right|}^{2}.
        \end{equation*}

        De forma recursiva, concluimos que

        \begin{equation*}
            \sum_{i=-\infty}^{\infty}
            {\left|u^{n+1}_{i}\right|}^{2}\leq
            {
                \left(
                \left|\alpha\right|+
                \left|\beta\right|
                \right)
            }^{2\left(n+1\right)}
            \sum_{i=-\infty}^{+\infty}
            {\left|u^{0}_{i}\right|}^{2}.
        \end{equation*}

        Por lo tanto, si
        \begin{math}
            \left|\alpha\right|+
            \left|\beta\right|\leq
            1
        \end{math}
        entonces el esquema numérico es estable.
        Sustituyendo los valores de $\alpha$ y $\beta$ y aplicando
        propiedades de la función valor absoluto, concluimos que si
        \begin{math}
            -1\leq
            c\frac{\Delta t}{\Delta x}\leq
            0
        \end{math}
        entonces el esquema numérico~\eqref{eq:downwindscheme2} es estable, como queríamos ver.
    \end{proof}
\end{frame}

\begin{frame}
    \frametitle{\secname}

    Emplearemos el subíndice $j$ para denotar
    $x_{j}=j\Delta x$ en lugar del subíndice $i$ para que no haya
    confusión con el número complejo $i=\sqrt{-1}$.
    En la celda $c_{j}$ y para el tiempo $t_{n}$, $u^{n}_{j}$ representa
    una función definida sobre una malla para el problema de Cauchy.
    Sea $\Delta x$ la longitud de las celdas de la malla,
    entonces su transformada de Fourier visto viene dada por
    \begin{equation}\label{eq:fouriertransform}
        \overline{u}
        \left(\xi\right)=
        \frac{1}{\sqrt{2\pi}}
        \sum_{j=-\infty}^{\infty}
        e^{-i\xi j\Delta x}
        u^{n}_{j}\Delta x,
    \end{equation}

    donde
    \begin{math}
        \xi\in
        \left[
            -\frac{\pi}{\Delta x},
            \frac{\pi}{\Delta x}
            \right]
    \end{math}
    y
    \begin{math}
        \overline{u}
        \left(
        -\frac{\pi}{\Delta x}
        \right)=
        \overline{u}^{n}
        \left(
        \frac{\pi}{\Delta x}
        \right)
    \end{math}.
    Y la transformada de Fourier inversa viene dada por
    \begin{equation}\label{eq:fourierinversetransform}
        u^{n}_{j}=
        \frac{1}{\sqrt{2\pi}}
        \int_{-\frac{\pi}{\Delta x}}^{\frac{\pi}{\Delta x}}
        e^{i\xi j\Delta x}
        \overline{u}^{n}
        \left(\xi\right)
        \dl\xi.
    \end{equation}

    Tomando la norma $L^{2}$ para funciones continuas y en la
    expresión~\eqref{eq:l2normdiscrete} para funciones discretas, se
    tiene el siguiente resultado.
\end{frame}

\begin{frame}
    \frametitle{\secname}

    \begin{theorem}[Identidad de Parseval]\normalfont
        Sea
        \begin{math}
            \overline{u}^{n}\colon
            \left[
                -\frac{\pi}{\Delta x},
                \frac{\pi}{\Delta x}
                \right]\to
            \mathbb{R}
        \end{math}
        una función de periodo $\frac{2\pi}{\Delta x}$, dada por la
        serie~\eqref{eq:fouriertransform} con coeficientes $u^{n}_{j}$
        dados por la igualdad~\eqref{eq:fourierinversetransform}.
        Entonces
        \begin{equation}\label{eq:parsevalidentity}
            {\left\|\overline{u}\left(\xi\right)\right\|}^{2}_{L^{2}\left(\mathbb{R}\right)}=
            {\left\|u^{n}\right\|}^{2}_{2}.
        \end{equation}
    \end{theorem}

    \begin{proofs}
        Apliquemos la norma $L^{2}\left(\mathbb{R}\right)$ a la función
        periódica $\overline{u}^{n}$ en el sentido de la definición 4.2, y
        usando propiedades de los números complejos, obtenemos
        \begin{align*}
            {\left\|
                \overline{u}^{n}
                \left(\xi\right)
            \right\|}^{2}_{L^{2}\left(\mathbb{R}\right)} & =
            \int_{-\frac{\pi}{\Delta x}}^{\frac{\pi}{\Delta x}}
            {\left|
                \overline{u}^{n}
                \left(\xi\right)
                \right|}^{2}
            \dl\xi.                                          \\
                                                         & =
            \int_{-\frac{\pi}{\Delta x}}^{\frac{\pi}{\Delta x}}
            \overline{\overline{u}^{n}\left(\xi\right)}
            \overline{u}^{n}
            \left(\xi\right)
            \dl\xi.                                          \\
                                                         & =
            \int_{-\frac{\pi}{\Delta x}}^{\frac{\pi}{\Delta x}}
            \overline{\overline{u}^{n}\left(\xi\right)}
            \frac{1}{\sqrt{2\pi}}
            \sum_{j=-\infty}^{+\infty}
            e^{-i\xi j\Delta x}
            u^{n}_{j}
            \Delta x
            \dl\xi.
        \end{align*}
    \end{proofs}
\end{frame}

\begin{frame}
    \frametitle{\secname}

    \begin{proof}[\proofname\ (Cont.)]
        \begin{align*}
            {\left\|
                \overline{u}^{n}
                \left(\xi\right)
                \right\|}^{2}_{L^{2}\left(\mathbb{R}\right)}
             & =
            \Delta x
            \sum_{j=-\infty}^{+\infty}
            \frac{1}{\sqrt{2\pi}}
            \int_{-\frac{\pi}{\Delta x}}^{\frac{\pi}{\Delta x}}
            e^{-i\xi j \Delta x}
            \overline{\overline{u}^{n}\left(\xi\right)}
            \dl\xi
            u^{n}_{j}. \\
             & =
            \Delta x
            \sum_{j=-\infty}^{+\infty}
            \frac{1}{\sqrt{2\pi}}
            \int_{-\frac{\pi}{\Delta x}}^{\frac{\pi}{\Delta x}}
            e^{i\xi j \Delta x}
            \overline{u}^{n}
            \left(\xi\right)
            \dl\xi
            u^{n}_{j}. \\
             & =
            \Delta x
            \sum_{j=-\infty}^{+\infty}
            \overline{u^{n}_{j}}
            u^{n}_{j}. \\
             & =
            {\left\|\overline{u}^{n}\right\|}^{2}_{2}.
            \tag*{\qedhere}
        \end{align*}
    \end{proof}
    De esta forma, por la
    desigualdad~\eqref{eq:stabilityonestepschemeequivalence} y por la
    identidad de Parseval~\eqref{eq:parsevalidentity}, si se demuestra la
    estabilidad para la transformada de Fourier, inmediatamente, está
    demostrada para la función original.
\end{frame}

\begin{frame}
    \frametitle{\secname}

    De forma genérica, sabemos que un esquema de volúmenes finitos de un
    paso se define como
    \begin{equation}\label{eq:onestepschemefinitevolume}
        u^{n+1}_{j}=
        u^{n}_{j}-
        c\frac{\Delta t}{\Delta x}
        \left(
        \phi\left(u^{n}_{j},u^{n}_{j+1}\right)-
        \phi\left(u^{n}_{j-1},u^{n}_{j}\right)
        \right).
    \end{equation}

    Para simplificar notación, denotamos
    \begin{math}
        \mu=
        c\frac{\Delta t}{\Delta x}
    \end{math}.
    Centrándonos en nuestro estudio de problemas lineales podemos
    escribir el esquema~\eqref{eq:onestepschemefinitevolume} como sigue
    \begin{equation}\label{eq:onestepschemefinitevolume2}
        u^{n+1}_{j}=
        u^{n}_{j}-
        \mu
        \left(
        \alpha u^{n}_{j-1}+
        \beta u^{n}_{j}+
        \gamma u^{n}_{j+1}
        \right)=
        \left(1-\mu\beta\right)
        u_j^n-
        \mu\alpha
        u^{n}_{j-1}-
        \mu\gamma
        u^{n}_{j+1},
    \end{equation}

    donde $\alpha,\beta,\gamma\in\mathbb{R}$, y sustituyendo la
    ecuación~\eqref{eq:fourierinversetransform} en la
    expresión~\eqref{eq:onestepschemefinitevolume2}, se obtiene que
    \begin{equation}\label{eq:onestepschemefinitevolume3}
        u^{n+1}_{j}=
        \frac{1}{\sqrt{2\pi}}
        \int_{-\frac{\pi}{\Delta x}}^{\frac{\pi}{\Delta x}}
        e^{i\xi j\Delta x}
        \left[
            \left(1-\mu\beta\right)-
            \mu\alpha
            e^{-i\xi\Delta x}-
            \mu\gamma
            e^{i\xi\Delta x}
            \right]
        \overline{u}^{n}\left(\xi\right)
        \dl\xi.
    \end{equation}

    Aplicando la definición de la transformada de Fourier inversa, dada
    por la ecuación~\eqref{eq:fourierinversetransform}, en el tiempo
    $t_{n+1}$, de la ecuación~\eqref{eq:onestepschemefinitevolume3}
    resulta que
    \begin{equation*}
        \overline{u}^{n+1}
        \left(\xi\right)=
        \left[
            \left(1-\mu\beta\right)-
            \mu\alpha
            e^{-i\xi\Delta x}-
            \mu\gamma e^{i\xi\Delta x}
            \right]
        \overline{u}^{n}
        \left(\xi\right)
    \end{equation*}

    Denotando
    \begin{math}
        g
        \left(\xi\Delta x\right)=
        \left(1-\mu\beta\right)-
        \mu\alpha
        e^{-i\xi\Delta x}-
        \mu\gamma
        e^{i\xi\Delta x}
    \end{math},
    entonces
    \begin{equation}\label{eq:amplificationfactor}
        \overline{u}^{n+1}
        \left(\xi\right)=
        g
        \left(\xi\Delta x\right)
        \overline{u}^{n}
        \left(\xi\right).
    \end{equation}
\end{frame}

\begin{frame}
    \frametitle{\secname}

    \begin{alertblock}{Observación}
        La ecuación~\eqref{eq:amplificationfactor} muestra que para esquemas
        numéricos de un paso, la solución en un tiempo concreto es igual a
        multiplicar la solución en el instante de tiempo previo por un factor
        denominado factor de amplificación, que corresponde con la magnitud
        en la que avanza la solución en un paso de tiempo.
        De la ecuación~\eqref{eq:amplificationfactor} de forma recursiva, se
        obtiene que
        \begin{equation*}
            \overline{u}^{n+1}
            \left(\xi\right)=
            {g\left(\xi\Delta x\right)}^{n}
            \overline{u}^{0}
            \left(\xi\right).
        \end{equation*}
    \end{alertblock}

    \begin{theorem}\normalfont
        Un esquema de volúmenes finitos de un paso es estable si y solo si
        \begin{math}
            {\left|
            {g\left(\xi\Delta x\right)}
            \right|}^{n}\leq
            1
        \end{math}.
    \end{theorem}

    \begin{proof}
        Ver en~\cite{Strikwerda2004}.
    \end{proof}
\end{frame}

\begin{frame}
    % \frametitle{\secname}
    \begin{example}
        Consideremos el esquema numérico~\eqref{eq:ftcsscheme}.
        Sustituimos la ecuación~\eqref{eq:fourierinversetransform} en el
        esquema numérico considerado y se obtiene que
        \vspace*{-.5\baselineskip}\setlength\belowdisplayshortskip{0pt}
        \begin{equation*}
            u^{n+1}_{j}=
            \frac{1}{\sqrt{2\pi}}
            \int_{-\frac{\pi}{\Delta x}}^{\frac{\pi}{\Delta x}}
            e^{i\xi j \Delta x}
            \left(
            1-c\frac{\Delta t}{\Delta x}
            e^{i \xi \Delta x}+
            c\frac{\Delta t}{\Delta x}
            e^{-i\xi\Delta x}\right)
            \overline{u}^{n}
            \left(\xi\right)
            \dl\xi.
        \end{equation*}
        Por tanto, resulta que
        \vspace*{-.5\baselineskip}\setlength\belowdisplayshortskip{0pt}
        \begin{equation*}
            \bar{u}^{n+1}
            \left(\xi\right)=
            \left(
            1-
            c\frac{\Delta t}{\Delta x}
            e^{i\xi\Delta x}+
            c\frac{\Delta t}{\Delta x}
            e^{-i\xi\Delta x}
            \right)
            \overline{u}^{n}
            \left(\xi\right),
        \end{equation*}
        donde el factor de amplificación viene dado por
        \vspace*{-.5\baselineskip}\setlength\belowdisplayshortskip{0pt}
        \begin{equation*}
            g
            \left(\xi\Delta x\right)=
            1-
            c\frac{\Delta t}{\Delta x}
            \left(
            e^{i\xi\Delta x}-
            e^{-i\xi\Delta x}
            \right)=
            1-
            c\frac{\Delta t}{\Delta x}
            \left(
            2i\sin
            \left(\xi\Delta x\right)
            \right).
        \end{equation*}
        Como el factor de amplificación es un número complejo, consideremos
        \begin{math}
            {\left|
                g\left(\xi\Delta x\right)
                \right|}^{2},
        \end{math}
        esto es,
        \vspace*{-.5\baselineskip}\setlength\belowdisplayshortskip{0pt}
        \begin{equation*}
            \left|
            g
            \left(\xi\Delta x\right)
            \right|^{2}=
            1+
            4c^{2}
            \frac{{\Delta t}^{2}}{{\Delta x}^{2}}
            \sin^{2}
            \left(\xi\Delta x\right).
        \end{equation*}
        De donde, concluimos que
        \begin{math}
            \left|
            g\left(\xi\Delta x\right)
            \right|>
            1
        \end{math}
        aún cumpliéndose la confición CFL.
        Entonces, concluimos que este esquema numérico para la ecuación de
        transporte es inestable.
    \end{example}
\end{frame}