\chapter{Convergencia, precisión y estabilidad}\label{ch:convergence}

El uso de un método numérico para resolver una ecuación diferencial
conlleva el estudio de la precisión y las propiedades de convergencia
del método.
En esta sección, se realizará un análisis teórico centrándonos
únicamente en el problema de Cauchy, esto es, sin considerar
condiciones de contorno.
Esto está por ejemplo justificado por el hecho de que consideramos
dominios infinitos (y datos iniciales con soporte compacto).

\section{Convergencia}

Para valorar si un esquema numérico proporciona una buena
aproximación de una ley de conservación, el método numérico
resultante debe ser convergente.
Tenemos que $u^{n}_{i}$ es una aproximación del valor medio de $u$
sobre la celda $c_{i}$ en el tiempo $t_{n}$ y de ahora en adelante,
denotaremos por

\begin{equation*}
  \widehat{u}^{n}_{i}\coloneqq
  \frac{1}{\Delta x}
  \int_{x_{i-\frac{1}{2}}}^{x_{i+\frac{1}{2}}}
  u\left(x,t_{n}\right)\dl x,
\end{equation*}

a la media de la solución exacta de la ecuación en derivadas
parciales en la celda $c_{i}$ en el tiempo $t_{n}$.

Usaremos $n$ para indicar el nivel de tiempo correspondiente al
instante $t_{n}=n\Delta t$.
En ese momento, para cada celda $c_{i}$, interesa conocer que tan
bien $u^{n}_{i}$ aproxima a $\widehat{u}^{n}_{i}$.
Entonces, definimos el siguiente concepto.

\begin{definition}
  El error global en el instante $t_{n}$ de un método numérico viene
  dado por la diferencia

  \begin{equation*}
    E^{n}=
    u^{n}-
    \widehat{u}^{n}.
  \end{equation*}

  Para cuantificar el error en un instante de tiempo fijo, debemos
  elegir alguna norma.
\end{definition}

\begin{definition}
  Sea $\Omega\subset\mathbb{R}^{n}$ un abierto.
  Se define el conjunto

  \begin{equation*}
    L^{2}\left(\Omega\right)=
    \left\{
    f\colon\Omega\to\mathbb{R}\mid
    f\text{ es medible y}
    \int_{\Omega}\left|f\right|^{2}<\infty
    \right\}.
  \end{equation*}

  Si $f\in L^{2}\left(\Omega\right)$ se define la

  \begin{equation*}
    {\left\|f\right\|}_{L^{2}\left(\Omega\right)}=
      {\left(
        \int_{\Omega}
        \left|f\left(x\right)\right|^{2}
        \right)}^{\frac{1}{2}}.
  \end{equation*}
\end{definition}

Para funciones discretas, como el error $E^{n}$, introducimos la
norma definida por

\begin{equation*}
  {\left\|E^{n}\right\|}_{2}=
    {
      \left(
      \Delta x
      \sum_{i=-\infty}^{\infty}
      {\left|E^{n}_{i}\right|}^{2}
      \right)
    }^{\frac{1}{2}},
\end{equation*}

denominada la norma discreta $L^{2}$.
Obsérvese que se introduce el factor $\Delta x$, permitiendo que la
serie converja a medida que la malla se refine, esto es, a medida que
$\Delta x$ disminuye.
Este factor $\Delta x$ hace que ambas normas definidas (4.1) y (4.2)
sean análogas.

Con el objetivo de simplificar la notación, asumimos que $\Delta t$ y
$\Delta x$ están relacionadas de manera fija.
Para problemas hiperbólicos, es razonable asumir que la razón
\begin{math}
  \frac{\Delta x}{\Delta t}
\end{math}
es fija y de hecho es usual definir $\Delta t$ en función de $\Delta x$
de forma que se verifique la condición CFL que estudiaremos en la sección 4.3.1.
De esta forma, cuando $\Delta t\to 0$ nos referiremos a que la malla se refina.
Entonces, introducimos el siguiente concepto.

\begin{definition}
  Se dice que el método numérico es convergente en el instante de
  tiempo $t^{n}$ para la norma ${\left\|\cdot\right\|}_{2}$ si

  \begin{equation*}
    \lim_{\Delta t\to0}
    {\left\|E^{n}\right\|}_{2}=
    0.
  \end{equation*}

  Se dirá que tiene orden de convergencia $s$ si

  \begin{equation*}
    \left\|
    E^{n}
    \right\|_{2}=
    \mathcal{O}
    \left({\Delta t}^{s}\right),
    \text{ cuando }
    \Delta t\to 0.
  \end{equation*}
\end{definition}

De forma general, después de una discretización en tiempo, es
imposible obtener una expresión simple para el error global.
Por tanto, en lugar de estudiar la convergencia de los métodos
numéricos mediante el cálculo directo del error global, procederemos
a estudiarla a partir de verificar las dos propiedades de los métodos
numéricos que se enuncian a continuación.

\begin{itemize}
  \item

        El método numérico debe ser consistente con la ecuación
        diferencial.
        En cada paso de tiempo, se estudia que existe una buena
        aproximación local.

  \item

        El método numérico debe ser estable: los errores locales
        que se producen en cada paso de tiempo no deben crecer
        demasiado rápido en los pasos de tiempos posteriores.
\end{itemize}


El estudio de estas dos propiedades depende del tipo de ecuación y
método.
En los dos apartados siguientes, estudiaremos cada una de ellas de
forma genérica y las analizaremos, siempre para la ecuación de
transporte, para los esquemas numéricos presentados a continuación:

\begin{align}
  u^{n+1}_{i} & =
  u^{n}_{i}-
  c\frac{\Delta t}{\Delta x}
  \left(
  u^{n}_{i}-
  u^{n}_{i-1}
  \right)         \\
  u^{n+1}_{i} & =
  u^{n}_{i}-
  c\frac{\Delta t}{\Delta x}
  \left(
  u^{n}_{i+1}-
  u^{n}_{i}
  \right)         \\
  u^{n+1}_{i} & =
  u^{n}_{i}-
  c\frac{\Delta t}{2\Delta x}
  \left(
  u^{n}_{i+1}-u^{n}_{i-1}
  \right)         \\
  u^{n+1}_{i} & =
  \frac{1}{2}
  \left(
  u^{n}_{i+1}+
  u^{n}_{i-1}
  \right)-
  c\frac{\Delta t}{2\Delta x}
  \left(
  u^{n}_{i+1}-
  u^{n}_{i-1}
  \right)         \\
  u^{n+1}_{i} & =
  u^{n}_{i}-
  c\frac{\Delta t}{2\Delta x}
  \left(
  u^{n}_{i+1}-
  u^{n}_{i-1}
  \right)+
  c^{2}
  \frac{{\Delta t}^{2}}{2{\Delta x}^{2}}
  \left(
  u^{n}_{i+1}-
  2u^{n}_{i}+
  u^{n}_{i-1}
  \right)
\end{align}

Los esquemas (4.3) y (4.4) son métodos descentrados, upwind y
downwind, respectivamente.
El esquema numérico (4.5) veremos que es inestable y es a partir del
cual se introdujo el esquema (4.6), denominado esquema de
Lax-Friedrichs.
Finalmente, el esquema numérico (4.7) es el conocido por
Lax-Wendroff.

En estos esquemas numéricos, la incógnita, $u^{n+1}$, se calcula
mediante el valor conocido, $u^{n}$, asociado a un paso de tiempo
previo.
Este tipo de esquemas numéricos se denominan esquemas numéricos de un
paso.
Por tanto, podemos definir la solución numérica en el instante de
tiempo $t_{n+1}$ como

\begin{equation*}
  u^{n+1}=
  \mathcal{N}\left(u^{n}\right).
\end{equation*}

donde $\mathcal{N}\left(\cdot\right)$ representa el operador numérico
que resulta la solución numérica en un instante de tiempo a partir
del instante de tiempo previo.
En esta memoria, nos centraremos en el estudio de este tipo de
esquemas numéricos.

\section{Error de truncamiento local y consistencia}

Recordemos que $\widehat{u}^{n}$ y $u^{n}$ representan la media de la
solución exacta de tiempo y de la solución aproximada de la ecuación
diferencial, respectivamente, en el paso de tiempo $n$.
Introducimos ahora conceptos necesarios para el estudio de la
consistencia.

\begin{definition}
  El error de truncamiento local o de consistencia en el paso
  $n$-ésimo de tiempo, $n=0,\dotsc,N-1$, es el número $\tau^{n}$ dado
  por

  \begin{equation*}
    \tau^{n}=
    \frac{1}{\Delta t}
    \left(
    \mathcal{N}\left(u^{n}\right)-
    \widehat{u}^{n+1}
    \right).
  \end{equation*}
\end{definition}

\begin{definition}
  Se dice que el método numérico es consistente si

  \begin{equation*}
    \lim_{\Delta t\to0}
    \max_{0\leq n\leq N-1}
    \left|\tau^{n}\right|
    =0.
  \end{equation*}

  Se dirá que tiene orden de consistencia $s$ si

  \begin{equation*}
    \tau^{n}=
    \mathcal{O}
    \left({\Delta t}^{s}\right).
  \end{equation*}
\end{definition}

\begin{remark}
  El error de consistencia indica la magnitud del error global
  siempre que el método numérico sea estable
  (concepto que introduciremos en la sección 4.3) y que no existan
  errores iniciales o bien estos sean suficientemente pequeños.
  Este tipo de razonamiento es usual en análisis numérico.
\end{remark}

En el siguiente ejemplo, estudiaremos la consistencia del método
descentrado upwind para la ecuación de transporte.

\begin{example}
  Calculamos el error de consistencia del esquema descentrado upwind
  para la ecuación de transporte
  \begin{math}
    u_{t}+
    cu_{x}=
    0.
  \end{math}
  Dicho esquema viene dado por la expresión (3.8).
  Teniendo en cuenta los desarrollos de Taylor correspondientes,
  cancelando términos comunes y sabiendo que $u$ es la solución
  exacta de la ecuación de transporte, es decir,
  $u_{t}+cu_{x}=0$, el error de consistencia viene dado por

  \begin{equation*}
    \tau^{n}=
    \frac{1}{2}
    c\Delta x
    u_{xx}
    \left(x_{i},t_{n}\right)-
    \frac{1}{2}\Delta t
    u_{tt}
    \left(x_{i},t_{n}\right)+
    \mathcal{O}
    \left(\Delta t^{2}\right).
  \end{equation*}

  Dado que $u_{t}+cu_{x}=0$, entonces $u_{t}=-cu_{x}$ y derivando
  dicha igualdad con respecto a $t$ y a $x$, obtenemos que
  $u_{tt}=-cu_{xt}$ y $u_{tx}=-cu_{xx}$, respectivamente.
  De aquí, $u_{tt}=c^{2}u_{xx}$ y sustituyendo en el error de
  consistencia (4.8), se tiene que

  \begin{equation*}
    \tau^{n}=
    \frac{1}{2}
    c\Delta x
    \left(
    1-c\frac{\Delta t}{\Delta x}
    \right)
    u_{xx}
    \left(x_{i},t_{n}\right)+
    \mathcal{O}
    \left(\Delta t^{2}\right).
  \end{equation*}

  Asumiremos que
  \begin{math}
    u\in
    \mathcal{C}^{2}
    \left(
    \mathbb{R}\times
    \mathbb{R}^{+}
    \right)
  \end{math}
  y que $\diffp[2]{u}{x}$ está acotada.
  Además, $\frac{\Delta t}{\Delta x}$ permanece constante.
  Entonces tenemos que $\lim_{\Delta t\to0}\tau^{n}=0$, y así queda
  probado que el esquema descentrado upwind para la
  ecuación de transporte es consistente, siendo el orden de
  convergencia $s=1$, pues el término de menor orden es $\Delta x$.
\end{example}

De forma similar, hallamos el error de consistencia del esquema (4.6),
esquema de Lax-Friedrichs y del esquema (4.7), esquema de
Lax-Wendroff, ambos, para la ecuación de transporte.
Respectivamente, vienen dados por

\begin{align*}
  \tau^{n} & =
  \frac{1}{2}
  \left(
  \frac{{\Delta x}^{2}}{\Delta t}-
  c^{2}\Delta t
  \right)
  u_{xx}
  \left(x_{i},t_{n}\right)+
  \mathcal{O}
  \left({\Delta t}^{2}\right), \\
  \tau^{n} & =
  \frac{1}{6}
  c
  \left(
  c^{2}{\Delta t}^{2}-
  {\Delta x}^{2}
  \right)
  u_{xxx}\left(x_{i},t_{n}\right)+
  \mathcal{O}\left({\Delta t}^{3}\right).
\end{align*}

Observamos que ambos métodos son consistentes si la solución exacta
es suficientemente regular.
Además, se tiene que el orden de convergencia del método de
Lax-Friedrichs es $s=1$ y el del método de Lax-Wendroff es $s=2$.

\section{Estabilidad}

El análisis de estabilidad es una evaluación fundamental del esquema
numérico a ser estudiado.
Si se determina que el esquema es inestable, debe buscarse otro
método numérico alternativo como solución.
Asumiendo que estamos estudiando leyes de conservación sin
término fuente, esto es, tratamos problemas de valores iniciales
homogéneos, introducimos el siguiente concepto de estabilidad.

\begin{definition}
  Un esquema numérico de un paso (3.6) para una ecuación en derivadas
  parciales de primer orden es estable si existe una constante $K$
  tal que

  \begin{equation*}
    \sum_{i=-\infty}^{+\infty}
    {\left|u^{n}_{i}\right|}^{2}\leq
    K
    \sum_{i=-\infty}^{+\infty}
    {\left|u^{0}_{i}\right|}^{2}.
  \end{equation*}
\end{definition}

Multiplicando la desigualdad (4.9) por $\Delta x$ y aplicando la
norma discreta $L^{2}$ dado por la expresión (4.2), tenemos que
la definición anterior es equivalente a

\begin{equation*}
  {\left\|u^{n}\right\|}^{2}_{2}\leq
  K
  {\left\|u^{0}\right\|}^{2}_{2},
\end{equation*}

para alguna constante $K$.
En particular, cada esquema numérico que estudiemos requerirá
un análisis de estabalidad. En algunos casos podemos determinar
condiciones suficientes que garantizan directamente la estabilidad a
partir de la expresión (4.9).
Veámoslo en el siguiente ejemplo para el esquema numérico (4.3).

\begin{example}
  Consideramos el esquema numérico downwind (4.4), esto es,

  \begin{equation*}
    u^{n+1}_{i}=
    \left(
    1+
    c\frac{\Delta t}{\Delta x}
    \right)
    u^{n}_{i}-
    c\frac{\Delta t}{\Delta x}
    u^{n}_{i+1}.
  \end{equation*}

  Denotemos
  \begin{math}
    \alpha=
    1+
    c\frac{\Delta t}{\Delta x}
  \end{math}
  y
  \begin{math}
    \beta=
    -c\frac{\Delta t}{\Delta x}
  \end{math},
  sustituyendo en la expresión (4.11) obtenemos que el esquema
  numérico viene dado por

  \begin{equation*}
    u^{n+1}_{i}=
    \alpha u^{n}_{i}+
    \beta u^{n}_{i+1}.
  \end{equation*}

  A continuación, vamos a demostrar que el esquema numérico es
  estable si
  \begin{math}
    \left|\alpha\right|+
    \left|\beta\right|\leq
    1
  \end{math}.

  \begin{align*}
    \sum_{i=-\infty}^{+\infty}
    {\left|u^{n+1}_{i}\right|}^{2} & =
    \sum_{i=-\infty}^{+\infty}
    {
    \left|\alpha u^{n}_{i}+
    \beta u^{n}_{i+1}\right|
    }^{2}.                                \\
                                   & =
    \sum_{i=-\infty}^{+\infty}
    \left|
    \alpha^{2}
    {\left(u^{n}_{i}\right)}^{2}+
    2\alpha\beta
    u^{n}_{i}
    u^{n}_{i+1}+
    \beta^{2}
    {\left(u^{n}_{i+1}\right)}^{2}
    \right|.                              \\
                                   & \leq
    \sum_{i=-\infty}^{+\infty}
    \left(
    \alpha^{2}
    {\left|u^{n}_{i}\right|}^{2}+
    2\left|\alpha\right|
    \left|\beta\right|
    \left|u^{n}_{i}\right|
    \left|u^{n}_{i+1}\right|+
    \beta^{2}
    {\left|u^{n}_{i+1}\right|}^{2}
    \right).
  \end{align*}

  A continuación, vamos a aplicar la desigualdad
  \begin{math}
    2
    \left|u^{n}_{i}\right|
    \left|u^{n}_{i+1}\right|\leq
    {\left|u^{n}_{i}\right|}^{2}+
    {\left|u^{n}_{i+1}\right|}^{2}
  \end{math}
  para separar, por un lado, los sumandos con aproximación de la
  solución en la celda $c_{i}$ y por otro lado, los sumandos con
  aproximación de la solución en la celda $c_{i+1}$.
  Finalmente, tomamos en todos los sumandos, la aproximación de la
  solución en la celda $c_{i}$:

  \begin{align*}
    \sum_{i=-\infty}^{+\infty}
    {\left|u^{n+1}_{i}\right|}^{2} & \leq
    \sum_{i=-\infty}^{+\infty}
    \left(
    \alpha^{2}
    {\left|u^{n}_{i}\right|}^{2}+
    \left|\alpha\right|
    \left|\beta\right|
    \left(
    \left|u^{n}_{i}\right|^{2}+
    \left|u^{n}_{i+1}\right|^{2}
    \right)+
    \beta^{2}
    {\left|u^{n}_{i+1}\right|}^{2}
    \right)                               \\
                                   & \leq
    \sum_{i=-\infty}^{+\infty}
    \left(
    \alpha^{2}
    \left|u^{n}_{i}\right|^{2}+
    \left|\alpha\right|
    \left|\beta\right|
    {\left|u^{n}_{i}\right|}^{2}
    \right)+
    \sum_{i=-\infty}^{+\infty}
    \left(
    \left|\alpha\right|
    \left|\beta\right|
    {\left|u^{n}_{i+1}\right|}^{2}+
    \beta^{2}
    {\left|u^{n}_{i+1}\right|}^{2}
    \right)                               \\
                                   & =
    \sum_{i=-\infty}^{+\infty}
    \left(
    \alpha^{2}
    {\left|u^{n}_{i}\right|}^{2}+
    \left|\alpha\right|
    \left|\beta\right|
    \left|u^{n}_{i}\right|^{2}
    \right)+
    \sum_{i=-\infty}^{+\infty}
    \left(
    \left|\alpha\right|
    \left|\beta\right|
    \left|u^{n}_{i}\right|^{2}+
    \beta^{2}
    {\left|u^{n}_{i}\right|}^{2}
    \right)                               \\
                                   & =
    \sum_{i=-\infty}^{+\infty}
    \left(
    \alpha^{2}+
    2\left|\alpha\right|
    \left|\beta\right|+
    \beta^{2}
    \right)
    \left|u^{n}_{i}\right|^{2}=
      {\left(\left|\alpha\right|+\left|\beta\right|\right)}^{2}
    \sum_{i=-\infty}^{+\infty}
    {\left|u^{n}_{i}\right|}^{2}.
  \end{align*}

  Entonces, hemos obtenido que

  \begin{equation*}
    \sum_{i=-\infty}^{+\infty}
    {\left|u^{n+1}_{i}\right|}^{2}\leq
    {
      \left(\left|\alpha\right|+
      \left|\beta\right|\right)
    }^{2}
    \sum_{i=-\infty}^{+\infty}
    {\left|u^{n}_{i}\right|}^{2}.
  \end{equation*}

  De forma recursiva, concluimos que

  \begin{equation*}
    \sum_{i=-\infty}^{\infty}
    {\left|u^{n+1}_{i}\right|}^{2}\leq
    {
      \left(
      \left|\alpha\right|+
      \left|\beta\right|
      \right)
    }^{2\left(n+1\right)}
    \sum_{i=-\infty}^{+\infty}
    {\left|u^{0}_{i}\right|}^{2}.
  \end{equation*}

  Por lo tanto, si
  \begin{math}
    \left|\alpha\right|+
    \left|\beta\right|\leq
    1
  \end{math}
  entonces el esquema numérico es estable.
  Sustituyendo los valores de $\alpha$ y $\beta$ y aplicando
  propiedades de la función valor absoluto, concluimos que si
  \begin{math}
    -1\leq
    c\frac{\Delta t}{\Delta x}\leq
    0
  \end{math}
  entonces el esquema numérico (4.11) es estable, como queríamos ver.
\end{example}

Hemos visto que a partir de la definición de estabilidad podemos
determinar condiciones suficientes que garantizan la estabilidad de
un esquema numérico.
A continuación, usaremos métodos del análisis de Fourier para
demostrar que además son condiciones necesarias.
En concreto, nos centramos en un procedimiento denominado el análisis
de Von Neumann, a partir del cual determinar la estabilidad de un
esquema numérico se reduce a consideraciones algebraicas.
Pero antes, introducimos la condición CFL, condición necesaria que
debe cumplirse en cualquier método de volumen finito si esperamos que
sea estable y que converja a la solución de la ecuación en derivadas
parciales a medida que se refine la malla.

Sea $\left(x,t\right)$ un punto fijo en el espacio-tiempo.
Como se vio en la expresión (2.10) la solución $u\left(x,t\right)$
del problema de Cauchy depende solo del dato inicial $u_{0}$ en
$x-ct$ siendo $c$ la velocidad de propagación de la ecuación
diferencial, originándose el siguiente concepto.

\begin{definition}
  El dominio de dependencia analítico o de una ecuación en derivadas
  parciales en el punto $\left(x,t\right)$ está definido por el
  conjunto

  \begin{equation*}
    D\left(x,t\right)=
    \left\{
    \left(x_{0},t_{0}\right)\in\mathbb{R}^{2}:
    x_{0}-ct_{0}=x-ct,
    0\leq t_{0}<t
    \right\}.
  \end{equation*}
\end{definition}

Se considera la ecuación de transporte, $u_t+cu_x=0$, con $c>0$, y se
aproxima su solución mediante un método explícito, por ejemplo, el
método descentrado.
Este esquema numérico viene dado por la ecuación (4.3), por tanto,
cada nuevo valor $u^{n+1}_{i}$ se calcula en función de las
aproximaciones de la solución asociadas a un paso de tiempo previo,
todas estas dentro de una región.
A esta región se le denomina el dominio de dependencia de un método
numérico, definido como el conjunto de datos iniciales que pueden
afectar a la solución numérica en el nivel de tiempo $n+1$.

En esta memoria, estudiamos esquemas numéricos de un paso.
Por lo tanto, la solución exacta que se traslada con velocidad
positiva $c$, se propaga una distancia $c\Delta t$, en un solo paso
de tiempo.
Para que en el esquema influyan solo las celdas adyacentes, se tiene
que cumplir que $c\Delta t\leq\Delta x$, esto es,
$c\frac{\Delta t}{\Delta x}\leq 1$, que es condición necesaria para
que la recta característica esté dentro del dominio de dependencia
numérico.
Además, incluyendo el caso en el que la velocidad de propagación sea
negativa, $c<0$, se tiene que cumplir que

\begin{equation*}
  \left|
  c\frac{\Delta t}{\Delta x}
  \right|\leq
  1,
\end{equation*}

donde
\begin{math}
  \left|
  c\frac{\Delta t}{\Delta x}
  \right|
\end{math}
se denomina número de Courant y mide la fracción de una celda de la
malla por la que se propaga la información en un paso de tiempo.
En general, veremos que una condición necesaria para la estabilidad
de un método de volúmenes finitos es que el dominio de dependencia
numérico contenga el dominio de dependencia analítico.
Esto se traduce en que la estabilidad requerirá una condición CFL del
tipo (4.12).
Así, en el ejemplo 4.2 vimos que el esquema numérico considerado es
estable si se verifica
\begin{math}
  -1\leq
  c\frac{\Delta t}{\Delta x}\leq
  0
\end{math}.

La condición CFL debe su nombre a los matemáticos Courant, Friedrichs
y Lewy que la introdujeron en el transcurso de la demostración de la
convergencia de un método numérico, en concreto, el método de
diferencias finitas.

\subsection{Análisis de la estabilidad de Von Neumann}

En este apartado, emplearemos el subíndice $j$ para denotar
$x_{j}=j\Delta x$ en lugar del subíndice $i$ para que no haya
confusión con el número complejo $i=\sqrt{-1}$.
En la celda $c_{j}$ y para el tiempo $t_{n}$, $u^{n}_{j}$ representa
una función definida sobre una malla para el problema de Cauchy.
Sea $\Delta x$ la longitud de las celdas de la malla,
entonces su transformada de Fourier visto viene dada por

\begin{equation*}
  \overline{u}
  \left(\xi\right)=
  \frac{1}{\sqrt{2\pi}}
  \sum_{j=-\infty}^{\infty}
  e^{-i\xi j\Delta x}
  u^{n}_{j}\Delta x,
\end{equation*}

donde