\chapter{Fundamentos básicos}\label{ch:fundaments}

Presentamos la ley de conservación lineal escalar unidimensional.
Se definen la solución asociada a un problema de Cauchy, que será
empleado para medir el error de las soluciones aproximadas.
Como problema modelo, se considera la ecuación de transporte escalar.

\section{Ley de conservación}

\begin{definition}
  Sean
  \begin{math}
    \Omega\subset
    \mathbb{R}
  \end{math}
  y
  \begin{math}
    f\colon\Omega\to
    \mathbb{R}
  \end{math}
  una
  función de clase
  \begin{math}
    \mathcal{C}^{1}
    \left(\Omega\right)
  \end{math}.
  Una ley de conservación unidimensional es

  \begin{equation}\label{eq:differentialconservationlaw}
    u_{t}
    \left(x,t\right)+
    {f\left(u\left(x,t\right)\right)}_{x}=
    0,
  \end{equation}

  donde la función

  \begin{align*}
    u\colon\mathbb{R}\times
    \left[0,+\infty\right) &
    \longrightarrow
    \Omega                   \\
    \left(x,t\right)       &
    \longmapsto
    u\left(x,t\right)
  \end{align*}

  se denomina variable conservativa, el conjunto $\Omega$ es el
  conjunto de los estados y la función $f$ es la función flujo.
\end{definition}

\begin{remark}
  La ecuación~\eqref{eq:differentialconservationlaw} se corresponde
  con la forma diferencial de la ley de conservación.
  La forma integral de la ley de conservación viene dada por la
  ecuación~\eqref{eq:integralconservationlaw}.

  \begin{equation}\label{eq:integralconservationlaw}
    \diffp{}{t}
    \int_{a}^{b}
    u\left(x,t\right)\dl x=
    F\left(a,t\right)-
    F\left(b,t\right).
  \end{equation}

  Esta refleja la base de la conservación, la variación de unas
  magnitudes en una región se debe solo a los flujos a través de los
  puntos extremos.
\end{remark}

\section{Problema de Cauchy}

Nos interesa estudiar la
ecuación~\eqref{eq:differentialconservationlaw} definida en un
dominio finito suplementada con una condición inicial que esté bien
planteada, es garantizar la existencia, unicidad de solución y la
continuidad con respecto a esta condición inicial.

\begin{definition}
  El problema de Cauchy consiste en encontrar una función

  \begin{equation*}
    u\colon\mathbb{R}\times
    \left[0,+\infty\right)\longrightarrow
    \Omega
  \end{equation*}

  que sea solución de la
  ecuación~\eqref{eq:differentialconservationlaw} y que verifique la
  condición inicial

  \begin{equation*}
    u\left(x,0\right)=
    u_{0}\left(x\right),\quad
    x\in\mathbb{R},
  \end{equation*}

  donde
  \begin{math}
    u_{0}\colon\mathbb{R}\to\Omega
  \end{math}
  es una función dada.
\end{definition}

Entonces, introducimos la solución al problema de Cauchy

\begin{definition}
  Una función
  \begin{math}
    u\colon\mathbb{R}\times
    \left[0,+\infty\right)\to
    \Omega
  \end{math}
  es una solución clásica del problema de valor inicial

  \begin{equation}\label{eq:IVPconservationlaw}
    \begin{cases}
      u_{t}
      \left(x,t\right)+
      {f\left(u\left(x,t\right)\right)}_{x}=
      0,              &
      \left(x,t\right)\in
      \mathbb{R}\times
      \left(0,+\infty\right), \\
      u\left(x,0\right)=
      u_{0}
      \left(x\right), &
      x\in\mathbb{R},
    \end{cases}
  \end{equation}

  si $u$ es una función de clase
  \begin{math}
    \mathcal{C}^{1}
    \left(
    \mathbb{R}\times
    \left[0,+\infty\right)
    \right)
  \end{math}
  que verifica el problema~\eqref{eq:IVPconservationlaw}
  puntualmente.
\end{definition}

\begin{definition}
  Sea
  \begin{math}
    u\in\mathcal{C}^{1}
    \left(
    \mathbb{R}\times
    \left[0,+\infty\right)
    \right)
  \end{math}
  una solución clásica.
  Defnimos las curvas características en
  \begin{math}
    \mathbb{R}\times
    \left[0,T\right]
  \end{math}
  como las curvas
  \begin{math}
    t\longmapsto
    \left(
    X\left(t\right),t
    \right)
  \end{math}
  dadas por la ecuación diferencial

  \begin{equation}\label{eq:characteristicsystem}
    \diff{X\left(t\right)}{t}=
    f^{\prime}
    \left(
    u\left(X\left(t\right),t\right)
    \right).
  \end{equation}
\end{definition}

Una de las particularidades de estas curvas características es que la
solución $u$, denominada variable característica, es constante a lo
largo de ellas ya que se tiene que

\begin{align*}
  \diff{
    u
    \left(
    X\left(t\right),t
    \right)
  }{t} & =
  u_{x}
  \left(
  X\left(t\right),t
  \right)\cdot
  X^{\prime}
  \left(t\right)+
  u_{t}
  \left(
  X\left(t\right),t
  \right)  \\
       & =
  u_{x}
  \left(
  X\left(t\right),t
  \right)\cdot
  f^{\prime}
  \left(
  u\left(X\left(t\right),t\right)
  \right)+
  u_{t}
  \left(
  X\left(t\right),t
  \right)  \\
       & =
  f_{x}
  \left(
  u\left(X\left(t\right),t\right)
  \right)+
  u_{t}
  \left(
  X\left(t\right),t
  \right)=
  0,
\end{align*}

por ser $u$ solución de~\eqref{eq:differentialconservationlaw}.

En el caso lineal, la función flujo de la ley de conservación viene
dada por $f\left(u\right)=cu$, con $c\in\mathbb{R}$.
Por tanto, tenemos que $f^{\prime}\left(u\right)=c$ y de la
ecuación~\eqref{eq:characteristicsystem} se deduce que las curvas
características son rectas dadas por $X\left(t\right)=x_{0}+ct$.
Como la solución $u$ es constante a lo largo de estas rectas, la
solución en el punto $\left(x,t\right)$ es igual a la solución en el
punto $\left(x_{0},0\right)$.
De forma que
\begin{math}
  \left(x,t\right)=
  \left(
  X\left(t\right),t
  \right)=
  \left(x_0+ct,t\right)
\end{math},
de donde $x_{0}=x-ct$.
Entonces, se obtiene

\begin{equation*}
  u\left(x,t\right)=
  u\left(x_{0},0\right)=
  u\left(x-ct,0\right)=
  u_{0}\left(x-ct\right),
\end{equation*}

que es la única solución del problema~\eqref{eq:IVPconservationlaw}.
Luego, para todos los valores iniciales existe una única solución que
tiene la misma regularidad.
A continuación, vemos un ejemplo de este tipo de problema.

\begin{example}
  Consideremos la siguiente ecuación en derivadas parciales

  \begin{equation*}
    u_{t}+
    \lambda u_{x}=
    0,
  \end{equation*}

  donde $x$ es la coordenada espacial y $t$ representa el tiempo.
  Esta es la llamada ecuación de transporte o de advección en una
  dimensión.
  Asumimos que la velocidad de propagación es positiva, $\lambda>0$.
  Como condición inicial, tomamos
  \begin{math}
    u\left(x,0\right)=
    u_{0}\left(x\right)
  \end{math}
  para $x\in\mathbb{R}$.
  Generalmente, se suele elegir para la condición inicial el instante
  inicial, $t=0$.

  La solución al problema de Cauchy

  \begin{equation*}
    \begin{cases}
      u_{t}
      \left(x,t\right)+
      \lambda
      u_{x}
      \left(x,t\right)=
      0,                   &
      \left(x,t\right)\in
      \mathbb{R}\times
      \left(0,+\infty\right), \\
      u\left(x,0\right)=
      u_{0}\left(x\right), &
      x\in\mathbb{R}.
    \end{cases}
  \end{equation*}
\end{example}

$u$ simplemente se transporta con velocidad constante $\lambda$ con
la evolución del tiempo, como se muestra en la Figura~\ref{fig:1}.
En este caso, se transporta hacia la derecha, puesto que
anteriormente asumimos que $\lambda>0$.
En particular, definimos $\lambda=1$ y la condición inicial dada por
\begin{math}
  u_{0}
  \left(x\right)=
  e^{-50{\left(x-0.3\right)}^{2}}
\end{math}.

\begin{figure}[ht!]
  \centering
  \includegraphics[width=.4\paperwidth]{figure1}
  \caption{
    Evolución de la solución de la ecuación de transporte con el
    tiempo.
  }
  \label{fig:1}
\end{figure}

\section{Problema de Riemann}

El problema de Cauchy correspondiente a una ley de conservación con
una condición inicial dada por una función constante a trozos con una
discontinuidad se conoce como problema de Riemann.

\begin{definition}
  El problema de Riemann consiste en encontrar una función

  \begin{equation*}
    u\colon\mathbb{R}\times
    \left[0,+\infty\right)\longrightarrow
    \Omega
  \end{equation*}

  solución de la ecuación~\eqref{eq:differentialconservationlaw} que
  verifica la condición inicial

  \begin{equation*}
    u_{0}\left(x\right)=
    \begin{cases}
      u_{l}, &
      \text{si } x<x_{0}, \\
      u_{r}, &
      \text{si } x>x_{0},
    \end{cases}
  \end{equation*}

  donde $u_{l},u_{r}\in\mathbb{R}$ vienen~dados.
  A este problema se le denota por
  $\operatorname{PR}\left(u_{l},u_{r}\right)$ centrado en $x_{0}$.
\end{definition}

Para una ecuación unidimensional, $u_{t}+cu_{x}=0$, la solución al
problema de Riemann centrado en el $x_{0}$ y con datos iniciales
$\left(u_{l},u_{r}\right)$ viene dada por
$u\left(x,t\right)=u_{0}\left(x-ct\right)$, es decir,

\begin{equation*}
  u
  \left(x,t\right)=
  \begin{cases}
    u_{l}, &
    \text{si } x-ct<x_{0}, \\
    u_{r}, &
    \text{si } x-ct>x_{0}.
  \end{cases}
\end{equation*}
