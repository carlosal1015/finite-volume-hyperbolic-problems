\chapter{Condición de salto Rankine-Hugoniot}

\begin{theorem}
  El salto $u_{r}-u_{l}$, a través de la discontinuidad, es un
  múltiplo escalar del autovector asociado al autovalor $\lambda$ de
  la ecuación $u_{t}+\lambda u_{x}=0$.
\end{theorem}

\begin{proof}
  Sabemos que $\lambda$ es la velocidad con la que se propaga esta
  discontinuidad, asumimos que es negativa.
  Estudiémosla en el rectángulo de espacio y tiempo
  \begin{math}
    \left[x_{1},x_{1}+\Delta x\right]\times
    \left[t_{1},t_{1}+\Delta t\right]
  \end{math}.
  Como se muestra en la Figura~\ref{fig:2}

  \begin{figure}[ht!]
    \centering
    \includegraphics[width=.4\paperwidth]{figure2}
    \caption{Región infinitesimal rectangular en el plano $xt$.}
    \label{fig:2}
  \end{figure}

  Aplicamos la forma integral de la ley de
  conservación~\eqref{eq:integralconservationlaw} a esta región,
  entonces obtenemos

  \begin{equation*}
    \diffp{}{t}
    \int_{x_{1}}^{x_{1}+\Delta x}
    u\left(x,t\right)\dl x=
    f\left(u\left(x_{1},t\right)\right)-
    f\left(u\left(x_{1}+\Delta x,t\right)\right).
  \end{equation*}

  Integrando en tiempo en el intervalo
  $\left[t_{1},t_{1}+\Delta t\right]$ tenemos
  \begin{equation*}
    \int_{x_{1}}^{x_{1}+\Delta x}
    u\left(x,t_{1}+\Delta t\right)-
    \int_{x_{1}}^{x_{1}+\Delta x}
    u\left(x,t_{1}\right)=
    \int_{t_{1}}^{t_{1}+\Delta t}
    f\left(u\left(x_{1},t\right)\right)
    \dl t-
    \int_{t_{1}}^{t_{1}+\Delta t}
    f\left(u\left(x_{1}+\Delta x,t\right)\right)
    \dl t
  \end{equation*}

  Entonces, como la función flujo es lineal y $u$ es constante en
  cada porción del plano,

  \begin{equation}\label{eq:difference}
    \Delta x u_{r}-
    \Delta x u_{l}=
    \Delta t f\left(u_{l}\right)-
    \Delta t f\left(u_{r}\right).
  \end{equation}

  Dado que $\lambda$ es la velocidad de propagación, negativa,
  entonces $\Delta x=-\lambda\Delta t$.
  Dividiendo la igualdad~\eqref{eq:difference} entre $-\Delta t$, y
  como $\Delta x=-\lambda\Delta t$, obtenemos

  \begin{equation*}
    \lambda\left(u_{r}-u_{l}\right)=
    f\left(u_{r}\right)-
    f\left(u_{l}\right),
  \end{equation*}

  y por lo tanto el salto $u_{r}-u_{l}$ es múltiplo del autovector
  asociado a $\lambda$ como queríamos ver.
  Esta ecuación se denomina condición de salto de Rankine-Hugoniot.
\end{proof}