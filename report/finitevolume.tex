\chapter{Método de Volúmenes Finitos}\label{ch:finitevolume}

En este capítulo estudiamos los métodos de volúmenes finitos para
resolver leyes de conservación, derivados de su forma integral, como
punto de partida.
Nos centraremos en los métodos de primer orden para ecuaciones
lineales, en concreto, en los métodos descentrados basados en la
solución del problema de Cauchy y en el método de Godunov basado en
la solución del problema de Riemann.

\section{Método de Volúmenes Finitos}

El Método de Volúmenes Finitos en una dimensión espacial consiste en
dividir el dominio espacial en intervalos, los volúmenes finitos,
también denominados celdas, y calcular en cada uno de ellos una
aproximación de la integral de la variable conservativa $u$.
En cada paso de tiempo, se actualizan estos valores utilizando
aproximaciones del flujo a través de las fronteras de los volúmenes
finitos.
Por tanto, este método numérico permite la resolución de leyes de
conservación haciendo uso de dicha conservación en las distintas
celdas en las que podemos dividir el dominio.
En esta sección vamos a ver cómo se lleva a cabo este método
numérico.

Comenzamos definiendo una partición del espacio
$\Omega\in\mathbb{R}$, es decir, se define un conjunto de celdas o
volúmenes finitos

\begin{equation*}
  c_{i}=
  \left(
  x_{i-\frac{1}{2}},
  x_{i+\frac{1}{2}}
  \right),\quad i=1,\cdots,N_{x}
\end{equation*}

cuyos puntos extremos serían

\begin{equation*}
  x_{\frac{1}{2}}<
  x_{1+\frac{1}{2}}<
  \cdots<
  x_{N_{x}-\frac{1}{2}}<
  x_{N_{x}+\frac{1}{2}},
\end{equation*}

y pertenecen al intervalo $\left(a,b\right)$.
Por simplicidad, asumimos que la longitud de las celdas es constante,
es decir,
\begin{math}
  \Delta x=
  \left|c_{i}\right|=
  x_{i+\frac{1}{2}}-
  x_{i-\frac{1}{2}}
\end{math}.
En la figura 3.1, se muestran dos ejemplos de celdas.


En esta memoria, denotamos por uni a la aproximación del valor medio
de $u$ sobre la celda $c_{i}$ en el tiempo $t_{n}$, es decir:

\begin{equation*}
  u^{n}_{i}\approx
  \frac{1}{\Delta x}
  \int_{c_{i}}
  u\left(x,t_{n}\right)\dl x=
  \frac{1}{\Delta x}
  \int_{x_{i-\frac{1}{2}}}^{x_{i+\frac{1}{2}}}
  u\left(x,t_{n}\right)\dl x.
\end{equation*}

El hecho de centrarnos en estos valores promedios nos facilita
deducir métodos numéricos que mimeticen propiedades importantes de
las leyes de conservación.
En particular, se puede asegurar que el método numérico es
conservativo en un sentido que imite a la solución exacta.
Esto es debido a que $\sum_{i=1}^{N}\Delta x\cdot u^{n}_{i}$
aproxima la integral de $u$ sobre el dominio espacial
$\left[a,b\right]$, es decir,

\begin{equation*}
  \int_{a}^{b}
  u\left(x,t_{n}\right)\dl x\approx
  \sum_{i=1}^{N}
  \Delta x\cdot u^{n}_{i}.
\end{equation*}

Veremos que si expresamos un método numérico en la forma
conservativa, la suma discreta anterior,
$\sum_{i=1}^{N}\Delta x\cdot u^{n}_{i}$, cambiará solamente debido a
los valores en las fronteras $x=a$ y $x=b$.
En concreto, la identidad~\eqref{eq:conservative} que aparece más
adelante, cuando el rango de valores de celdas definido por $I$, $J$
recubre toda la partición, puede interpretarse como la conservación
de la solución discreta.

Ahora podemos expresar la forma integral de la ley de conservación en
la celda $c_{i}$ como

\begin{equation}\label{eq:integralformcell}
  \diffp{}{x}
  \int_{c_{i}}
  u\left(x,t\right)\dl x=
  f\left(u\left(x_{i-\frac{1}{2}}\right),t\right)-
  f\left(u\left(x_{i+\frac{1}{2}}\right),t\right).
\end{equation}

Una vez que se ha discretizado el dominio espacial, procedemos a
hacerlo con el temporal.
Consideremos una partición uniforme en el tiempo,

\begin{equation*}
  0=t_{0}<
  t_{1}<
  \cdots<
  t_{N_{t}}=
  T,
\end{equation*}

siendo $\Delta t=t_{n+1}-t_{n}$.
El paso en tiempo lo consideraremos constante.
Ahora, integramos la expresión~\eqref{eq:integralformcell} en cada
intervalo de tiempo $\left[t_{n},t_{n+1}\right]$:

\begin{equation}\label{eq:integrationeachinterval}
  \int_{c_{i}}
  u\left(x,t_{n+1}\right)\dl x-
  \int_{c_{i}}
  u\left(x,t_{n}\right)\dl x=
  \int_{t_{n}}^{t_{n+1}}
  \left[
    f\left(u\left(x_{i-\frac{1}{2}}\right),t\right)-
    f\left(u\left(x_{i+\frac{1}{2}}\right),t\right)
    \right]\dl t.
\end{equation}

Dividiendo la expresión~\eqref{eq:integrationeachinterval} entre
$\Delta x$ y reordenando los términos, se obtiene

\begin{gather}\label{eq:expressiondividedoverdeltax}
  \begin{split}
    \frac{1}{\Delta x}
    \int_{c_{i}}
    u\left(x,t_{n+1}\right)\dl x
     & =
    \frac{1}{\Delta x}
    \int_{c_{i}}
    u\left(x,t_{n}\right)\dl x \\
     & \phantom{=}
    -\frac{1}{\Delta x}
    \int_{t_{n}}^{t_{n+1}}
    \left[
      f\left(u\left(x_{i-\frac{1}{2}}\right)\right)-
      f\left(u\left(x_{i+\frac{1}{2}}\right)\right)
      \right]\dl t.
  \end{split}
\end{gather}

Debido a que, en general, no podemos evaluar exactamente las
integrales del segundo miembro de la
expresión~\eqref{eq:expressiondividedoverdeltax}, ya que
\begin{math}
  u
  \left(x_{i}+1,t\right)
\end{math}
y
\begin{math}
  u
  \left(x_{i}-1,t\right)
\end{math}
varían con el
tiempo a lo largo de cada lado de la celda, y a que no conocemos la
solución exacta, vamos a estudiar métodos numéricos de la forma

\begin{equation}\label{eq:godunovscheme}
  u^{n+1}_{i}=
  u^{n}_{i}-
  \frac{\Delta t}{\Delta x}
  \left(
  F^{n}_{i+\frac{1}{2}}-
  F^{n}_{i-\frac{1}{2}}
  \right),
\end{equation}

donde $F^{n}_{i+\frac{1}{2}}$ es una aproximación del valor medio del
flujo en el plano $xt$ a lo largo de la recta $x=x_{i}+\frac{1}{2}$
con $t$ variando entre $t_{n}$ y $t_{n+1}$,es decir,

\begin{equation}\label{eq:numericalflux}
  F^{n}_{i+\frac{1}{2}}\approx
  \frac{1}{\Delta t}
  \int_{t_{n}}^{t_{n+1}}
  f
  \left(
  u
  \left(x_{i+\frac{1}{2}},t\right)
  \right)
  \dl t.
\end{equation}

Como se ha podido comprobar la expresión~\eqref{eq:integralformcell}
ha sido utilizada para el desarrollo de un método explícito en
tiempo: dado $u^{n}_{i}$, es decir, dadas las aproximaciones
promedios en las celdas en el tiempo $t_{n}$, aproximamos
$u^{n+1}_{i}$ que son los valores promedios en las celdas en el
tiempo $t_{n+1}$,esto es, después de un paso de tiempo de longitud
$\Delta t=t_{n+1}-t_{n}$.

Una consideración interesante es que, reordenando los términos en la
expresión~\eqref{eq:godunovscheme}, obtenemos

\begin{equation*}
  \frac{u^{n+1}_{i}-u^{n}_{i}}{\Delta t}+
  \frac{F^{n}_{i+\frac{1}{2}}-F^{n}_{i-\frac{1}{2}}}{\Delta x}=
  0,
\end{equation*}

de forma que este método puede verse como una aproximación en
Diferencias Finitas de la ley de conservación
$u_{t}+f\left(u\right)_{x}=0$.

Dado que en los problemas hiperbólicos la información se propaga a
velocidad finita, parece razonable suponer que podemos obtener
$F^{n}_{i+\frac{1}{2}}$ basándonos únicamente en los valores de
$u^{n}_{i-1}$ y $u^{n}_{i}$, valores promedios de $u^{n}$ en las
celdas $c_{i-1}$ y $c_{i}$ que tienen extremo común
$x_{i-\frac{1}{2}}$.
Entonces, podríamos escribir

\begin{equation*}
  F^{n}_{i-\frac{1}{2}}=
  \phi
  \left(
  u^{n}_{i-1},
  u^{n}_{i}
  \right),
\end{equation*}

donde $\phi$ es una cierta función flujo numérico.
Por tanto, el esquema numérico~\eqref{eq:godunovscheme} lo podemos
expresar de la siguiente manera

\begin{equation}\label{eq:onestepscheme}
  u^{n+1}_{i}=
  u^{n}_{i}-
  \frac{\Delta t}{\Delta x}
  \left(
  \phi\left(u^{n}_{i},u^{n}_{i+1}\right)-
  \phi\left(u^{n}_{i-1},u^{n}_{i}\right)
  \right).
\end{equation}

En las secciones posteriores, se estudiarán distintos métodos
numéricos y se verá que todos ellos pueden expresarse según la
elección de la expresión de $\phi$.
Algunas de las propiedades que deberían verificar este tipo de flujo
numérico son las siguientes:

\begin{itemize}
  \item

        El flujo numérico debería proximar la integral~\eqref{eq:numericalflux}.
        En particular, si la variable conservativa es constante en
        $x$, esto es, $u\left(x,t\right)\equiv\overline{u}$ con
        $\overline{u}\in\mathbb{R}$, entonces la
        integral~\eqref{eq:numericalflux} se reduce a
        $f\left(\overline{u}\right)$.
        Por lo tanto, se tiene que
        \begin{math}
          \phi
          \left(
          \overline{u},
          \overline{u}
          \right)=
          f\left(\overline{u}\right).
        \end{math}

  \item

        La función flujo numérico,
        \begin{math}
          \phi\left(u^{n}_{i},u^{n+1}_{i}\right)
        \end{math}
        tiene que ser continua en función de la
        variación de $u^{n}_{i}$ y $u^{n+1}_{i}$,
        esto es,

        \begin{equation*}
          \lim_{u^{n}_{i},u^{n+1}_{i}\overline{u}}
          \phi\left(u^{n}_{i},u^{n+1}_{i}\right)=
          f\left(\overline{u}\right).
        \end{equation*}
\end{itemize}

El principio básico de una ley de conservación es que la cantidad
total de una variable conservada solamente cambia en función del
flujo en la frontera.
A continuación, vamos a comprobar que los esquemas en forma
conservativa también cumplen una propiedad análoga.
Partiendo del esquema~\eqref{eq:godunovscheme} sumemos $u^{n+1}_{i}$
sobre cualquier subconjunto de celdas $i=I,\dotsc,J$ y multipliquemos
por $\Delta x$:

\begin{equation*}
  \Delta x
  \sum_{i=I}^{J}
  u^{n+1}_{i}=
  \Delta x
  \sum_{i=I}^{J}
  u^{n}_{i}-
  \Delta t
  \left(
  \sum_{i=I}^{J}
  F^{n}_{i+\frac{1}{2}}-
  \sum_{i=I}^{J}
  F^{n}_{i-\frac{1}{2}}
  \right).
\end{equation*}

Como la suma de los flujos se cancela salvo los flujos extremos
$x=x_{I-\frac{1}{2}}$ y $x=x_{J+\frac{1}{2}}$, obtenemos que

\begin{equation}\label{eq:conservative}
  \Delta x
  \sum_{i=I}^{J}
  u^{n+1}_{i}=
  \Delta x
  \sum_{i=I}^{J}
  u^{n}_{i}-
  \Delta t
  \left(
  F^{n}_{J+\frac{1}{2}}-
  F^{n}_{I-\frac{1}{2}}
  \right),
\end{equation}

razón por la cual decimos que el método de volúmenes finitos es
conservativo.

\section{Métodos descentrados}

La solución del Problema de Valores Iniciales (2.8) es constante a lo
largo de curvas características.
En el caso lineal, a lo largo de rectas características, pudiéndose
diferenciar dos casos en función del signo de la velocidad de
propagación (figuras 3.2 y 3.3).


Las rectas características son usadas para determinar mejores
funciones de flujo numérico, dando lugar a los métodos descentrados.
Veamos un ejemplo de este tipo de método en la ecuación de
transporte.

\subsection*{Método descentrado para la ecuación de transporte}

La ecuación de transporte viene dada por $u_{t}+cu_{x}=0$.
En primer lugar, supongamos que la información se propaga con
velocidad positiva, $c>0$.
Como se muestra en la figura 3.2, el flujo a través del extremo
izquierdo de la celda $c_{i}$ está completamente determinado por el
valor $u^{n}_{i-1}$.
Por tanto, el flujo numérico a lo largo de la recta
$x=x_{i}-\frac{1}{2}$ se define como

\begin{equation*}
  F^{n}_{i-\frac{1}{2}}=
  cu^{n}_{i-1}.
\end{equation*}

En cambio, el flujo a través del extremo derecho de la celda $c_{i}$
está completamente determinado por el valor uni en esta misma celda.
Luego, el flujo numérico a lo largo de la recta $x=x_{i+\frac{1}{2}}$
se define como

\begin{equation*}
  F^{n}_{i+\frac{1}{2}}=
  cu^{n}_{i}.
\end{equation*}

Sustituyendo en la ecuación~\eqref{eq:godunovscheme}, el método
descentrado para la ecuación de transporte con velocidad de
propagación positiva, $c>0$, viene dado por

\begin{equation}\label{eq:upwind}
  u^{n+1}_{i}=
  u^{n}_{i}-
  \frac{c\Delta t}{\Delta x}
  \left(
  u^{n}_{i}-
  u^{n}_{i-1}
  \right).
\end{equation}

denominado método descentrado upwind.

Por otro lado, en el caso de que la información se propague con
velocidad negativa $c<0$, como se muestra en la figura 3.3, siguiendo
el mismo razonamiento que para el caso anterior, resulta que el
método descentrado viene dado por el siguiente esquema numérico

\begin{equation}\label{eq:downwind}
  u^{n+1}_{i}=
  u^{n}_{i}-
  \frac{c\Delta t}{\Delta x}
  \left(
  u^{n}_{i+1}-
  u^{n}_{i}
  \right),
\end{equation}

donde
\begin{math}
  F^{n}_{i+\frac{1}{2}}=
  cu^{n}_{i+1}
\end{math}
y
\begin{math}
  F^{n}_{i-\frac{1}{2}}=
  cu^{n}_{i}
\end{math}
denominado método descentrado downwind.

A medida que el tiempo evoluciona, como la solución es constante a lo
largo de las rectas características, se tiene que

\begin{equation*}
  u^{n+1}_{i}\approx
  u
  \left(x_{i},t_{n+1}\right)=
  u
  \left(
  x_{i}-
  c\left(t_{n+1}-t_{n}\right),
  t_{n}
  \right)=
  u
  \left(
  x_{i}-
  c\Delta t,t_{n}
  \right).
\end{equation*}

Asumiendo que $c>0$, el salto
\begin{math}
  W_{i}-\frac{1}{2}\coloneqq
  u^{n}_{i}-
  u_{i-1}^{n}
\end{math}
en cada uno de los bordes de las celdas, $c_{i}$, sera el mismo a
medida que el tiempo evolucione pero desplazado una distancia
$c\Delta t$ en cada paso de tiempo.
Por tanto, el esquema~\eqref{eq:upwind} se puede escribir de esta
otra forma

\begin{equation*}
  u^{n+1}_{i}=
  u^{n}_{i}-
  \frac{c\Delta t}{\Delta x}
  W_{i-\frac{1}{2}}.
\end{equation*}

Por otro lado, en el caso que $c<0$, definimos el salto
$W_{i+\frac{1}{2}}\coloneqq u^{n}_{i+1}-u^{n}_{i}$, resultando el
esquema~\eqref{eq:downwind} como

\begin{equation*}
  u^{n+1}_{i}=
  u^{n}_{i}-
  \frac{c\Delta t}{\Delta x}
  W_{i+\frac{1}{2}}.
\end{equation*}

En la figura 3.4, se muestra lo estudiado en esta sección asumiendo
que $c>0$.
Obsérvese que el uso de los mismos colores a medida que el tiempo
evoluciona, representa la solución constante para cada celda
considerada, esto es,
\begin{math}
  u
  \left(
  x_{i},
  t_{n+1}
  \right)=
  u
  \left(
  x_{i}-c\Delta t,
  t_{n}
  \right)
\end{math}.
Por tanto, el salto producido en cada borde entre celdas será el
mismo en cada uno de los instantes de tiempo.

\section{Método de Godunov}

En esta sección, se presenta el método de Godunov en sus dos
versiones y se explica detalladamente cómo puede aplicarse en el
caso lineal.
La particularidad de dicho método es el uso de la estructura de ondas
determinada por la solución del problema de Riemann.

\subsection{Primera versión}

Se describe el método en tres pasos, pudiéndose referenciar cada uno
de ellos con una palabra, respectivamente: reconstruir, evolucionar
y promediar.

\begin{description}
  \item[Paso 1]

        Se reconstruye una función constante a trozos definida por:

        \begin{equation*}
          \widetilde{u}
          \left(x,t_{n}\right)=
          u^{n}_{i},
        \end{equation*}

        para $x$ en cada celda $c_{i}$ , entendiéndose como el
        promedio de la solución en cada celda, en el tiempo $t_{n}$,
        como se muestra en la figura 3.5.

  \item[Paso 2]

        En el nivel de tiempo $n$, en cada frontera
        $x_{i+\frac{1}{2}}$ entre las celdas $c_{i}$ y $c_{i+1}$,
        tenemos el problema local de Riemann centrado en la frontera
        $x_{i+\frac{1}{2}}$ y con datos iniciales
        \begin{math}
          \left(
          u^{n}_{i},
          u^{n}_{i+1}
          \right)
        \end{math}.
        El objetivo es encontrar la solución del problema global en
        un nivel de tiempo posterior $n+1$, es decir, evolucionar en
        el tiempo.
        Para ello, Godunov propone resolver dos problemas de Riemann
        locales, por un lado, el problema de Riemann centrado en
        $x_{i-\frac{1}{2}}$ con datos iniciales
        \begin{math}
          \left(
          u^{n}_{i-1},u^{n}_{i}
          \right)
        \end{math}
        y, por otro lado, el problema de Riemann centrado en
        $x_{i+\frac{1}{2}}$ con datos soluciones iniciales
        \begin{math}
          \left(
          u^{n}_{i},
          u^{n}_{i+1}
          \right)
        \end{math}.
        Denotamos
        \begin{math}
          \widetilde{u}
          \left(
          x_{i-\frac{1}{2}},
          t_{n+1}
          \right)
        \end{math}
        y
        \begin{math}
          \widetilde{u}
          \left(
          x_{i+\frac{1}{2}},
          t_{n+1}
          \right)
        \end{math},
        respectivamente, las soluciones citadas.

  \item[Paso 3]

        Se obtiene $u^{n+1}_{i}$ como el promedio de las soluciones
        de los problemas de Riemann correspondientes en la celda
        $c_{i}$, es decir:

        \begin{equation}\label{eq:godunovmean}
          u^{n+1}_{i}=
          \frac{1}{\Delta x}
          \int_{x_{i-\frac{1}{2}}}^{x_{i}}
          \widetilde{u}
          \left(x_{i-\frac{1}{2}},t_{n+1}\right)
          \dl x+
          \frac{1}{\Delta x}
          \int_{x_{i}}^{x_{i+\frac{1}{2}}}
          \widetilde{u}
          \left(
          x_{i+\frac{1}{2}},t_{n+1}
          \right)
          \dl x.
        \end{equation}
\end{description}

Todo este proceso se repite en el siguiente paso de tiempo.
Proceso correspondido a la primera versión del método de Godunov, en
el cual se ha definido un nuevo valor promedio $u^{n+1}_{i}$ en el
tiempo $t_{n+1}=t_{n}+\Delta t$ dentro de cada celda $c_{i}$.
Pero para realizar el promedio, ninguna onda puede tener interación
con otra dentro de cada celda $c_{i}$ en el tiempo
$t\in\left[t_{n},t_{n+1}\right]$.
Esto implica que el paso de tiempo $\Delta t$ sea lo suficientemente
pequeño para que las ondas de dos problemas de Riemann adyacentes no
interactúen entre sí.
Este es uno de los principales inconvenientes de esta versión debido
a que la condición CFL, que estudiaremos en la sección 4.3.1, tiene
que ser muy restringida, tomar $\Delta t$ muy pequeño.
Además, tener que realizar la evaluación de las
integrales~\eqref{eq:godunovmean}, puede ser complicado.
Luego, estas dos desventajas conllevan la introducción de una segunda
versión del método.

\subsection{Segunda versión}

En esta versión, se desarrolla un método de volúmenes finitos basado
en la función constante por partes reconstruida en el paso 1 de la
versión anterior.
Resulta fácil determinar la función flujo numérico correspondiente al
método de Godunov.

Como ya sabemos $F^{n}_{i+\frac{1}{2}}$ es una aproximación del valor
medio del flujo en el plano $xt$ a lo largo de la recta
$x=x_{i+\frac{1}{2}}$ con $t$ variando entre $t_{n}$ y $t_{n+1}$,
es decir,

\begin{equation*}
  F^{n}_{i+\frac{1}{2}}\approx
  \frac{1}{\Delta t}
  \int_{t_{n}}^{t_{n+1}}
  f\left(u\left(x_{i+\frac{1}{2}},t\right)\right)
  \dl t.
\end{equation*}

Ahora, se reemplaza $u\left(x,t\right)$ por en la función
$\widetilde{u}\left(x,t_{n}\right)$, que era la reconstrucción
constante por partes de Godunov, definida en el paso 1 de la versión
anterior.
De esta forma, el integrando
\begin{math}
  f
  \left(
  u
  \left(
    x_{i+\frac{1}{2}},
    t
    \right)
  \right)
\end{math}
en cada frontera depende de la solución exacta
\begin{math}
  \widetilde{u}
  \left(
  x_{i+\frac{1}{2}},t
  \right)
\end{math}
del problema de Riemann correspondiente en el intervalo de tiempo
\begin{math}
  \left[
    t_{n},
    t_{n+1}
    \right]
\end{math}.
Entonces, la solución se evalúa a lo largo del límite entre celdas.
Si denotamos
\begin{math}
  u^{\downarrow}
  \left(u_{i},u_{i+1}\right)\coloneqq
  \widetilde{u}
  \left(
  x_{i+\frac{1}{2}},
  t
  \right)
\end{math},
el flujo numérico $F^{n}_{i+\frac{1}{2}}$ viene dado por

\begin{equation}\label{eq:godunovfluxright}
  F^{n}_{i+\frac{1}{2}}=
  \frac{1}{\Delta t}
  \int_{t_{n}}^{t_{n+1}}
  f
  \left(
  u^{\downarrow}
  \left(u_{i},u_{i+1}\right)
  \right)
  \dl t=
  f
  \left(
  u^{\downarrow}
  \left(u_{i},u_{i+1}\right)
  \right)
\end{equation}

Razonando de igual forma, el flujo numérico $F^{n}_{i-\frac{1}{2}}$
viene dado por

\begin{equation}\label{eq:godunovfluxleft}
  F_{i-\frac{1}{2}}^{n}=
  \frac{1}{\Delta t}
  \int_{t_{n}}^{t_{n+1}}
  f
  \left(
  u^{\downarrow}
  \left(u_{i-1},u_{i}\right)
  \right)
  \dl t=
  f
  \left(
  u^{\downarrow}
  \left(u_{i-1},u_{i}\right)
  \right).
\end{equation}

Por tanto, el esquema numérico~\eqref{eq:godunovscheme} para el método de Godunov resulta:

\begin{equation*}
  u^{n+1}_{i}=
  u^{n}_{i}-
  \frac{\Delta t}{\Delta x}
  \left(
  f
  \left(
    u^{\downarrow}
    \left(u_{i},u_{i+1}\right)
    \right)-
  f
  \left(
    u^{\downarrow}
    \left(u_{i-1},u_{i}\right)
    \right)
  \right).
\end{equation*}

En esta última versión, han sido resueltos los problema de Riemann
centrados en $x_{i-\frac{1}{2}}$ y $x_{i+\frac{1}{2}}$ para obtener
$u^{\downarrow}\left(u_{i-1},u_{i}\right)$ y
$u^{\downarrow}\left(u_{i},u_{i+1}\right)$, respectivamente.
Entonces, son definidos los flujos $F^{n}_{i-\frac{1}{2}}$ y
$F^{n}_{i+\frac{1}{2}}$ por las
expresiones~\eqref{eq:godunovfluxleft} y~\eqref{eq:godunovfluxright}
respectivamente, y se obtiene os flujos $F^{n}_{i-\frac{1}{2}}$ y
$F^{n}_{i+\frac{1}{2}}$ por las
expresiones~\eqref{eq:godunovfluxleft} y~\eqref{eq:godunovfluxright}
respectivamente, y se obtiene $u^{n+1}_{i}$ mediante el esquema
numérico~\eqref{eq:godunovscheme}.