\chapter{Conclusiones}\label{ch:conclusions}

En este trabajo hemos estudiado el método de volúmenes finitos para
problemas hiperbólicos.
Para ello, que en el primer capítulo se ha definido el problema de
Cauchy, así como el de Riemann en particular en el contexto de una
ley conservativa.
En ambos problemas hemos hallado la solución analítica.
En el segundo capítulo, hemos realizado el estudio de los métodos de
volúmenes finitos derivados de la forma integral de la ley de
conservación, como punto de partida.
Nos hemos centrado en los métodos de un paso para problemas
hiperbólicos lineales.
Más concretamente en los esquemas descentrados, deducidos por conocer
el comportamiento de la solución del problema de Cauchy a lo largo de
las curvas características.
Además del método de Godunov, que nos permite también conectar con
los resultados estudiados en el capítulo anterior, basado en la
solución del problema de Riemann.
En el tercer capítulo una vez estudiado el método de volúmenes
finitos, se han introducido una variedad de esquemas numéricos y
hemos estudiado con qué grado de precisión se aproximan a la solución
de la ecuación de transporte, ley de conservación, para cada uno de
ellos.
En ella se establecieron propiedades importantes que un método
numérico eficiente debe verificar, como lo son el error de
truncamiento local, la consistencia, la estabilidad y
consecuentemente, la convergencia.
Estas propiedades han sido de interés a la hora de realizar los
distintos experimentos numéricos.
En el cuarto capítulo, una vez finalizado el marco teórico de este
trabajo, comenzamos su aplicación práctica para resolver ciertos
problemas.
En primer lugar, se resolvieron los esquemas numéricos, para la
ecuación de transporte, que fueron introducidos en el tercer
capítulo.
Todos estos experimentos numéricos son programados usando el lenguaje
de programación Python.
En los test numéricos presentados, comparamos la solución exacta de
la ecuación de transporte y su solución numérica mediante cada uno de
los esquemas numéricos estudiados.
En el primer ejemplo, resolviendo un problema de Cauchy con dato
inicial muy regular y en el segundo, un problema de Riemann, por en
cuanto se muestran la ventaja que tienen cada uno de los esquemas.