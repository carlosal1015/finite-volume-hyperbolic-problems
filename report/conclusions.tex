\chapter{Conclusiones}\label{ch:conclusions}

En este trabajo hemos estudiado el método de volúmenes finitos para
problemas hiperbólicos.
Para ello, en el capítulo 2, se ha definido el problema de Cauchy
para una ley de conservación y de Riemann en particular.
Para ambos problemas, se ha estudiado su solución.
Posteriormente, en el capítulo 3, comenzamos con el estudio de los
métodos de volúmenes finitos derivados de la forma integral de la ley
de conservación, como punto de partida.
Nos hemos centrado en los métodos de un paso para problemas
hiperbólicos lineales.
Más concretamente en los esquemas descentrados, deducidos por conocer
el comportamiento de la solución del problema de Cauchy a lo largo de
las curvas características.
Además del método de Godunov, que nos permite también conectar con
los resultados estudiados en el capítulo anterior, basado en la
solución del problema de Riemann.
Una vez estudiado el método de volúmenes finitos, se han introducido
una variedad de esquemas numéricos y hemos estudiado con qué grado de
precisión se aproximan a la solución de la ecuación de transporte,
ley de conservación, para cada uno de ellos (capítulo 4).
Entonces, se establecieron propiedades importantes que un método
numérico eficiente debe verificar, como lo son el error de
consistencia, la consistencia, la estabilidad y consecuentemente, la
convergencia.
Estas propiedades han sido de interés a la hora de realizar los
distintos experimentos numéricos.
Por último, una vez finalizado el marco teórico de este trabajo,
comenzamos su aplicación práctica para resolver ciertos problemas.
En primer lugar se resolvieron los esquemas numéricos, para la
ecuación de transporte, que fueron introducidos en el capítulo 4.
Todos estos experimentos numéricos son programados usando el lenguaje
de programación Python.
En los test numéricos presentados, comparamos la solución exacta de
la ecuación de transporte y su solución numérica mediante cada uno de
los esquemas numéricos estudiados.
En el primer ejemplo, resolviendo un problema de Cauchy con dato
inicial muy regular y en el segundo, un problema de Riemann, entonces
se muestra la ventaja que tienen cada uno de los esquemas.