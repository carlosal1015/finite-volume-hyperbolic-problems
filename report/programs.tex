\chapter{Programas empleados}

En este apéndice mostraremos los programas en el lenguaje de
programación Python empleados para realizar las simulaciones
numéricas del capítulo~\ref{ch:results}.

\section{Test numéricos}

Primero, veremos los programas correspondientes a los test numéricos
de la sección 5.3.1.
Para la resolución numérica se han empleado los métodos descentrados
downwind para el problema de Cauchy, y upwind para el problema de
Riemann.
También se ha calculado la solución numérica de ambos problemas
mediante el método de Lax-Friedrichs y el de Lax-Wendroff.
Entonces, se presentan los programas de la resolución numérica
para cada uno de estos métodos numéricos empleados.

\subsection{Método descentrado upwind}

En esta sección, encontraremos el programa empleado para la
resolución numérica de la ecuación de transporte mediante el método
descentrado upwind.

\begin{listing}[ht!]
    \tiny
    \centering
    \inputminted[firstline=1,lastline=2]{python}{upwind.py}
    \inputminted[firstline=4,lastline=7]{python}{upwind.py}
    \inputminted[firstline=10,lastline=11]{python}{upwind.py}
    \inputminted[firstline=35,lastline=35]{python}{upwind.py}
    \inputminted[firstline=38,lastline=53]{python}{upwind.py}
\end{listing}

\subsection{Método descentrado downwind}

En esta sección, encontraremos el programa empleado para la
resolución numérica de la ecuación de transporte mediante el método
descentrado downwind.

\begin{listing}[ht!]
    \tiny
    \centering
    \inputminted[firstline=1,lastline=2]{python}{downwind.py}
    \inputminted[firstline=4,lastline=7]{python}{downwind.py}
    \inputminted[firstline=10,lastline=11]{python}{downwind.py}
    \inputminted[firstline=35,lastline=35]{python}{downwind.py}
    \inputminted[firstline=38,lastline=53]{python}{downwind.py}
\end{listing}

\subsection{Método de Lax-Friedrichs}

En esta sección, encontraremos el programa empleado para la
resolución numérica de la ecuación de transporte mediante el método
de Lax-Friedrichs.

\begin{listing}[ht!]
    \tiny
    \centering
    \inputminted[firstline=1,lastline=2]{python}{laxfriedrichs.py}
    \inputminted[firstline=4,lastline=7]{python}{laxfriedrichs.py}
    % \inputminted[firstline=10,lastline=11]{python}{laxfriedrichs.py}
    % \inputminted[firstline=35,lastline=35]{python}{laxfriedrichs.py}
    \inputminted[firstline=38,lastline=56]{python}{laxfriedrichs.py}
\end{listing}

\subsection{Método de Lax-Wendroff}

En esta sección, encontraremos el programa empleado para la
resolución numérica de la ecuación de transporte mediante el método
de Lax-Wendroff.

\begin{listing}[ht!]
    \tiny
    \centering
    \inputminted[firstline=1,lastline=2]{python}{laxwendroff.py}
    \inputminted[firstline=4,lastline=7]{python}{laxwendroff.py}
    % \inputminted[firstline=10,lastline=11]{python}{laxwendroff.py}
    % \inputminted[firstline=35,lastline=35]{python}{laxwendroff.py}
    \inputminted[firstline=38,lastline=56]{python}{laxwendroff.py}
\end{listing}