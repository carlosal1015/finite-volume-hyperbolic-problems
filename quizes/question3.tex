Considere la ecuación de transporte de coeficiente constante
\begin{math}
    \difcp{u}{t}+
    c\difcp{u}{x}=
    f\left(x,t\right)
\end{math}.
Reemplazaremos la derivada espacial $\difcp{u}{x}$ por una
aproximación de diferencia finita para obtener un sistema
semidiscretizado donde
\begin{math}
    w_{j}\left(t\right)=
    u\big(x_{j},t\big)
\end{math}
en una malla uniforme $\Omega_{h}$
\begin{equation*}
    \forall j=1,\dotsc,m:
    w^{\prime}_{j}\left(t\right)=
    \dfrac{c}{2h}
    \left(
    w_{j-1}\left(t\right)-
    w_{j+1}\left(t\right)
    \right),
\end{equation*}
con
\begin{math}
    w_{0}\left(t\right)=
    w_{m}\left(t\right)
\end{math}
y
\begin{math}
    w_{m+1}\left(t\right)=
    w_{1}\left(t\right)
\end{math}.
Así, tenemos
\begin{math}
    w^{\prime}\left(t\right)=Aw\left(t\right)
\end{math}.
Verifique que los autovalores de la matriz antisimétrica
circulante
\begin{equation*}
    A=
    \frac{c}{2h}
    \begin{bmatrix}
        0      & -1     & 0      & \cdots & 1      \\
        1      & 0      & -1     & \ddots & \vdots \\
        0      & \ddots & \ddots & \ddots & 0      \\
        \vdots & \ddots & \ddots & 0      & -1     \\
        -1     & 0      & \cdots & 1      & 0
    \end{bmatrix}\in\mathbb{R}^{m\times m}
\end{equation*}

tiene $m$ autovalores con la parte imaginaria igual a cero.