%! arara: clean: {
%! arara: --> extensions:
%! arara: --> ['aux', 'bbl', 'bcf', 'blg', 'log', 'nav', 'out', 'pdf', 'run.xml', 'snm', 'toc']
%! arara: --> }
%! arara: lualatex: {
%! arara: --> shell: yes,
%! arara: --> draft: yes,
%! arara: --> interaction: batchmode
%! arara: --> }
%! arara: biber
% arara: lualatex: {
% arara: --> shell: yes,
% arara: --> draft: no,
% arara: --> interaction: batchmode
% arara: --> }
% arara: lualatex: {
% arara: --> shell: yes,
% arara: --> draft: no,
% arara: --> interaction: batchmode
% arara: --> }
%! arara: clean: {
%! arara: --> extensions:
%! arara: --> ['aux', 'bbl', 'bcf', 'blg', 'log', 'nav', 'out', 'run.xml', 'snm', 'toc']
%! arara: --> }
\documentclass[spanish,addpoints,answers,a4paper,8pt]{exam}
\usepackage[T1]{fontenc}
\usepackage[spanish]{babel}
\usepackage{libertine}
\usepackage{geometry}
\geometry{
	left = 2cm,
	right = 2cm,
	top = 2cm,
	bottom = 3cm
}
\usepackage[shortlabels]{enumitem}
\usepackage{multicol}
\usepackage{hyperref}
\usepackage{amsmath,amsthm}
\theoremstyle{definition}
\newtheorem{definition}{Definición}
\pagestyle{headandfoot}

\usepackage{minted}
\renewcommand{\solutiontitle}{\noindent\textbf{Solución}\par\noindent}
\renewcommand\listingscaption{Listado}
\newcommand{\unmarkedfntext}[1]{%
	\begingroup
	\renewcommand\thefootnote{}\footnote{#1}%
	\addtocounter{footnote}{-1}%
	\endgroup
}

\pointpoints{Punto}{Puntos}
\bracketedpoints
\begin{document}
\begin{center}
    \sffamily\bfseries\Large
    Métodos Numéricos para Ecuaciones Diferenciales Parciales\quad CM--032\\
    Primera práctica calificada
\end{center}

\begin{center}
    \fbox{\fbox{\parbox{5.5in}
            {\centering
                Responda las preguntas en los espacios provistos en las hojas de preguntas. Los argumentos y la claridad de las respuestas se considerarán en la puntuación final.
            }}}
\end{center}
\vspace{0.1in}
\makebox[\textwidth]{Nombres y apellidos:\enspace\hrulefill}
\vspace{0.2in}
\makebox[\textwidth]{Nombres y apellidos del profesor:\enspace MSc. Nombre.\hfill}

\begin{questions}
    \header{Math 115}{Second Exam}{July 4, 1776}
    \question[5] Dibuje todos los grafos no isomorfos $3$-regulares de seis vértices.

    \begin{definition}{\em Grafo $d$-regular}

        Sea $d\geq0$ un entero. Un grafo $d$-regular es un grafo $G$ tal que cualquier vértice tiene como grado precisamente igual a $d$.
    \end{definition}

    \question Sea $G=(V,E)$ un grafo. Una \emph{orientación} de $G$ es cualquier grafo orientado $G^{\prime}=(V,E^{\prime})$ que se obtiene al reemplazar cada rama $\{u,v\}\in E$ por la rama dirigida $(u,v)$ o bien por la rama dirigida $(v,u)$.
    \begin{parts}
        \part[2\half] Pruebe que si todos los grados de $G$ son pares, entonces existe una orientación $H$ de $G$ con $\deg^{+}_{H}(v)=\deg^{-}_{H}(v)$ para todos los vértices $v\in V(G)$.
        \part[2\half] Pruebe que un grafo dirigido $G$ que satisface $\deg^{+}_{H}(v)=\deg^{-}_{H}(v)$ para todo vértice $v$ es fuertemente conexo si y sólo si es débilmente conexo.
    \end{parts}

    \begin{definition}{\em Grafo fuertemente conexo}

        Un grafo dirigido es fuertemente conexo si cada par de aristas es \emph{alcanzanble} el uno del otro.
    \end{definition}

    \begin{definition}{\em Grafo débilmente conexo}

        Un grafo dirigido es débilmente conexo si al convertir el grafo dirigido en uno no dirigido, el grafo resultante es conexo.
    \end{definition}
    \question[5] Pruebe que cualquier grafo $G=(V,E)$ que no tenga ciclos y que satisfaga $|V|=|E|+1$, es un árbol.

    \begin{definition}{\em Árbol}

        Un árbol $T$ es un grafo no dirigido que es conexo y no posee ciclos.
    \end{definition}

    \question[5] Encuentre dos árboles no isomorfos con igual secuencia de grados.
\end{questions}
\vfill
\begin{flushright}\bfseries
    Facultad de Ciencias\\[2mm]
    Universidad Nacional de Ingeniería\\[2mm]
    \today%\unmarkedfntext{Cada pregunta vale 5 puntos.}
\end{flushright}
\end{document}

