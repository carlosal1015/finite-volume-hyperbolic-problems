% arara: clean: {
% arara: --> extensions:
% arara: --> ['aux', 'log', 'nav', 'out', 'pdf', 'snm', 'toc']
% arara: --> }
% arara: lualatex: {
% arara: --> shell: yes,
% arara: --> draft: yes,
% arara: --> interaction: batchmode
% arara: --> }
% arara: lualatex: {
% arara: --> shell: yes,
% arara: --> draft: no,
% arara: --> interaction: batchmode
% arara: --> }
% arara: clean: {
% arara: --> extensions:
% arara: --> ['aux', 'log', 'nav', 'out', 'snm', 'toc']
% arara: --> }
\documentclass[aspectratio=1610,spanish,8pt]{beamer}
\usetheme[light]{Mirage}
\usepackage[spanish]{babel}
\usepackage{mathtools}
\usepackage{diffcoeff}
% The glowy light in the footnline
% \setlength{\MirageGlowRadius}{0.25ex}

\title{Métodos Numéricos para las Ecuaciones Diferenciales Parciales} % CM 032
\subtitle{Prueba de Entrada}
\author{Nombre}
\institute{Universidad Nacional de Ingeniería}
\date{\today}

\usepackage[lining,tabular]{carlito}
\usepackage{caladea}

\usepackage{unicode-math}
%\setmathfont{STIX Two Math}
\setmathfont{Erewhon Math}
%\setmathfont{Fira Math}
\difdef{c}{L}{op-symbol=\mathop{}\!\mathbin\bigtriangleup}

\begin{document}

\frame{\maketitle}

\section{Pregunta $1$}

\begin{frame}
    \frametitle{\secname}
    \question

Dada la siguiente tabla
\begin{table}[ht!]
    \centering
    \begin{tabular}{|C|C|C|C|C|C|}
        \hline
        x_{i}               & 2            & 5           & 7           & 11          & 20           \\
        \hline
        f\left(x_{i}\right) & \frac{1}{16} & \frac{1}{8} & \frac{5}{8} & \frac{1}{8} & \frac{1}{16} \\
        \hline
    \end{tabular}
\end{table}

\begin{parts}
    \part[5]

    Halle el valor de
    \begin{math}\displaystyle
        \mu=
        \sum_{i=1}^{5}
        x_{i}f\left(x_{i}\right)
    \end{math}.

    \part[5]

    Analice la convergencia de
    \begin{math}\displaystyle
        I=
        \int_{0}^{\infty}
        xe^{-\big(\frac{x-\mu}{2}\big)^{2}}\dl x.
    \end{math}
\end{parts}

% \begin{figure}[ht!]
%     \centering
%     \begin{tikzpicture}[xscale=0.5,yscale=0.5]
%         \draw[help lines] (0,0) grid (32,28);
%     \end{tikzpicture}
% \end{figure}

\begin{solutionordottedlines}
    \begin{parts}
        \part

        \textbf{Cálculo de} $\mu$.

        \begin{math}
            \mu=
            \sum_{i=1}^{5}
            x_{i}f\left(x_{i}\right)=
            2\times\dfrac{1}{16}+
            5\times\dfrac{1}{8}+
            7\times\dfrac{5}{8}+
            11\times\dfrac{1}{8}+
            20\times\dfrac{1}{16}=
            \dfrac{124}{16}=
            7.75
        \end{math}.

        \part

        \textbf{Convergencia de la integral}.

        Si la sustitución es $z=\frac{x-\mu}{2}$, entonces
        $x=2z+\mu$ y $\dl z=\frac{\dl x}{2}$.

        \begin{equation*}
            I=
            \int_{0}^{\infty}
            xe^{-\big(\frac{x-\mu}{2}\big)^{2}}
            \dl x=
            \int_{0}^{\infty}
            \left(2z+\mu\right)e^{-z^{2}}2\dl z=
            4\int_{0}^{\infty}
            ze^{-z^{2}}\dl z+
            2\mu\int_{0}^{\infty}e^{-z^{2}}\dl z<
            \infty.
        \end{equation*}

        Como cada integral converge, $I$ es convergente.
    \end{parts}
\end{solutionordottedlines}
\end{frame}

\begin{frame}
    \frametitle{\secname}

    \begin{proof}[Solución]
        Buscamos soluciones separadas de la forma
        \begin{math}
            u\left(r,\theta\right)=
            R\left(r\right)
            \Theta\left(\theta\right)
        \end{math},
        para las cuales la EDP requiere que
        \begin{equation*}
            R^{\prime\prime}\Theta+
            \frac{1}{r}
            R^{\prime}
            \Theta+
            \frac{1}{r^{2}}
            R\Theta^{\prime\prime}=
            0.
        \end{equation*}

        Dividiendo por $R\Theta$ y multiplicando por $r^{2}$,
        obtenemos
        \begin{equation*}
            r^{2}
            \frac{R^{\prime\prime}}{R}+
            r\frac{R^{\prime}}{R}=
            -\frac{\Theta^{\prime\prime}}{\Theta}=
            \lambda.
        \end{equation*}
        \begin{equation*}
            \lambda_{n}=n^{2},\quad
            \Theta_{n}\left(\Theta\right)=
            A_{n}
            \cos\left(n\theta\right)+
            B_{n}\sin\left(n\theta\right),\quad
            n=1,\dotsc,
        \end{equation*}
        y para $n=0$, obtenemos que $\Theta_{0}\left(\theta\right)=A_{0}$.

        \begin{equation*}
            R_{n}\left(r\right)=
            c_{n}r^{n}+
            d_{n}r^{-n}.
        \end{equation*}

        \begin{equation*}
            u\left(x,y\right)=
            \frac{1}{2\pi a}
            \int_{y\in\partial D}
            g\left(y\right)
            \frac{a^{2}-{\left\|x\right\|}^{2}}{{\left\|x-y\right\|}^{2}}\dl{s}.
        \end{equation*}
    \end{proof}
\end{frame}

\begin{frame}
    \frametitle{\secname}

    \begin{equation*}
        \left\{
        \begin{aligned}
            x & =
            r\cos\theta. \\
            y & =
            r\sin\theta.
        \end{aligned}
        \right.
        \iff
        \left\{
        \begin{aligned}
            r      & =
            \sqrt{x^{2}+y^{2}}. \\
            \theta & =
            \arctan
            \left(
            \frac{y}{x}
            \right).
        \end{aligned}
        \right.
    \end{equation*}

    \begin{align*}
        \diffp{}{x} & =
        \diffp{r}{x}
        \diffp{}{r}+
        \diffp{\theta}{x}
        \diffp{}{\theta}=
        \frac{x}{\sqrt{x^{2}+y^{2}}}
        \diffp{}{r}-
        \frac{y}{x^{2}+y^{2}}
        \diffp{}{\theta}=
        \cos\theta
        \diffp{}{r}-
        \frac{\sin\theta}{r}
        \diffp{}{\theta}. \\
        \diffp{}{y} & =
        \diffp{r}{x}
        \diffp{}{r}+
        \diffp{\theta}{x}
        \diffp{}{\theta}=
        \frac{y}{\sqrt{x^{2}+y^{2}}}
        \diffp{}{r}+
        \frac{x}{x^{2}+y^{2}}
        \diffp{}{\theta}=
        \sin\theta
        \diffp{}{r}+
        \frac{\cos\theta}{r}
        \diffp{}{\theta}.
    \end{align*}

    \begin{definition}{Operador Laplaciano}
        \begin{equation*}
            \difc.L.{u}{}\coloneqq
            \sum_{i=1}^{d}
            \difcp[2]{u}{x_{i}}.
        \end{equation*}
    \end{definition}
\end{frame}

\section{Pregunta $2$}

\begin{frame}
    \frametitle{\secname}

    \begin{itemize}
        \item

              Pruebe que
\begin{math}
    \forall x\in\mathbb{R}^{d}:
    \left\|x\right\|_{1}\leq
    \sqrt{d}{\left\|x\right\|}_{2}
\end{math}.

        \item

              Pruebe que
\begin{math}
    \forall x\in\mathbb{R}^{d}:
    \left\|x\right\|_{2}\leq
    \sqrt{d}{\left\|x\right\|}_{\infty}
\end{math}.
    \end{itemize}

    \begin{definition}[Normas vectoriales]
        \begin{align*}
            \left\|x\right\|_{1}      & \coloneqq
            \sum_{j=1}^{d}\left|x_{j}\right|.                  \\
            \left\|x\right\|_{2}      & \coloneqq
            {\Big(\sum_{j=1}^{d}x^{2}_{j}\Big)}^{\frac{1}{2}}. \\
            \left\|x\right\|_{\infty} & \coloneqq
            \max_{1\leq j\leq d}\left|x_{j}\right|.            \\
        \end{align*}
    \end{definition}
\end{frame}

\section{Pregunta $3$}

\begin{frame}
    \frametitle{\secname}

    \question

Sean $\Omega$ el espacio muestral y $A,B\subset\Omega$ dos sucesos
tales que $\mathbb{P}\left(A\right)=\frac{1}{4}$,
$\mathbb{P}\left(B\mid A\right)=\frac{1}{2}$ y
$\mathbb{P}\left(A\mid B\right)=\frac{1}{4}$.

Indique y justifique si las siguientes afirmaciones son verdaderas o falsas.

\begin{parts}
	\part[2]

	$A$ y $B$ son mutuamente excluyentes.

	\part[2]

	$\mathbb{P}\left(\overline{A}\mid\overline{B}\right)=\frac{3}{4}$.

	\part[2]

	$\mathbb{P}\left(A\mid B\right)+\mathbb{P}\left(A\mid\overline{B}\right)=1$.
\end{parts}

\begin{figure}[ht!]
	\centering
	\begin{tikzpicture}[xscale=0.5,yscale=0.5]
		\draw[help lines] (0,0) grid (32,12);
	\end{tikzpicture}
\end{figure}

\end{frame}

\begin{frame}
    \begin{proof}[Solución]
        Los $m$ autovalores $\lambda_{k}$ de la matriz
        \begin{equation*}
            B=
            \begin{bmatrix}
                0      & -1     & 0      & \cdots & 1      \\
                1      & 0      & -1     & \ddots & \vdots \\
                0      & \ddots & \ddots & \ddots & 0      \\
                \vdots & \ddots & \ddots & 0      & -1     \\
                -1     & 0      & \cdots & 1      & 0
            \end{bmatrix}\in\mathbb{R}^{m\times m}
        \end{equation*}
        son
        \begin{equation*}
            \forall k=0,\dotsc,m-1:
            \lambda_{k}=
            \sum_{j=0}^{m-1}
            c_{j}
            \exp
            \left(
            2\pi i\frac{kj}{m}
            \right).
        \end{equation*}
    \end{proof}
\end{frame}

\begin{frame}
\end{frame}

\end{document}
