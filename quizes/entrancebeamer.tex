\documentclass[aspectratio=1610,spanish,8pt]{beamer}
\usetheme{Mirage} % [light]
\usepackage[spanish]{babel}
\usepackage{mathtools}
\usepackage{diffcoeff}
% The glowy light in the footnline
% \setlength{\MirageGlowRadius}{0.25ex}

\title{Métodos Numéricos para las Ecuaciones Diferenciales Parciales} % CM 032
\subtitle{Prueba de Entrada}
\author{Nombre}
\institute{Universidad Nacional de Ingeniería}
\date{\today}

\usepackage[lining,tabular]{carlito}
\usepackage{caladea}

\usepackage{unicode-math}
%\setmathfont{STIX Two Math}
\setmathfont{Erewhon Math}
%\setmathfont{Fira Math}
\difdef{c}{L}{op-symbol=\mathop{}\!\mathbin\bigtriangleup}

\begin{document}

\frame{\maketitle}

\section{Pregunta $1$}

\begin{frame}
    \frametitle{\secname}

    Considere el problema de valor de frontera Dirichlet en un disco
    \begin{math}
        D=
        \left\{
        \left(x,y\right)\in
        \mathbb{R}^{2}\mid
        x^{2}+y^{2}<a^{2}
        \right\}
    \end{math}
    \begin{equation*}
        \begin{cases}
            \difc.L.{u}{}=0, & x\in D,         \\
            u=g,             & x\in\partial D.
        \end{cases}
    \end{equation*}

    Resolvamos este problema mediante la separación de variables.
    Pasando a coordenadas polares, sea $u=u\left(r,\theta\right)$,
    y notemos que $x\in\partial D$ es lo mismo que $r=a$, de modo que
    $u=g\left(\theta\right)$ en $r=a$.
    Reescriba el operador $\difc.L.{u}{}$ en términos de $r$ y
    $\theta$ obtenga
    \begin{equation*}
        \difc.L.{u}{}=
        \difcp[2]{u}{r}+
        \frac{1}{r}
        \difcp{u}{r}+
        \frac{1}{r^{2}}
        \difcp{u}{\theta}.
    \end{equation*}

    Luego, resuelva
    \begin{equation*}
        \begin{cases}
            \difcp[2]{u}{r}+
            \dfrac{1}{r}
            \difcp{u}{r}+
            \dfrac{1}{r^{2}}
            \difcp{u}{\theta}=0,    & r<a, \\
            u=g\left(\theta\right), & r=a.
        \end{cases}
    \end{equation*}
\end{frame}

\begin{frame}
    \frametitle{\secname}

    \begin{proof}[Solución]
        Buscamos soluciones separadas de la forma
        \begin{math}
            u\left(r,\theta\right)=
            R\left(r\right)
            \Theta\left(\theta\right)
        \end{math},
        para las cuales la EDP requiere que
        \begin{equation*}
            R^{\prime\prime}\Theta+
            \frac{1}{r}
            R^{\prime}
            \Theta+
            \frac{1}{r^{2}}
            R\Theta^{\prime\prime}=
            0.
        \end{equation*}

        Dividiendo por $R\Theta$ y multiplicando por $r^{2}$,
        obtenemos
        \begin{equation*}
            r^{2}
            \frac{R^{\prime\prime}}{R}+
            r\frac{R^{\prime}}{R}=
            -\frac{\Theta^{\prime\prime}}{\Theta}=
            \lambda.
        \end{equation*}
        \begin{equation*}
            \lambda_{n}=n^{2},
            \Theta_{n}\left(\Theta\right)=
            A_{n}
            \cos\left(n\theta\right)+
            B_{n}\sin\left(n\theta\right),\quad
            n=1,\dotsc,
        \end{equation*}
        y para $n=0$, obtenemos que $\Theta_{0}\left(\theta\right)=A_{0}$.

        \begin{equation*}
            R_{n}\left(r\right)=
            c_{n}r^{n}+
            d_{n}r^{-n}.
        \end{equation*}

        \begin{equation*}
            u\left(x,y\right)=
            \frac{1}{2\pi a}
            \int_{y\in\partial D}
            g\left(y\right)
            \frac{a^{2}-{\left\|x\right\|}^{2}}{{\left\|x-y\right\|}^{2}}\dl{s}
        \end{equation*}
    \end{proof}
\end{frame}

\begin{frame}
    \frametitle{\secname}

    \begin{equation*}
        \left\{
        \begin{aligned}
            x & =
            r\cos\theta. \\
            y & =
            r\sin\theta.
        \end{aligned}
        \right.
        \iff
        \left\{
        \begin{aligned}
            r      & =
            \sqrt{x^{2}+y^{2}}. \\
            \theta & =
            \arctan
            \left(
            \frac{y}{x}
            \right).
        \end{aligned}
        \right.
    \end{equation*}

    \begin{align*}
        \diffp{}{x} & =
        \diffp{r}{x}
        \diffp{}{r}+
        \diffp{\theta}{x}
        \diffp{}{\theta}=
        \frac{x}{\sqrt{x^{2}+y^{2}}}
        \diffp{}{r}-
        \frac{y}{x^{2}+y^{2}}
        \diffp{}{\theta}=
        \cos\theta
        \diffp{}{r}-
        \frac{\sin\theta}{r}
        \diffp{}{\theta}. \\
        \diffp{}{y} & =
        \diffp{r}{x}
        \diffp{}{r}+
        \diffp{\theta}{x}
        \diffp{}{\theta}=
        \frac{y}{\sqrt{x^{2}+y^{2}}}
        \diffp{}{r}+
        \frac{x}{x^{2}+y^{2}}
        \diffp{}{\theta}=
        \sin\theta
        \diffp{}{r}+
        \frac{\cos\theta}{r}
        \diffp{}{\theta}.
    \end{align*}

    \begin{definition}{Operador Laplaciano}
        \begin{equation*}
            \difc.L.{u}{}\coloneqq
            \sum_{i=1}^{d}
            \difcp[2]{u}{x_{i}}.
        \end{equation*}
    \end{definition}
\end{frame}

\section{Pregunta $2$}

\begin{frame}
    \frametitle{\secname}

    \begin{itemize}
        \item

              Pruebe que
              \begin{math}
                  \forall x\in\mathbb{R}^{d}:
                  \left\|x\right\|_{1}\leq
                  \sqrt{d}{\left\|x\right\|}_{2}
              \end{math}.

        \item

              Pruebe que
              \begin{math}
                  \forall x\in\mathbb{R}^{d}:
                  \left\|x\right\|_{2}\leq
                  \sqrt{d}{\left\|x\right\|}_{\infty}
              \end{math}.
    \end{itemize}
\end{frame}

\section{Pregunta $3$}

\begin{frame}
    \frametitle{\secname}

    Considere la ecuación de transporte de coeficiente constante
    \begin{math}
        \difcp{u}{t}+
        c\difcp{u}{x}=
        f\left(x,t\right)
    \end{math}.
    Reemplazaremos la derivada espacial $\difcp{u}{x}$ por una
    aproximación de diferencia finita para obtener un sistema
    semidiscretizado donde $w_{j}\left(t\right)=u\left(x_{j},t\right)$
    en una malla uniforme $\Omega_{h}$
    \begin{equation*}
        \forall j=1,\dotsc,m:
        w^{\prime}_{j}\left(t\right)=
        \dfrac{c}{2h}
        \left(
        w_{j-1}\left(t\right)-
        w_{j+1}\left(t\right)
        \right),
    \end{equation*}
    con
    \begin{math}
        w_{0}\left(t\right)=
        w_{m}\left(t\right)
    \end{math}
    y
    \begin{math}
        w_{m+1}\left(t\right)=
        w_{1}\left(t\right)
    \end{math}.
    Así, tenemos
    \begin{math}
        w^{\prime}\left(t\right)=Aw\left(t\right)
    \end{math}.
    Verifique que los autovalores de la matriz antisimétrica
    circulante
    \begin{equation*}
        A=
        \frac{c}{2h}
        \begin{bmatrix}
            0      & -1     & 0      & \cdots & 1      \\
            1      & 0      & -1     & \ddots & \vdots \\
            0      & \ddots & \ddots & \ddots & 0      \\
            \vdots & \ddots & \ddots & 0      & -1     \\
            -1     & 0      & \cdots & 1      & 0
        \end{bmatrix}\in\mathbb{R}^{m\times m}
    \end{equation*}

    tiene $m$ autovalores con parte imaginaria igual a cero.
\end{frame}

\begin{frame}
    \begin{definition}[Normas vectoriales]
        \begin{align*}
            \left\|x\right\|_{1}      & \coloneqq
            \sum_{j=1}^{d}\left|x_{j}\right|.                  \\
            \left\|x\right\|_{2}      & \coloneqq
            {\Big(\sum_{j=1}^{d}x^{2}_{j}\Big)}^{\frac{1}{2}}. \\
            \left\|x\right\|_{\infty} & \coloneqq
            \max_{1\leq j\leq d}\left|x_{j}\right|.            \\
        \end{align*}
    \end{definition}
\end{frame}
\end{document}
