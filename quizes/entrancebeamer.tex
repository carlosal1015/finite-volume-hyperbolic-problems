% arara: clean: {
% arara: --> extensions:
% arara: --> ['aux', 'log', 'nav', 'out', 'pdf', 'snm', 'toc']
% arara: --> }
% arara: lualatex: {
% arara: --> shell: yes,
% arara: --> draft: yes,
% arara: --> interaction: batchmode
% arara: --> }
% arara: lualatex: {
% arara: --> shell: yes,
% arara: --> draft: no,
% arara: --> interaction: batchmode
% arara: --> }
% arara: clean: {
% arara: --> extensions:
% arara: --> ['aux', 'log', 'nav', 'out', 'snm', 'toc']
% arara: --> }
\documentclass[aspectratio=1610,spanish,8pt]{beamer}
\usetheme[light]{Mirage}
\usepackage[spanish]{babel}
\usepackage{mathtools}
\usepackage{diffcoeff}
% The glowy light in the footnline
% \setlength{\MirageGlowRadius}{0.25ex}

\title{Métodos Numéricos para las Ecuaciones Diferenciales Parciales} % CM 032
\subtitle{Prueba de Entrada}
\author{Nombre}
\institute{Universidad Nacional de Ingeniería}
\date{\today}

\usepackage[lining,tabular]{carlito}
\usepackage{caladea}

\usepackage{unicode-math}
%\setmathfont{STIX Two Math}
\setmathfont{Erewhon Math}
%\setmathfont{Fira Math}
\difdef{c}{L}{op-symbol=\mathop{}\!\mathbin\bigtriangleup}

\begin{document}

\frame{\maketitle}

\section{Pregunta $1$}

\begin{frame}
    \frametitle{\secname}
    Considere el problema de valor de frontera Dirichlet en un disco
\begin{math}
    D=
    \left\{
    \left(x,y\right)\in\mathbb{R}^{2}\mid
    x^{2}+y^{2}<a^{2}
    \right\}
\end{math}
\begin{equation*}
    \begin{cases}
        \difc.L.{u}{}=0, & x\in D,         \\
        u=g,             & x\in\partial D.
    \end{cases}
\end{equation*}

Resolvamos este problema mediante la separación de variables.
Pasando a coordenadas polares, sea $u=u\left(r,\theta\right)$,
y notemos que $x\in\partial D$ es lo mismo que $r=a$, de modo que
$u=g\left(\theta\right)$ en $r=a$.
Reescriba el operador $\difc.L.{u}{}$ en términos de $r$ y
$\theta$ obtenga
\begin{equation*}
    \difc.L.{u}{}=
    \difcp[2]{u}{r}+
    \frac{1}{r}
    \difcp{u}{r}+
    \frac{1}{r^{2}}
    \difcp{u}{\theta}.
\end{equation*}

Luego, resuelva
\begin{equation*}
    \begin{cases}
        \difcp[2]{u}{r}+
        \dfrac{1}{r}
        \difcp{u}{r}+
        \dfrac{1}{r^{2}}
        \difcp{u}{\theta}=0,    & r<a, \\
        u=g\left(\theta\right), & r=a.
    \end{cases}
\end{equation*}
\end{frame}

\begin{frame}
    \frametitle{\secname}

    \begin{proof}[Solución]
        Buscamos soluciones separadas de la forma
        \begin{math}
            u\left(r,\theta\right)=
            R\left(r\right)
            \Theta\left(\theta\right)
        \end{math},
        para las cuales la EDP requiere que
        \begin{equation*}
            R^{\prime\prime}\Theta+
            \frac{1}{r}
            R^{\prime}
            \Theta+
            \frac{1}{r^{2}}
            R\Theta^{\prime\prime}=
            0.
        \end{equation*}

        Dividiendo por $R\Theta$ y multiplicando por $r^{2}$,
        obtenemos
        \begin{equation*}
            r^{2}
            \frac{R^{\prime\prime}}{R}+
            r\frac{R^{\prime}}{R}=
            -\frac{\Theta^{\prime\prime}}{\Theta}=
            \lambda.
        \end{equation*}
        \begin{equation*}
            \lambda_{n}=n^{2},\quad
            \Theta_{n}\left(\Theta\right)=
            A_{n}
            \cos\left(n\theta\right)+
            B_{n}\sin\left(n\theta\right),\quad
            n=1,\dotsc,
        \end{equation*}
        y para $n=0$, obtenemos que $\Theta_{0}\left(\theta\right)=A_{0}$.

        \begin{equation*}
            R_{n}\left(r\right)=
            c_{n}r^{n}+
            d_{n}r^{-n}.
        \end{equation*}

        \begin{equation*}
            u\left(x,y\right)=
            \frac{1}{2\pi a}
            \int_{y\in\partial D}
            g\left(y\right)
            \frac{a^{2}-{\left\|x\right\|}^{2}}{{\left\|x-y\right\|}^{2}}\dl{s}.
        \end{equation*}
    \end{proof}
\end{frame}

\begin{frame}
    \frametitle{\secname}

    \begin{equation*}
        \left\{
        \begin{aligned}
            x & =
            r\cos\theta. \\
            y & =
            r\sin\theta.
        \end{aligned}
        \right.
        \iff
        \left\{
        \begin{aligned}
            r      & =
            \sqrt{x^{2}+y^{2}}. \\
            \theta & =
            \arctan
            \left(
            \frac{y}{x}
            \right).
        \end{aligned}
        \right.
    \end{equation*}

    \begin{align*}
        \diffp{}{x} & =
        \diffp{r}{x}
        \diffp{}{r}+
        \diffp{\theta}{x}
        \diffp{}{\theta}=
        \frac{x}{\sqrt{x^{2}+y^{2}}}
        \diffp{}{r}-
        \frac{y}{x^{2}+y^{2}}
        \diffp{}{\theta}=
        \cos\theta
        \diffp{}{r}-
        \frac{\sin\theta}{r}
        \diffp{}{\theta}. \\
        \diffp{}{y} & =
        \diffp{r}{x}
        \diffp{}{r}+
        \diffp{\theta}{x}
        \diffp{}{\theta}=
        \frac{y}{\sqrt{x^{2}+y^{2}}}
        \diffp{}{r}+
        \frac{x}{x^{2}+y^{2}}
        \diffp{}{\theta}=
        \sin\theta
        \diffp{}{r}+
        \frac{\cos\theta}{r}
        \diffp{}{\theta}.
    \end{align*}

    \begin{definition}{Operador Laplaciano}
        \begin{equation*}
            \difc.L.{u}{}\coloneqq
            \sum_{i=1}^{d}
            \difcp[2]{u}{x_{i}}.
        \end{equation*}
    \end{definition}
\end{frame}

\section{Pregunta $2$}

\begin{frame}
    \frametitle{\secname}

    \begin{itemize}
        \item

              Pruebe que
\begin{math}
    \forall x\in\mathbb{R}^{d}:
    \left\|x\right\|_{1}\leq
    \sqrt{d}{\left\|x\right\|}_{2}
\end{math}.

        \item

              Pruebe que
\begin{math}
    \forall x\in\mathbb{R}^{d}:
    \left\|x\right\|_{2}\leq
    \sqrt{d}{\left\|x\right\|}_{\infty}
\end{math}.
    \end{itemize}

    \begin{definition}[Normas vectoriales]
        \begin{align*}
            \left\|x\right\|_{1}      & \coloneqq
            \sum_{j=1}^{d}\left|x_{j}\right|.                  \\
            \left\|x\right\|_{2}      & \coloneqq
            {\Big(\sum_{j=1}^{d}x^{2}_{j}\Big)}^{\frac{1}{2}}. \\
            \left\|x\right\|_{\infty} & \coloneqq
            \max_{1\leq j\leq d}\left|x_{j}\right|.            \\
        \end{align*}
    \end{definition}
\end{frame}

\section{Pregunta $3$}

\begin{frame}
    \frametitle{\secname}

    \question[10]

Determine
\begin{math}\displaystyle
    \diff{}{x}
    \left(
    \int_{3x-1}^{0}
    \frac{\dl t}{t+4}
    \right).
\end{math}

% \begin{figure}[ht!]
%     \centering
%     \begin{tikzpicture}[xscale=0.5,yscale=0.5]
%         \draw[help lines] (0,0) grid (32,40);
%     \end{tikzpicture}
% \end{figure}

\begin{solutionordottedlines}

    Defina las siguientes funciones y calcule sus derivadas
    \begin{equation}
        \begin{split}
            f\left(x\right) & \coloneqq
            \int_{0}^{x}
            \frac{\dl t}{t+4}.          \\
            g\left(x\right) & \coloneqq
            3x-1.
        \end{split}
        \xRightarrow{\quad\text{Teorema Fundamental del Cálculo}\quad}
        \begin{split}
            \diff{}{x}f\left(x\right) & =
            \frac{1}{x+4}.                \\
            \diff{}{x}g\left(x\right) & =
            3.
        \end{split}
    \end{equation}
    Note que
    \begin{align*}
        \left(f\circ g\right)\left(x\right) & =
        f\left(3x-1\right)=
        \int_{0}^{3x-1}
        \frac{\dl t}{t+4}.
        \shortintertext{Aplique la regla de la cadena}
        \diff{}{x}
        \left(f\circ g\right)\left(x\right)
                                            & =
        \diff{}{x}f\left(g\left(x\right)\right)
        \diff{}{x}g\left(x\right)=
        \frac{3}{3x-1+4}=
        \frac{1}{x+1}.
        \shortintertext{Así,}
        -\frac{1}{x+1}=
        -\diff{}{x}
        \left(f\circ g\right)\left(x\right)
                                            & =
        -\diff{}{x}
        \left(
        \int_{0}^{3x-1}
        \frac{\dl t}{t+4}
        \right)=
        \diff{}{x}
        \left(
        \int_{3x-1}^{0}
        \frac{\dl t}{t+4}
        \right).
    \end{align*}
\end{solutionordottedlines}
\end{frame}

\begin{frame}
    \begin{proof}[Solución]
        Los $m$ autovalores $\lambda_{k}$ de la matriz
        \begin{equation*}
            B=
            \begin{bmatrix}
                0      & -1     & 0      & \cdots & 1      \\
                1      & 0      & -1     & \ddots & \vdots \\
                0      & \ddots & \ddots & \ddots & 0      \\
                \vdots & \ddots & \ddots & 0      & -1     \\
                -1     & 0      & \cdots & 1      & 0
            \end{bmatrix}\in\mathbb{R}^{m\times m}
        \end{equation*}
        son
        \begin{equation*}
            \forall k=0,\dotsc,m-1:
            \lambda_{k}=
            \sum_{j=0}^{m-1}
            c_{j}
            \exp
            \left(
            2\pi i\frac{kj}{m}
            \right).
        \end{equation*}
    \end{proof}
\end{frame}

\begin{frame}
\end{frame}

\end{document}
