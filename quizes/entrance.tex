% arara: clean: {
% arara: --> extensions:
% arara: --> ['aux', 'log', 'out', 'pdf']
% arara: --> }
% arara: lualatex: {
% arara: --> shell: yes,
% arara: --> draft: no,
% arara: --> interaction: batchmode
% arara: --> }
% arara: clean: {
% arara: --> extensions:
% arara: --> ['aux', 'log', 'out']
% arara: --> }
\documentclass[
	12pt,
	oneside,
	appendixprefix=true
]{scrbook}
\usepackage[spanish]{babel}
\decimalpoint
\usepackage{diffcoeff}
\usepackage[intlimits]{mathtools}
\usepackage{amssymb}
\usepackage{amsthm}
\usepackage{graphicx}
\usepackage{minted}
\usepackage[
	citestyle=numeric,
	style=numeric,
	backend=biber,
]{biblatex}
\usepackage{hyperref}

\title{
	Una introducción acerca de los métodos de volúmenes finitos para
	una ecuación de transporte
}
\author{\scshape\name}
\date{\today}

% \addtokomafont{section}{\centering}
% \addtokomafont{subsection}{\centering}

\theoremstyle{definition}
\newtheorem{theorem}{Teorema}
\newtheorem{definition}{Definición}
\newtheorem{remark}{Observación}
\newtheorem{example}{Ejemplo}
\addbibresource{references.bib}

\providecommand{\name}{Carlos Aznarán} % Nombre
\renewcommand{\listingscaption}{Programa}
\renewcommand{\listoflistingscaption}{Lista de \listingscaption s}

\hypersetup{
	pdfencoding=auto,
	linktocpage=true,
	colorlinks=true,
	linkcolor=blue,
	urlcolor=magenta,
	pdfpagelabels,
	pdftex,
	pdfauthor={Carlos Aznarán Laos},
	pdftitle={Una introducción acerca de los métodos de volúmenes
			finitos para una ecuación de transporte},
	pdfsubject={Lecture},
	pdfkeywords={finite volume, pde, transport equation},
	pdfproducer={LuaHBTeX, Version 1.18.0 (TeX Live 2024/Arch Linux)},
	% bookmark=false
}

\clearpage
\pagestyle{empty}
\renewcommand*{\chapterpagestyle}{empty}


\begin{document}
\begin{center}
    \sffamily\bfseries\Large
    Métodos Numéricos para Ecuaciones Diferenciales Parciales~CM 032\\
    Prueba de entrada
\end{center}

\begin{center}
    \fbox{\fbox{\parbox{5.5in}
            {\centering
                Responda las preguntas en los espacios provistos en
                las hojas de preguntas.
                Los argumentos y la claridad de las respuestas se
                considerarán en la puntuación final.
            }}}
\end{center}
\vspace{0.1in}
\makebox[\textwidth]{Nombres y apellidos del alumno:\enspace\hrulefill}
\vspace{0.2in}
\makebox[\textwidth]{Nombres y apellidos del profesor:\enspace MSc. Nombre.\hfill}

\begin{questions}
    \header{Math 115}{Second Exam}{July 4, 1776}
    \question[10]

    Considere el problema de valor de frontera Dirichlet en un disco
\begin{math}
    D=
    \left\{
    \left(x,y\right)\in\mathbb{R}^{2}\mid
    x^{2}+y^{2}<a^{2}
    \right\}
\end{math}
\begin{equation*}
    \begin{cases}
        \difc.L.{u}{}=0, & x\in D,         \\
        u=g,             & x\in\partial D.
    \end{cases}
\end{equation*}

Resolvamos este problema mediante la separación de variables.
Pasando a coordenadas polares, sea $u=u\left(r,\theta\right)$,
y notemos que $x\in\partial D$ es lo mismo que $r=a$, de modo que
$u=g\left(\theta\right)$ en $r=a$.
Reescriba el operador $\difc.L.{u}{}$ en términos de $r$ y
$\theta$ obtenga
\begin{equation*}
    \difc.L.{u}{}=
    \difcp[2]{u}{r}+
    \frac{1}{r}
    \difcp{u}{r}+
    \frac{1}{r^{2}}
    \difcp{u}{\theta}.
\end{equation*}

Luego, resuelva
\begin{equation*}
    \begin{cases}
        \difcp[2]{u}{r}+
        \dfrac{1}{r}
        \difcp{u}{r}+
        \dfrac{1}{r^{2}}
        \difcp{u}{\theta}=0,    & r<a, \\
        u=g\left(\theta\right), & r=a.
    \end{cases}
\end{equation*}

    \question

    \begin{parts}
        \part[2\half]

        Pruebe que
\begin{math}
    \forall x\in\mathbb{R}^{d}:
    \left\|x\right\|_{1}\leq
    \sqrt{d}{\left\|x\right\|}_{2}
\end{math}.

        \part[2\half]

        Pruebe que
\begin{math}
    \forall x\in\mathbb{R}^{d}:
    \left\|x\right\|_{2}\leq
    \sqrt{d}{\left\|x\right\|}_{\infty}
\end{math}.

    \end{parts}
    % Sugerencia: Use la desigualdad de Cauchy-Schwarz.

    \question[5]

    \question[10]

Determine
\begin{math}\displaystyle
    \diff{}{x}
    \left(
    \int_{3x-1}^{0}
    \frac{\dl t}{t+4}
    \right).
\end{math}

% \begin{figure}[ht!]
%     \centering
%     \begin{tikzpicture}[xscale=0.5,yscale=0.5]
%         \draw[help lines] (0,0) grid (32,40);
%     \end{tikzpicture}
% \end{figure}

\begin{solutionordottedlines}

    Defina las siguientes funciones y calcule sus derivadas
    \begin{equation}
        \begin{split}
            f\left(x\right) & \coloneqq
            \int_{0}^{x}
            \frac{\dl t}{t+4}.          \\
            g\left(x\right) & \coloneqq
            3x-1.
        \end{split}
        \xRightarrow{\quad\text{Teorema Fundamental del Cálculo}\quad}
        \begin{split}
            \diff{}{x}f\left(x\right) & =
            \frac{1}{x+4}.                \\
            \diff{}{x}g\left(x\right) & =
            3.
        \end{split}
    \end{equation}
    Note que
    \begin{align*}
        \left(f\circ g\right)\left(x\right) & =
        f\left(3x-1\right)=
        \int_{0}^{3x-1}
        \frac{\dl t}{t+4}.
        \shortintertext{Aplique la regla de la cadena}
        \diff{}{x}
        \left(f\circ g\right)\left(x\right)
                                            & =
        \diff{}{x}f\left(g\left(x\right)\right)
        \diff{}{x}g\left(x\right)=
        \frac{3}{3x-1+4}=
        \frac{1}{x+1}.
        \shortintertext{Así,}
        -\frac{1}{x+1}=
        -\diff{}{x}
        \left(f\circ g\right)\left(x\right)
                                            & =
        -\diff{}{x}
        \left(
        \int_{0}^{3x-1}
        \frac{\dl t}{t+4}
        \right)=
        \diff{}{x}
        \left(
        \int_{3x-1}^{0}
        \frac{\dl t}{t+4}
        \right).
    \end{align*}
\end{solutionordottedlines}
\end{questions}
\vfill
\begin{flushright}\bfseries
    Facultad de Ciencias\\[2mm]
    Universidad Nacional de Ingeniería\\[2mm]
    \today%\unmarkedfntext{Cada pregunta vale 5 puntos.}
\end{flushright}
\end{document}

