%! arara: clean: {
%! arara: --> extensions:
%! arara: --> ['aux', 'bbl', 'bcf', 'blg', 'log', 'nav', 'out', 'pdf', 'run.xml', 'snm', 'toc']
%! arara: --> }
%! arara: lualatex: {
%! arara: --> shell: yes,
%! arara: --> draft: yes,
%! arara: --> interaction: batchmode
%! arara: --> }
%! arara: biber
% arara: lualatex: {
% arara: --> shell: yes,
% arara: --> draft: no,
% arara: --> interaction: batchmode
% arara: --> }
% arara: lualatex: {
% arara: --> shell: yes,
% arara: --> draft: no,
% arara: --> interaction: batchmode
% arara: --> }
%! arara: clean: {
%! arara: --> extensions:
%! arara: --> ['aux', 'bbl', 'bcf', 'blg', 'log', 'nav', 'out', 'run.xml', 'snm', 'toc']
%! arara: --> }
\documentclass[spanish,addpoints,answers,a4paper,8pt]{exam}
\usepackage[T1]{fontenc}
\usepackage[spanish]{babel}
\usepackage{libertine}
\usepackage{geometry}
\geometry{
	left = 2cm,
	right = 2cm,
	top = 2cm,
	bottom = 3cm
}
\usepackage{mathtools}
\usepackage{amssymb}
\usepackage{diffcoeff}
\usepackage{amsthm}
\usepackage[shortlabels]{enumitem}
\usepackage{multicol}
\usepackage{hyperref}
\theoremstyle{definition}
\newtheorem{definition}{Definición}
\pagestyle{headandfoot}

\usepackage{minted}
\renewcommand{\solutiontitle}{\noindent\textbf{Solución}\par\noindent}
\renewcommand\listingscaption{Listado}
\newcommand{\unmarkedfntext}[1]{%
	\begingroup
	\renewcommand\thefootnote{}\footnote{#1}%
	\addtocounter{footnote}{-1}%
	\endgroup
}

\difdef{c}{L}{op-symbol=\mathop{}\!\mathbin\bigtriangleup}

\pointpoints{Punto}{Puntos}
\bracketedpoints

\begin{document}
\begin{center}
    \sffamily\bfseries\Large
    Métodos Numéricos para Ecuaciones Diferenciales Parciales~CM 032\\
    Prueba de entrada
\end{center}

\begin{center}
    \fbox{\fbox{\parbox{5.5in}
            {\centering
                Responda las preguntas en los espacios provistos en
                las hojas de preguntas.
                Los argumentos y la claridad de las respuestas se
                considerarán en la puntuación final.
            }}}
\end{center}
\vspace{0.1in}
\makebox[\textwidth]{Nombres y apellidos del alumno:\enspace\hrulefill}
\vspace{0.2in}
\makebox[\textwidth]{Nombres y apellidos del profesor:\enspace MSc. Nombre.\hfill}

\begin{questions}
    \header{Math 115}{Second Exam}{July 4, 1776}
    \question[10]

    Considere el problema de valor de frontera Dirichlet en un disco
    \begin{math}
        D=
        \left\{
        \left(x,y\right)\in\mathbb{R}^{2}\mid
        x^{2}+y^{2}<a^{2}
        \right\}
    \end{math}
    \begin{equation*}
        \begin{cases}
            \difc.L.{u}{}=0, & x\in D,         \\
            u=g,             & x\in\partial D.
        \end{cases}
    \end{equation*}

    Resolvamos este problema mediante la separación de variables.
    Pasando a coordenadas polares, sea $u=u\left(r,\theta\right)$,
    y notemos que $x\in\partial D$ es lo mismo que $r=a$, de modo que
    $u=h\left(\theta\right)$ en $r=a$.
    Reescriba el operador $\difc.L.{u}{}$ en términos de $r$ y
    $\theta$ obtenga
    \begin{equation*}
        \difc.L.{u}{}=
        \difcp[2]{u}{r}+
        \frac{1}{r}
        \difcp{u}{r}+
        \frac{1}{r^{2}}
        \difcp{u}{\theta}.
    \end{equation*}

    Luego, resuelva
    \begin{equation*}
        \begin{cases}
            \difcp[2]{u}{r}+
            \dfrac{1}{r}
            \difcp{u}{r}+
            \dfrac{1}{r^{2}}
            \difcp{u}{\theta}=0,    & r<a, \\
            u=g\left(\theta\right), & r=a.
        \end{cases}
    \end{equation*}

    \question

    \begin{parts}
        \part[2\half]

        Pruebe que
        \begin{math}
            \forall x\in\mathbb{R}^{d}:
            \left\|x\right\|_{1}\leq
            \sqrt{d}{\left\|x\right\|}_{2}
        \end{math}.

        \part[2\half]

        Pruebe que
        \begin{math}
            \forall x\in\mathbb{R}^{d}:
            \left\|x\right\|_{2}\leq
            \sqrt{d}{\left\|x\right\|}_{\infty}
        \end{math}.
    \end{parts}
    Sugerencia: Use la desigualdad de Cauchy-Schwarz.

    \question[5]

    Considere la ecuación de transporte de coeficiente constante
    \begin{math}
        \difcp{u}{t}+
        c\difcp{u}{x}=
        f\left(x,t\right)
    \end{math}.
    Reemplazaremos la derivada espacial $\difcp{u}{x}$ por una
    aproximación de diferencia finita para obtener un sistema
    semidiscretizado donde $w_{j}\left(t\right)=u\left(x_{j},t\right)$
    en una malla uniforme $\Omega_{h}$
    \begin{equation*}
        \forall j=1,\dotsc,m:
        w^{\prime}_{j}\left(t\right)=
        \dfrac{c}{2h}
        \left(
        w_{j-1}\left(t\right)-
        w_{j+1}\left(t\right)
        \right),
    \end{equation*}
    con
    \begin{math}
        w_{0}\left(t\right)=
        w_{m}\left(t\right)
    \end{math}
    y
    \begin{math}
        w_{m+1}\left(t\right)=
        w_{1}\left(t\right)
    \end{math}.
    Así, tenemos
    \begin{math}
        w^{\prime}\left(t\right)=Aw\left(t\right)
    \end{math}.
    Verifique que los autovalores de la matriz antisimétrica
    circulante
    \begin{equation*}
        A=
        \frac{c}{2h}
        \begin{bmatrix}
            0      & -1     & 0      & \cdots & 1      \\
            1      & 0      & -1     & \ddots & \vdots \\
            0      & \ddots & \ddots & \ddots & 0      \\
            \vdots & \ddots & \ddots & 0      & -1     \\
            -1     & 0      & \cdots & 1      & 0
        \end{bmatrix}\in\mathbb{R}^{m\times m}
    \end{equation*}

    tiene $m$ autovalores con parte imaginaria igual a cero.
\end{questions}
\vfill
\begin{flushright}\bfseries
    Facultad de Ciencias\\[2mm]
    Universidad Nacional de Ingeniería\\[2mm]
    \today%\unmarkedfntext{Cada pregunta vale 5 puntos.}
\end{flushright}
\end{document}

