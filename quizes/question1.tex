Considere el problema de valor de frontera Dirichlet en un disco
\begin{math}
    D=
    \left\{
    \left(x,y\right)\in\mathbb{R}^{2}\mid
    x^{2}+y^{2}<a^{2}
    \right\}
\end{math}
\begin{equation*}
    \begin{cases}
        \difc.L.{u}{}=0, & x\in D,         \\
        u=g,             & x\in\partial D.
    \end{cases}
\end{equation*}

Resolvamos este problema mediante la separación de variables.
Pasando a coordenadas polares, sea $u=u\left(r,\theta\right)$,
y notemos que $x\in\partial D$ es lo mismo que $r=a$, de modo que
$u=g\left(\theta\right)$ en $r=a$.
Reescriba el operador $\difc.L.{u}{}$ en términos de $r$ y
$\theta$ obtenga
\begin{equation*}
    \difc.L.{u}{}=
    \difcp[2]{u}{r}+
    \frac{1}{r}
    \difcp{u}{r}+
    \frac{1}{r^{2}}
    \difcp{u}{\theta}.
\end{equation*}

Luego, resuelva
\begin{equation*}
    \begin{cases}
        \difcp[2]{u}{r}+
        \dfrac{1}{r}
        \difcp{u}{r}+
        \dfrac{1}{r^{2}}
        \difcp{u}{\theta}=0,    & r<a, \\
        u=g\left(\theta\right), & r=a.
    \end{cases}
\end{equation*}