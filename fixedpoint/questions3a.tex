\subsection*{Teoremas de punto fijo en espacios métricos completos}

\begin{questions}
	\question

	Sean $\left(X,d\right)$ un espacio métrico completo y
	$f\colon X\to X$ una aplicación.
	Si $\exists k\in\mathbb{N}$ tal que
	$f^{\left(k\right)}\colon X\to X$ es contractiva, entonces
	\begin{math}
		\operatorname{Fix}\left(f\right)=
		\left\{x^{\ast}\right\}
	\end{math}
	y $\forall x\in X$:
	\begin{math}
		f^{\left(n\right)}\left(x\right)
		\xrightarrow[n\to\infty]{}
		x^{\ast}
	\end{math}.

	\textbf{Sugerencia}:
	Sean $X\neq\emptyset$ y $f\colon X\to X$.
	Si $\exists k\in\mathbb{N}$ tal que
	\begin{math}
		\operatorname{Fix}\left(f^{\left(k\right)}\right)=
		\left\{x^{\ast}\right\}
	\end{math}, entonces
	\begin{math}
		\operatorname{Fix}\left(f\right)=
		\left\{x^{\ast}\right\}
	\end{math}.

	\begin{solutionordottedlines}
	\end{solutionordottedlines}

	\question

	Sea $\left(X,d\right)$ un espacio métrico completo y
	\begin{math}
		\left\{
		f_{n}
		\right\}_{n\in\mathbb{N}}\subset
		C\left(X\right)
	\end{math}.
	Suponga que $\forall n\in\mathbb{N}$:
	$x_{n}\in\operatorname{Fix}\left(f_{n}\right)$.
	\begin{parts}
		\part

		Sean ${\left\{x_{n}\right\}}_{n\in\mathbb{N}}\subset X$,
		$x^{\ast}\in X$ y $f_{n}\rightrightarrows f$ convergente
		uniformemente en $X$.
		\begin{enumerate}[(i)]
			\item

			      Si $x_{n}\xrightarrow[n\to\infty]{}x^{\ast}$ o si
			      \begin{math}
				      f\left(x_{n}\right)\xrightarrow[n\to\infty]{}
				      x^{\ast}
			      \end{math},
			      entonces
			      $x^{\ast}\in\operatorname{Fix}\left(f\right)$.

			\item

			      Si $f\colon X\to X$ es contractiva y
			      \begin{math}
				      \operatorname{Fix}\left(f\right)=
				      \left\{x^{\ast}\right\}
			      \end{math},
			      entonces
			      $x_{n}\xrightarrow[n\to\infty]{}x^{\ast}$.
		\end{enumerate}

		\part

		Sean $f_{n}\xrightarrow[n\to\infty]{}f$ una sucesión de funciones
		lipschitzianas con $L\left(f_{n}\right)\leq M<\infty$ que
		converge puntualmente en $X$.
		\begin{enumerate}[(i)]
			\item

			      $f\colon X\to X$ es lipschitziana con
			      $L\left(f\right)\leq M$.

			\item

			      Si $x_{n}\xrightarrow[n\to\infty]{}x^{\ast}$, entonces
			      $x^{\ast}\in\operatorname{Fix}\left(f\right)$.

			\item

			      Si $M<1$, entonces
			      $x_{n}\xrightarrow[n\to\infty]{}x^{\ast}$ y
			      \begin{math}
				      \operatorname{Fix}\left(f\right)=
				      \left\{x^{\ast}\right\}
			      \end{math}.
		\end{enumerate}

		\part

		Si $\forall n\in\mathbb{N}$: $L\left(f_{n}\right)\leq M<1$ en
		(b)~(\textsc{iii}) no puede ser relajado a
		$L\left(f_{n}\right)<1$ incluso si $L\left(f\right)<1$.
		Defina
		\vspace*{-\baselineskip}\setlength\belowdisplayshortskip{0pt}
		\begin{align*}
			f_{n}\colon\ell^{2}                    & \longrightarrow\ell^{2} \\
			\left(x_{1},\dotsc,x_{n},\dotsc\right) & \longmapsto
			\left(
			0,\dotsc,\left(1-\frac{1}{n}\right)x_{n}+\frac{1}{n},0,\dotsc
			\right).
		\end{align*}
		Entonces, $\forall n\in\mathbb{N}$: $L\left(f_{n}\right)<1$,
		\begin{math}
			\left\|e_{n}\right\|=
			1
		\end{math}
		y $f_{n}\xrightarrow[n\to\infty]{}0$ converge puntualmente a la
		aplicación nula.
	\end{parts}

	\begin{solutionordottedlines}
	\end{solutionordottedlines}

	\question

	Sea $\left(X,d\right)$ un espacio métrico completo localmente
	compacto y $f\colon X\to X$ una aplicación contractiva.
	Si
	\begin{math}
		{\left\{f_{n}\right\}}_{n\in\mathbb{N}}\xrightarrow[n\to\infty]{}f
	\end{math}
	es una sucesión de aplicaciones contractivas convergente
	puntualmente, $x^{\ast}\in\operatorname{Fix}\left(f\right)$ y
	$\forall n\in\mathbb{N}$:
	$x_{n}\in\operatorname{Fix}\left(f_{n}\right)$,
	entonces $x_{n}\xrightarrow[n\to\infty]{}x^{\ast}$.
	(\cite{Nadler1968}~\Citeauthor{Nadler1968},~\citeyear{Nadler1968}).

	\question(Versión parametrizada del teorema de Banach).
	Sea $\left(X,d\right)$ un espacio métrico completo.
	$\left(\Lambda,\varrho\right)$ un espacio métrico y
	${\left\{H_{\lambda}\right\}}_{\lambda\in\Lambda}$ una familia de
	aplicaciones contractivas de $X$ en sí mismo.
	Suponga que $H\colon X\times\Lambda\to X$ es continua en la segunda
	variable y $\forall\lambda\in\Lambda$:
	\begin{math}
		\operatorname{Fix}\left(H_{\lambda}\right)=
		\left\{x_{\lambda}\right\}
	\end{math}.
	\begin{parts}
		\part

		Si $\forall\lambda\in\Lambda$: $H_{\lambda}$ son
		$\alpha$-contractiva con $0<\alpha<1$, entonces
		$\lambda\longmapsto x_{\lambda}$ es continua.

		\part

		Si además, $\left(X,d\right)$ es un espacio métrico localmente
		compacto, entonces $\lambda\longmapsto x_{\lambda}$ es continua.
	\end{parts}
	\textbf{Sugerencia}: Para (b), use la pregunta~3.

	\question

	Sea $\left(X,d\right)$ un espacio métrico completo,
	$f\colon X\to X$ una aplicación y
	\begin{math}
		{\left\{\alpha_{n}\right\}}_{n\in\mathbb{N}}\subset
		\mathbb{R}_{\geq0}
	\end{math}
	con $\sum_{n=1}^{\infty}\alpha_{n}<\infty$.
	Si $\forall n\in\mathbb{N}$:
	$\forall x,y\in X$:
	\begin{math}
		d\left(
		f^{\left(n\right)}\left(x\right),
		f^{\left(n\right)}\left(y\right)
		\right)\leq
		\alpha_{n}d\left(x,y\right)
	\end{math},
	entonces
	$\operatorname{Fix}\left(f\right)=\left\{x^{\ast}\right\}$ y
	$\forall x\in X$:
	$f^{\left(n\right)}\left(x\right)\xrightarrow[n\to\infty]{}x^{\ast}$
	(\cite{Weissinger1952}~\Citeauthor{Weissinger1952},~\citeyear{Weissinger1952}).

	\question

	Sean $\left(X,d\right)$ un espacio métrico completo y
	$f\colon X\to X$ una aplicación.
	Si $\forall S\subset X$ cerrado con\linebreak
	$\operatorname{diam}\left(S\right)>0$:
	$\exists q\in\left[0,1\right)$ tal que
	\begin{math}
		\operatorname{diam}\left(f\left(S\right)\right)\leq
		q\operatorname{diam}\left(S\right)
	\end{math},
	entonces $\operatorname{Fix}\left(f\right)\neq\emptyset$
	(\cite{Amann1982}~\Citeauthor{Amann1982},~\citeyear{Amann1982}).

	\question

	Sean $\left(X,d\right)$ un espacio métrico completo y
	$f\colon X\to X$ una aplicación tal que $\forall x,y\in X$ con $x\neq y$:
	\begin{math}
		d\left(f\left(x\right),f\left(y\right)\right)<
		d\left(x,y\right)
	\end{math}.
	\begin{parts}
		\part

		Si $\exists x_{0}\in X$ tal que
		\begin{math}
			{\left\{f^{\left(n\right)}\left(x_{0}\right)\right\}}_{n\in\mathbb{N}}
		\end{math}
		posee una subsucesión convergente, entonces
		$\left|\operatorname{Fix}\left(f\right)\right|=1$.

		\part

		Si $\overline{f\left(X\right)}$ es compacto, entonces
		\begin{math}
			\operatorname{Fix}\left(f\right)=
			\left\{x^{\ast}\right\}
		\end{math}
		y $\forall x\in X$:
		\begin{math}
			f^{\left(n\right)}\left(x\right)\xrightarrow[n\to\infty]{}
			x^{\ast}
		\end{math}.

		\part

		Construya una aplicación $f\colon X\to X$ que satisface la
		desigualdad de arriba excepto en los puntos fijos y tal que
		$\exists x_{0},y_{0}\in X$, la sucesión
		\begin{math}
			d\left(
			f^{\left(n\right)}\left(x_{0}\right),
			f^{\left(n\right)}\left(y_{0}\right)\right)
			\nrightarrow 0
		\end{math}.

		\textbf{Sugerencia}:
		Considere la aplicación
		$x\longmapsto\ln\left(1+\exp\left(x\right)\right)$ de $\mathbb{R}$
		en sí mismo.
	\end{parts}

	\question

	Sea $\left(Y,d\right)$ un espacio métrico completo.
	$f\colon Y\to Y$ es expansivo si y solo si
	$\exists\beta>1$ tal que $\forall x,y\in Y$:
	\begin{math}
		d\left(f\left(x\right),f\left(y\right)\right)\geq
		\beta d\left(x,y\right)
	\end{math}.
	Suponga que $f\colon Y\to Y$ es sobreyectiva y expansiva.
	\begin{parts}
		\part

		$f$ es biyectiva.

		\part

		\begin{math}
			\operatorname{Fix}\left(f\right)=
			\left\{x^{\ast}\right\}
		\end{math}
		y $\forall y\in Y$:
		$f^{-n}\left(y\right)\xrightarrow[n\to\infty]{}x^{\ast}$.
	\end{parts}

	\question

	Sean $\left(E,\left\|\cdot\right\|\right)$ un espacio de Banach y
	$f\colon E\to E$ un operador lineal tal que
	$\exists{\left(I-f\right)}^{-1}\colon E\to E$.

	\begin{parts}
		\part

		Si $g\colon E\to E$ es lipschitziana con
		$\left\|{\left(I-f\right)}^{-1}\right\|<\frac{1}{L\left(g\right)}$,
		entonces $\left|\operatorname{Fix}\left(f+g\right)\right|=1$.

		\part

		Sean $r>0$, $q\in\left(0,1\right)$ y $K=K\left(0,r\right)$.
		Suponga que $g\colon K\left(0,r\right)\to E$ es lipschitziana tal
		que
		\begin{math}
			\left\|g\left(0\right)\right\|\leq
			\frac{\left(1-q\right)r}{\left\|{\left(I-F\right)}^{-1}\right\|}
		\end{math}.
		Si
		\begin{math}
			\left\|\left(I-F\right)^{-1}\right\|<
			\frac{q}{L\left(g\right)}
		\end{math},
		entonces $\left|\operatorname{Fix}\left(f+g\right)\right|=1$.
	\end{parts}

	\question

	Sean $\left(X,d\right)$ un espacio métrico completo,
	$f\colon X\to X$ una aplicación $\alpha$-contractiva y
	$x^{\ast}\in\operatorname{Fix}\left(f\right)$.
	Si $\forall\varepsilon>0$: $\exists\beta>0$ con $\alpha+\beta<1$
	tal que $\forall g\colon X\to X$ es
	$\left(\alpha+\beta\right)$-contractiva, entonces $\forall x\in X$
	con $d\left(f\left(x\right),g\left(x\right)\right)<\beta$:
	\begin{math}
		y_{0}=
		g\left(x^{\ast}\right)\in
		B\left(x^{\ast},\varepsilon\right)
	\end{math}.

	\question

	Sean $\left(E,\left\|\cdot\right\|\right)$ un espacio de Banach,
	$U\subset E$ un conjunto abierto y acotado tal que $0\in U$.
	Si $f,g\colon\overline{U}\to E$ son dos aplicaciones contractivas
	tales que $f\big|_{\partial U}=g\big|_{\partial U}$.
	Entonces,
	\begin{math}
		\operatorname{Fix}(f)\neq\emptyset\iff
		\operatorname{Fix}\left(g\right)\neq\emptyset
	\end{math}.

	\question

	Sean $\left(E,\left\|\cdot\right\|\right)$ un espacio de Banach y
	$U\subset E$ un conjunto abierto y acotado tal que $U=-U$ y
	$V\subset U$ es una vecindad abierta del origen tal que
	$\overline{V}\subset U$.
	Si $f\colon\overline{U}\to E$ es una aplicación contractiva tal que
	\begin{parts}
		\part

		$\forall x\in\partial U$:
		$f\left(x\right)=-f\left(-x\right)$.

		\part
		$\exists x_{0}$ tal que $\forall x\in\partial V$:
		$\forall\lambda\geq0$: $x\neq f\left(x\right)+\lambda x_{0}$.
	\end{parts}
	Entonces,
	\begin{math}
		\operatorname{Fix}\left(f\right)=
		\left\{x^{\ast}\right\}\subset
		\overline{U}-V
	\end{math}.

	\question

	Sean $\left(E=A\oplus B,\left\|\cdot\right\|\right)$ un espacio de
	Banach representado como una suma directa de dos subespacios
	vectoriales cerrados $A$ y $B$ con proyecciones
	$\Pi_{A}\colon E\to A$ y $\Pi_{B}\colon E\to B$,
	$F\colon A\to E$ y $G\colon B\to E$ dos aplicaciones lipschitzianas
	y $f\colon A\to E$, $g\colon B\to E$ dados por
	$a\longmapsto a-F\left(a\right)$ y
	$b\longmapsto b-G\left(b\right)$, respectivamente.
	Si
	\begin{math}
		\left\|\Pi_{A}\right\|
		L\left(F\right)+
		\left\|\Pi_{B}\right\|
		L\left(G\right)<1
	\end{math},
	entonces $\left|f\left(A\right)\cap g\left(B\right)\right|=1$.

	\textbf{Sugerencia}:
	Pruebe que si $H\colon E\to E$ es contractiva, entonces
	$x\longmapsto x+H\left(x\right)$ es un homeomorfismo.

	\question
	(Teorema de Banach discreto).
	Sea $Y\neq\emptyset$ un conjunto y
	\begin{math}
		\left\{R_{n}\right\}_{n\in\mathbb{N}\cup\left\{0\right\}}\subset
		Y\times Y
	\end{math}
	una sucesión de relaciones de equivalencias tales que
	\begin{parts}
		\part

		$\forall n\in\mathbb{N}\cup\left\{0\right\}$:
		$R_{n+1}\subset R_{n}$, donde $R_{0}=Y\times Y$.

		\part

		\begin{math}
			\bigcap_{n=0}^{\infty}R_{n}=
			\left\{\left(y_{1},y_{2}\right)\in
			Y\times Y\mid y_{1}=y_{2}\right\}
		\end{math}.

		\part

		Si
		\begin{math}
			\forall{\left\{y_{n}\right\}}_{n\in\mathbb{N}\cup\left\{0\right\}}\subset
			Y
		\end{math}
		tal que $\forall n\in\mathbb{N}\cup\left\{0\right\}$: $\left(y_{n},y_{n+1}\right)\in R_{n}$,
		entonces $\exists y\in Y$ tal que $\forall n\in\mathbb{N}\cup\left\{0\right\}$:
		$\left(y_{n},y\right)\in R_{n}$.
	\end{parts}
	Sea $f\colon Y\to Y$ una aplicación tal que
	$\forall\left(x,y\right)\in R_{n}$:
	$\left(f\left(x\right),f\left(y\right)\right)\in R_{n+1}$.
	Entonces,
	$\operatorname{Fix}\left(f\right)=\left\{x^{\ast}\right\}$ y
	$\forall n\in\mathbb{N}$: $\forall y\in Y$:
	$\left(f^{n}\left(y\right),x^{\ast}\right)\in R_{n}$.
	(\cite{Eilenberg1946}~\Citeauthor{Eilenberg1946},~\citeyear{Eilenberg1946}).
\end{questions}
