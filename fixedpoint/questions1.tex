\question

Si
\begin{math}
	f\colon\left[0,1\right]\to
	\left[0,1\right]
\end{math}
es una función continua, entonces
\begin{math}
	\operatorname{Fix}\left(f\right)\coloneqq
	\left\{x\in\operatorname{dom}\left(f\right)\mid
	f\left(x\right)=x\right\}\neq
	\emptyset
\end{math}.

\begin{solutionordottedlines}
	Vea \url{https://math.stackexchange.com/q/512786}.
\end{solutionordottedlines}


\question

Sea
\begin{math}
	f\colon\left[0,1\right]\to
	\left[0,1\right]
\end{math}
una función continua.
Si $\forall x\in\operatorname{Fix}\left(f\right)$: $f^{\prime}\left(x\right)<1$,
entonces
\begin{math}
	\left|\operatorname{Fix}\left(f\right)\right|=
	1
\end{math}.

\begin{solutionordottedlines}
	Vea \url{https://math.stackexchange.com/q/4516410}.
\end{solutionordottedlines}

\question

Sea
\begin{math}
	f\colon\mathbb{R}\to\mathbb{R}
\end{math}
una función.
Si $\left|\operatorname{Fix}\left(f\circ f\right)\right|=1$, entonces
$\left|\operatorname{Fix}\left(f\right)\right|=1$.

\begin{solutionordottedlines}
	Vea \url{}.
\end{solutionordottedlines}

\question

Sean
\begin{math}
	A=
	\begin{bmatrix}
		\alpha & -\beta \\
		\beta  & \alpha
	\end{bmatrix}\in
	\mathbb{R}^{2\times 2}
\end{math}
y
\begin{math}
	b=
	\begin{bmatrix}
		\gamma \\
		\delta
	\end{bmatrix}\in
	\mathbb{R}^{2}
\end{math}.
Se define $\forall x\in\mathbb{R}^{2}$: $Tx=Ax+b$.
Demuestre que
$\operatorname{Fix}\left(T\right)\neq\emptyset$ si y solo si $\det\left(A\right)=1$.

\begin{solutionordottedlines}
	Vea \url{}.
\end{solutionordottedlines}

\question

Sean
\begin{math}
	B=
	\left\{x\in\mathbb{R}^{n}\mid{\left\|x\right\|}_{2}\leq 1\right\}
\end{math}
y $f\colon B\to B$ una función.
Si $\forall x,y\in B$: ${\left\|f\left(x\right)-f\left(y\right)\right\|}_{2}\leq{\left\|x-y\right\|}_{2}$, entonces
$\operatorname{Fix}\left(f\right)\neq\emptyset$.

\begin{solutionordottedlines}
	Vea \url{https://math.stackexchange.com/q/1745971}.
\end{solutionordottedlines}

\question

Sea $\left(X,d\right)$ un espacio métrico.
Si $X$ es compacto y $\forall x,y\in X\text{ distintos}$:
$d\left(T\left(x\right),T\left(y\right)\right)<d\left(x,y\right)$,
entonces $\left|\operatorname{Fix}\left(T\right)\right|=1$.

\begin{solutionordottedlines}
	Vea \url{}.
\end{solutionordottedlines}

% \question

% Sea $\kappa\colon\left[a,b\right]\times\left[a,b\right]\to\mathbb{R}$
% una función continua.
% Dada $g\colon\left[a,b\right]\to\mathbb{R}$ considere

% \begin{solutionordottedlines}
% 	Vea \url{}.
% \end{solutionordottedlines}

% \question

% Sea $\Omega\subset\mathbb{R}^{3}$ un conjunto

% \begin{solutionordottedlines}
% 	Vea \url{}.
% \end{solutionordottedlines}

\question

Sea la sucesión dada por $x_{n+1}=\sqrt{2+\sqrt{x_{n}}}$ con
$x_{1}=\sqrt{2}$.
Use el Teorema del Punto Fijo de Banach para probar que
$\left\{x_{n}\right\}_{n\in\mathbb{N}}$ converge a la raíz del
polinomio $x^{4}-4x^{2}-x+4=0$ que cae entre $\sqrt{3}$ y $2$.

\begin{solutionordottedlines}
	Vea \url{}.
\end{solutionordottedlines}

\question

¿Es cierto que cualquier polinomio de grado impar con coeficiente
real tendrá al menos un punto fijo?

\question

Sea $\left(X,d\right)$ un espacio métrico completo y
$\left\{f_{n}\right\}_{n\in\mathbb{N}}$ una sucesión de contracciones
con constantes de Lipschitz $q_{i}$.
Entonces, existe un único conjunto compacto  tal que
$K=\bigcup_{i=1}^{n}f_{i}\left(K\right)$.

\begin{solutionordottedlines}
	Vea \url{https://math.stackexchange.com/q/4984016}.
\end{solutionordottedlines}

\question

Encuentre los valores de $p$, $q$ y $r$ en
$x_{k+1}=px_{k}+\frac{qc}{x_{k}}+\frac{rc^{2}}{x^{3}_{k}}$
para garantizar que la sucesión converge a $\sqrt{c}$ con la
mayor tasa de convergencia.

\begin{solutionordottedlines}
	Vea \url{https://math.stackexchange.com/q/4975765}.
\end{solutionordottedlines}

% \question

% Sea $f\in C^{2}\left[a,b\right]$ tal que $\exists x_{0}\in\mathbb{R}$.

% \begin{solutionordottedlines}
% 	Vea \url{}.
% \end{solutionordottedlines}

% \question

% Considere $f\left(z\right)=z^{2}+1$.

% \begin{solutionordottedlines}
% 	Vea \url{}.
% \end{solutionordottedlines}

\question

Sea $D\left(0,1\right)=\left\{z\in\mathbb{C}\mid\left|z\right|<1\right\}$.
Pruebe que cualquier función holomorfa
$f\colon\overline{D\left(0,1\right)}\to\overline{D\left(0,1\right)}$ tiene un punto fijo.

% \begin{solutionordottedlines}
% 	Vea \url{}.
% \end{solutionordottedlines}

\question

Muestre que $f\left(t\right)=t+\frac{1}{1+e^{t}}$ no tiene punto fijo, aunque se cumpla que
$\forall t\in\mathbb{R}$: $0<f^{\prime}\left(t\right)<1$.

\begin{solutionordottedlines}
	Vea \url{https://math.stackexchange.com/q/2290165}.
\end{solutionordottedlines}

\question

Sea $g\colon D\subset\mathbb{R}^{n}\to\mathbb{R}^{n}$.
Se define $x^{\left(k+1\right)}\coloneqq g\left(x^{\left(k\right)}\right)$.
Si $D$ es un conjunto cerrado y $\forall x\in D: g\left(x\right)\in D$ y
$\forall x,y\in D$: $\exists q<1$ tal que $\left\|g\left(x\right)-g\left(y\right)\right\|\leq q\left\|x-y\right\|$,
entonces $\exists\,! x^{\ast}\in D$ con $g\left(x^{\ast}\right)=x^{\ast}$, $\forall k\in\mathbb{N}$ se tiene que
\begin{align}
	\left\|x^{\left(k\right)}-x^{\ast}\right\| & \leq
	\frac{q^{k}}{1-q}
	\left\|x^{\left(1\right)}-x^{\left(0\right)}\right\|.\tag{Estimación de error a priori} \\
	\left\|x^{\left(k\right)}-x^{\ast}\right\| & \leq
	\frac{q}{1-q}
	\left\|x^{\left(k\right)}-x^{\left(k-1\right)}\right\|.\tag{Estimación de error a posteriori}
\end{align}
% https://terpconnect.umd.edu/~petersd/666/fixedpoint.pdf
