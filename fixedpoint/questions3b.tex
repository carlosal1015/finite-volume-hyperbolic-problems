\subsection*{Extensiones del teorema de Banach}

\begin{questions}
	\question

	Un espacio métrico $\left(X,d\right)$ es $\varepsilon$-encadenable
	si y solo si $\forall x,y\in X$:
	\begin{math}
		\exists
		\left\{
		x=x_{0},
		x_{1},
		\ldots,
		x_{n},
		x_{n+1}=y
		\right\}\subset
		X
	\end{math}
	tal que $\forall i\in\left\{0,\dotsc,n\right\}$:
	$d\left(x_{i},x_{i+1}\right)<\varepsilon$.
	Sea $\left(X,d\right)$ un espacio métrico completo y
	$f\colon X\to X$ una aplicación.
	Si $\left(X,d\right)$ es $\varepsilon$-encadenable y
	$\exists q\in\left[0,1\right)$ tal que
	$d\left(x, y\right)<\varepsilon$:
	\begin{math}
		d\left(f\left(x\right),f\left(y\right)\right)\leq
		qd\left(x,y\right)
	\end{math},
	entonces $\left|\operatorname{Fix}\left(f\right)\right|=1$
	(\cite{Edelstein1961}~\Citeauthor{Edelstein1961},~\citeyear{Edelstein1961}).

	\question

	Sea $\left(X,d\right)$ un espacio métrico completo y
	$f\colon X\to X$ una aplicación tal que $\forall x,y\in X$:
	$d\left(f\left(x\right),f\left(y\right)\right)\leq\varphi[d\left(x,y\right)]$
	donde $\varphi\colon\mathbb{R}^{+}\to\mathbb{R}^{+}$ es
	cualquier función tal que
	\begin{parts}
		\part

		$\varphi$ es no decreciente.

		\part

		$\forall t>0$: $\varphi\left(t\right)<t$.

		\part

		$\varphi$ es continua por la derecha.
	\end{parts}
	Si
	\begin{math}
		\operatorname{Fix}\left(f\right)=
		\left\{x^{\ast}\right\}
	\end{math}
	y $\forall x\in X$:
	\begin{math}
		f^{\left(n\right)}\left(x\right)
		\xrightarrow[n\to\infty]{}
		x^{\ast}
	\end{math}.
	(\cite{Browder1968}~\Citeauthor{Browder1968},~\citeyear{Browder1968}).

	\textbf{Sugerencia}:
	Aplique el Teorema~\ref{thm:1} mostrando que $\forall t>0$:
	$\varphi^{\left(n\right)}\left(t\right)\xrightarrow[n\to\infty]{}0$.
	\begin{theorem}\label{thm:1}
		Sean $\left(X,d\right)$ un espacio métrico completo y
		$f\colon X\to X$ una aplicación tal que $\forall x,y\in X$:
		\begin{math}
			d\left(f\left(x\right),f\left(y\right)\right)\leq
			\varphi\left[d\left(x,y\right)\right]
		\end{math},
		donde $\varphi\colon\mathbb{R}^{+}\to\mathbb{R}^{+}$ es cualquier
		función no decreciente tal que $\forall t>0$:
		\begin{math}
			\varphi^{\left(n\right)}\left(t\right)
			\xrightarrow[n\to\infty]{}0
		\end{math}.
		Si
		\begin{math}
			\operatorname{Fix}\left(f\right)=
			\left\{x^{\ast}\right\}
		\end{math}
		y $\forall x\in X$:
		\begin{math}
			f^{\left(n\right)}\left(x\right)
			\xrightarrow[n\to\infty]{}
			x^{\ast}
		\end{math}.
	\end{theorem}

	\question

	Sea $\left(X,d\right)$ un espacio métrico completo y
	$f\colon X\to X$ una aplicación continua.
	Si $\exists q\in\left[0,1\right)$ tal que
	$\forall x, y \in X$:
	\begin{math}
		d\left(f\left(x\right),f\left(y\right)\right)\leq
		q
		\max\left\{
		d\left(x,y\right),
		d\left(x,f\left(x\right)\right),
		d\left(y,f\left(y\right)\right),
		d\left(x,f\left(y\right)\right),
		d\left(y,f\left(x\right)\right)
		\right\}
	\end{math},
	entonces
	\begin{math}
		\operatorname{Fix}\left(f\right)=
		\left\{x^{\ast}\right\}
	\end{math}
	y $\forall x\in X$:
	\begin{math}
		f^{\left(n\right)}\left(x\right)
		\xrightarrow[n\to\infty]{}
		x^{\ast}
	\end{math}
	(\cite{Ciric1971}~\Citeauthor{Ciric1971},~\citeyear{Ciric1971}).

	\textbf{Sugerencia}:
	Use
	\begin{math}
		\operatorname{diam}
		\left(
		\operatorname{orbit}
		f^{n}\left(x\right)
		\right)\leq
		k
		\operatorname{diam}
		\left(
		\operatorname{orbit}
		f^{n-1}\left(x\right)
		\right)
	\end{math}.

	\question

	Sea $\left(X,d\right)$ un espacio métrico completo y
	$f\colon X\to X$ una aplicación continua.
	Suponga que $\forall\varepsilon>0$: $\forall x,y\in X$:
	$\exists n\in\mathbb{N}$ tal que
	\begin{math}
		d\left(
		f^{\left(n\right)}\left(x\right),
		f^{\left(n\right)}\left(y\right)
		\right)<
		\varepsilon
	\end{math}.
	Si $\varphi\colon X\to\mathbb{R}^{+}$ es una función definida por
	\begin{math}
		\varphi\left(x\right)\coloneqq
		d\left(x,f\left(x\right)\right)
	\end{math}
	tal que
	$\forall a>0$:
	\begin{math}
		\inf
		\left\{
		\varphi\left(x\right)+
		\varphi\left(y\right)\mid
		d\left(x,y\right)\geq a
		\right\}>0
	\end{math},
	entonces $\operatorname{Fix}\left(f\right)\neq\emptyset$
	(\cite{Bailey1966}~\Citeauthor{Bailey1966},~\citeyear{Bailey1966}).

	\question

	Sean $\left(X,d\right)$ un espacio métrico completo y
	$f\colon X\to X$ una aplicación continua.
	Si $\exists n\in\mathbb{Z}$ y $q\in\left[0,1\right)$ tales que
	$\forall x,y,z\in X$:
	\begin{math}
		d\left(f\left(x\right),f\left(y\right)\right)\leq
		q
		\left[
			d\left(x,f^{\left(n\right)}\left(z\right)\right)+
			d\left(y,f^{\left(n\right)}\left(z\right)\right)
			\right]
	\end{math},
	entonces $\left|\operatorname{Fix}\left(f\right)\right|=1$
	(\cite{Pittnauer1975}~\Citeauthor{Pittnauer1975},~\citeyear{Pittnauer1975}).

	\question

	Sean $\left(X,d\right)$ un espacio métrico y $A\subset X$ un
	conjunto compacto.
	Si $\varphi\colon X\to\mathbb{R}^{+}$ es una función tal que
	$\forall a>0$:
	\begin{math}
		\inf
		\left\{
		\varphi\left(x\right)\mid
		d\left(x,A\right)\geq a
		\right\}>0
	\end{math},
	entonces $\forall{\left\{x_{n}\right\}}_{n\in\mathbb{N}}\subset X$
	con $\varphi\left(x_{n}\right)\xrightarrow[n\to\infty]{}0$
	posee una subsucesión convergente hacia algún punto de $A$.

	\question

	Sea $\left(X,d\right)$ un espacio métrico y $f\colon X\to X$ una
	aplicación tal que $\forall x,y\in X$, $x\neq y$:
	\begin{math}
		d\left(f\left(x\right),f\left(y\right)\right)<
		d\left(x,y\right)
	\end{math}.
	Si $\exists z\in X$ tal que
	\begin{math}
		{
			\left\{f^{\left(n\right)}\left(z\right)\right\}
		}_{n\in\mathbb{N}}\subset X
	\end{math}
	posee una subsucesión convergente a $x^{\ast}\in X$,
	entonces $x^{\ast}\in\operatorname{Fix}\left(f\right)$.

	\textbf{Sugerencia}:
	Use la pregunta 6 con
	\begin{math}
		\varphi\left(x\right)\coloneqq
		d\left(x,f\left(x\right)\right)-
		d\left(f\left(x\right),f^{\left(2\right)}\left(x\right)\right)+
		d\left(x,u\right)
	\end{math}.

	\question

	Sean $\left(X,d\right)$ es un espacio métrico y $f\colon X\to X$
	una aplicación.
	Denotamos el diámetro de la órbita
	\begin{math}
		\left\{
		f^{\left(n\right)}\left(x\right)\mid
		n\in\mathbb{N}\cup\left\{0\right\}
		\right\}
	\end{math}
	de $x\in X$
	por $\delta\left(x\right)$ y
	decimos que $f$ contrae órbitas si y solo si
	$\forall x\in X$ con $\delta\left(x\right)>0$:
	$\exists n\in\mathbb{N}\cup\left\{0\right\}$ tal que
	\begin{math}
		\delta\left(
		f^{\left(n\right)}\left(x\right)
		\right)<
		\delta\left(x\right)
	\end{math}.

	Suponga que $\left(X,d\right)$ un espacio métrico acotado y
	$f\colon X\to X$ una aplicación tal que $\forall x,y\in X$:
	\begin{math}
		d\left(f\left(x\right),f\left(y\right)\right)\leq
		d\left(x,y\right)
	\end{math}.
	Si $\exists z\in X$ tal que
	\begin{math}
		{
			\left\{f^{\left(n\right)}\left(z\right)\right\}
		}_{n\in\mathbb{N}}
	\end{math}
	posee una
	subsucesión convergente a $x^{\ast}$ y $f$ contrae órbitas,
	entonces $x^{\ast}\in\operatorname{Fix}\left(f\right)$.

	\textbf{Sugerencia}:
	Muestre que $x\longmapsto\delta\left(x\right)$ es continua en $X$,
	entonces aplique la pregunta 6 usando $\forall s\geq 1$:
	\begin{math}
		\varphi\left(x\right)\coloneqq
		\delta\left(x\right)-
		\delta\left(f^{\left(s\right)}\left(x\right)\right)+
		d\left(x,x^{\ast}\right)
	\end{math}.

	\question

	Sean $\left(X,d\right)$ un espacio métrico y $f\colon X\to X$ una
	aplicación continua.
	Si $\exists z\in X$ tal que
	\begin{math}
		{\left\{f^{\left(n\right)}\left(z\right)\right\}}_{n\in\mathbb{N}}
	\end{math}
	posee una subsucesión convergente
	\begin{math}
		\left\{f^{\left(n_{k}\right)}\left(z\right)\right\}_{k\in\mathbb{N}}
	\end{math}
	con
	\begin{math}
		d\left(
		f^{\left(n_{k}\right)}\left(z\right),
		f^{\left(1+n_{k}\right)}\left(z\right)
		\right)
		\xrightarrow[k\to\infty]{}0
	\end{math},
	entonces $\operatorname{Fix}\left(f\right)\neq\emptyset$.

	\textbf{Sugerencia}:
	Use la pregunta 6 con
	\begin{math}
		\varphi\left(x\right)\coloneqq
		d\left(x,x^{\ast}\right)+
		d\left(f\left(x\right),x^{\ast}\right)
	\end{math},
	donde
	\begin{math}
		x^{\ast}=
		\lim_{k\to\infty}
		f^{\left(n_{k}\right)}\left(x\right)
	\end{math}.

	\question

	Sean $\left(X,d\right)$ un espacio métrico y $f\colon X\to X$ una
	aplicación continua.
	Suponga que $\exists V\colon X\times X\to\mathbb{R}^{+}$ una
	función continua con
	$V^{-1}\left(0\right)\subset\operatorname{graph}\left(f\right)$
	tal que $\inf\left\{V\left(x,x\right)\mid x\in X\right\}=0$.
	\begin{parts}
		\part

		Si $A\subset X$ es un conjunto compacto y
		$\varphi\colon X\to\mathbb{R}^{+}$ es una función definida por
		$\varphi\left(x\right)\coloneqq V\left(x,x\right)$ tal que
		$\forall a>0$:
		\begin{math}
			\inf
			\left\{
			\varphi\left(x\right)\mid
			d\left(x,A\right)\geq a
			\right\}>0
		\end{math},
		entonces $\operatorname{Fix}\left(f\right)\neq\emptyset$.

		\part

		Si además $\left(X,d\right)$ es un espacio métrico completo y
		$\varphi\colon X\to\mathbb{R}^{+}$ es una función definida por
		$\varphi\left(x\right)\coloneqq V\left(x,x\right)$ tal que
		$\forall a>0$:
		\begin{math}
			\inf
			\left\{
			\varphi\left(x\right)+
			\varphi\left(y\right)\mid
			d\left(x,y\right)\geq a
			\right\}>0
		\end{math},
		entonces $\operatorname{Fix}\left(f\right)\neq\emptyset$.
	\end{parts}
\end{questions}
