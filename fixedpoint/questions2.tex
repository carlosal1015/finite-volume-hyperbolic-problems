\question

Sea $I=\left[a,b\right]\subset\mathbb{R}$.
Si $f\colon I\to I$ es continua, derivable en
$\operatorname{Int}\left(I\right)$ y $\exists q\in\left[0,1\right)$
tal que $\forall\xi\in\operatorname{Int}\left(I\right)$:
$\left|f^{\prime}\left(\xi\right)\right|\leq q$, entonces
$\left|\operatorname{Fix}\left(f\right)\right|=1$.

\begin{solutionordottedlines}
\end{solutionordottedlines}

\question

Sea $q\in\left[0,1\right)$.
Se define $\forall t\in\left(0,\infty\right)$: $f\left(t\right)=qt$.
Demuestre que $\operatorname{Fix}\left(f\right)\neq\emptyset$.

\begin{solutionordottedlines}
\end{solutionordottedlines}

\question

Sea
\begin{math}
  B_{r}=
  \left\{x\in\mathbb{R}^{n}\mid{\left\|x\right\|}_{2}<r\right\}
\end{math}.
Si $g\colon\overline{B}_{r}\to\mathbb{R}^{n}$ es una función tal que
$\forall x\in\partial B_{r}$:
${\left\langle g\left(x\right),x\right\rangle}_{2}\geq 0$, entonces
$\operatorname{Fix}\left(g\right)\neq\emptyset$.

\begin{solutionordottedlines}
\end{solutionordottedlines}

\question

Sea $\left(E,\left\|\cdot\right\|\right)$ un espacio de Banach,
$C\subset E$ un conjunto abierto y convexo con $0\in C$.
Demuestre que $\forall x\in E\setminus C$:
$\exists!q\in\left(0,1\right]$ tal que $qx\in\partial C$.

\begin{solutionordottedlines}
\end{solutionordottedlines}

\question

Sea $\left(E,\left\|\cdot\right\|\right)$ un espacio de Banach,
$K\subset X$ un conjunto compacto y $f\colon K\to E$ es una función.
Si $\forall n\in\mathbb{N}$: $f_{n}\colon K\to E$ son funciones
continuas y
\begin{math}
  \lim\limits_{n\to\infty}
  \sup\limits_{x\in K}
  \left\|f_{n}\left(x\right)-f\left(x\right)\right\|=
  0
\end{math},
entonces $f$ es continua.
Además, si $\forall n\in\mathbb{N}$:
$\operatorname{Fix}\left(f_{n}\right)\neq\emptyset$, entonces
$\operatorname{Fix}\left(f\right)\neq\emptyset$.

\begin{solutionordottedlines}
\end{solutionordottedlines}

\question

Sea
\begin{math}
  \left(
    C\left(\left[0,1\right]\right),
    {\left\|\cdot\right\|}_{\text{unif}}
  \right)
\end{math}
un espacio de Banach.
Defina la función no lineal
\begin{equation*}
  \forall x\in C\left(\left[0,1\right]\right):
  \forall t\in[0,1]:
  G\left(x\right)\left(t\right)\coloneqq
  \int_{0}^{1}
  \kappa\left(t,s\right)
  \alpha\left(s,x\left(s\right)\right)\dl s,
\end{equation*}
donde $k\in C\left(\left[0,1\right]\times\left[0,1\right]\right)$ y
$\alpha\in C\left(\left[0,1\right]\times\mathbb{R}\right)$.
Demuestre que si cumplen las desigualdades con $L>0$
\begin{align*}
  \forall s\in\left[0,1\right]:
  \forall u,v\in\mathbb{R}:
  \left|f\left(s,u\right)-f\left(s,v\right)\right|&\leq
  L\left|u-v\right|.\\
  \max_{t\in\left[0,1\right]}
  \int_{0}^{1}
  \left|\kappa\left(t,s\right)\right|\dl s&<
  \frac{1}{L}.
\end{align*}
Entonces
\begin{math}
  G\colon C\left(\left[0,1\right]\right)\to
  C\left(\left[0,1\right]\right)
\end{math}
es una contracción.

\begin{solutionordottedlines}
\end{solutionordottedlines}

\question

Sea $\left(E,\left\|\cdot\right\|\right)$ un espacio de Banach,
$K\subset X$ un conjunto compacto y $f\colon K\to K$ es contractiva,
es decir,
\begin{equation*}
  \forall x,y\in K\text{ distintos}:
  \left\|f\left(x\right)-f\left(y\right)\right\|<
  \left\|x-y\right\|.
\end{equation*}
Demuestre que $\left|\operatorname{Fix}\left(f\right)\right|=1$.
Además, demuestre que la función
$x\longmapsto\left\|x-f\left(x\right)\right\|$ es monótona
decreciente sobre los puntos generados por la sucesión convergente
dada por
\begin{equation*}
  \left\{
    \begin{aligned}
      x^{\left(k+1\right)}=f\left(x^{\left(k\right)}\right).\\
      x^{\left(0\right)}=x\in E.
    \end{aligned}
    \right.
  \end{equation*}

  \begin{solutionordottedlines}
  \end{solutionordottedlines}

  \question

  Si $\left(X,d\right)X$ es un espacio métrico compacto, entonces $X$
  es separable y también es completo.

  \begin{solutionordottedlines}
  \end{solutionordottedlines}

  \question

  Sea
  \begin{math}
    \left(
      C\left(\left[a,b\right]\right),
      \left\|\cdot\right\|_{\max}
    \right)
  \end{math}.
  Demuestre que $C\left(\left[a,b\right]\right)$ es separable, pero
  no es compacto.

  \begin{solutionordottedlines}
  \end{solutionordottedlines}

  \question

  Encuentre una reformulación como una ecuación de punto fijo en
  $\mathbb{R}^{2}$ y resuelva el sistema de ecuaciones no lineales
  por el teorema de punto fijo de Brouwer.
  \begin{equation*}
    \left\{
      \begin{aligned}
        2x^{2}_{1}-x^{2}_{2}-8x_{1}&=0.\\
        x^{2}_{1}+x_{1}x_{2}-4x_{2}+1&=0.
      \end{aligned}
      \right.
    \end{equation*}

    \begin{solutionordottedlines}
    \end{solutionordottedlines}

    \question

    Considere la siguiente función no lineal
    \begin{align*}
      f\colon\mathbb{R}^{2}&\longrightarrow\mathbb{R}^{2}\\
      \left(x_{1},x_{2}\right)&\longmapsto
      \frac{1}{6}
      \left(
        x_{1}e^{-x^{2}_{2}}+x_{1}x_{2}+3,
        \ln\left(1+x_{1}^{2}+
        x^{2}_{2}\right)-1
      \right)
    \end{align*}

    Considere la norma
    \begin{math}
      {\left\|\left(x_{1},x_{2}\right)\right\|}_{\infty}=
      \max\left\{\left|x_{1}\right|,\left|x_{2}\right|\right\}
    \end{math}
    y demuestre cada una de las siguientes proposiciones.

    \begin{parts}
      \part

      $f$ es Lipschitz continua en
      $\left[0,1\right]\times\left[0,1\right]$ con constante de
      Lipschitz $L=\frac{5}{6}$.

      \part

      $\left|\operatorname{Fix}\left(f\right)\right|=1$.

      \part

      ¿Cuántos pasos iterativos $k\in\mathbb{N}$ se necesita para
      obtener una precisión de al menos $10^{-3}$ usando la iteración
      de punto fijo
      \begin{math}
        x^{\left(k+1\right)}\coloneqq
        f\left(x^{\left(k\right)}\right)
      \end{math}
      iniciando con $x^{\left(0\right)}\coloneqq\left(0,0\right)$?
    \end{parts}

    \begin{solutionordottedlines}
    \end{solutionordottedlines}

    \question

    Sea $I=\left[0,\infty\right)$.
    Se define $\forall t\in I$:
    $f\left(t\right)=\frac{t+\frac{1}{2}}{t+1}$ es una contracción.
    Calcule el punto fijo de $f$ y una aproximación
    $x^{\left(1\right)}$ usando el punto inicial
    $x^{\left(0\right)}=1$ en la iteración de punto fijo.

    \begin{solutionordottedlines}
    \end{solutionordottedlines}

    \question

    Demuestre que el siguiente sistema de ecuaciones no lineales
    tiene solución en $\mathbb{R}^{2}$.
    \begin{equation*}
      \left\{
        \begin{aligned}
          \sen\left(x_{1}+x_{2}\right)-x_{2}&=
          0.\\
          \cos\left(x_{1}+x_{2}\right)-x_{1}&=
          0.
        \end{aligned}
        \right.
      \end{equation*}

      \begin{solutionordottedlines}
      \end{solutionordottedlines}

      \question

      Sea $A=
      \begin{bmatrix}a_{ij}
      \end{bmatrix}\in\mathbb{R}^{n\times n}$ una matriz.
      Si $\forall i,j\in\left\{1,\dotsc,n\right\}$: $a_{ij}\geq 0$,
      entonces
      \begin{math}
        \exists\left(\lambda,v\right)\in
        \left[0,\infty\right)\times\mathbb{R}^{n}
      \end{math}
      tal que $Av=\lambda v$ y
      $\forall i\in\left\{1,\dotsc,n\right\}$: $v_{i}\geq 0$.

      \begin{solutionordottedlines}
      \end{solutionordottedlines}

      \question

      Sea $f\colon\mathbb{R}^{n}\to\mathbb{R}^{n}$ una función
      continua y $\exists c>1$ tal que $\forall x,y\in\mathbb{R}^{n}$:
      \begin{math}
        \left\|f\left(x\right)-f\left(y\right)\right\|\geq
        c\left\|x-y\right\|
      \end{math}.
      Demuestre que si $f$ es sobreyectiva, entonces
      $\left|\operatorname{Fix}\left(f\right)\right|=1$.

      \question

      Sea
      \begin{math}
        \left(
          C\left(\left[0,1\right]\right),
          {\left\|\cdot\right\|}_{\infty}
        \right)
      \end{math}
      y
      \begin{math}
        S\coloneqq
        \left\{x\in C\left(\left[0,1\right]\right)\mid
          0=x\left(0\right)\leq
          x\left(t\right)\leq
          x\left(1\right)=1
        \right\}
      \end{math}.
      Defina la función
      \begin{equation*}
        \forall x\in C\left(\left[0,1\right]\right):
        \forall t\in\left[0,1\right]:
        T\left(x\right)\left(t\right)\coloneqq
        tx\left(t\right).
      \end{equation*}
      Demuestre cada una de las siguientes proposiciones.
      \begin{parts}
        \part

        $T$ es una función lineal no expansiva con
        ${\left\|T\right\|}_{\infty}=1$.

        \part

        $\forall x\in S$: $T\left(x\right)\in S$.

        \part

        $S$ es un conjunto acotado, cerrado y convexo.

        \part

        Si
        \begin{math}
          \widetilde{T}=T\big|_{C\left(\left[0,1\right]\right)\setminus S}
        \end{math},
        entonces
        \begin{math}
          \left|\operatorname{Fix}\left(\widetilde{T}\right)\right|=
          1
        \end{math}.

        \part

        $\exists x_{t}\in S$ tal que
        $x_{t}=tu+\left(1-t\right)T\left(x_{t}\right)$ donde $u\in S$,
        $u\left(t\right)\coloneqq t$.

        \part

        \begin{math}
          \inf\limits_{x\in S}
          \left\|x-T\left(x\right)\right\|=
          0
        \end{math}.

        \part

        \begin{math}
          \lim\limits_{n\to\infty}
          {\left\|x^{\left(0\right)}-T^{\left(n\right)}\left(x^{\left(0\right)}\right)\right\|}_{\infty}=
          \operatorname{diam}\left(S\right).
        \end{math}
      \end{parts}

      \question

      Sea $\left(E,\left\|\cdot\right\|\right)$ un espacio de Banach,
      $K\subset E$ un conjunto compacto y $f\colon E\to E$ una
      función continua.
      Demuestre que
      \begin{equation*}
        \inf_{x\in E}\left\|x-f\left(x\right)\right\|=
        0\iff
        \operatorname{Fix}\left(f\right)\neq\emptyset.
      \end{equation*}

      \question

      Sea $\left(E,\left\|\cdot\right\|\right)$ un espacio de Banach
      y $K\subset E$ un conjunto compacto y convexo.
      Si $f\colon E\to E$ es una función no expansiva, entonces
      $\left|\operatorname{Fix}\left(f\right)\right|=1$.

      \question

      Se definen los siguientes conjuntos
      \begin{align*}
        C_{0}&\coloneqq
        \left\{\left(x,y\right)\in\mathbb{R}^{2}\mid
        x^{2}+y^{2}\leq 1\right\}&
        C_{2}&\coloneqq
        \left\{
          \left(x,y\right)\in\mathbb{R}^{2}\mid
          x\leq 0,
          \left|y\right|\leq\left|x\right|
        \right\}.\\
        C_{1}&\coloneqq
        \left\{
          \left(x,y\right)\in\mathbb{R}^{2}\mid
          x\geq 0,
          \left|y\right|\leq x
        \right\}&
        C&\coloneqq
        \left(C_{0}\cap C_{1}\right)\cup
        \left(C_{0}\cap C_{2}\right).
      \end{align*}
      Demuestre que
      \begin{math}
        C\subset
        \operatorname{co}
        \left(
          \left\{
            P_{C}\left(x,y\right):
            \left(x,y\right)\in
            \mathbb{R}^{2}\setminus C
          \right\}
        \right)
      \end{math}.

      \question

      Sea $\left(E,\left\|\cdot\right\|\right)$ un espacio de Banach
      y $U<E$ un subespacio cerrado.
      Considere el espacio cociente
      \begin{equation*}
        E/U\coloneqq
        \left\{\left[x\right]\mid x\in E\right\}
        \text { donde }
        \left[x\right]\coloneqq
        \left\{x+u:u\in U\right\}.
      \end{equation*}
      Demuestre que $\forall x\in E$:
      $\left\|\left[x\right]\right\|\coloneqq d\left(x,U\right)$
      define una norma en $E/U$.

      \question

      Sea $\left(E,\left\|\cdot\right\|\right)$ un espacio de Banach
      y $U<E$ un subespacio propio cerrado.
      Demuestre que $\forall\varepsilon>0$:
      $\exists x\in\overline{B}_{1}$ tal que
      $d\left(x,U\right)\geq 1-\varepsilon$.
