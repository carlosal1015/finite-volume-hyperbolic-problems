\question

Justifique la veracidad o falsedad de las siguientes proposiciones.

\begin{parts}
	\part[2]

	Sea $f\colon\mathbb{R}\to\mathbb{R}$ una función definida por
	$f\left(x\right)=\frac{x^{5}}{1+x^{6}}$.
	Entonces $f^{\left(2003\right)}\left(0\right)=-2003!$

	\part[2]

	Sea $f\colon\left[a,b\right]\to\mathbb{R}$ una función integrable y
	se define $F\colon\left[a,b\right]\to\mathbb{R}$ como
	\begin{math}
		F\left(x\right)=
		\int_{a}^{x}f\left(t\right)\dl t
	\end{math}.
	¿Es $F$ Lipschitz?

	\part[1]

	Sea $f\colon\left[a,b\right]\to\mathbb{R}$ una función definida por
	\begin{math}
		f\left(x\right)=
		\begin{cases}
			x, & \text{si }x\in\mathbb{Q}.                    \\
			0, & \text{si }x\in\mathbb{R}\setminus\mathbb{Q}.
		\end{cases}
	\end{math}
	¿Es $f$ integrable?
\end{parts}

\question

Sea
\begin{math}
	f\in
	C^{\infty}
	\left(
	\left]-1,1\right[,\mathbb{R}
	\right)
\end{math}
una función tal que
$\forall n\in\mathbb{N}$:
$\forall x\in\left(-1,1\right)$:
$\left|f^{\left(n\right)}\left(x\right)\right|\leq 1$.

\begin{parts}
	\part[2\half]

	Sea $T_{m}\left(x\right)$ el $m$-ésimo polinomio de Taylor de $f$ alrededor de $0$.
	Demuestre que
	\begin{equation*}
		\left|f\left(x\right)-T_{m}\left(x\right)\right|=
		\left|R_{m}\left(x\right)\right|
		\leq\frac{1}{m!}.
	\end{equation*}

	\part[2\half]

	Además, pruebe que $\forall x\in\left]-1,1\right[$:
	\begin{math}\displaystyle
		\lim_{m\to\infty}T_{m}\left(x\right)=f\left(x\right).
	\end{math}
\end{parts}

\begin{solutionordottedlines}
\end{solutionordottedlines}

\question[5]

Si $f,g\colon\left[a,c\right]\to\mathbb{R}$ son funciones acotadas, entonces
\begin{equation*}
	% \underline{\int_{a}^{c}}f\left(x\right)\dl x & =
	% \underline{\int_{a}^{b}}f\left(x\right)\dl x+
	% \underline{\int_{b}^{c}}f\left(x\right)\dl x.    \\
	\forall b\in\left(a,c\right):
	\overline{\int_{a}^{c}}f\left(x\right)\dl x=
	\overline{\int_{a}^{b}}f\left(x\right)\dl x+
	\overline{\int_{b}^{c}}f\left(x\right)\dl x.
\end{equation*}

\begin{solutionordottedlines}
\end{solutionordottedlines}

\question[5]

Sea $f\colon\left[0,1\right]\to\mathbb{R}$ una función definida como
\begin{equation*}
	f\left(x\right)=
	\begin{cases}
		0,               & \text{si }x=0.                                                                                    \\
		\frac{1}{2^{n}}, & \text{si }x\in\left(\frac{1}{2^{n+1}},\frac{1}{2^{n}}\right], n\in\mathbb{N}\cup\left\{0\right\}.
	\end{cases}
\end{equation*}
Pruebe que $f$ es integrable y calcule $\int_{0}^{1}f\left(x\right)\dl x$.
\begin{solutionordottedlines}
\end{solutionordottedlines}
% \bonusquestion

% Justifique la veracidad o falsedad de las siguientes proposiciones.

% \begin{parts}
% 	\part[1]

% 	Sea
% 	\begin{math}
% 		f\left(x\right)=
% 		\begin{cases}
% 			x^{2}\sen\left(\frac{1}{x}\right), & x\neq 0. \\
% 			0,                                 & x=0.
% 		\end{cases}
% 	\end{math}
% 	Entonces,
% 	\begin{math}
% 		f^{\prime}\left(x\right)=
% 		\begin{cases}
% 			2x\sen\left(\frac{1}{x}\right)-\cos\left(\frac{1}{x}\right), & x\neq 0, \\
% 			0,                                                           & x=0,
% 		\end{cases}
% 	\end{math}
% 	es discontinua en el origen.

% 	\part[1]

% 	Si
% 	\begin{math}
% 		f\left(x\right)=
% 		\begin{cases}
% 			x^{2}\sen\left(\frac{1}{x^{2}}\right), & x\neq 0. \\
% 			0,                                     & x=0.
% 		\end{cases}
% 	\end{math}
% 	Entonces,
% 	\begin{math}
% 		f^{\prime}\left(x\right)=
% 		\begin{cases}
% 			2x\sen\left(\frac{1}{x^{2}}\right)-\frac{1}{x^{2}}, & x\neq 0, \\
% 			0,                                                  & x=0,
% 		\end{cases}
% 	\end{math}
% 	es acotada en $\left[-1,1\right]$.

% 	\part[2]

% 	Si
% 	\begin{math}
% 		f\left(x\right)=
% 		\begin{cases}
% 			\exp\left(-\frac{1}{x^{2}{\left(1-x\right)}^{2}}\right), & x\in\left(0,1\right).                    \\
% 			0,                                                       & x\in\mathbb{R}\setminus\left(0,1\right).
% 		\end{cases}
% 	\end{math}.
% 	Entonces, $f$ es positiva e infinitamente diferenciable en $\left(0,1\right)$.
% \end{parts}

% \begin{solutionordottedlines}
% \end{solutionordottedlines}
