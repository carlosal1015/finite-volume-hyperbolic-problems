\question

Sean $X\subset\mathbb{R}$ un conjunto abierto y
$f\colon X\to\mathbb{R}$ una función derivable hasta (al menos) el
orden $n\in\mathbb{N}$.
Dado $a\in X$, definimos el $n$-ésimo polinomio de Taylor de $f$
alrededor de $a$ por
\vspace*{-.5\baselineskip}\setlength\belowdisplayshortskip{0pt}
\begin{equation*}
	T_{n}\left(x\right)\coloneqq
	\sum_{k=0}^{n}
	\frac{f^{\left(k\right)}\left(a\right)}{k!}
	\left(x-a\right)^{k},
\end{equation*}
donde $f^{\left(0\right)}=f$ y $f^{\left(k\right)}$ denota la
$k$-ésima derivada de $f$.
El $n$-ésimo resto de Taylor de $f$ alrededor de $a$ es
\vspace*{-.5\baselineskip}\setlength\belowdisplayshortskip{0pt}
\begin{equation*}
	R_{n}\left(x\right)\coloneqq
	f\left(x\right)-
	T_{n}\left(x\right)\text{ tal que }
	\frac{R_{n}\left(x\right)}{{\left(x-a\right)}^{k}}
	\xrightarrow[x\to a]{}0
\end{equation*}

Pruebe que si $f$ es $n$-veces derivable en $a$, entonces
$\forall k\in\left\{0,\dotsc,n\right\}$:
\begin{math}
	T^{\left(k\right)}_{n}\left(a\right)=
	f^{\left(k\right)}\left(a\right).
\end{math}

\question

Sea $I\subset\mathbb{R}$ un intervalo abierto tal que $a\in I$.
Si $f\colon I\to\mathbb{R}$ es derivable,
entonces $\forall x\in I\setminus\left\{a\right\}$,
\begin{math}
	\exists c\in
	\left(
	\min\left\{a,x\right\},
	\max\left\{a,x\right\}
	\right)
\end{math}
tal que
\begin{math}
	R_{0}\left(x\right)=
	f^{\prime}\left(c\right)
	\left(x-a\right)
\end{math}.

\question

Use el Teorema~\ref{thm:taylor} y $T_{n}\left(x\right)$ de
$\exp\left(x\right)$ alrededor de $0$ para probar que
\vspace*{-.5\baselineskip}\setlength\belowdisplayshortskip{0pt}
\begin{equation*}
	\forall x\in\mathbb{R}:
	\exp\left(x\right)=
	\sum_{k=0}^{\infty}\frac{x^{k}}{k!}=
	\lim_{n\to\infty}\sum_{k=0}^{n}\frac{x^{k}}{k!}.
\end{equation*}
Además, pruebe que la constante de Euler
$e\in\mathbb{R}\setminus\mathbb{Q}$, con ayuda del
Teorema~\ref{thm:taylor}, es decir,
\vspace*{-.5\baselineskip}\setlength\belowdisplayshortskip{0pt}
\begin{equation*}
	\exists c_{n}\in
	\left(0,1\right)\text{ tal que }
	e=
	\sum_{k=0}^{n}
	\frac{1}{k!}+
	\frac{e^{c_{n}}}{\left(n+1\right)!}.
\end{equation*}

\begin{theorem}[Teorema de Taylor\footnote{Brook Taylor (18 de agosto de 1685 - 29 de diciembre de 1731).}]\label{thm:taylor}
	Sea $I\subset\mathbb{R}$ un intervalo abierto tal que $a,x\in I$
	con $x\neq a$.
	Si $f\colon I\to\mathbb{R}$ es $n$-veces derivable y
	\begin{math}
		\forall t\in
		\left(
		\min\left\{a,x\right\},
		\max\left\{a,x\right\}
		\right)
	\end{math}:
	$\exists f^{\left(n+1\right)}\left(t\right)\in\mathbb{R}$,
	entonces
	\vspace*{-.5\baselineskip}\setlength\belowdisplayshortskip{0pt}
	\begin{equation*}
		\exists c\in I\text{ tal que }
		R_{n}\left(x\right)=
		\frac{f^{\left(n+1\right)}\left(c\right)}{\left(n+1\right)!}
		{\left(x-a\right)}^{n+1}.
	\end{equation*}
\end{theorem}

\question

Encuentre los polinomios de Taylor
$T_{2n}\left(x\right)$ y $T_{2n+1}\left(x\right)$
de $f\left(x\right)=\cos\left(x\right)$ alrededor de $0$ y pruebe que
\vspace*{-.5\baselineskip}\setlength\belowdisplayshortskip{0pt}
\begin{equation*}
	\forall x\in\left(0,\pi\right):
	1-\frac{x^{2}}{2!}+\frac{x^{4}}{4!}>
	\cos\left(x\right)>
	1-\frac{x^{2}}{2!}+\frac{x^{4}}{4!}-\frac{x^{6}}{6!}.
\end{equation*}

\question

Use el Teorema~\ref{thm:taylor} para probar que
\vspace*{-.5\baselineskip}\setlength\belowdisplayshortskip{0pt}
\begin{multicols}{2}
	\begin{parts}
		\part

		\begin{math}
			\forall x\in\mathbb{R}
		\end{math}:
		\begin{math}\displaystyle
			\sen\left(x\right)=
			\sum_{n=0}^{\infty}
			\frac{{\left(-1\right)}^{n}x^{2n+1}}{\left(2n+1\right)!}
		\end{math}.

		\part

		\begin{math}
			\forall x\in\mathbb{R}
		\end{math}:
		\begin{math}\displaystyle
			\cos\left(x\right)=
			\sum_{n=0}^{\infty}
			\frac{{\left(-1\right)}^{n}x^{2n}}{\left(2n\right)!}
		\end{math}.


		\part

		\begin{math}
			\forall x\in\left[1,2\right)
		\end{math}:
		\begin{math}\displaystyle
			\ln\left(x\right)=
			\sum_{n=1}^{\infty}
			\frac{{\left(-1\right)}^{n-1}{\left(x-1\right)}^{n}}{n}
		\end{math}.

		\part

		\begin{math}
			\forall\left|x\right|\leq 1
		\end{math}:
		\begin{math}\displaystyle
			\arctan\left(x\right)=
			\sum_{n=1}^{\infty}
			\frac{{\left(-1\right)}^{n}{x}^{2n+1}}{2n+1}
		\end{math}.
	\end{parts}
\end{multicols}

\question

Sea $I\subset\mathbb{R}$ un intervalo abierto tal que $a\in I$.
Si $f\colon I\to\mathbb{R}$ es $n$-veces derivable y $\exists M>0$
tal que $\forall x\in I$:
$\forall n\in\mathbb{N}$:
$\left|f^{\left(n\right)}\left(x\right)\right|\leq M^{n}$,
entonces
\vspace*{-.5\baselineskip}\setlength\belowdisplayshortskip{0pt}
\begin{equation*}
	\lim_{n\to\infty}
	T_{n}\left(x\right)=
	f\left(x\right)=
	\sum_{n=0}^{\infty}
	\frac{f^{\left(n\right)}\left(a\right)}{n!}{\left(x-a\right)}^{n}.
\end{equation*}

% \question

% Sean $n\geq 2$, $a\in\mathbb{R}$ y $\varepsilon>0$.
% Si $f\colon\left(a-\varepsilon,a+\varepsilon\right)\to\mathbb{R}$ es
% $n$-veces derivable y $\forall k\in\left\{1,\dotsc,n-1\right\}$:
% $f^{\left(k\right)}\left(a\right)=0$ y
% $f^{\left(n\right)}\left(a\right)\neq 0$.
% \begin{parts}
% 	\part

% 	Si $n$ es par y $f^{\left(n\right)}\left(a\right)>0$, entonces
% 	$f$ tiene un mínimo local en $a$.

% 	\part

% 	Si $n$ es par y $f^{\left(n\right)}\left(a\right)<0$, entonces
% 	$f$ tiene un máximo local en $a$.

% 	\part

% 	Si $n$ es impar, entonces $f$ no tiene ni mínimo ni máximo local
% 	en $a$.
% \end{parts}

% \question

% Suponga que $f,g,s,c\colon\mathbb{R}\to\mathbb{R}$ son funciones
% derivables tales que
% \vspace*{-.5\baselineskip}\setlength\belowdisplayshortskip{0pt}
% \begin{align*}
% 	f^{\prime}      & =g.                 &
% 	g^{\prime}      & =-f.                &
% 	s^{\prime}      & =c.                   \\
% 	c^{\prime}      & =-s.                &
% 	f\left(0\right) & =s\left(0\right)=0. &
% 	g\left(0\right) & =c\left(0\right)=1
% \end{align*}
% Use el Teorema~\ref{thm:taylor} para $\left(f-s\right)\left(x\right)$
% y que $\forall n\in\mathbb{N}$: $\exists B>0$ tal que
% \begin{math}
% 	\left|\left(f-s\right)\left(x\right)\right|\leq
% 	\frac{B\left|x^{n+1}\right|}{\left(n+1\right)!}
% \end{math}
% para probar que $f=s$ y $g=c$.

% \question

% Si $f,g\colon\left[a,b\right]\to\mathbb{R}$ son continuas y
% derivables en $\left(a,b\right)$, entonces
% $\exists c\in\left(a,b\right)$ tal que
% \vspace*{-.5\baselineskip}\setlength\belowdisplayshortskip{0pt}
% \begin{equation*}
% 	g^{\prime}\left(c\right)
% 	\left[f\left(b\right)-f\left(a\right)\right]=
% 	f^{\prime}\left(c\right)
% 	\left[g\left(b\right)-g\left(a\right)\right].
% \end{equation*}
% \vspace*{-2\baselineskip}\setlength\belowdisplayshortskip{0pt}
% \begin{figure}[ht!]
% 	\centering
% 	\includegraphics[width=.55\paperwidth]{cauchy}
% 	\caption{Geométricamente, esto significa que existe una recta tangente
% 		a la gráfica de la curva
% 		\begin{math}
% 			\begin{aligned}
% 				\left[a,b\right] & \longrightarrow\mathbb{R}^{2}                           \\
% 				t                & \longmapsto\left(f\left(t\right),g\left(t\right)\right)
% 			\end{aligned}
% 		\end{math}
% 		que es paralela a la recta definida por los puntos
% 		$\left(f\left(a\right),g\left(a\right)\right)$ y
% 		$\left(f\left(b\right),g\left(b\right)\right)$.
% 	}
% \end{figure}

\question

Sean $X\subset\mathbb{R}$ un conjunto abierto.
Use el principio de inducción matemática para probar la regla de
Leibniz\footnote{Gottfried Wilhelm Leibniz (1 de junio de 1646 - 14 de noviembre de 1716).}.
$\forall n\in\mathbb{N}$: si $f$ y $g$ son $n$-veces derivables,
entonces
\vspace*{-.5\baselineskip}\setlength\belowdisplayshortskip{0pt}
\begin{equation*}
	{\left(fg\right)}^{\left(n\right)}\left(x\right)=
	\sum_{k=0}^{n}
	\binom{n}{k}
	f^{\left(k\right)}\left(x\right)
	g^{\left(n-k\right)}\left(x\right),
\end{equation*}
donde $f^{\left(k\right)}$ y $g^{\left(k\right)}$ denotan la
$k$-ésima derivada de $f$ y $g$, $f^{\left(0\right)}=f$,
$g^{\left(0\right)}=g$, y
$\binom{n}{k}\coloneqq\frac{n!}{k!\left(n-k\right)!}$.

% \question

% Defina la función de Thomae
% \begin{equation*}
% 	T\left(x\right)=
% 	\left\{
% 	\begin{aligned}
% 		 & \frac{1}{n}, & \text{ si }x=\frac{m}{n}\neq 0,\text{donde }m\in\mathbb{Z},n\in\mathbb{N}\text{ y }\operatorname{mcd}\left(m,n\right)=1 \\
% 		 & 1,           & \text{ si }x=0                                                                                                          \\
% 		 & 0,           & \text{ si }x\in\mathbb{R}\setminus\mathbb{Q}
% 	\end{aligned}
% 	\right\}
% \end{equation*}
% \vspace*{-.5\baselineskip}\setlength\belowdisplayshortskip{0pt}
% \begin{figure}[ht!]
% 	\centering
% 	\includegraphics[width=.7\paperwidth]{thomae}
% \end{figure}

% \begin{multicols}{2}
% 	\begin{parts}
% 		\part

% 		$T$ es continua en $\mathbb{R}\setminus\mathbb{Q}$ y es
% 		discontinua en $\mathbb{Q}$.

% 		\part

% 		$\forall x\in\mathbb{R}$:
% 		$\lim_{y\to x}T\left(y\right)=0$.

% 		\part

% 		$T$ es integrable en intervalos compactos.

% 		\part

% 		$T$ no tiene primitiva en subintervalos de $\left[0,1\right]$.

% 		\part

% 		$\int_{0}^{x}T$ es derivable en $\left[0,1\right]$.

% 		\part

% 		$\forall x\in D\subset\left[0,1\right]$ denso:
% 		\begin{math}
% 			\left(\int_{0}^{x}T\right)^{\prime}\left(x\right)\neq
% 			T\left(x\right)
% 		\end{math}.
% 	\end{parts}
% \end{multicols}

% \question

% Encuentre un ejemplo de una función
% $f\colon\mathbb{R}\to\mathbb{R}$ estrictamente creciente
% y $\forall x\in\mathbb{R}$:
% $f^{\prime}\left(x\right)>0$.

% \question

% Pruebe que no existe una función derivable $f$ definida en cualquier
% intervalo abierto que contenga al cero tal que $\forall x\in I$:
% \begin{math}
% 	f^{\prime}\left(x\right)=
% 	\left\{
% 	\begin{aligned}
% 		1, & \text{si }x\geq 0, \\
% 		2, & \text{si }x<0
% 	\end{aligned}
% 	\right\}
% \end{math}.

% \question

% Pruebe que si $f$ es derivable en un intervalo abierto $I$ y
% $\forall x\in I$: $f^{\prime}\left(x\right)\neq 0$, entonces
% $f^{\prime}\left(x\right)$ tiene el mismo signo a lo largo del
% intervalo $I$.

\question

Suponga que $f$ es definida y acotada en $\left[a,b\right]$ y $L,M\in\mathbb{R}$.
\begin{parts}
	\part

	Si $\exists\left\{\mathcal{P}_{n}\right\}_{n\in\mathbb{N}}$ una
	sucesión de particiones de $\left[a,b\right]$ tal que
	$\underline{S}\left(f,\mathcal{P}_{n}\right)\xrightarrow[n\to\infty]{}L$, entonces
	$\overline{\int_{a}^{b}}f\geq L$.

	\part
	Si $\exists\left\{\mathcal{Q}_{n}\right\}_{n\in\mathbb{N}}$ una
	sucesión de particiones de $\left[a,b\right]$ tal que
	$\overline{S}\left(f,\mathcal{Q}_{n}\right)\xrightarrow[n\to\infty]{}M$, entonces
	$\overline{\int_{a}^{b}}f\leq M$.

	\part
	Si $\exists\left\{\mathcal{P}_{n}\right\}_{n\in\mathbb{N}}$ y
	$\left\{\mathcal{Q}_{n}\right\}_{n\in\mathbb{N}}$ sucesiones de particiones de $\left[a,b\right]$ tales que
	$\underline{S}\left(f,\mathcal{P}_{n}\right)\xrightarrow[n\to\infty]{}L$ y\linebreak
	$\overline{S}\left(f,\mathcal{Q}_{n}\right)\xrightarrow[n\to\infty]{}L$, entonces $f$ es integrable y $\int_{a}^{b}f=L$.
\end{parts}

\question

Una función acotada $f\colon\left[a,b\right]\to\mathbb{R}$ es integrable en $\left[a,b\right]$ si y solo si
$\exists K>0$ tal que $\forall\varepsilon>0$:
$\exists$ una partición $\mathcal{P}$ de $\left[a,b\right]$ tal que
$\overline{S}\left(f,\mathcal{P}\right)-\underline{S}\left(f,\mathcal{P}\right)<K\varepsilon$.

\question

Una función acotada $f\colon\left[a,b\right]\to\mathbb{R}$ es integrable en $\left[a,b\right]\iff\exists$ un número y solo un número $I\in\mathbb{R}$
tal que $\forall$ particiones $\mathcal{P}$ de $\left[a,b\right]$:
$\overline{S}\left(f,\mathcal{P}\right)\leq I\leq\underline{S}\left(f,\mathcal{P}\right)$.
En este caso, $I=\int_{a}^{b}f\left(x\right)\dl x$.

\question

Si $f$ es monótona en $\left[a,b\right]$, entonces $f$ es integrable en $\left[a,b\right]$.

\question

Suponga que $f\colon\left[a,b\right]\to\mathbb{R}$ es acotada y no negativa en $\left[a,b\right]$.
Pruebe que
\begin{parts}
	\part

	$\int_{a}^{b}f\left(x\right)\dl x\geq 0$.

	\part

	Si $\exists x_{0}\in\left(a,b\right)$ tal que $f$ es continua y $f\left(x_{0}\right)>0$, entonces
	$\underline{\int_{a}^{b}}f\left(x\right)\dl x>0$.

	\part

	Si $f$ es continua en $\left[a,b\right]$, entonces $\underline{\int_{a}^{b}}f\left(x\right)\dl x=0\iff\forall x\in\left[a,b\right]$: $f\left(x\right)=0$.

\end{parts}

% \question

% Suponga que $f,g\colon\left[a,b\right]\to\mathbb{R}$ son acotadas en $\left[a,b\right]$ y $\forall x\in\left[a,b\right]$: $f\left(x\right)\leq g\left(x\right)$.
% Pruebe que $\underline{\int_{a}^{b}}f\leq\underline{\int_{a}^{b}}g$ y $\overline{\int_{a}^{b}}f\leq\overline{\int_{a}^{b}}g$.


% \question

% Si $f$ es no negativa e integrable en $\left[a,b\right]$ y $\forall r\in\mathbb{Q}\cap\left[a,b\right]$: $f\left(r\right)=0$,
% entonces $\int_{a}^{b}f=0$.

\question

Suponga que ${\left\{\mathcal{P}_{n}\right\}}_{n\in\mathbb{N}}$ es
una sucesión de particiones de $\left[a,b\right]$, donde cada una de
ellas es refinada por su sucesor, es decir, $\forall n\in\mathbb{N}$:
$\mathcal{P}_{n}\subset\mathcal{P}_{n+1}$.
Pruebe que para cualquier función acotada $f\colon\left[a,b\right]\to\mathbb{R}$,
\begin{parts}
	\part

	ambas sucesiones ${\left\{\underline{S}\left(f,\mathcal{P}_{n}\right)\right\}}_{n\in\mathbb{N}}$ y
	${\left\{\overline{S}\left(f,\mathcal{P}_{n}\right)\right\}}_{n\in\mathbb{N}}$ convergen y
	$\lim_{n\to\infty}\underline{S}\left(f,\mathcal{P}_{n}\right)\leq\lim_{n\to\infty}\overline{S}\left(f,\mathcal{P}_{n}\right)$.

	\part

	Si $\lim_{n\to\infty}\underline{S}\left(f,\mathcal{P}_{n}\right)=\lim_{n\to\infty}\overline{S}\left(f,\mathcal{P}_{n}\right)=L$,
	entonces $f$ es integrable en $\left[a,b\right]$ y $\int_{a}^{b}f\left(x\right)\dl x=L$.

	\part

	Muestre un contraejemplo donde la integrabilidad de $f$ en $\left[a,b\right]$ no garantiza que
	$\lim_{n\to\infty}\underline{S}\left(f,\mathcal{P}_{n}\right)=\lim_{n\to\infty}\overline{S}\left(f,\mathcal{P}_{n}\right)$.
\end{parts}

% \question

% La malla de una partición $\mathcal{P}$ de $\left[a,b\right]$ es la
% longitud del subintervalo más largo $\left[x_{i-1},x_{i}\right]$
% entre puntos consecutivos de la partición $\mathcal{P}$,
% \vspace*{-.5\baselineskip}\setlength\belowdisplayshortskip{0pt}
% \begin{equation*}
% 	\left\|\mathcal{P}\right\|=
% 	\max\left\{x_{i}-x_{i-1}\mid 1\leq i\leq n\right\}.
% \end{equation*}

% Una función acotada $f\colon\left[a,b\right]\to\mathbb{R}$ es integrable en $\left[a,b\right]\iff\exists k>0$
% tal que $\forall\varepsilon>0$: $\exists\delta>0$ tal que $\forall$ particiones $\mathcal{P}$ de $\left[a,b\right]$ con
% $\left\|\mathcal{P}\right\|<\delta$: $\overline{S}\left(f,\mathcal{P}\right)-\underline{S}\left(f,\mathcal{P}\right)<k\varepsilon$.

% \question

% Una partición estrellada $\mathcal{P}^{\ast}$ de $\left[a,b\right]$ es una partición
% $\mathcal{P}=\left\{x_{0},\dotsc,x_{n}\right\}$ de $\left[a,b\right]$ junto con un conjunto de puntos
% $\forall i\in\left\{1,\dotsc,n\right\}$: $x^{\ast}_{i}\in\left[x_{i-1},x_{i}\right]$.
% La suma $R\left(f,\mathcal{P}^{\ast}\right)=\sum_{i=1}^{n}f\left(x^{\ast}_{i}\right)\left[x_{i}-x_{i-1}\right]$
% es la suma de Riemann de $f$ sobre $\mathcal{P}^{\ast}$.

% Pruebe que para cualquier partición $\mathcal{P}$ de $\left[a,b\right]$ y cualquier conjunto
% de puntos $\left\{x^{\ast}_{1},\dotsc,x^{\ast}_{n}\right\}$ en sus respectivos subintervalos
% $\left[x_{i-1},x_{i}\right]$, se cumple que $\underline{S}\left(f,\mathcal{P}\right)\leq R\left(f,\mathcal{P}^{\ast}\right)\leq\overline{S}\left(f,\mathcal{P}\right)$.

% Esto es, para una partición dada $\mathcal{P}$ de $\left[a,b\right]$, todas las sumas de Riemann para $f$ sobre $\mathcal{P}$
% caen entre

% \question

% Para cualquier $f\colon\left[a,b\right]\to\mathbb{R}$, $f$ es integrable en $\left[a,b\right]$ y $\int_{a}^{b}f=I\iff$
% $\exists k>0$ tal que $\forall\varepsilon>0$: $\exists\delta>0$ tal que $\forall\mathcal{P}^{\ast}$ de $\left[a,b\right]$ con $\left\|\mathcal{P}^{\ast}\right\|<\delta$:
% $\left|R\left(f,\mathcal{P}^{\ast}\right)-I\right|<k\varepsilon$.

% Así, si el límite existe y es independiente de los puntos $x^{\ast}_{i}$.
% Escribimos
% \vspace*{-.5\baselineskip}\setlength\belowdisplayshortskip{0pt}
% \begin{equation*}
% 	\int_{a}^{b}f=
% 	\lim_{\left\|\mathcal{P}^{\ast}\right\|\to0}
% 	R\left(f,\mathcal{P}^{\ast}\right)=
% 	\lim_{\left\|\mathcal{P}^{\ast}\right\|\to0}
% 	\sum_{i=1}^{n}
% 	f\left(x^{\ast}_{i}\right)
% 	\left(x_{i}-x_{i-1}\right).
% \end{equation*}

\question

Si $f$ es integrable en $\left[a,b\right]$ y ${\left\{\mathcal{P}_{n}\right\}}_{n\in\mathbb{N}}$
es una sucesión de particiones de $\left[a,b\right]$ tal que $\left\|\mathcal{P}_{n}\right\|\to0$.
Entonces,
\begin{parts}
	\part

	$\underline{S}\left(f,\mathcal{P}_{n}\right)\to\int_{a}^{b}f$ y
	$\overline{S}\left(f,\mathcal{P}_{n}\right)\to\int_{a}^{b}f$.

	\part

	$\forall\mathcal{P}^{\ast}$ partición de $\left[a,b\right]$:
	$R\left(f,\mathcal{P}^{\ast}_{n}\right)\xrightarrow[n\to\infty]{}\int_{a}^{b}f$ sin importar
	la elección de los puntos $x^{\ast}_{i}$s.
\end{parts}

\question

Si $f$ es integrable en $\left[0,1\right]$, entonces
$\lim_{n\to\infty}\frac{1}{n}\sum_{k=1}^{n}f\left(\frac{k}{n}\right)=\int_{0}^{1}f\left(x\right)\dl x$.

\question

Si $f$ es integrable en $\left[a,b\right]$ y $\mathcal{Q}_{n}=\left\{x_{0},\dotsc,x_{n}\right\}$
es una partición regular de $\left[a,b\right]$, es decir, $\left\|\mathcal{Q}_{n}\right\|=\frac{b-a}{n}$,
así, $\forall i\in\left\{0,\dotsc,n\right\}$: $x_{i}=a+i\left\|\mathcal{Q}_{n}\right\|$.
Entonces,
\vspace*{-.5\baselineskip}\setlength\belowdisplayshortskip{0pt}
\begin{equation*}
	\int_{a}^{b}f\left(x\right)\dl x=
	\lim_{n\to\infty}
	\frac{b-a}{2n}
	\left[
		f\left(x_{0}\right)+
		2f\left(x_{1}\right)+
		\cdots+
		2f\left(x_{n-1}\right)+
		f\left(x_{n}\right)
		\right].
\end{equation*}

Note que la expresión entre corchetes es
$2\left[f\left(x_{0}\right)+f\left(x_{1}\right)+\cdots+f\left(x_{n-1}\right)+f\left(x_{n}\right)\right]-\left[f\left(b\right)+f\left(a\right)\right]$.

\question

Si $f$ es integrable en $\left[a,b\right]$ y $\mathcal{Q}_{n}=\left\{x_{0},\dotsc,x_{n}\right\}$
es una partición regular de $\left[a,b\right]$ en un número par de subintervalos.
Entonces,
\vspace*{-.5\baselineskip}\setlength\belowdisplayshortskip{0pt}
\begin{equation*}
	\int_{a}^{b}f\left(x\right)\dl x=
	\lim_{n\to\infty}
	\frac{b-a}{3n}
	\left[
		f\left(x_{0}\right)+
		4f\left(x_{1}\right)+
		2f\left(x_{2}\right)+
		4f\left(x_{3}\right)+
		\cdots+
		4f\left(x_{n-1}\right)+
		f\left(x_{n}\right)
		\right].
\end{equation*}

\question

Si $f\colon\left[a,b\right]\to\mathbb{R}$ es acotada y $a<c<b$, entonces
\vspace*{-.5\baselineskip}\setlength\belowdisplayshortskip{0pt}
\begin{equation*}
	\overline{\int_{a}^{b}}f\left(x\right)\dl x=
	\overline{\int_{a}^{c}}f\left(x\right)\dl x+
	\overline{\int_{c}^{b}}f\left(x\right)\dl x
	\quad\text{ y }\quad
	\underline{\int_{a}^{b}}f\left(x\right)\dl x=
	\underline{\int_{a}^{c}}f\left(x\right)\dl x+
	\underline{\int_{c}^{b}}f\left(x\right)\dl x.
\end{equation*}

% \question

% Si $f\colon\left[a,b\right]\to\mathbb{R}$ es integrable y $a<c<b$, entonces
% $\forall c\in\left(a,b\right)$: $f$ es integrable en $\left[a,c\right]$ y $\left[c,b\right]$ y
% $\int_{a}^{b}f=\int_{a}^{c}f+\int_{c}^{b}f$.

% \question

% Si $f$ es integrable en $\left[a,b\right]$ y $\mathcal{P}=\left\{x_{0},x_{1},\dotsc,x_{n}\right\}$
% es cualquier partición de $\left[a,b\right]$, entonces $f$ es integrable en $\left[x_{0},x_{1}\right]$,
% $\left[x_{1},x_{2}\right],\dotsc$, y $\left[x_{n-1},x_{n}\right]$ y
% \vspace*{-.5\baselineskip}\setlength\belowdisplayshortskip{0pt}
% \begin{equation*}
% 	\int_{a}^{b}f=
% 	\sum_{i=1}^{n}
% 	\int_{x_{i-1}}^{x_{i}}f.
% \end{equation*}

% \question

% Si $f$ es integrable en $\left[a,b\right]$, entonces $f$ es integrable en cualquier subintervalo cerrado
% $\left[c,d\right]\subset\left[a,b\right]$.

% \question

% Si $f$ es integrable en $\left[a,c\right]$ y $\left[c,b\right]$, donde $a<c<b$,
% entonces $f$ es integrable en $\left[a,b\right]$ y $\int_{a}^{b}f=\int_{a}^{c}f+\int_{c}^{b}f$.

% \question

% Si $\mathcal{P}=\left\{x_{0},x_{1},\dotsc,x_{n}\right\}$
% es cualquier partición de $\left[a,b\right]$ y $f$ es integrable
% en cada subintervalo $\left[x_{i-1},x_{i}\right]$ creado por su partición, entonces
% $f$ es integrable en $\left[a,b\right]$ y
% \vspace*{-.5\baselineskip}\setlength\belowdisplayshortskip{0pt}
% \begin{equation*}
% 	\int_{a}^{b}f=
% 	\sum_{i=1}^{n}
% 	\int_{x_{i-1}}^{x_{i}}f.
% \end{equation*}

% \question

% Una función acotada $f\colon\left[a,b\right]\to\mathbb{R}$ se dice que es
% continua por tramos si y solo si $\exists\mathcal{P}=\left\{x_{0},x_{1},\dotsc,x_{n}\right\}$ partición de $\left[a,b\right]$
% tal que $\forall i\in\left\{1,\dotsc,n\right\}$: $f$ es continua en $\left(x_{i-1},x_{i}\right)$.
% Similarmente, una función acotada $f\colon\left[a,b\right]\to\mathbb{R}$ se dice que es
% monótona por tramos si y solo si $\exists\mathcal{P}=\left\{x_{0},x_{1},\dotsc,x_{n}\right\}$ partición de $\left[a,b\right]$
% tal que $\forall i\in\left\{1,\dotsc,n\right\}$: $f$ es monótona en $\left(x_{i-1},x_{i}\right)$.

% Una función $\tau\colon\left[a,b\right]\to\mathbb{R}$ se dice que es
% una función paso si y solo si $\exists\mathcal{P}=\left\{x_{0},x_{1},\dotsc,x_{n}\right\}$ partición de $\left[a,b\right]$
% y $\exists\left\{c_{1},\dotsc,c_{n}\right\}\subset\mathbb{R}$
% tales que $\forall i\in\left\{1,\dotsc,n\right\}$: $\tau\left(x\right)=c_{i}$ si $x\in\left(x_{i-1},x_{i}\right)$.

% Pruebe que todas las funciones continuas por tramos, monótonas por tramos y funciones paso relativos a una partición
% $\mathcal{P}=\left\{x_{0},x_{1},\dotsc,x_{n}\right\}$ partición de $\left[a,b\right]$ son integrables en $\left[a,b\right]$
% y
% \vspace*{-.5\baselineskip}\setlength\belowdisplayshortskip{0pt}
% \begin{equation*}
% 	\int_{a}^{b}f=
% 	\sum_{i=1}^{n}
% 	\int_{x_{i-1}}^{x_{i}}f.
% \end{equation*}

\question

Si $f\colon\left[a,b\right]\to\mathbb{R}$ es integrable en $\left[a,b\right]$,
$f\left(\left[a,b\right]\right)\subset\left[c,d\right]$ y $g\colon\left[c,d\right]\to\mathbb{R}$
es continua, entonces $g\circ f\colon\left[a,b\right]\to\mathbb{R}$ es integrable en $\left[a,b\right]$.

\question

Si $f$ es integrable en $\left[a,b\right]$, entonces $\left|f\right|$ es integrable.
Más aún,
\begin{math}
	\left|
	\int_{a}^{b}
	f\left(x\right)\dl x
	\right|\leq
	\int_{a}^{b}
	\left|
	f\left(x\right)
	\right|
	\dl x\leq
	M\left(b-a\right)
\end{math}
donde
$M$ es cualquier cota superior para $\left|f\right|$ en $\left[a,b\right]$.

% \question

% Si $f$ y $g$ son integrables en $\left[a,b\right]$, entonces

% \begin{parts}
% 	\part $\forall n\in\mathbb{N}$: $f^{n}$ es integrable en $\left[a,b\right]$.
% 	\part Si $f$ es positivo y acotado fuera de $0$ en $\left[a,b\right]$, entonces $\frac{1}{f}$ es integrable en $\left[a,b\right]$.
% 	\part Sea $n\in\mathbb{N}$. Si $\forall x\in\left[a,b\right]$: $f^{\frac{1}{n}}$ existe, entonces $f^{\frac{1}{n}}$ es integrable en $\left[a,b\right]$.
% 	\part $\sen\left(f\left(x\right)\right)$, $\cos\left(f\left(x\right)\right)$ y $\exp\left(f\left(x\right)\right)$ son integrables en $\left[a,b\right]$.
% 	\part Si $f$ es positivo y acotado fuera de $0$ en $\left[a,b\right]$, entonces $\ln\left(f\left(x\right)\right)$ es integrable en $\left[a,b\right]$.
% 	\part $fg$ es integrable en $\left[a,b\right]$.
% 	\part $\max\left\{f,g\right\}$ y $\min\left\{f,g\right\}$ son integrables en $\left[a,b\right]$.
% \end{parts}

% \question

% Si $f,g\colon\left[a,b\right]\to\mathbb{R}$ son acotadas en $\left[a,b\right]$, entonces
% $\underline{\int_{a}^{b}}f+g\geq\underline{\int_{a}^{b}}f+\underline{\int_{a}^{b}}g$.

\question

Si $f_{1},\dotsc,f_{n}$ son integrables en $\left[a,b\right]$ y
$\left\{c_{1},\dotsc,c_{n}\right\}\subset\mathbb{R}$, entonces
$\sum_{i=1}^{n}c_{i}f_{i}$ es integrable en $\left[a,b\right]$ y
\vspace*{-.5\baselineskip}\setlength\belowdisplayshortskip{0pt}
\begin{equation*}
	\int_{a}^{b}
	\sum_{i=1}^{n}
	c_{i}f_{i}\left(x\right)\dl x=
	\sum_{i=1}^{n}
	c_{i}\int_{a}^{b}f_{i}\left(x\right)\dl x.
\end{equation*}

% \question

% Encuentre una función $f\colon\left[0,1\right]\to\mathbb{R}$ tal que $f^{2}$ es integrable en $\left[0,1\right]$, pero $f$ no es integrable.

% \question

% Si $f,g\colon\left[a,b\right]\to\mathbb{R}$ son integrables en $\left[a,b\right]$ y $\forall x\in\left[a,b\right]$:
% $f\left(x\right)\leq h\left(x\right)\leq g\left(x\right)$.
% Pruebe que si $\int_{a}^{b}f=\int_{a}^{b}g$, entonces $h$ es integrable en $\left[a,b\right]$ y $\int_{a}^{b}h=\int_{a}^{b}f$.

\question

Si $f\colon\left[a,b\right]\to\mathbb{R}$ es integrable en $\left[a,b\right]$ y $k\in\mathbb{R}$.
Defina $g\colon\left[a+k,b+k\right]\to\mathbb{R}$ por $g\left(x\right)=f\left(x-k\right)$.
Pruebe que $g$ es integrable en $\left[a+k,b+k\right]$ y
$\int_{a+k}^{b+k}g\left(x\right)\dl x=\int_{a}^{b}f\left(x\right)\dl x$.

\question

Si $f\colon\left[a,b\right]\to\mathbb{R}$ es continua en $\left[a,b\right]$, entonces
$\exists c\in\left(a,b\right)$ tal que $\int_{a}^{b}f\left(x\right)\dl x=f\left(c\right)\left(b-a\right)$.

\question

Si $f,g\colon\left[a,b\right]\to\mathbb{R}$, $f$ es continua en $\left[a,b\right]$ y $g$ es integrable en $\left[a,b\right]$
tal que no cambia de signo en $\left[a,b\right]$, entonces
$\exists c\in\left(a,b\right)$ tal que $\int_{a}^{b}\left(fg\right)\left(x\right)\dl x=f\left(c\right)\int_{a}^{b}g\left(x\right)\dl x$.

% \question

% Pruebe que $\forall x\in\mathbb{R}$: $\int_{-1}^{x}\operatorname{sgn}\left(x\right)\dl x=\left|x\right|-1$.

\question

Si $f$ es integrable en $\left[a,b\right]$ y $f\left(x\right)\geq 0$ en $\left[a,b\right]$, entonces
$F\left(x\right)=\int_{a}^{x}f\left(t\right)\dl t$ es monótona creciente.

% \question

% Si $f$ y $F$ son continuas en $\left[a,b\right]$ y $F\left(a\right)=0$, entonces
% $F^{\prime}=f$ en $\left[a,b\right]$ es equivalente a $\forall x\in\left[a,b\right]$: $F\left(x\right)=\int_{a}^{x}f$.

% \question

% Suponga que $f$ es continua y $g,h$ son derivables.
% Aplique el segundo teorema fundamental del cálculo y la regla de la cadena

% \begin{multicols}{4}
% 	\begin{parts}
% 		\part

% 		\begin{math}\displaystyle
% 			\diff*{\int_{x}^{a}f}{x}
% 		\end{math}.

% 		\part

% 		\begin{math}\displaystyle
% 			\diff*{\int_{a}^{g\left(x\right)}f}{x}
% 		\end{math}.

% 		\part

% 		\begin{math}\displaystyle
% 			\diff*{\int_{g\left(x\right)}^{a}f}{x}
% 		\end{math}.

% 		\part

% 		\begin{math}\displaystyle
% 			\diff*{\int_{g\left(x\right)}^{h\left(x\right)}f}{x}
% 		\end{math}.
% 	\end{parts}
% \end{multicols}

% \question

% La única función $F\colon\mathbb{R}\to\mathbb{R}$ tal que
% \begin{align*}
% 	\forall x\in\mathbb{R}: F^{\prime\prime}\left(x\right) & =-F\left(x\right), &
% 	F\left(0\right)                                        & =0,                &
% 	F^{\prime}\left(0\right)                               & =1,
% \end{align*}
% es la función $F\left(x\right)=\sen\left(x\right)$.

% \question

% La única función $F\colon\mathbb{R}\to\mathbb{R}$ tal que
% \begin{align*}
% 	\forall x\in\mathbb{R}: F^{\prime\prime}\left(x\right) & =-F\left(x\right), &
% 	F\left(0\right)                                        & =1,                &
% 	F^{\prime}\left(0\right)                               & =0,
% \end{align*}
% es la función $F\left(x\right)=\cos\left(x\right)$.

% https://universitytime.home.blog/wp-content/uploads/2020/04/1.vector-cal-4.pdf
