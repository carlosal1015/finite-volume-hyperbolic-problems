\question

Justifique la veracidad o falsedad de las siguientes proposiciones.

\begin{parts}
	\part[2]

	Sea
	\begin{math}
		{\left\{F_{n}\right\}}_{n=1}^{\infty}\subset
		\mathbb{R}
	\end{math}
	una familia de conjuntos cerrados.
	Si
	\begin{math}
		A\subset
		\bigcap_{n=1}^{\infty}
		F_{n}
	\end{math},
	entonces
	\begin{math}
		\overline{A}\subset
		\bigcap_{n=1}^{\infty}
		F_{n}
	\end{math}.

	\part[1]

	$\exists A,B,C\subset\mathbb{R}$ conexos no vacíos y distintos
	entre sí tales que
	\begin{math}
		\overline{A\cap B\cap C}\neq
		\overline{A}\cap\overline{B}\cap\overline{C}
	\end{math}.

	\part[2]

	Si $X\subset\mathbb{R}$ es un conjunto tal que todo subconjunto
	infinito de $X$ tiene un punto de acumulación en $X$, entonces $X$
	es compacto.
\end{parts}

\begin{solutionordottedlines}
\end{solutionordottedlines}

\question[5]

Decimos que $f\colon X\subset\mathbb{R}\to\mathbb{R}$ es
uniformemente continua en $X$ si y solo si
\begin{equation*}
	\forall\varepsilon>0:
	\exists\delta>0
	\text{ tal que }
	\forall x,y\in X:
	\left|x-y\right|<
	\delta\implies
	\left|f\left(x\right)-f\left(y\right)\right|<
	\varepsilon.
\end{equation*}
Demuestre que si $X\subset\mathbb{R}$ es un conjunto acotado,
entonces la función $f\colon X\to\mathbb{R}$ es uniformemente
continua en $X\iff$ para cualquier sucesión de Cauchy
${\left\{x_{n}\right\}}_{n\in\mathbb{N}}\subset X$,
\begin{math}
	{\left\{f\left(x_{n}\right)\right\}}_{n\in\mathbb{N}}\subset
	\mathbb{R}
\end{math}
es una sucesión de Cauchy.

\begin{solutionordottedlines}
\end{solutionordottedlines}

\question[5]

Sean $f\colon X\subset\mathbb{R}\to\mathbb{R}$ una función y
$a\in\mathbb{R}$ es un punto de acumulación de
$X\cap\left(-\infty,a\right)$.
Demuestre que las siguientes afirmaciones son equivalentes:
\begin{parts}
	\part

	\begin{math}
		\lim_{x\to a^{-}}f\left(x\right)=
		L\in\mathbb{R}
	\end{math}.

	\part

	\begin{math}
		\forall
		{\left\{x_{n}\right\}}_{n\in\mathbb{N}}\subset X\text{ sucesión creciente}:
		x_{n}\xrightarrow[n\to\infty]{}a\implies
		f\left(x_{n}\right)\xrightarrow[n\to\infty]{}L
	\end{math}.
\end{parts}
\begin{solutionordottedlines}
\end{solutionordottedlines}

\question

Sean $A$, $B\subset\mathbb{R}$ dos conjuntos no vacíos.
Definimos su distancia como
\begin{equation*}
	\operatorname{dist}\left(A,B\right)\coloneqq
	\inf
	\left\{
	\left|x-y\right|\in\mathbb{R}\mid
	\left(x,y\right)\in A\times B
	\right\}.
\end{equation*}

\begin{parts}
	\part[2]


	Si $A$ es compacto, $B$ es cerrado y $A\cap B=\emptyset$,
	entonces $\operatorname{dist}\left(A,B\right)>0$.

	\part[2]

	Si $A$ es compacto y $B$ es cerrado, entonces
	$\exists\left(x_{0},y_{0}\right)\in A\times B$ tal que
	\begin{math}
		\operatorname{dist}
		\left(A,B\right)=
		\left|x_{0}-y_{0}\right|
	\end{math}.

	\part[1]

	\begin{math}
		\operatorname{dist}
		\left(A,B\right)=
		\operatorname{dist}
		\left(\overline{A},\overline{B}\right)
	\end{math}.
\end{parts}

\begin{solutionordottedlines}
\end{solutionordottedlines}
