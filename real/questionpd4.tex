\question

Si $f$ es uniformemente continua en un conjunto acotado, entonces $f$
es acotada.
Concluya que $x\longmapsto\frac{1}{x^{2}}$ no es uniformemente
continua en $\left(0,1\right)$.

% \question

% Sean $\emptyset\neq X\subset\mathbb{R}$ un conjunto compacto y
% $f\colon X\to\mathbb{R}$ una función.
% Si $f$ es continua, entonces $f$ es uniformemente continua.
% Este resultado se conoce como el teorema de Heine-Cantor.

% \question

% Si $f\colon X\subset\mathbb{R}\to\mathbb{R}$ es uniformemente
% continua, entonces $\forall S\subset X$: $f\colon S\to\mathbb{R}$ es
% continua.

% \question

% Sea $I=\left(a,b\right)\subset\mathbb{R}$ un intervalo no degenerado
% y acotado.
% $f\colon I\to\mathbb{R}$ es uniformemente continua si y solo si
% ambos límites $\lim\limits_{x\to a^{+}}f\left(x\right)$ y
% $\lim\limits_{x\to b^{-}}f\left(x\right)$ existen.

% \question

% Sea $I=\left(a,b\right)\subset\mathbb{R}$ un intervalo no degenerado
% y acotado.
% Si $f\colon I\to\mathbb{R}$ es continua, monótona y acotada, entonces
% $f$ es uniformemente continua.

\question

Sean $f\colon S\to\mathbb{R}$ una función y $S\subset X$ un conjunto.
Decimos que $g\colon X\to\mathbb{R}$ es una extensión de $f$ hacia
$X$ si y solo si $\forall x\in S$: $g\left(x\right)=f\left(x\right)$.
Esto es, las dos funciones coinciden en $S$, es decir,
$g\big|_{S}=f$.

Suponga que $a<b$. Una función $f\colon\left(a,b\right)\to\mathbb{R}$
es uniformemente continua $\iff f$ tiene una extensión continua
$g\colon\left[a,b\right]\to\mathbb{R}$.

\question

Determine cuáles de las siguientes funciones $f$ son uniformemente
continuas en el conjunto dado.

\begin{multicols}{2}
	\begin{parts}
		\part

		$f\left(x\right)=5x^{2}-3x+7$	en $\left[1,3\right]$.

		\part

		$f\left(x\right)=\frac{1}{x^{2}}$ en $\left(1,5\right)$.

		\part

		$f\left(x\right)=\tan\left(x\right)$ en
		$\left(-\frac{\pi}{2},\frac{\pi}{2}\right)$.

		\part

		$f\left(x\right)=5x^{2}-3x+7$ en $\left(1,3\right)$.

		\part

		$f\left(x\right)=x\sen\left(x\right)$ en
		$\left(0,\frac{\pi}{2}\right)$.

		\part

		$f\left(x\right)=\tan\left(x\right)$ en
		$\left(-\frac{\pi}{4},\frac{\pi}{4}\right)$.
	\end{parts}
\end{multicols}

% \question

% $x\longmapsto\sqrt{x}$ es uniformemente continua en
% $\left[0,\infty\right)$.

% \textbf{Sugerencia}: Pruebe que
% $\left|\sqrt{x}-\sqrt{y}\right|\leq\sqrt{\left|x-y\right|}$.

% \question

% Si $f,g\colon X\subset\mathbb{R}\to\mathbb{R}$ son uniformemente
% continuas, entonces $f+g$ es uniformemente continua.

% \question

% Sean $f,g\colon X\subset\mathbb{R}\to\mathbb{R}$ uniformemente
% continuas.
% $\exists S\subset X$ tal que $fg\colon S\to\mathbb{R}$ no es
% uniformemente continua.

% \question

% Muestre que las funciones $\tan\left(x\right)$ y $\sec\left(x\right)$
% son continuas, pero no son uniformemente continuas en
% $\left(-\frac{\pi}{2},\frac{\pi}{2}\right)$,
% mientras que $\cot\left(x\right)$ y $\csc\left(x\right)$ son
% continuas, pero no son uniformemente continuas en $\left(0,\pi\right)$.

% \question

% Sea $f\colon I\to\mathbb{R}$ una función definida en un intervalo no
% degenerado y acotado.
% $\exists f$ continua y acotada tal que $f$ no es uniformemente
% continua.

\question

Suponga que $f$ y $g$ son uniformemente continuas en un conjunto
$X\subset\mathbb{R}$.
Pruebe que
\begin{parts}
	\part

	si $f$ y $g$ son acotadas en $X$, entonces $fg$ es uniformemente
	continua en $X$;

	\part

	si $X$ es acotado, entonces $fg$ es uniformemente continua en $X$.
\end{parts}

\question

Sea $X\subset\mathbb{R}$ un conjunto no vacío.
Si $f$ es uniformemente continua en $X$ y $g$ es uniformemente
continua en $f\left(X\right)$, entonces $g\circ f$ es uniformemente
continua en $X$.

\question

Sea $I$ un intervalo abierto no degenerado.
Se dice que $f\colon I\to\mathbb{R}$ satisface la
condición de Lipschitz de orden $\alpha\iff\exists M>0$ tal
que $\forall x,y\in I$:
\begin{math}
	\left|f\left(x\right)-f\left(y\right)\right|\leq
	M{\left|x-y\right|}^{\alpha}
\end{math}.

\begin{parts}
	\part

	Si $f\colon I\to\mathbb{R}$ satisface una condición Lipschitz de
	orden $\alpha$ para algún número real $\alpha>1$, entonces $f$ es
	derivable y $f^{\prime}\left(x\right)=0$.

	\part

	Encuentre un ejemplo de una función $f\colon I\to\mathbb{R}$ que
	satisface la condición Lipschitz de orden $\alpha=1$, pero $f$ no
	es derivable.
\end{parts}

\question

$\forall\alpha\in\left(0,1\right)$: $x\longmapsto x^{\alpha}$ es
uniformemente continua en $\left[0,1\right]$, pero no satisface la
condición Lipschitz.

Por lo tanto, una condición Lipschitz es estrictamente más fuerte que
la continuidad uniformemente.

% \question

% Sean $p>0$ y $f\colon\mathbb{R}\to\mathbb{R}$ una función.
% Si $\forall x\in\mathbb{R}$: $f\left(x+p\right)=f\left(x\right)$ y
% $f$ es continua en cualquier intervalo compacto de la forma
% $\left[a,a+p\right]$, entonces debe ser continua y uniformemente
% continua en $\mathbb{R}$.

% \question

% Basado en la pregunta anterior, se podría formular la siguiente
% conjetura: Si $f$ es uniformemente continua en conjuntos disjuntos
% $A$ y $B$, entonces $f$ es uniformemente continua en $A\cup B$.

% \begin{parts}
% 	\part

% 	Encuentre una función $f$ y dos intervalos abiertos acotados que
% 	pruebe que esta conjetura es falsa.

% 	\part

% 	Pruebe que la conjetura es verdadera si $A$ y $B$ son acotados y
% 	$\sup A<\inf B$.
% \end{parts}

% \question

% Sean $f\colon X\to\mathbb{R}$ una función y $S\subset X$.
% Definimos la oscilación de $f\big|_{S}$ como
% $\sup\left(A\right)-\inf\left(A\right)$ es acotada.

% \question

% El conjunto de discontinuidades de una función $f\colon X\to\mathbb{R}$
% es unión numerable de conjuntos cerrados.

% \question

% $\mathbb{R}$ y $\mathbb{R}\setminus\mathbb{Q}$ son de segunda categoría.

\question

Si $f$ es derivable, entonces $\forall c\in\mathbb{R}$:
\begin{math}
	\diff*{{\left[f\left(x\right)\right]}^{c}}{x}=
	c{\left[f\left(x\right)\right]}^{c-1}
	f^{\prime}\left(x\right)
\end{math}.

\question

Sea $a\geq 1$. Entonces $\forall x\in\mathbb{R}$, definimos
$a^{x}\coloneqq\lim\limits_{n\to\infty}a^{r_{n}}$, donde
${\left\{r_{n}\right\}}_{n\in\mathbb{N}}\subset\mathbb{Q}$ es
cualquier sucesión monótona creciente con
$r_{n}\xrightarrow[n\to\infty]{}x$.
Si $a\in\left(0,1\right)$, entonces $a^{-1}>1$, así que
$\forall x\in\mathbb{R}$, definimos
$a^{x}\coloneqq\frac{1}{{\left(a^{-1}\right)}^{x}}$.
Sea $a>0$, $a\neq 1$. Definimos la función
$f\left(x\right)=\log_{a}x$ como la inversa de la función
$g\left(x\right)=a^{x}$.
Esto es, $y=\log_{a}x\iff a^{y}=x$.
Demuestre que $\forall x\in\mathbb{R}\setminus\left\{0\right\}$:
$\diff*{\log_{a}\left|x\right|}{x}=\frac{1}{x\ln a}$.

\question

Si $f$ y $g$ son derivables y $f$ es positiva, entonces
\begin{math}
	\diff*{{\left[f\left(x\right)\right]}^{g\left(x\right)}}{x}=
	\left[f\left(x\right)\right]^{g\left(x\right)}
	\left[
		g^{\prime}\left(x\right)\ln f\left(x\right)+
		g\left(x\right)\frac{f^{\prime}\left(x\right)}{f\left(x\right)}
		\right]
\end{math}.

\question

$f\colon X\to\mathbb{R}$ es derivable en
$x_{0}\in\operatorname{int}\left(X\right)$ con derivada
\begin{math}
	f^{\prime}\left(x_{0}\right)\in\mathbb{R}\iff
	\forall{\left\{x_{n}\right\}}_{n\in\mathbb{N}}\subset
	X\setminus\left\{x_{0}\right\}
\end{math}
con $x_{n}\xrightarrow[n\to\infty]{}x_{0}$:
\begin{math}
	\frac{f\left(x_{n}\right)-f\left(x_{0}\right)}{x_{n}-x_{0}}
	\xrightarrow[n\to\infty]{}
	f^{\prime}\left(x_{0}\right)
\end{math}.

\question

Sea
\begin{math}
	f\left(x\right)=
	\begin{cases}
		x^{2}\sen\left(\frac{1}{x}\right), & \text{ si }x\neq 0. \\
		0,                                 & \text{ si }x=0.
	\end{cases}
\end{math}.
Pruebe que $f$ es derivable en $\mathbb{R}$ y que $f^{\prime}$ es
continua excepto en $0$.

\question

Sea
\begin{math}
	f\left(x\right)=
	\begin{cases}
		x^{2}, & \text{ si }x\in\mathbb{Q}.                    \\
		0,     & \text{ si }x\in\mathbb{R}\setminus\mathbb{Q}.
	\end{cases}
\end{math}
es derivable en $0$.
¿$f$ es derivable en algún otro punto?
Explique.

\question

Sea
\begin{math}
	f\left(x\right)=
	\begin{cases}
		x^{r}\sen\left(\frac{1}{x}\right), & \text{ si }x\neq 0. \\
		0,                                 & \text{ si }x=0.
	\end{cases}
\end{math}
Pruebe que $f$ es continua por la derecha en $0\iff r>0$ y que $f$ es
derivable por la derecha en $0\iff r>1$.

\question

\begin{parts}
	\part

	Si $f$ es derivable en $x_{0}$, entonces
	\begin{math}
		\lim\limits_{h\to 0}
		\frac{f\left(x_{0}+h\right)-f\left(x_{0}-h\right)}{2h}
	\end{math}
	existe y es igual a $f^{\prime}\left(x_{0}\right)$.

	\part

	Encuentre un ejemplo de una función $f$ y un punto $x_{0}$ tales
	que el
	\begin{math}
		\lim\limits_{h\to 0}
		\frac{f\left(x_{0}+h\right)-f\left(x_{0}-h\right)}{2h}
	\end{math}
	exista, pero $f$ no es derivable en $x_{0}$.
\end{parts}

\question

Sin usar la regla de la cadena, pruebe que si $f$ es derivable en
$x_{0}$, entonces $\forall n\in\mathbb{N}$: $f^{n}$ es derivable en
$x_{0}$, y
\begin{math}
	\diff*{{\left[f\left(x\right)\right]}^{n}}{x}=
	n\left[f\left(x\right)\right]^{n-1}f^{\prime}\left(x\right)
\end{math}.
Use el principio de inducción matemática.

% \question

% Si $f$ es derivable en $x_{0}$ y $f\left(x_{0}\right)\neq 0$, entonces
% \begin{math}
% 	\diff*{\left[\frac{1}{f\left(x\right)}\right]}{x}=
% 	-\frac{f^{\prime}\left(x\right)}{f^{2}\left(x\right)}
% \end{math}.

% \question

% Establezca y pruebe la regla de la cadena para $h\circ g\circ f$.

% \question

% Determine donde la función $\sqrt{x+\sqrt{x+\sqrt{x}}}$ es derivable
% y encuentre su derivada.

\question

Suponga que $f\colon X\subset\mathbb{R}\to\mathbb{R}$ es derivable en
$x_{0}\in\operatorname{int}\left(X\right)$ y
$g\colon Y\subset\mathbb{R}\to\mathbb{R}$ es derivable en
$f\left(x_{0}\right)\in\operatorname{int}\left(Y\right)$.
Encuentre la falla en la siguiente ``demostración'' de la regla de la
cadena:
\begin{align*}
	\lim_{x\to x_{0}}
	\frac{
	\left(g\circ f\right)\left(x\right)-
	\left(g\circ f\right)\left(x_{0}\right)
	}{x-x_{0}} & =
	\lim_{x\to x_{0}}
	\left[
	\frac{g\left(f\left(x\right)\right)-g\left(f\left(x_{0}\right)\right)}{f\left(x\right)-f\left(x_{0}\right)}\cdot
	\frac{f\left(x\right)-f\left(x_{0}\right)}{x-x_{0}}
	\right].                                                                 \\
	           & =
	\lim_{f\left(x\right)\to f\left(x_{0}\right)}
	\frac{g\left(f\left(x\right)\right)-g\left(f\left(x_{0}\right)\right)}{f\left(x\right)-f\left(x_{0}\right)}\cdot
	\lim_{x\to x_{0}}
	\frac{f\left(x\right)-f\left(x_{0}\right)}{x-x_{0}}.                     \\
	           & \text{usando el teorema 1 de cambio de variable en límites} \\
	           & =
	g^{\prime}\left(f\left(x_{0}\right)\right)\cdot f^{\prime}\left(x_{0}\right).
\end{align*}
Entonces, muestre que la prueba es válida si $f$ es estrictamente monótona en una vecindad de $x_{0}$.

\begin{theorem}[Cambio de variable en límites]
	Sean $f\colon X\subset\mathbb{R}\to\mathbb{R}$ y $g\colon Y\subset\mathbb{R}\to\mathbb{R}$ dos funciones.
	Si $\lim\limits_{x\to x_{0}}g\left(x\right)=u_{0}$ y $\lim\limits_{u\to u_{0}}f\left(u\right)=L$, donde $x_{0}$ y $u_{0}$
	son dos puntos de acumulación de $X$ e $Y$, respectivamente, y $\forall x\in V^{\prime}_{\delta}\left(x_{0}\right)\cap Y$:
	$g\left(x\right)\in X\setminus\left\{u_{0}\right\}$, entonces
	\begin{equation*}
		\lim_{x\to x_{0}}f\left(g\left(x\right)\right)=
		\lim_{u\to u_{0}}f\left(u\right)=L.
	\end{equation*}
	\vspace*{-2\baselineskip}\setlength\belowdisplayshortskip{0pt}
	\begin{figure}[ht!]
		\centering
		\includegraphics[width=.7\paperwidth]{changevariable}
	\end{figure}
\end{theorem}


% \question

% Sean $X\subset\mathbb{R}$ un conjunto abierto.
% Use el principio de inducción matemática para probar la regla de
% Leibniz.
% $\forall n\in\mathbb{N}$: si $f$ y $g$ son $n$-veces derivables,
% entonces
% \begin{math}
% 	{\left(fg\right)}^{\left(n\right)}\left(x\right)=
% 	\sum_{k=0}^{n}
% 	\binom{n}{k}
% 	f^{\left(k\right)}\left(x\right)
% 	g^{\left(n-k\right)}\left(x\right)
% \end{math},
% donde $f^{\left(k\right)}$ y $g^{\left(k\right)}$ denotan la
% $k$-ésima derivada de $f$ y $g$, $f^{\left(0\right)}=f$,
% $g^{\left(0\right)}=g$, y
% $\binom{n}{k}\coloneqq\frac{n!}{k!\left(n-k\right)!}$.

% \question

% Si $f\colon\mathbb{R}\to\mathbb{R}$ es periódica con período $p$ y
% derivable en algún intervalo $\left[a,a+p\right]$,
% entonces $f$ es derivable en todo $\mathbb{R}$ y $f^{\prime}$ es
% periódica con período $p$.

% \question

% Si $f\colon\left[a,b\right)\to\mathbb{R}$.
% Defina $\widehat{f}\colon\mathbb{R}\to\mathbb{R}$ por
% \begin{math}
% 	\widehat{f}\left(x\right)=
% 	f\left(
% 	x-\left(b-a\right)
% 	\left\lfloor\frac{x-a}{b-a}\right\rfloor
% 	\right)
% \end{math},
% donde $\left\lfloor x\right\rfloor$ denota la función máximo entero. Pruebe que
% \begin{parts}
% 	\part

% 	$\forall x\in\mathbb{R}$:
% 	\begin{math}
% 		x-\left(b-a\right)
% 		\left\lfloor\frac{x-a}{b-a}\right\rfloor\in
% 		\left[a,b\right)
% 	\end{math}.

% 	\part

% 	$\widehat{f}$ es periódica con período $b-a$ y
% 	$\widehat{f}\big|_{\left[a,b\right)}=f$.
% 	Llamamos $\widehat{f}$ la extensión periódica de $f$ hacia
% 	$\mathbb{R}$.

% 	\part

% 	Si $f$ es continua en $\left[a,b\right)$ y
% 	$\lim\limits_{x\to b^{-}}f\left(x\right)=f\left(a\right)$, entonces
% 	$\widehat{f}$ es continua en $\mathbb{R}$.

% 	\part

% 	Si $f$ es derivable en $\left(a,b\right)$, derivable por la derecha
% 	en $a$ y
% 	$\lim\limits_{x\to b^{-}}f^{\prime}\left(x\right)=\lim\limits_{x\to a^{+}}f^{\prime}\left(x\right)$,
% 	entonces $\widehat{f}$ es derivable en todo $\mathbb{R}$.
% \end{parts}

% \question

% Defina la función de Thomae
% \begin{equation*}
% 	T\left(x\right)=
% 	\left\{
% 	\begin{aligned}
% 		 & \frac{1}{n}, & \text{ si }x=\frac{m}{n}\neq 0,\text{donde }m\in\mathbb{Z},n\in\mathbb{N}\text{ y }\operatorname{mcd}\left(m,n\right)=1 \\
% 		 & 1,           & \text{ si }x=0                                                                                                          \\
% 		 & 0,           & \text{ si }x\in\mathbb{R}\setminus\mathbb{Q}
% 	\end{aligned}
% 	\right\}
% \end{equation*}
% \vspace*{-2.5\baselineskip}\setlength\belowdisplayshortskip{0pt}
% \begin{figure}[ht!]
% 	\centering
% 	\includegraphics[width=.7\paperwidth]{thomae}
% \end{figure}

% \begin{parts}
% 	\part $T$ es continua en todo número irracional, pero es discontinua en todo número racional.
% 	\part $\forall x\in\mathbb{R}$: $\lim\limits_{y\to x}T\left(y\right)=0$.
% \end{parts}

% \question

% Encuentre un ejemplo de una función
% $f\colon\mathbb{R}\to\mathbb{R}$ que es estrictamente creciente,
% pero que $f^{\prime}\left(x\right)>0$ en todas partes.

% \question

% Pruebe que no existe una función derivable $f$ definida en cualquier
% intervalo abierto que contenga al cero tal que $\forall x\in I$:
% \begin{math}
% 	f^{\prime}\left(x\right)=
% 	\left\{
% 	\begin{aligned}
% 		1, & \text{si }x\geq 0, \\
% 		2, & \text{si }x<0
% 	\end{aligned}
% 	\right\}
% \end{math}.

% \question

% Pruebe que si $f$ es derivable en un intervalo abierto $I$ y
% $\forall x\in I$: $f^{\prime}\left(x\right)\neq 0$, entonces
% $f^{\prime}\left(x\right)$ tiene el mismo signo a lo largo del
% intervalo $I$.

\question

En cada uno de los siguientes casos, dé un ejemplo de una función
que cumpla las condiciones dadas y para la cual no se cumpla la
conclusión del teorema de Rolle.

\begin{parts}
	\part

	$f$ es continua en $\left[a,b\right]$ y
	$f\left(a\right)=f\left(b\right)$.

	\part

	$f$ es derivable en $\left(a,b\right)$ y
	$f\left(a\right)=f\left(b\right)$.

	\part

	$f$ es continua en $\left[a,b\right]$ y derivable en
	$\left(a,b\right)$.
\end{parts}

\question

Sea $I$ un intervalo no vacío $f$.
Pruebe que si $\forall x\in I:$
$f^{\prime}\left(x\right)\neq 0$, entonces $f$ es
inyectiva en $I$.

\question

Si $\forall x\in\mathbb{R}$: $f^{\prime\prime}\left(x\right)=0$,
entonces $f$ debe ser un polinomio y tiene grado menor o igual que
$1$.

\question

Si $\forall x\in\mathbb{R}$:
$f^{\prime\prime\prime}\left(x\right)=0$, entonces $f$ debe ser un
polinomio y tiene grado menor o igual que $2$.

\question

Si $f\colon\mathbb{R}\to\mathbb{R}$ y $\forall x,y\in\mathbb{R}$:
$\left|f\left(x\right)-f\left(y\right)\right|\leq{\left|x-y\right|}^{2}$,
entonces $f$ es cualquier función constante.

\question

Si $f$ es diferenciable, $f\left(-1\right)=5$, $f\left(0\right)=0$ y
$f\left(1\right)=10$, entonces $\exists c,d\in\left(-1,1\right)$
tales que $f^{\prime}\left(c\right)=-3$ y
$f^{\prime}\left(d\right)=3$.

\question

Use el teorema del valor medio para probar las siguientes desigualdades.

\begin{multicols}{2}
	\begin{parts}
		\part

		$\forall x,y\in\mathbb{R}$:
		$\left|\cos\left(x\right)-\cos\left(y\right)\right|\leq\left|x-y\right|$.

		\part

		$\forall x\in\left(0,\frac{\pi}{2}\right)$: $\tan\left(x\right)>x$.

		\part

		$\forall 0<x<y$: $\frac{y-x}{y}<\ln\left(\frac{y}{x}\right)<\frac{y-x}{x}$.

		\part

		$\forall x>1$: $\frac{x-1}{x}<\ln\left(x\right)<x-1$.

		\part

		$\forall x>1$: $e^{x}>ex$.

		\part

		$\forall x\in\left(0,\frac{\pi}{2}\right]$: $\sen\left(x\right)>\frac{2x}{\pi}$
	\end{parts}
\end{multicols}

\question

Pruebe que $x\longmapsto\frac{\sen\left(x\right)}{x}$ es estrictamente decreciente en $\left(0,\frac{\pi}{2}\right]$.

% \question

% $\forall x\neq 0$: $e^{x}>ex$.

\question

Muestre que existe un intervalo compacto en el cual la función
\begin{math}
	f\left(x\right)=
	\begin{cases}
		x^{2}\sen\left(\frac{1}{x^{2}}\right), & \text{si }x\neq 0, \\
		0,                                     & \text{si }x=0,
	\end{cases}
\end{math}
es derivable, pero $f^{\prime}$ no es acotada.

\question

\begin{parts}
	\part

	Si $f^{\prime}$ existe y es acotada en un intervalo $I$ (posiblemente infinito), entonces
	$f$ satisface la condición Lipschitz de orden $1$ en $I$.

	\part

	Use el resultado anterior para probar que si $a>0$, entonces
	$\forall n\in\mathbb{N}$: $f\left(x\right)=\frac{1}{x^{n}}$ es uniformemente continua en $\left[a,\infty\right)$.
\end{parts}

\question

Si $f$ satisface la condición Lipschitz de orden $\alpha>1$ en un intervalo $I$, entonces $f$ es constante en $I$.

\question

Si $f\colon\mathbb{R}\to\mathbb{R}$ es continua, entonces $f$ es
uniformemente continua $\iff\left|f\right|$ es uniformemente
continua.
