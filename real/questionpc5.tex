\question

Justifique la veracidad o falsedad de las siguientes proposiciones.

\begin{parts}
	\part[2]

	Sea $K\subset\mathbb{R}$ un conjunto compacto.
	Si $f\colon K\to\mathbb{R}$ es continua, entonces
	$\forall\varepsilon>0$: $\exists c_{\varepsilon}>0$ tal que
	\begin{math}
		\left|y-x\right|\geq
		\varepsilon\implies
		\left|f\left(y\right)-f\left(x\right)\right|\leq
		c_{\varepsilon}\left|y-x\right|
	\end{math}.

	\part[1\half]
	Cualquier función $f\colon\mathbb{N}\to\mathbb{R}$ es uniformemente
	continua.

	\part[1\half]
	Sea $K\subset\mathbb{R}$ un conjunto compacto.
	Si $f\colon\mathbb{R}\to\mathbb{R}$ es derivable en $K$ y
	$g\colon\mathbb{R}\to\mathbb{R}$ no es derivable en $K$,
	entonces $f+g$ no es derivable en $K$.
\end{parts}

\begin{solutionordottedlines}
\end{solutionordottedlines}

\question

Sean las funciones $f,g\colon X\subset\mathbb{R}\to\mathbb{R}$
uniformemente continuas.
\begin{parts}
	\part[2\half]

	Si $f$ y $g$ son acotadas, entonces $fg$ es uniformemente
	continua.

	\part[2\half]

	Si $X$ es acotado, entonces $fg$ es uniformemente continua.
\end{parts}

\begin{solutionordottedlines}
\end{solutionordottedlines}

\question[5]

Sean $f\colon X\subset\mathbb{R}\to\mathbb{R}$ una función y
$a\in\operatorname{int}\left(X\right)$.
Si $f^{\prime}\left(a\right)$ existe y las sucesiones
$x_{n}\xrightarrow[n\to\infty]{}a$ y
$y_{n}\xrightarrow[n\to\infty]{}a$ son tales que
$\forall n\in\mathbb{N}$: $a\in \left(x_{n},y_{n}\right)$, entonces
\begin{math}
	\frac{f\left(y_{n}\right)-f\left(x_{n}\right)}{y_{n}-x_{n}}\xrightarrow[n\to\infty]{}
	f^{\prime}\left(a\right)
\end{math}.
\begin{solutionordottedlines}
\end{solutionordottedlines}

\question

Una función $f$ definida en un intervalo abierto $I$ se dice que
satisface la condición de Lipschitz de orden $\alpha$ en $I$ si y solo si
$\exists M>0$ tal que $\forall x,y\in I$:
\begin{math}
	\left|f\left(x\right)-f\left(y\right)\right|\leq
	M{\left|x-y\right|}^{\alpha}
\end{math}.

\begin{parts}
	\part[2\half]

	Pruebe que si $f$ satisface una condición Lipschitz de orden
	$\alpha$ en $I$ para algún número real $\alpha>1$, entonces $f$ es
	derivable en $I$ y $f^{\prime}\left(x\right)=0$ en $I$.

	\part[2\half]

	Encuentre un ejemplo de una función $f\colon I\to\mathbb{R}$ que
	satisface la condición Lipschitz de orden $\alpha=1$ en $I$, pero
	$f$ no es derivable en $I$.
\end{parts}

% \bonusquestion

% Justifique la veracidad o falsedad de las siguientes proposiciones.

% \begin{parts}
% 	\part[1]

% 	Sea
% 	\begin{math}
% 		f\left(x\right)=
% 		\begin{cases}
% 			x^{2}\sen\left(\frac{1}{x}\right), & x\neq 0. \\
% 			0,                                 & x=0.
% 		\end{cases}
% 	\end{math}
% 	Entonces,
% 	\begin{math}
% 		f^{\prime}\left(x\right)=
% 		\begin{cases}
% 			2x\sen\left(\frac{1}{x}\right)-\cos\left(\frac{1}{x}\right), & x\neq 0, \\
% 			0,                                                           & x=0,
% 		\end{cases}
% 	\end{math}
% 	es discontinua en el origen.

% 	\part[1]

% 	Si
% 	\begin{math}
% 		f\left(x\right)=
% 		\begin{cases}
% 			x^{2}\sen\left(\frac{1}{x^{2}}\right), & x\neq 0. \\
% 			0,                                     & x=0.
% 		\end{cases}
% 	\end{math}
% 	Entonces,
% 	\begin{math}
% 		f^{\prime}\left(x\right)=
% 		\begin{cases}
% 			2x\sen\left(\frac{1}{x^{2}}\right)-\frac{1}{x^{2}}, & x\neq 0, \\
% 			0,                                                  & x=0,
% 		\end{cases}
% 	\end{math}
% 	es acotada en $\left[-1,1\right]$.

% 	\part[2]

% 	Si
% 	\begin{math}
% 		f\left(x\right)=
% 		\begin{cases}
% 			\exp\left(-\frac{1}{x^{2}{\left(1-x\right)}^{2}}\right), & x\in\left(0,1\right).                    \\
% 			0,                                                       & x\in\mathbb{R}\setminus\left(0,1\right).
% 		\end{cases}
% 	\end{math}.
% 	Entonces, $f$ es positiva e infinitamente diferenciable en $\left(0,1\right)$.
% \end{parts}

% \begin{solutionordottedlines}
% \end{solutionordottedlines}
