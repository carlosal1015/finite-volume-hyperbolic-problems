\chapter{Implementación de las técnicas}

A continuación, pensamos dos estrategias para tentar resolver por el
método de los volúmenes finitos de un
\textcolor{DarkRed}{sistema EDP complicado}.
Se trata de resolver un sistema EDP a la vez, de menor a mayor
dificultad.
Suponga que $\Omega\subset\mathbb{R}^{3}$ es un conjunto abierto y
$\symbf{a},\symbf{b}\in\mathbb{R}^{3}\setminus\left\{\symbf{0}\right\}$.

\section{Primera estrategia}

\begin{equation*}
	\eqref{eq:advectionsystem}\implies
	\eqref{eq:advectionreactionsystem}\implies
	\eqref{eq:advectionreactionsystemquaslinearnonhomogeneous}\implies
	\mathcolor{DarkRed}{\eqref{eq:complicatedsystem}}.
\end{equation*}

\section{Segunda estrategia}

\begin{equation*}
	\eqref{eq:advectionsystem}\implies
	\eqref{eq:advectionreactionsystemquaslinear}\implies
	\eqref{eq:advectionreactionsystemquaslinearnonhomogeneous}\implies
	\mathcolor{DarkRed}{\eqref{eq:complicatedsystem}}.
\end{equation*}

\subsection*{Sistema EDP de advección lineal homogéneo}

Encuentre
\begin{math}
	\symbf{u}\in
	\symbf{C}^{1}\left(I\times\left[0,T\right],\Omega\right)
\end{math}
en el problema de valor inicial y de frontera~\eqref{eq:advectionsystem}
\begin{equation}\label{eq:advectionsystem}
	\begin{cases}
		\diffp{\symbf{u}}{t}+\diffp{}{x}\symbf{f}\left(\symbf{u}\right)=
		\symbf{0}     &
		\text{ en }I\times\left(0,T\right].   \\
		\symbf{u}                                                      =
		\symbf{u}_{0} &
		\text{ en }I\times\left\{t=0\right\}. \\
		\symbf{u}                                                      =
		\symbf{0}     &
		\text{ en }\partial I\times\left[0,T\right].
	\end{cases}
\end{equation}
Donde
\begin{math}
	\symbf{u}_{0}\in
	{\symbf{L}^{\infty}_{\text{loc}}\left(I\right)}^{3}
\end{math}
es conocida y
\begin{math}
	\symbf{f}\in
	\symbf{C}^{1}\left(\Omega,\mathbb{R}^{3}\right)
\end{math}
es dada por
\begin{math}
	\symbf{f}\left(\symbf{u}\right)=
	\symbf{a}\odot\symbf{u}
\end{math}.

\subsection*{Sistema EDP de advección reacción lineal}

Encuentre
\begin{math}
	\symbf{u}\in
	\symbf{C}^{1}\left(I\times\left[0,T\right],\Omega\right)
\end{math}
en el problema de valor inicial y de frontera~\eqref{eq:advectionreactionsystem}
\begin{equation}\label{eq:advectionreactionsystem}
	\begin{cases}
		\diffp{\symbf{u}}{t}+
		\diffp{}{x}\symbf{f}\left(\symbf{u}\right)               =
		\symbf{s}\left(\symbf{u}\right) &
		\text{ en }I\times\left(0,T\right].   \\
		\symbf{u}                                                                     =
		\symbf{u}_{0}                   &
		\text{ en }I\times\left\{t=0\right\}. \\
		\symbf{u}                                                                     =
		\symbf{0}                       &
		\text{ en }\partial I\times\left[0,T\right].
	\end{cases}
\end{equation}
Donde
\begin{math}
	\symbf{u}_{0}\in
	{\symbf{L}^{\infty}_{\text{loc}}\left(I\right)}^{3}
\end{math},
\begin{math}
	\symbf{s}\colon\Omega\to
	\mathbb{R}^{3}
\end{math}
son conocidas y
\begin{math}
	\symbf{f}\in
	\symbf{C}^{1}\left(\Omega,\mathbb{R}^{3}\right)
\end{math}
es dada por
\begin{math}
	\symbf{f}\left(\symbf{u}\right)=
	\symbf{a}\odot\symbf{u}
\end{math},
\begin{math}
	\symbf{s}\left(\symbf{u}\right)=
	\symbf{b}\odot\symbf{u}
\end{math}.

\subsection*{Sistema EDP de advección reacción cuasilineal I}

Encuentre
\begin{math}
	\symbf{u}\in
	\symbf{C}^{1}\left(I\times\left[0,T\right],\Omega\right)
\end{math}
en el problema de valor inicial y de frontera~\eqref{eq:advectionreactionsystemquaslinear}
\begin{equation}\label{eq:advectionreactionsystemquaslinear}
	\begin{cases}
		\diffp{\symbf{u}}{t}+\diffp{}{x}\symbf{f}\left(\symbf{u}\right)=
		\symbf{s}\left(\symbf{u}\right) & \text{ en }I\times\left(0,T\right].          \\
		\symbf{u}                                                      =
		\symbf{u}_{0}                   & \text{ en }I\times\left\{t=0\right\}.        \\
		\symbf{u}                                                      =
		\symbf{0}                       & \text{ en }\partial I\times\left[0,T\right].
	\end{cases}
\end{equation}
Donde
\begin{math}
	\symbf{u}_{0}\in
	{\symbf{L}^{\infty}_{\text{loc}}\left(I\right)}^{3}
\end{math},
\begin{math}
	\symbf{s}\colon\Omega\to
	\mathbb{R}^{3}
\end{math}
son conocidas y
\begin{math}
	\symbf{f}\in
	\symbf{C}^{1}\left(\Omega,\mathbb{R}^{3}\right)
\end{math}
es dada por
\begin{math}
	\symbf{f}\left(\symbf{u}\right)=
	\symbf{a}\odot\symbf{u}+
	\mathcolor{DarkRed}{a_{1}u_{1}\left(u_{2}-1\right)\symbf{e_{1}}}
\end{math},
\begin{math}
	\symbf{s}\left(\symbf{u}\right)=
	\symbf{b}\odot\symbf{u}
\end{math}.
% -\int^{x}\symbf{b}\odot\symbf{u}\dl y

\subsection*{Sistema EDP de advección reacción cuasilineal II}

Encuentre
\begin{math}
	\symbf{u}\in
	\symbf{C}^{1}\left(I\times\left[0,T\right],\Omega\right)
\end{math}
en el problema de valor inicial y de frontera~\eqref{eq:advectionreactionsystemquaslinearnonhomogeneous}
\begin{equation}\label{eq:advectionreactionsystemquaslinearnonhomogeneous}
	\begin{cases}
		\diffp{\symbf{u}}{t}+\diffp{}{x}\symbf{f}\left(\symbf{u}\right)=
		\symbf{s}\left(\symbf{u}\right) & \text{ en }I\times\left(0,T\right].          \\
		\symbf{u}                                                      =
		\symbf{u}_{0}                   & \text{ en }I\times\left\{t=0\right\}.        \\
		\symbf{u}                                                       =
		\symbf{0}                       & \text{ en }\partial I\times\left[0,T\right].
	\end{cases}
\end{equation}
Donde
\begin{math}
	\symbf{u}_{0}\in
	{\symbf{L}^{\infty}_{\text{loc}}\left(I\right)}^{3}
\end{math},
\begin{math}
	\symbf{s}\colon\Omega\to
	\mathbb{R}^{3}
\end{math}
son conocidas y
\begin{math}
	\symbf{f}\in
	\symbf{C}^{1}\left(\Omega,\mathbb{R}^{3}\right)
\end{math}
es dada por
\begin{math}
	\symbf{f}\left(\symbf{u}\right)=
	\symbf{a}\odot\symbf{u}+
	\mathcolor{DarkRed}{a_{1}u_{1}\left(u_{2}-1\right)\symbf{e_{1}}}
\end{math},
\begin{math}
	\symbf{s}\left(\symbf{u}\right)=
	u_{2}u_{3}\symbf{b}
	% \begin{bmatrix}
	% 	b_{1}u_{2}u_{3}  \\
	% 	-b_{2}u_{2}u_{3} \\
	% 	-b_{3}u_{2}u_{3}
	% \end{bmatrix}
\end{math}.

\subsection*{Sistema EDP de advección reacción cuasilineal III}

Encuentre
\begin{math}
	\symbf{u}\in
	\symbf{C}^{1}\left(I\times\left[0,T\right],\Omega\right)
\end{math}
en el problema de valor inicial y de frontera~\eqref{eq:advectionreactionsystemquaslinearnonhomogeneous}
\begin{equation}\label{eq:advectionreactionsystemquaslinearnonhomogeneous}
	\begin{cases}
		\diffp{\symbf{u}}{t}+\diffp{}{x}\symbf{f}\left(\symbf{u}\right)=
		\symbf{s}\left(\symbf{u}\right) & \text{ en }I\times\left(0,T\right].          \\
		\symbf{u}                                                      =
		\symbf{u}_{0}                   & \text{ en }I\times\left\{t=0\right\}.        \\
		\symbf{u}                                                       =
		\symbf{0}                       & \text{ en }\partial I\times\left[0,T\right].
	\end{cases}
\end{equation}
Donde
\begin{math}
	\symbf{u}_{0}\in
	{\symbf{L}^{\infty}_{\text{loc}}\left(I\right)}^{3}
\end{math},
\begin{math}
	\symbf{s}\colon\Omega\to
	\mathbb{R}^{3}
\end{math}
son conocidas y
\begin{math}
	\symbf{f}\in
	\symbf{C}^{1}\left(\Omega,\mathbb{R}^{3}\right)
\end{math}
es dada por
\begin{math}
	\symbf{f}\left(\symbf{u}\right)=
	\symbf{a}\odot\symbf{u}+
	\mathcolor{DarkRed}{a_{1}u_{1}\left(u_{2}-1\right)\symbf{e_{1}}}
\end{math},
\begin{math}
	\symbf{s}\left(\symbf{u}\right)=
	u_{2}u_{3}\symbf{b}-\beta u_{1}\symbf{e}_{1}
	% \begin{bmatrix}
	% 	b_{1}u_{2}u_{3}-\beta u_{1} \\
	% 	-b_{2}u_{2}u_{3}            \\
	% 	-b_{3}u_{2}u_{3}
	% \end{bmatrix}
\end{math}.

\subsection*{\color{DarkRed}Sistema EDP complicado}

Encuentre
\begin{math}
	\symbf{u}\in
	\symbf{C}^{1}\left(I\times\left[0,T\right],\Omega\right)
\end{math}
en el problema de valor inicial y de frontera~\eqref{eq:complicatedsystem}
\begin{equation}\label{eq:complicatedsystem}
	\begin{cases}
		\diffp{\symbf{u}}{t}+\diffp{}{x}\symbf{f}\left(\symbf{u}\right)=
		\symbf{s}\left(\symbf{u}\right) & \text{ en }I\times\left(0,T\right].          \\
		\symbf{u}                                                      =
		\symbf{u}_{0}                   & \text{ en }I\times\left\{t=0\right\}.        \\
		\symbf{u}                                                       =
		\symbf{0}                       & \text{ en }\partial I\times\left[0,T\right].
	\end{cases}
\end{equation}
Donde
\begin{math}
	\symbf{u}_{0}\colon I\to
	\mathbb{R}^{3}
\end{math},
\begin{math}
	\symbf{s}\colon\Omega\to
	\mathbb{R}^{3}
\end{math}
son conocidas y
\begin{math}
	\symbf{f}\colon\Omega\to
	\mathbb{R}^{3}
\end{math}
es dada por
\begin{math}
	\symbf{f}\left(\symbf{u}\right)=
	\symbf{a}\odot\symbf{u}+
	\mathcolor{DarkRed}{a_{1}u_{1}\left(u_{3}-1\right)\symbf{e_{1}}}
\end{math},
\begin{math}
	\symbf{s}\left(\symbf{u}\right)=
	u_{2}u_{3}\Phi\symbf{b}-\beta u_{1}\symbf{e}_{1}
	% \begin{bmatrix}
	% 	b_{1}u_{2}u_{3}\Phi-\beta u_{1} \\
	% 	-b_{2}u_{2}u_{3}\Phi            \\
	% 	-b_{3}u_{2}u_{3}\Phi
	% \end{bmatrix}
\end{math}.

\chapter{Resultados}

%https://mladenivkovic.github.io/work.html
% \begin{figure}[ht!]
% 	\centering
% 	\includegraphics[width=.8\paperwidth]{1}
% 	\includegraphics[width=.8\paperwidth]{2}
% 	\includegraphics[width=.8\paperwidth]{3}
% \end{figure}

