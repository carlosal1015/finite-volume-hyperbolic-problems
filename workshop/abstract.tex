\chapter*{Resumen}

En este manuscrito, presentamos la síntesis de los dos trabajos
previos de Proyecto de Tesis, en el cual nos interesa extender el
método de los volúmenes finitos a las ecuaciones más importantes de la
dinámica de fluidos, cuando estas se expresan como un sistema de
ecuaciones diferenciales parciales regidas sobre un conjunto de
estados arbitrario.
En la primera parte, generalizamos el concepto de una solución
clásica y definimos una convergencia en el sentido distribucional
acerca de cuál solución débil corresponde a la solución físicamente
coherente.
También, presentamos métodos populares del esquema de volúmenes
finitos, la convergencia, la consistencia y la estabilidad son
definidos en el contexto vectorial.
En la segunda parte, mostramos los sistemas hiperbólicos a analizar
y desarrollar, con ayuda del programa \textsc{Clawpack} (Conservation
Laws Package)~\cite{Clawpack2025,Mandli2016} y modelamos el flujo de
un fluido de dos fases en un medio poroso.
Finalmente, se discute la calidad de las soluciones aproximadas con
respecto a otro enfoque popular de semidiscretización como el método
de las líneas.

\noindent\textcolor{DarkBlue}{\bfseries\sffamily Palabras clave}:
Método de volúmenes finitos, sistema hiperbólico de leyes de
conservación, Clawpack.

\chapter*{Abstract}

In this manuscript, we present a synthesis of two previous thesis
projects, in which we are interested in extending the finite-volume
method to the most important equations of fluid dynamics, when these
are expressed as a system of partial differential equations governed
by an arbitrary set of states.
In the first part, we generalize the concept of a classical solution
and define convergence in the distributional sense, determining which
weak solution corresponds to the physically coherent solution.
We also present popular methods of the finite-volume scheme;
convergence, consistency, and stability are defined in the vector
context.
In the second part, we show the hyperbolic systems to be analyzed and
developed using the \textsc{Clawpack} (Conservation Laws Package)
software~\cite{Clawpack2025,Mandli2016}, and we model the flow of a
two-phase fluid in a porous medium.
Finally, the quality of the approximate solutions is discussed with
respect to another popular semi-discretization approach, such as the
method of lines.

\noindent\textcolor{DarkBlue}{\bfseries\sffamily Keywords}:
Finite volume methods, hyperbolic system of conservation laws,
Clawpack.
