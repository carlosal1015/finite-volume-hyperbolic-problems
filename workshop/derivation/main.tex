% arara: clean: {
% arara: --> extensions:
% arara: --> ['aux', 'bbl', 'bcf', 'blg', 'log', 'nav',
% arara: --> 'out', 'pdf', 'run.xml', 'snm', 'toc', 'vrb']
% arara: --> }
% arara: lualatex: {
% arara: --> shell: yes,
% arara: --> draft: yes,
% arara: --> interaction: batchmode
% arara: --> }
% arara: biber
% arara: lualatex: {
% arara: --> shell: yes,
% arara: --> draft: no,
% arara: --> interaction: batchmode
% arara: --> }
% arara: lualatex: {
% arara: --> shell: yes,
% arara: --> draft: no,
% arara: --> interaction: batchmode
% arara: --> }
% arara: clean: {
% arara: --> extensions:
% arara: --> ['aux', 'bbl', 'bcf', 'blg', 'log', 'nav',
% arara: --> 'out', 'run.xml', 'snm', 'toc', 'vrb']
% arara: --> }
\PassOptionsToPackage{svgnames}{xcolor}
\documentclass[
    8pt,
    aspectratio=1610,
    c,
    intlimits,
		handout,
    leqno,
    professionalfonts,
]{beamer}

\usepackage{mathtools}
\usepackage{unicode-math}
\usepackage{diffcoeff}
\usepackage{minted}
\usepackage{newunicodechar}
\usepackage[
	citestyle=numeric,
	style=numeric,
	backend=biber,
]{biblatex}
\addbibresource{beamer.bib}

\addtobeamertemplate{theorem begin}{\normalfont}{}
\usefonttheme[onlymath]{serif}
\setbeamertemplate{navigation symbols}{}
\setbeamercolor{structure}{fg=DarkBlue}
\setbeamertemplate{frametitle}[default][center]
\setbeamertemplate{items}[ball]

\DeclareMathAlphabet{\mathbb}{U}{msb}{m}{n}
\DeclareMathAlphabet{\mathcal}{OMS}{cmsy}{m}{n}

\begin{document}

\begin{frame}
	\begin{block}{El fluido fluye en una boquilla con sección transversal variable}
		El siguiente sistema describe la evolución de un flujo de fluido
		en una boquilla con sección transversal variable:
		\begin{columns}
			\begin{column}{.35\paperwidth}
				\begin{equation}\label{eq:1}
					\begin{aligned}
						\diffp{}{t}
						\left(a\rho\right)+
						\diffp{}{x}
						\left(a\rho u\right)                    & =
						0.                                          \\
						\diffp{}{t}
						\left(a\rho u\right)+
						\diffp{}{x}
						\left(a\left(\rho u^{2}+p\right)\right) & =
						p\diffp{a}{x}                               \\
						\diffp{}{t}
						\left(a\rho e\right)+
						\diffp{}{x}
						\left(au\left(\rho e+p\right)\right)    & =
						0.
					\end{aligned}
				\end{equation}
			\end{column}
			\begin{column}{.35\paperwidth}
				\begin{figure}[ht!]
					\centering
					\includegraphics[width=.35\paperwidth]{noizzy}
					\caption{.}
				\end{figure}
			\end{column}
		\end{columns}
		Donde las notaciones $\rho$, $\varepsilon$, $T$, $S$ y $p$
		representan las variables termodinámicas: densidad, energía
		interna, temperatura absoluta, entropía y presión,
		respectivamente; $u$ es la velocidad y
		$e=\varepsilon+\frac{u^{2}}{2}$ es la energía total.
		La función $a=a\left(x\right)>0$, $x\in\mathbb{R}$ es el área de
		la sección transversal.
	\end{block}
\end{frame}

\begin{frame}
	\begin{block}{Hiperbolicidad e hiperbolicidad no estricta}
		Para investigar las propiedades básicas del sistema~\eqref{eq:1},
		podemos complementarlo con la ecuación trivial $\diffp{a}{t}=0$,
		para obtener el siguiente sistema de leyes equilibradas
		\begin{equation}\label{eq:2}
			\begin{aligned}
				\diffp{}{t}
				\left(a\rho\right)+
				\diffp{}{x}
				\left(a\rho u\right)                 & =
				0.                                       \\
				\diffp{}{t}
				\left(a\rho u\right)+
				\diffp{}{x}
				\left(a\left(\rho u^{2}+p\right)\right)
				                                     & =
				p\diffp{a}{x}.                           \\
				\diffp{}{t}
				\left(a\rho e\right)+
				\diffp{}{x}
				\left(au\left(\rho e+p\right)\right) & =
				0.                                       \\
				\diffp{a}{t}                         & =
				0.
			\end{aligned}
		\end{equation}
		Consideremos el par de variables termodinámicas independientes
		$\left(\rho, S\right)$.
		Las ecuaciones de estado tendrán entonces la forma
		\begin{equation*}
			p=
			p\left(\rho,S\right),\quad
			\varepsilon=
			\varepsilon\left(\rho, S\right),\quad
			T=
			T\left(\rho,S\right).
		\end{equation*}
		Por lo tanto, el sistema~\eqref{eq:2} puede expandirse para la
		función desconocida $U=\left(\rho,u,S,a\right)$ de la siguiente
		manera.
		Considérese la primera ecuación de~\eqref{eq:2}, que puede
		reescribirse como
		\begin{equation*}
			\rho\diffp{a}{t}+
			a\diffp{\rho}{t}+
			a\diffp{}{x}\left(\rho u\right)+
			\diffp{a}{x}\rho u=
			0.
		\end{equation*}
		Usando la tercera ecuación en~\eqref{eq:2}, dividiendo la última
		ecuación por $a$, obtenemos
		\begin{equation}\label{eq:3}
			\diffp{\rho}{t}+
			u\diffp{\rho}{x}+
			\rho\diffp{u}{x}+
			\frac{\rho u}{a}\diffp{a}{x}=
			0.
		\end{equation}
	\end{block}
\end{frame}

\begin{frame}
	La segunda ecuación en~\eqref{eq:2} se puede expandir mediante
	\begin{equation*}
		u\diffp{}{t}\left(a\rho\right)+
		a\rho\diffp{u}{t}+
		a\rho u\diffp{u}{x}+
		u\diffp{}{x}\left(a\rho u\right)+
		p\diffp{a}{x}+
		a\diffp{p}{x}=
		p\diffp{a}{x}.
	\end{equation*}
	Simplificando y luego reordenando los términos de la última
	ecuación obtenemos
	\begin{equation*}
		u\left[
			\diffp{}{t}\left(a\rho\right)+
			\diffp{}{x}\left(a\rho u\right)
			\right]+
		a\rho\left[
			\diffp{u}{t}+
			u\diffp{u}{x}+
			\frac{1}{\rho}
			\left(\diffp{p}{\rho}+\diffp{p}{S}\diffp{S}{x}\right)
			\right]=
		0.
	\end{equation*}
	Usando la primera ecuación en~\eqref{eq:2}, podemos ver que el
	primer término de la última ecuación se anula.
	Descartando el primer término de la última ecuación y luego
	dividiéndolo entre $a\rho$, obtenemos
	\begin{equation}\label{eq:4}
		\diffp{u}{t}+
		\frac{1}{\rho}\diffp{p}{\rho}\diffp{\rho}{x}+
		u\diffp{u}{x}+
		\frac{1}{p}
		\diffp{p}{S}
		\diffp{S}{x}=0.
	\end{equation}
	A continuación, la tercera ecuación en~\eqref{eq:2} se puede
	escribir como
	\begin{equation*}
		a\rho\diffp{e}{t}+
		e\diffp{}{t}\left(a\rho\right)+
		a\rho u\diffp{e}{x}+
		e\diffp{}{x}\left(a\rho u\right)+
		\diffp{}{x}\left(aup\right)=
		0.
	\end{equation*}
	El primer término de la última ecuación se anula, gracias a la
	primera ecuación en~\eqref{eq:2}.
	Dado que
	\begin{equation*}
		e=\varepsilon+\frac{u^{2}}{2},
	\end{equation*}
	la última ecuación se convierte en
	\begin{equation*}
		a\rho\left(\diffp{\varepsilon}{t}+u\diffp{\varepsilon}{x}\right)+
		a\rho u\left(\diffp{u}{t}+u\diffp{u}{x}\right)+
		\diffp{}{x}\left(aup\right)=0.
	\end{equation*}
\end{frame}

\begin{frame}
	De~\eqref{eq:4}, encontramos
	\begin{equation*}
		\diffp{u}{t}+
		u\diffp{u}{x}=
		-\frac{1}{p}
		\diffp{p}{x}.
	\end{equation*}
	De las dos últimas ecuaciones, obtenemos
	\begin{math}
		a\rho\left(\diffp{\varepsilon}{t}+u\diffp{\varepsilon}{x}\right)-
		au\diffp{p}{x}+
		\diffp{}{x}\left(aup\right)=
		0
	\end{math}.
	Utilizando la identidad termodinámica
	\begin{equation*}
		\dl\varepsilon=
		T\dl S-p\dl v,\quad
		v=
		\frac{1}{\rho}
	\end{equation*}
	de modo que
	\begin{align*}
		\diffp{\varepsilon}{t} & =
		T\diffp{S}{t}-p\diffp{v}{t}. \\
		\diffp{\varepsilon}{x} & =
		T\diffp{S}{x}-p\diffp{v}{x}.
	\end{align*}
	Obtenemos de la última ecuación
	\begin{equation*}
		a\rho T\left(\diffp{S}{t}+u\diffp{S}{x}\right)+
		a\rho\frac{p}{\rho^{2}}
		\left(\diffp{\rho}{t}+u\diffp{\rho}{x}\right)+
		p\diffp{au}{x}=0,
	\end{equation*}
	Por otra parte, de~\eqref{eq:3} se sigue que
	\begin{equation*}
		\diffp{\rho}{t}+
		u\diffp{\rho}{x}=
		-\left[\rho\diffp{u}{x}+\frac{\rho u}{a}\diffp{a}{x}\right],
	\end{equation*}
	que, al sustituir en la última ecuación, da
	\begin{equation*}
		a\rho T
		\left(\diffp{S}{t}+u\diffp{S}{x}\right)-
		ap
		\left(\diffp{u}{x}+\frac{u}{a}\diffp{a}{x}\right)+
		p\left(\diffp{a}{x}u+u\diffp{a}{x}\right)=
		0.
	\end{equation*}
	Es decir, tenemos
	\begin{equation*}
		a\rho T\left(\diffp{S}{t}+u\diffp{S}{x}\right)-
		ap\left(\diffp{u}{x}+\frac{u}{a}\diffp{a}{x}\right)+
		p\left(\diffp{a}{x}u+u\diffp{a}{x}\right)=
		0.
	\end{equation*}
\end{frame}

\begin{frame}
	Simplificando la última ecuación obtenemos
	\begin{equation*}
		a\rho T
		\left(\diffp{S}{t}+u\diffp{S}{x}\right)=
		0
	\end{equation*}
	Dividiendo ambos lados de la última ecuación por $a\rho T$,
	obtenemos $\diffp{S}{t}+u\diffp{S}{x}=0$.
	Así, el sistema~\eqref{eq:2} puede escribirse en forma matricial
	mediante
	\begin{equation*}
		\diffp{U}{t}+
		A\left(U\right)\diffp{U}{x}=
		0,
	\end{equation*}
	donde
	\begin{equation}\label{eq:6}
		U=\begin{bmatrix}
			\rho \\
			u    \\
			S    \\
			a
		\end{bmatrix}\qquad
		A\left(U\right)=
		\begin{bmatrix}
			u                                                & \rho & 0                                             & \frac{u\rho}{a} \\
			\frac{1}{\rho}\diffp{p}{\rho}\left(\rho,S\right) & u    & \frac{1}{\rho}\diffp{p}{S}\left(\rho,S\right) & 0               \\
			0                                                & 0    & u                                             & 0               \\
			0                                                & 0    & 0                                             & 0
		\end{bmatrix}.
	\end{equation}
	La ecuación característica asociada a la matriz $A\left(U\right)$
	en~\eqref{eq:6} está dada por
	\begin{equation*}
		\lambda\left(\lambda-u\right)
		\left(\left(u-\lambda\right)-\diffp{p}{\rho}\left(\rho,S\right)\right)=
		0.
	\end{equation*}
	Siempre que $\diffp{p}{\rho}\left(\rho,S\right)>0$
	la matriz $A\left(U\right)$ admite cuatro valores propios reales,
	\begin{equation*}
		\lambda_{1}\left(U\right)=u-c,\quad
		\lambda_{2}\left(U\right)=u,\quad
		\lambda_{3}\left(U\right)=u+c,\quad
		\lambda_{4}\left(U\right)=0,
	\end{equation*}
	donde $c$ es la rapidez del sonido definido por
	$c\coloneqq\sqrt{\diffp{p}{\rho}\left(\rho, S\right)}$.
\end{frame}

\begin{frame}
	\begin{align}
		\diffp{\theta}{t}+
		a_{1}
		\diffp{
			\big(
			\theta
			S_{\text{o}}
			\big)
		}{x}                    & =
		b_{1}
		S_{\text{o}}
		S_{y}
		\Phi-
		\beta\theta.
		\label{eq:6}                \\
		\diffp{S_{y}}{t}+
		a_{2}
		\diffp{S_{y}}{x}        & =
		-b_{2}
		S_{\text{o}}
		S_{y}
		\Phi.
		\label{eq:7}                \\
		\diffp{S_{\text{o}}}{t}+
		a_{3}
		\diffp{S_{\text{o}}}{x} & =
		-b_{3}
		S_{\text{o}}
		S_{y}
		\Phi.
		\label{eq:8}
	\end{align}
\end{frame}
\end{document}
