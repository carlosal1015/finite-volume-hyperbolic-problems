\chapter{Introducción}
\section{Acerca de esta tesis}

El ánimo de esta tesis es presentar resultados que mejoran y amplían
la aplicabilidad del método de volúmenes finitos de diversas maneras.

% Aproximadamente el primer tercio de esta tesis (capítulos 2 a 4) se
% centra en los aspectos de implementación de la DG.
% El capítulo 3 describe cómo implementar el método combinando los
% objetivos, a menudo opuestos,
% de utilizar prácticas de ingeniería de software consolidadas y
% lograr un alto rendimiento.
% El capítulo 5 explica cómo se puede aumentar el rendimiento
% computacional en un orden de magnitud o más mediante el uso de
% hardware computacional masivamente paralelo para el mercado de masas.
% Para respaldar este último avance, se presentan algunas herramientas
% de programación paralela creadas específicamente para ello, pero que
% han encontrado un uso mucho más amplio en la comunidad científica
% (capítulo 4).

Los matemáticos aplicados realizan experimentos computacionales a
gran escala a diario.
Lamentablemente, la mayoría de estos experimentos son improvisados y
se mantienen registros deficientes (o nulos) al respecto.
Parte de este problema radica en que los físicos y químicos reciben
formación rutinaria en la tarea de realizar experimentos exhaustivos,
bien documentados y reproducibles, mientras que los matemáticos
aplicados no.

Me gustaría sugerir las siguientes pautas que he intentado seguir en
mi trabajo en esta tesis:
\begin{description}
	\item[\color{DarkBlue}Disponibilidad en formato de código fuente]

	\item[\color{DarkBlue}Legibilidad]

	\item[\color{DarkBlue}Dependencias reducidas del software de pago]
\end{description}

\section{El método científico y el experimento computacional}

En la segunda parte de esta tesis, se adquieren nuevos conocimientos
mediante el examen del comportamiento de métodos recién introducidos
en experimentos computacionales seleccionados a propósito.
En matemáticas aplicadas, estos experimentos suelen ser criticados
por su escasa especificación, su dificultad o imposibilidad de
reproducirse y, por estas razones, su valor cuestionable.

A diferencia de (por ejemplo) la física o la química, las matemáticas
aplicadas no cuentan con una cultura madura de experimentación, y
mucho menos con una rama experimental plenamente aceptada.
Antes de la llegada de las computadoras, los experimentos matemáticos
solían ser tediosos y poco prácticos, por lo que la cultura
matemática se ha desarrollado principalmente en torno a resultados
teóricos.
Pero ahora que existe la posibilidad, en mi opinión, el campo haría
bien en imitar a las demás ciencias y adoptar la experimentación como
uno de sus métodos aceptados.
Debería hacerlo con el fin de captar conocimiento que de otro modo no
estaría disponible, pero también con cautela.

\section{Reproducibilidad de los resultados de esta tesis}

Para garantizar la reproducibilidad de mis resultados, utilizaré esta
sección para especificar tanto el entorno computacional en el que he
realizado mis experimentos como las versiones precisas del software
que he utilizado.
En primer lugar, la Tabla 1.1 ofrece un resumen completo de los
componentes de terceros que realizaron las operaciones que condujeron
a los resultados que se muestran en este trabajo.
Si se indican varios números de versión de un componente, no se
observaron cambios significativos en los resultados en todas las
versiones especificadas.
Además, como ya he indicado, todo mi código está disponible
gratuitamente.
Puede descargarse de mis repositorios de control de versiones en la
dirección
\begin{center}
	\url{https://github.com/carlosal1015/finite-volume-methods}
\end{center}
