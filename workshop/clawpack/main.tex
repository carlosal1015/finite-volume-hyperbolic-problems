% arara: clean: {
% arara: --> extensions:
% arara: --> ['aux', 'bbl', 'bcf', 'blg', 'log', 'nav',
% arara: --> 'out', 'pdf', 'run.xml', 'snm', 'toc', 'vrb']
% arara: --> }
% arara: lualatex: {
% arara: --> shell: yes,
% arara: --> draft: yes,
% arara: --> interaction: batchmode
% arara: --> }
% arara: biber
% arara: lualatex: {
% arara: --> shell: yes,
% arara: --> draft: no,
% arara: --> interaction: batchmode
% arara: --> }
% arara: lualatex: {
% arara: --> shell: yes,
% arara: --> draft: no,
% arara: --> interaction: batchmode
% arara: --> }
% arara: clean: {
% arara: --> extensions:
% arara: --> ['aux', 'bbl', 'bcf', 'blg', 'log', 'nav',
% arara: --> 'out', 'run.xml', 'snm', 'toc', 'vrb']
% arara: --> }
\documentclass{scrletter}
\usepackage{graphicx}
\usepackage[shortlabels]{enumitem}
\usepackage{mathtools}
% \usepackage{plantuml}
\usepackage{hyperref}

\usepackage[
	citestyle=numeric,
	style=numeric,
	backend=biber,
]{biblatex}

\addbibresource{references.bib}

\newcommand{\MVAt}{{\usefont{U}{mvs}{m}{n}\symbol{`@}}}
\renewcommand{\thesection}{\Roman{section}}
\renewcommand{\thesubsection}{\thesection.\Roman{subsection}}


\begin{document}

\frame{
  \maketitle
}

\begin{frame}
  \begin{columns}
    \begin{column}{.6\paperwidth}
      \begin{block}{¿Qué es Clawpack?}
        \begin{quoting}
          Clawpack significa ``Paquete de Leyes de Conservación'' y se
          desarrolló inicialmente para sistemas hiperbólicos lineales y no
          lineales de leyes de conservación, con el objetivo de implementar
          métodos de tipo Godunov de alta resolución mediante limitadores
          en un marco general aplicable a diversas aplicaciones.
          Estos métodos de volumen finito requieren un ``solucionador de
          Riemann'' para resolver la discontinuidad de salto en la interfaz
          entre dos celdas de la malla en ondas que se propagan a las
          celdas vecinas.
          La formulación de Clawpack permite una fácil extensión a la
          solución de problemas hiperbólicos que no están en forma de
          conservación.
        \end{quoting}
      \end{block}
    \end{column}
    \begin{column}{.3\paperwidth}
      \begin{figure}[ht!]
        \centering
        \includegraphics[width=.3\paperwidth]{ClawpackLogoTitleFrame}
        \includegraphics[width=.3\paperwidth]{GeoClawLogoTitleFrame}
      \end{figure}
    \end{column}
  \end{columns}
\end{frame}

\begin{frame}
  \begin{block}{Partes de Clawpack}
    \begin{description}
      \item[ARMClaw]

        Refinamiento de malla adaptativo.

      \item[Classic]

        Rutinas clásicas de Fortran de una sola malla.

      \item[GeoClaw]

        Flujos geofísicos.

        \url{http://www.geoclaw.org}

      \item[PyClaw]

        Versión Python de los solucionadores de EDPs hiperbólicas
        que permite resolver el problema en Python sin utilizar
        explícitamente ningún código Fortran.

        \url{https://www.clawpack.org/gallery/pyclaw/gallery/gallery_all.html}

      \item[Riemann]

        El solucionador de Riemann define la ecuación hiperbólica
        que se está resolviendo y realiza la mayor parte del trabajo
        computacional: se llama en cada interfaz de celda en cada
        paso de tiempo y devuelve la información sobre las ondas
        y las velocidades que se necesita para actualizar la
        solución.

      \item[VisClaw]

        Utilidades para graficar usando Python.
    \end{description}
  \end{block}
\end{frame}

\begin{frame}[fragile]
  \frametitle{Classic}
  \inputminted[fontsize=\tiny,firstline=5,lastline=27]{bash}{script.sh}
\end{frame}

\begin{frame}[fragile]
  \frametitle{ARMClaw}
  \inputminted[fontsize=\tiny,firstline=29,lastline=51]{bash}{script.sh}
\end{frame}

\section{Ejemplos del manual~\cite{Ancey2025}}

\begin{frame}
  \inputminted[fontsize=\tiny,firstline=1,lastline=27]{python}{chapter1.py}
\end{frame}

\begin{frame}
  \frametitle{Referencias}

  \nocite{*}
  \printbibliography[heading=none]
\end{frame}
\end{document}
