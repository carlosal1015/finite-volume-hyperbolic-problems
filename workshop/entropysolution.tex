\begin{example}[]
	Sea.
	\begin{equation*}
		\left\{
		\begin{aligned}
			\diffp{u}{t}+
			\diffp{}{x}
			\frac{u^{2}}{2} & =
			0.                  \\
			u_{0}\left(x\right)=
			\begin{cases}
				u_{l}, & x<0. \\
				u_{r}, & x>0.
			\end{cases}
		\end{aligned}
		\right.
	\end{equation*}
\end{example}

\begin{equation*}
	\diffp{}{t}
	U\left(\symbf{u}\right)+
	\diffp{}{x}
	F\left(\symbf{u}\right)=
	0.
\end{equation*}


\section{Soluciones de entropía}

\begin{definition}
	Sea $\Omega\subset\mathbb{R}^{p}$ un conjunto convexo.
	$U\in\symbf{C}^{1}\left(\Omega,\mathbb{R}\right)$ es una
	\textcolor{DarkBlue}{\bfseries entropía}\index{entropía}
	de~\eqref{eq:systemofconservationlaw} si y solo si
	$\exists F\in\symbf{C}^{1}\left(\Omega,\mathbb{R}\right)$ tal que
	\begin{equation*}
		{\begin{bmatrix}
				\diffp{U}{u_{k}}
			\end{bmatrix}}_{1\leq k\leq p}
		{\begin{bmatrix}
				\diffp{f_{i}}{u_{k}}
			\end{bmatrix}}_{\substack{1\leq i\leq p\\1\leq k\leq p}}
		\left(\symbf{u}\right)=
		{\begin{bmatrix}
			\diffp{F}{u_{k}}
		\end{bmatrix}}_{1\leq k\leq p}.
	\end{equation*}
	% \begin{equation*}
	%   \diffp{}{t}
	%   U\left(\symbf{u}\right)+
	%   \sum_{j=1}^{d}
	%   \diffp{}{x_{j}}
	%   F_{j}\left(\symbf{u}\right)=0.
	% \end{equation*}
\end{definition}

\begin{example}[$p$-sistema]
	.
\end{example}

\section{Método de viscosidad}

Dado $\varepsilon>0$, asociamos
con~\eqref{eq:systemofconservationlaw} el sistema parabólico
\begin{equation}\label{eq:parabolicsystemofconservationlaw}
	\diffp{\symbf{u}_{\varepsilon}}{t}+
	\diffp{}{x}
	\symbf{f}
	\left(\symbf{u}_{\varepsilon}\right)-
	\varepsilon
	\Delta\symbf{u}_{\varepsilon}=
	\symbf{0},
\end{equation}
donde $-\varepsilon\Delta\symbf{u}_{\varepsilon}$ puede ser visto
como un término de viscosidad.

\begin{theorem}
	Sea
	\begin{math}
		\left\{\symbf{u}_{\varepsilon}\right\}_{\varepsilon}
	\end{math}
	una sucesión de soluciones suaves
	de~\eqref{eq:parabolicsystemofconservationlaw}.
	Si~\eqref{eq:systemofconservationlaw} admite una entropía
	$U$ con flujo entropía $F$, $\exists C>0$ tal que
	\begin{math}
		{
			\left\|\symbf{u}_{\varepsilon}\right\|
		}_{
			{\symbf{L}^{\infty}\left(\mathbb{R}\times\left[0,\infty\right[\right)}^{p}
		}\leq C
	\end{math}
	y $\lim\limits_{\varepsilon\to0}\symbf{u}_{\varepsilon}=\symbf{u}$
	casi en todas partes en $\mathbb{R}\times\left[0,\infty\right[$.
	Entonces, $\symbf{u}$ es una solución débil
	de~\eqref{eq:systemofconservationlaw} y
	\begin{equation*}
		\forall\varphi\in
		\symbf{C}^{\infty}_{0}
		\left(\mathbb{R}\times\left]0,\infty\right[\right),
		\varphi\geq 0:
		\int_{0}^{\infty}
		\int_{-\infty}^{\infty}
		\left[
			U\left(\symbf{u}\right)
			\diffp{\varphi}{t}+
			F\left(\symbf{u}\right)
			\diffp{\varphi}{x}
			\right]
		\dl x\dl t\geq
		0.
	\end{equation*}
	% \begin{equation*}
	% 	\diffp{}{t}
	% 	U\left(\symbf{u}\right)+
	% 	\diffp{}{x}
	% 	F\left(\symbf{u}\right)\leq
	% 	0
	% \end{equation*}
\end{theorem}

\begin{definition}
	$\symbf{u}$ es una
	\textcolor{DarkBlue}{\bfseries solución de entropía}
	\index{solución de entropía} si y solo si
	\begin{equation*}
		\forall U:
		\forall\varphi\in
		\symbf{C}^{1}_{0}
		\left(\mathbb{R}\times\left[0,\infty\right[\right),
		\varphi\geq 0:
		\int_{0}^{\infty}
		\int_{-\infty}^{\infty}
		\left[
			U\left(\symbf{u}\right)
			\diffp{\varphi}{t}+
			F\left(\symbf{u}\right)
			\diffp{\varphi}{x}
			\right]
		\dl x\dl t+
		\int_{-\infty}^{\infty}
		U\left(\symbf{u}_{0}\left(x\right)\right)
		\varphi\left(x,0\right)\dl x\geq
		0.
	\end{equation*}
\end{definition}

\begin{equation*}
	\left[F\left(\symbf{u}\right)\right]\leq
	s\left[U\left(\symbf{u}\right)\right].
\end{equation*}

\begin{theorem}
	Si $u_{0}\in\symbf{L}^{\infty}\left(\mathbb{R}\right)$, entonces
	\begin{equation*}
		{\left\|
			u\left(\cdot, t\right)
			\right\|}_{\symbf{L}^{\infty}\left(\mathbb{R}\right)}
		\leq
		{\left\|u_{0}\right\|}_{\symbf{L}^{\infty}\left(\mathbb{R}\right)}.
	\end{equation*}
	Si
	\begin{math}
		u_{0}\in
		\symbf{L}^{\infty}
		\left(\mathbb{R}\right)\cap
		\symbf{BV}\left(\mathbb{R}\right)
	\end{math},
	entonces
	\begin{math}
		u\left(\cdot, t\right)\in
		\symbf{BV}\left(\mathbb{R}\right)
	\end{math}
	con
	\begin{equation*}
		\operatorname{TV}
		\left(u\left(\cdot, t\right)\right)\leq
		\operatorname{TV}
		\left(u_{0}\right).
	\end{equation*}
\end{theorem}

\begin{equation*}
	\diffp{\symbf{u}}{t}+
	\symbf{A}
	\diffp{\symbf{u}}{t}=
	\symbf{0}.
\end{equation*}

$\symbf{u}\in\mathbb{R}^{p}$, $\symbf{A}\in\mathbb{R}^{p\times p}$

\begin{equation*}
	\diffp{\symbf{u}}{t}+
	\symbf{A}
	\diffp{\symbf{u}}{x}=
	\sum_{k=1}^{p}
	\left(
	\diffp{\alpha_{k}}{t}+
	\lambda_{k}
	\diffp{\alpha_{k}}{x}
	\right)
	\symbf{r}_{k}
\end{equation*}

\begin{equation*}
	\symbf{T}^{-1}
	\diffp{\symbf{u}}{t}+
	\symbf{T}^{-1}
	\symbf{A}
	\diffp{\symbf{u}}{x}=
	\diffp{\symbf{v}}{t}+
	\symbf{\Lambda}\diffp{\symbf{v}}{x}=
	\symbf{0}.
\end{equation*}

\begin{equation*}
	\alpha_{k0}\left(x\right)=
	\symbf{l}^{T}_{k}
	\symbf{u}_{0}\left(x\right)
\end{equation*}

\begin{equation*}
	\alpha_{k}\left(x,t\right)=
	\alpha_{k0}\left(x-\lambda_{k}t\right)=
	\symbf{l}^{T}_{k}\symbf{u}_{0}\left(x-\lambda_{k}t\right),
\end{equation*}

\begin{equation*}
	\symbf{u}\left(x,t\right)=
	\sum_{k=1}^{p}
	\symbf{l}^{T}_{k}
	\symbf{u}_{0}\left(x-\lambda_{k}t\right)
	\symbf{r}_{k}.
\end{equation*}

\begin{equation*}
	\symbf{w}_{R}\left(\frac{x}{t};\symbf{u}_{L},\symbf{u}_{R}\right)=
	\begin{cases}
		\symbf{w}_{0}=\symbf{u}_{L}, & \frac{x}{t}<\lambda_{1}.               \\
		\symbf{w}_{1},               & \lambda_{1}<\frac{x}{t}<\lambda_{2}.   \\
		\vdots,                      &                                        \\
		\symbf{w}_{p-1},             & \lambda_{p-1}<\frac{x}{t}<\lambda_{p}. \\
		\symbf{w}_{p}=\symbf{u}_{R}, & \frac{x}{t}>\lambda_{p}.               \\
	\end{cases}
\end{equation*}

\begin{equation*}
	\symbf{A}\left(\symbf{w}_{m}-\symbf{w}_{m-1}\right)=
	\lambda_{m}\left(\symbf{w}_{m}-\symbf{w}_{m-1}\right).
\end{equation*}

\subsection{Ondas de choques y discontinuidades de contacto}

\begin{equation*}
	\sigma\left[\symbf{u}\right]=
	\left[\symbf{f}\left(\symbf{u}\right)\right].
\end{equation*}

\begin{equation*}
	\symbf{u}\left(x,t\right)=
	\begin{cases}
		\symbf{u}_{L}, & x<\sigma t. \\
		\symbf{u}_{R}, & x>\sigma t.
	\end{cases}
\end{equation*}

\begin{equation*}
	\symbf{f}\left(\symbf{u}_{R}\right)-
	\symbf{f}\left(\symbf{u}_{L}\right)=
	\sigma
	\left(\symbf{u}_{R}-\symbf{u}_{L}\right).
\end{equation*}

\begin{definition}
	El conjunto de Rankine-Hugoniot de $\symbf{u}_{0}$ es
	\begin{equation*}
		\operatorname{R-K}\left(\symbf{u}_{0}\right)\coloneqq
		\left\{
		\symbf{u}\in\Omega\mid\exists\sigma\left(\symbf{u}_{0},\symbf{u}\right)\in\mathbb{R}\text{ tal que }
		\sigma\left(\symbf{u}_{0},\symbf{u}\right)\left(\symbf{u}-\symbf{u}_{0}\right)=
		\symbf{f}\left(\symbf{u}\right)-\symbf{f}\left(\symbf{u}_{0}\right)
		\right\}.
	\end{equation*}
\end{definition}

\begin{definition}
	$\exists k\in\left\{1,\dotsc,p\right\}$ tal que
	\begin{equation*}
		\begin{cases}
			\lambda_{k}\left(\symbf{u}_{+}\right)<\sigma<\lambda_{k+1}\left(\symbf{u}_{+}\right),
			\lambda_{k-1}\left(\symbf{u}_{-}\right)<\sigma<\lambda_{k}\left(\symbf{u}_{-}\right).
		\end{cases}
	\end{equation*}

	\begin{equation*}
		\lambda_{k}\left(\symbf{u}_{-}\right)=\sigma=\lambda_{k}\left(\symbf{u}_{+}\right).
	\end{equation*}
\end{definition}
