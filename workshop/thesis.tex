% arara: clean: {
% arara: --> extensions:
% arara: --> ['aux', 'bbl', 'bcf', 'blg', 'log', 'out', 'pdf', 'run.xml', 'toc']
% arara: --> }
% arara: lualatex: {
% arara: --> shell: yes,
% arara: --> draft: yes,
% arara: --> interaction: batchmode
% arara: --> }
% arara: biber
% arara: lualatex: {
% arara: --> shell: yes,
% arara: --> draft: no,
% arara: --> interaction: batchmode
% arara: --> }
% arara: lualatex: {
% arara: --> shell: yes,
% arara: --> draft: no,
% arara: --> interaction: batchmode
% arara: --> }
% arara: clean: {
% arara: --> extensions:
% arara: --> ['aux', 'bbl', 'bcf', 'blg', 'log', 'out', 'run.xml', 'toc']
% arara: --> }
\documentclass{scrletter}
\usepackage{graphicx}
\usepackage[shortlabels]{enumitem}
\usepackage{mathtools}
% \usepackage{plantuml}
\usepackage{hyperref}

\usepackage[
	citestyle=numeric,
	style=numeric,
	backend=biber,
]{biblatex}

\addbibresource{references.bib}

\newcommand{\MVAt}{{\usefont{U}{mvs}{m}{n}\symbol{`@}}}
\renewcommand{\thesection}{\Roman{section}}
\renewcommand{\thesubsection}{\thesection.\Roman{subsection}}


\begin{document}

\maketitle

\chapter{Agradecimientos}

Con la culminación de este trabajo, deseo expresar mi más sincero
agradecimiento a todas las personas que hicieron posible su
realización.
En primer lugar, a mi supervisor, Fidel Jara Huanca, profesor de la
Facultad de Ciencias de la Universidad Nacional de Ingeniería, por la
oportunidad de integrarme a este Taller de Investigación, vinculado a
su proyecto doctoral.
Le estoy profundamente agradecido por su invaluable apoyo y guía
durante los últimos dos años, por sus acertados consejos, su
motivación constante y, especialmente, por sus exhaustivas revisiones
y sugerencias, que enriquecieron enormemente este manuscrito.

Asimismo, extiendo mi agradecimiento al matemático Jose Alcalde, del
Instituto de Matemática y Ciencias Afines, por su valiosa
colaboración y el fructífero trabajo en conjunto con el profesor Jara
y mi persona.
De manera muy especial, quiero agradecer a la estudiante de
matemática Karen Lizeth Silvera de la Universidad Nacional de San
Antonio Abad del Cusco, mi compañera en este camino.
A ella le doy las gracias por su apoyo incondicional, su paciencia,
su aliento y por ser un pilar fundamental en cada etapa de este
proceso.

Por último, pero no menos importante, agradezco a mis padres,
José Aznarán y Miriam Laos, quienes, durante toda mi vida, me han
brindado su apoyo, sus consejos y su amor incondicional.

\chapter*{Resumen}

En este manuscrito, presentamos la síntesis de los dos trabajos
previos de Proyecto de Tesis, en el cual nos interesa extender el
método de los volúmenes finitos a las ecuaciones más importantes de la
dinámica de fluidos, cuando estas se expresan como un sistema de
ecuaciones diferenciales parciales regidas sobre un conjunto de
estados arbitrario.
En la primera parte, generalizamos el concepto de una solución
clásica y definimos una convergencia en el sentido distribucional
acerca de cuál solución débil corresponde a la solución físicamente
coherente.
También, presentamos métodos populares del esquema de volúmenes
finitos, la convergencia, la consistencia y la estabilidad son
definidos en el contexto vectorial.
En la segunda parte, mostramos los sistemas hiperbólicos a analizar
y desarrollar, con ayuda del programa \textsc{Clawpack} (Conservation
Laws Package)~\cite{Clawpack2025,Mandli2016} y modelamos el flujo de
un fluido de dos fases en un medio poroso.
Finalmente, se discute la calidad de las soluciones aproximadas con
respecto a otro enfoque popular de semidiscretización como el método
de las líneas.

\noindent\textcolor{DarkBlue}{\bfseries\sffamily Palabras clave}:
Método de volúmenes finitos, sistema hiperbólico de leyes de
conservación, Clawpack.

\chapter*{Abstract}

In this manuscript, we present a synthesis of two previous thesis
projects, in which we are interested in extending the finite-volume
method to the most important equations of fluid dynamics, when these
are expressed as a system of partial differential equations governed
by an arbitrary set of states.
In the first part, we generalize the concept of a classical solution
and define convergence in the distributional sense, determining which
weak solution corresponds to the physically coherent solution.
We also present popular methods of the finite-volume scheme;
convergence, consistency, and stability are defined in the vector
context.
In the second part, we show the hyperbolic systems to be analyzed and
developed using the \textsc{Clawpack} (Conservation Laws Package)
software~\cite{Clawpack2025,Mandli2016}, and we model the flow of a
two-phase fluid in a porous medium.
Finally, the quality of the approximate solutions is discussed with
respect to another popular semi-discretization approach, such as the
method of lines.

\noindent\textcolor{DarkBlue}{\bfseries\sffamily Keywords}:
Finite volume methods, hyperbolic system of conservation laws,
Clawpack.

\tableofcontents

\part{Teoría}

\chapter{Sistema de leyes de conservación}

Presentamos un sistema de leyes de conservación en una dimensión
espacial, así como los ejemplos más destacados en la física de los
medios continuos.

\begin{definition}
	Sea $\Omega\subset\mathbb{R}^{p}$ un conjunto abierto.
	Un \textcolor{DarkBlue}{\bfseries sistema de leyes de conservación}
	\index{sistema de leyes de conservación} es
	\begin{equation}\label{eq:systemofconservationlaw}
		\diffp{\symbf{u}}{t}+
		\diffp{}{x}
		\symbf{f}\left(\symbf{u}\right)=
		\symbf{s}\left(\symbf{u},x\right).
	\end{equation}
	Donde $\Omega$ es el conjunto de estados,
	\begin{math}
		\symbf{f}\in\symbf{C}^{1}
		\left(\Omega,\mathbb{R}^{p}\right)
	\end{math}
	es la función flujo,
	\begin{math}
		\symbf{s}\in
		\symbf{C}^{1}
		\left(\Omega\times\mathbb{R},\mathbb{R}^{p}\right)
	\end{math}
	es la función fuente sin términos de derivadas de $\symbf{u}$ y
	\begin{math}
		\symbf{u}\in
		\symbf{C}^{1}
		\left(\mathbb{R}\times\left[0,\infty\right[,\Omega\right)
	\end{math} es la solución de~\eqref{eq:systemofconservationlaw}.
	\begin{align*}
		\symbf{f}\colon\Omega                                &
		\longrightarrow\mathbb{R}^{p}                        &
		\symbf{s}\colon\Omega\times\mathbb{R}                &
		\longrightarrow\mathbb{R}^{p}                        &
		\symbf{u}\colon\mathbb{R}\times\left[0,\infty\right[ &
		\longrightarrow\Omega                                  \\
		\begin{bmatrix}
			u_{1}  \\
			\vdots \\
			u_{p}
		\end{bmatrix}                                      &
		\longmapsto
		\begin{bmatrix}
			f_{1}  \\
			\vdots \\
			f_{p}
		\end{bmatrix},                                      &
		\left(\symbf{u},x\right)                             &
		\longmapsto
		\begin{bmatrix}
			s_{1}\left(\symbf{u},x\right) \\
			\vdots                        \\
			s_{p}\left(\symbf{u},x\right)
		\end{bmatrix},                     &
		\left(x,t\right)                                     &
		\longmapsto
		\begin{bmatrix}
			u_{1}\left(x,t\right) \\
			\vdots                \\
			u_{p}\left(x,t\right)
		\end{bmatrix}=
		\begin{bmatrix}
			u_{1}  \\
			\vdots \\
			u_{p}
		\end{bmatrix}.
	\end{align*}
\end{definition}
Si $I\subset\mathbb{R}$, entonces
de~\eqref{eq:systemofconservationlaw} se obtiene la ecuación de
balance que expresa que la variación en el tiempo de la cantidad
total en el medio es igual al flujo neto a través de la interface
más la contribución del término fuente.
\begin{equation*}
	\diff{}{t}
	\int_{I}\symbf{u}\dl x+
	{\symbf{f}\left(\symbf{u}\right)\Big|}_{\partial I}=
	\int_{I}\symbf{s}\left(\symbf{u},x\right)\dl x.
\end{equation*}

\begin{definition}
	Un sistema~\eqref{eq:systemofconservationlaw} es
	\textcolor{DarkBlue}{\bfseries hiperbólico}\index{hiperbólico} si y
	solo si
	\begin{equation}
		\forall\symbf{u}\in\Omega\!:
		\forall\omega\in\mathbb{R}\setminus\left\{0\right\}\!:
		\exists
		{\left\{
			\left(\lambda_{k},\symbf{r}_{k}\right)
			\right\}}^{p}_{k=1}\subset
		\mathbb{R}\times\mathbb{R}^{p}
		\text{ tal que }
		\symbf{A}\left(\symbf{u},\omega\right)
		\symbf{r}_{k}=
		\lambda_{k}
		\symbf{r}_{k}.
	\end{equation}
	Donde
	\begin{math}
		\symbf{A}\left(\symbf{u},\omega\right)\coloneqq
		\omega
		{
			\begin{bmatrix}
				\diffp{f_{i}}{u_{k}}
				\left(\symbf{u}\right)
			\end{bmatrix}}_{\substack{1\leq i\leq p\\1\leq k\leq p}}
	\end{math}
	es un múltiplo de la matriz jacobiana de $\mathbf{f}$.
\end{definition}

\begin{definition}
	Un \textcolor{DarkBlue}{\bfseries problema de Riemann}
	\index{problema de Riemann} es un problema de valor inicial
	asociado a~\eqref{eq:systemofconservationlaw}
	\begin{equation}\label{eq:cauchysystemofconservationlaw}
		\begin{cases}
			\diffp{\symbf{u}}{t}+
			\diffp{}{x}
			\symbf{f}\left(\symbf{u}\right)=
			\symbf{s}\left(\symbf{u},x\right) &
			\text{en }\mathbb{R}\times\left(0,\infty\right). \\
			\symbf{u}=\symbf{u}_{0}           &
			\text{en }\mathbb{R}\times\left\{t=0\right\}.
		\end{cases}
	\end{equation}
	Donde $\symbf{u}_{l},\symbf{u}_{r}\in\Omega$ son los estados y
	\begin{equation*}
		\symbf{u}_{0}\left(x\right)=
		\begin{cases}
			\symbf{u}_{l}, & x<0. \\
			\symbf{u}_{r}, & x>0.
		\end{cases}
	\end{equation*}
\end{definition}

\begin{definition}
	El problema de valor inicial y de frontera asociado
	a~\eqref{eq:systemofconservationlaw} es
	\begin{equation}
		\begin{dcases}
			\diffp{\symbf{u}}{t}+
			\diffp{}{x}\symbf{f}\left(\symbf{u}\right)=
			\symbf{s}\left(\symbf{u},x\right) &
			\text{ en }I\times\left(0,T\right].         \\
			\symbf{u}=\symbf{u}_{0}           &
			\text{c.t.p. en }I\times\left\{t=0\right\}. \\
			\symbf{u}=\symbf{g}               &
			\text{ en }\partial I\times\left[0,T\right].
		\end{dcases}
	\end{equation}
\end{definition}

\begin{example}[La ecuación de Bateman-Burgers no viscosa]\index{ecuación de Burgers}
	\begin{equation}
		\begin{dcases}
			\diffp{u}{t}+
			\frac{1}{2}\diffp{u^{2}}{x}=0 &
			\text{ en }I\times\left(0,T\right].         \\
			u=u_{0}                       &
			\text{c.t.p. en }I\times\left\{t=0\right\}. \\
			u=g                           &
			\text{ en }\partial I\times\left[0,T\right].
		\end{dcases}
	\end{equation}
	% \begin{equation*}
	% 	\diffp{u}{t}+
	% 	u\diffp{u}{x}-
	% 	\nu\diffp[2]{u}{x}=
	% 	0.
	% \end{equation*}
\end{example}

\begin{example}%[Ecuación de Buckley-Leverett]
	La ecuación clásica de
	Buckley-Leverett\index{ecuación de Buckley-Leverett}
	es un modelo simple para un flujo de fluido de dos fases en un medio
	poroso.
	Una aplicación es la recuperación secundaria mediante impulsión de
	agua en la simulación de yacimientos de petróleo.
	\begin{equation}
		\begin{dcases}
			\diffp{u}{t}+
			\diffp{}{x}
			f\left(u\right)=0 &
			\text{ en }I\times\left(0,T\right].         \\
			u=u_{0}                       &
			\text{c.t.p. en }I\times\left\{t=0\right\}. \\
			u=g                           &
			\text{ en }\partial I\times\left[0,T\right].
		\end{dcases}
	\end{equation}
	\begin{equation*}
		f\left(s\right)=
		\frac{\frac{\kappa_{\text{rel,water}}\left(s\right)}{\mu_{\text{water}}}}{
			\frac{\kappa_{\text{rel,water}}\left(s\right)}{\mu_{\text{water}}}+
			\frac{\kappa_{\text{rel,oil}}\left(s\right)}{\mu_{\text{oil}}}
		}.
	\end{equation*}
\end{example}

\begin{example}[$p$-sistema]\index{$p$-sistema}
	\begin{equation*}
		\begin{cases}
			\diffp{v}{t}-\diffp{u}{x}=0               &
			\text{en }\mathbb{R}\times\left(0,\infty\right). \\
			\diffp{u}{t}+\diffp{}{x}p\left(v\right)=0 &
			\text{en }\mathbb{R}\times\left(0,\infty\right).
		\end{cases}
	\end{equation*}
\end{example}


% Sistema de fluidos % de la dinámica de gases
\begin{example}[Ecuaciones de Euler]\index{ecuaciones de Euler}
	\begin{equation*}
		\begin{dcases}
			\diffp{\rho}{t}+
			\sum_{j=1}^{3}
			\diffp{}{x_{j}}
			\left(\rho u_{j}\right)=0                   &
			\text{en }\mathbb{R}\times\left(0,\infty\right). \\
			\forall i\in\left\{1,2,3\right\}:
			\diffp{}{t}
			\left(\rho u_{i}\right)+
			\sum_{j=1}^{3}
			\diffp{}{x_{j}}
			\left(\rho u_{i}u_{j}+p\delta_{ij}\right)=0 &
			\text{en }\mathbb{R}\times\left(0,\infty\right). \\
			\diffp{}{t}
			\left(\rho e\right)+
			\sum_{j=1}^{3}
			\diffp{}{x_{j}}
			\left(\left(\rho e+p\right)u_{j}\right)=0   &
			\text{en }\mathbb{R}\times\left(0,\infty\right).
		\end{dcases}
	\end{equation*}
\end{example}

\chapter{Esquemas de volúmenes finitos}

Considere un problema de valor inicial frontera para un sistema
hiperbólico de leyes de conservación.
\begin{equation*}
	\diffp{\symbf{u}}{t}+
	\diffp{}{x}
	\symbf{f}\left(\symbf{u}\right)=
	\symbf{s}\left(\symbf{u}\right).
	\symbf{u}\left(x,0\right)
	\symbf{u}\left(a,t\right)=
	\symbf{u}\left(b,t\right).
\end{equation*}
La forma integral de las leyes de conservación es
\begin{equation*}
	\int_{x_{l}}^{x_{r}}
	\symbf{u}\left(x,t_{2}\right)\dl x=
	\int_{x_{l}}^{x_{r}}
	\symbf{u}\left(x,t_{1}\right)\dl x+
	\int_{t_{1}}^{t_{2}}
	\symbf{f}\left(\symbf{u}\left(x_{l},t\right)\right)\dl t-
	\int_{t_{1}}^{t_{2}}
	\symbf{f}\left(\symbf{u}\left(x_{r},t\right)\right)\dl t+
	\int_{t_{1}}^{t_{2}}
	\int_{x_{l}}^{x_{r}}
	\symbf{s}\left(\symbf{u}\left(x,t\right)\right)\dl x\dl t.
\end{equation*}

\begin{equation*}
	V^{n}_{i}=
	\left[x_{i-\frac{1}{2}},x_{i+\frac{1}{2}}\right]\times
	\left[t_{n},t_{n+1}\right]
\end{equation*}
con $\Delta x=x_{i+\frac{1}{2}}-x_{i-\frac{1}{2}}$ y $\Delta t=t_{n+1}-t_{n}$.

\begin{equation*}
	\symbf{u}^{n+1}_{i}=
	\symbf{u}^{n}_{i}-
	\frac{\Delta t}{\Delta x}
	\left(\symbf{f}_{i+\frac{1}{2}}-\symbf{f}_{i-\frac{1}{2}}\right)+
	\Delta t\symbf{s}_{i}.
\end{equation*}

\begin{align*}
	\symbf{u}^{n}_{i}             & =
	\frac{1}{\Delta x}
	\int_{x_{i-\frac{1}{2}}}^{x_{i+\frac{1}{2}}}
	\symbf{u}\left(x,t_{n}\right)\dl x.                                   \\
	\symbf{f}^{n}_{i+\frac{1}{2}} & =
	\frac{1}{\Delta t}
	\int_{t_{n}}^{t_{n+1}}
	\symbf{f}\left(\symbf{u}\left(x_{i+\frac{1}{2}},t\right)\right)\dl t. \\
	\symbf{s}_{i}                 & =
	\frac{1}{\Delta t\Delta x}
	\int_{t_{n}}^{t_{n+1}}
	\int_{x_{i-\frac{1}{2}}}^{x_{i+\frac{1}{2}}}
	\symbf{s}\left(\symbf{u}\left(x,t\right)\right)\dl x\dl t.            \\
\end{align*}

Considere el problema de Cauchy~\eqref{eq:cauchysystemofconservationlaw}
donde~\eqref{eq:systemofconservationlaw} es hiperbólico.
Consideremos $C_{j}=\left[x_{j-\frac{1}{2}},x_{j+\frac{1}{2}}\right]\subset\mathbb{R}$
\begin{equation*}
	\symbf{v}^{n}_{j}\approx
	\frac{1}{\left|C_{j}\right|}
	\int_{C_{j}}
	\symbf{u}\left(x,t_{n}\right)\dl{\symbf{x}}.
\end{equation*}

\begin{equation*}
	\diff{}{t}
	\int_{C_{j}}
	\symbf{u}\left(x,t\right)\dl x+
	\symbf{f}\left(\symbf{u}\left(x_{j+\frac{1}{2}},t\right)\right)-
	\symbf{f}\left(\symbf{u}\left(x_{j-\frac{1}{2}},t\right)\right)=
	\symbf{0}.
\end{equation*}

\begin{equation*}
	\diff{}{t}\symbf{v}_{j}+
	\frac{1}{\Delta x}
	\left(
	\symbf{g}_{j+\frac{1}{2}}\left(t\right)-
	\symbf{g}_{j-\frac{1}{2}}\left(t\right)
	\right)=
	\symbf{0}.
\end{equation*}

\begin{equation*}
	\symbf{v}^{n+1}_{j}=
	\symbf{H}\left(\symbf{v}^{n}_{j-k},\dotsc,\symbf{v}^{n}_{j+k}\right).
\end{equation*}

\begin{equation*}
	\symbf{g}_{j+\frac{1}{2}}\left(t\right)=
	\symbf{g}\left(\symbf{v}\right)
\end{equation*}

\section{Método de Godunov}

\begin{equation*}
	\symbf{w}\left(x,t\right)=
	\symbf{w}_{R}
	\left(\frac{x}{t};\symbf{u}_{L},\symbf{u}_{R}\right)
\end{equation*}

\section{Método de Roe}

\section{Método de Rusanov}

% clawpack, fipy
\part{Aplicaciones}
\chapter{Implementación de las técnicas}

A continuación, pensamos dos estrategias para tentar resolver por el
método de los volúmenes finitos de un
\textcolor{DarkRed}{sistema EDP complicado}.
Se trata de resolver un sistema EDP a la vez, de menor a mayor
dificultad.
Suponga que $\Omega\subset\mathbb{R}^{3}$ es un conjunto abierto.

\section{Primera estrategia}

\begin{equation*}
	\eqref{eq:advectionsystem}\implies
	\eqref{eq:advectionreactionsystem}\implies
	\eqref{eq:advectionreactionsystemquaslinearnonhomogeneous}\implies
	\mathcolor{DarkRed}{\eqref{eq:complicatedsystem}}.
\end{equation*}

\section{Segunda estrategia}

\begin{equation*}
	\eqref{eq:advectionsystem}\implies
	\eqref{eq:advectionreactionsystemquaslinear}\implies
	\eqref{eq:advectionreactionsystemquaslinearnonhomogeneous}\implies
	\mathcolor{DarkRed}{\eqref{eq:complicatedsystem}}.
\end{equation*}

\subsection*{Sistema EDP de advección lineal homogéneo}

Encuentre
\begin{math}
	\symbf{u}\in
	C^{1}\left(I\times\left[0,T\right],\Omega\right)
\end{math}
en el problema de valor inicial y de frontera~\eqref{eq:advectionsystem}
\begin{equation}\label{eq:advectionsystem}
	\begin{cases}
		\diffp{\symbf{u}}{t}+\diffp{}{x}\symbf{f}\left(\symbf{u}\right)=
		\symbf{0}     &
		\text{ en }I\times\left(0,T\right].   \\
		\symbf{u}                                                      =
		\symbf{u}_{0} &
		\text{ en }I\times\left\{t=0\right\}. \\
		\symbf{u}                                                      =
		\symbf{0}     &
		\text{ en }\partial I\times\left[0,T\right].
	\end{cases}
\end{equation}
Donde
\begin{math}
	\symbf{u}_{0}\colon I\to
	\mathbb{R}^{3}
\end{math}
es conocida y
\begin{math}
	\symbf{f}\colon\Omega\to
	\mathbb{R}^{3}
\end{math}
es dada por
\begin{math}
	\symbf{f}\left(\symbf{u}\right)=
	\symbf{a}\odot\symbf{u}
\end{math},
siendo $\symbf{a}\in\mathbb{R}^{3}\setminus\left\{\symbf{0}\right\}$.

\subsection*{Sistema EDP de advección reacción lineal homogéneo}

Encuentre
\begin{math}
	\symbf{u}\in
	C^{1}\left(I\times\left[0,T\right],\Omega\right)
\end{math}
en el problema de valor inicial y de frontera~\eqref{eq:advectionreactionsystem}
\begin{equation}\label{eq:advectionreactionsystem}
	\begin{cases}
		\diffp{\symbf{u}}{t}+\diffp{}{x}\symbf{f}\left(\symbf{u}\right)               =
		-\symbf{b}\odot\symbf{u} &
		\text{ en }I\times\left(0,T\right].   \\
		\symbf{u}                                                                     =
		\symbf{u}_{0}            &
		\text{ en }I\times\left\{t=0\right\}. \\
		\symbf{u}                                                                     =
		\symbf{0}                &
		\text{ en }\partial I\times\left[0,T\right].
	\end{cases}
\end{equation}
Donde
\begin{math}
	\symbf{u}_{0}\colon I\to
	\mathbb{R}^{3}
\end{math}
es conocida y
\begin{math}
	\symbf{f}\colon\Omega\to
	\mathbb{R}^{3}
\end{math}
es dada por
\begin{math}
	\symbf{f}\left(\symbf{u}\right)=
	\symbf{a}\odot\symbf{u}
\end{math},
siendo $\symbf{a},\symbf{b}\in\mathbb{R}^{3}\setminus\left\{\symbf{0}\right\}$.

\subsection*{Sistema EDP de advección reacción cuasilineal homogéneo}

Encuentre
\begin{math}
	\symbf{u}\in
	C^{1}\left(I\times\left[0,T\right],\Omega\right)
\end{math}
en el problema de valor inicial y de frontera~\eqref{eq:advectionreactionsystemquaslinear}
\begin{equation}\label{eq:advectionreactionsystemquaslinear}
	\begin{cases}
		\diffp{\symbf{u}}{t}+\diffp{}{x}\symbf{f}\left(\symbf{u}\right)=
		-\symbf{b}\odot\symbf{u} & \text{ en }I\times\left(0,T\right].          \\
		\symbf{u}                                                      =
		\symbf{u}_{0}            & \text{ en }I\times\left\{t=0\right\}.        \\
		\symbf{u}                                                      =
		\symbf{0}                & \text{ en }\partial I\times\left[0,T\right].
	\end{cases}
\end{equation}
Donde
\begin{math}
	\symbf{u}_{0}\colon I\to
	\mathbb{R}^{3}
\end{math}
es conocida y
\begin{math}
	\symbf{f}\colon\Omega\to
	\mathbb{R}^{3}
\end{math}
es dada por
\begin{math}
	\symbf{f}\left(\symbf{u}\right)=
	\symbf{a}\odot\symbf{u}+
	\mathcolor{DarkRed}{a_{1}u_{1}\left(u_{2}-1\right)\symbf{e_{1}}}
\end{math},
siendo $\symbf{a},\symbf{b}\in\mathbb{R}^{3}\setminus\left\{\symbf{0}\right\}$.

\subsection*{Sistema EDP de advección reacción cuasilineal no homogéneo}

Encuentre
\begin{math}
	\symbf{u}\in
	C^{1}\left(I\times\left[0,T\right],\Omega\right)
\end{math}
en el problema de valor inicial y de frontera~\eqref{eq:advectionreactionsystemquaslinearnonhomogeneous}
\begin{equation}\label{eq:advectionreactionsystemquaslinearnonhomogeneous}
	\begin{cases}
		\diffp{\symbf{u}}{t}+\diffp{}{x}\symbf{f}\left(\symbf{u}\right)=
		-\symbf{b}\odot\symbf{u}-\symbf{h} & \text{ en }I\times\left(0,T\right].          \\
		\symbf{u}                                                      =
		\symbf{u}_{0}                      & \text{ en }I\times\left\{t=0\right\}.        \\
		\symbf{u}                                                       =
		\symbf{0}                          & \text{ en }\partial I\times\left[0,T\right].
	\end{cases}
\end{equation}
Donde
\begin{math}
	\symbf{u}_{0}\colon I\to
	\mathbb{R}^{3}
\end{math},
\begin{math}
	\symbf{h}\colon I\times\left[0,T\right]\to
	\mathbb{R}^{3}
\end{math}
son conocidas y
\begin{math}
	\symbf{f}\colon\Omega\to
	\mathbb{R}^{3}
\end{math}
es dada por
\begin{math}
	\symbf{f}\left(\symbf{u}\right)=
	\symbf{a}\odot\symbf{u}+
	\mathcolor{DarkRed}{a_{1}u_{1}\left(u_{2}-1\right)\symbf{e_{1}}}
\end{math},
siendo $\symbf{a},\symbf{b}\in\mathbb{R}^{3}\setminus\left\{\symbf{0}\right\}$.

\subsection*{\color{DarkRed}Sistema EDP complicado}

Encuentre
\begin{math}
	\symbf{u}\in
	C^{1}\left(I\times\left[0,T\right],\Omega\right)
\end{math}
en el problema de valor inicial y de frontera~\eqref{eq:complicatedsystem}
\begin{equation}\label{eq:complicatedsystem}
	\begin{cases}
		\diffp{\symbf{u}}{t}+\diffp{}{x}\symbf{f}\left(\symbf{u}\right)=
		\symbf{g}\left(\symbf{u}\right) & \text{ en }I\times\left(0,T\right].          \\
		\symbf{u}                                                      =
		\symbf{u}_{0}                   & \text{ en }I\times\left\{t=0\right\}.        \\
		\symbf{u}                                                       =
		\symbf{0}                       & \text{ en }\partial I\times\left[0,T\right].
	\end{cases}
\end{equation}
Donde
\begin{math}
	\symbf{u}_{0}\colon I\to
	\mathbb{R}^{3}
\end{math}
es conocida y
\begin{math}
	\symbf{f}\colon\Omega\to
	\mathbb{R}^{3}
\end{math}
es dada por
\begin{math}
	\symbf{f}\left(\symbf{u}\right)=
	\symbf{a}\odot\symbf{u}+
	\mathcolor{DarkRed}{a_{1}u_{1}\left(u_{3}-1\right)\symbf{e_{1}}}
\end{math},
\begin{math}
	\symbf{g}\left(\symbf{u}\right)=
	\begin{bmatrix}
		b_{1}u_{2}u_{3}\Phi-\beta u_{0} \\
		-b_{2}u_{2}u_{3}\Phi            \\
		-b_{3}u_{2}u_{3}\Phi
	\end{bmatrix}
\end{math}
siendo $\symbf{a},\symbf{b}\in\mathbb{R}^{3}\setminus\left\{\symbf{0}\right\}$.

\chapter{Resultados}

%https://mladenivkovic.github.io/work.html
% \begin{figure}[ht!]
% 	\centering
% 	\includegraphics[width=.8\paperwidth]{1}
% 	\includegraphics[width=.8\paperwidth]{2}
% 	\includegraphics[width=.8\paperwidth]{3}
% \end{figure}



\appendix

\chapter{Símbolos}

% https://ntrs.nasa.gov/api/citations/19880008959/downloads/19880008959.pdf
El producto de Hadamard se define como
\begin{align*}
	\odot\colon\mathbb{R}^{n}\times\mathbb{R}^{n} & \longrightarrow\mathbb{R}^{n} \\
	\left(
	\begin{bmatrix}
			a_{1}  \\
			\vdots \\
			a_{n}
		\end{bmatrix},
	\begin{bmatrix}
			b_{1}  \\
			\vdots \\
			b_{n}
		\end{bmatrix}
	\right)                                       & \longmapsto
	\begin{bmatrix}
		a_{1}b_{1} \\
		\vdots     \\
		a_{n}b_{n}
	\end{bmatrix}.
\end{align*}
El producto exterior se define como
\begin{align*}
	\otimes\colon\mathbb{R}^{m}\times\mathbb{R}^{n} & \longrightarrow\mathbb{R}^{m\times n} \\
	\left(
	\begin{bmatrix}
			a_{1}  \\
			\vdots \\
			a_{m}
		\end{bmatrix},
	\begin{bmatrix}
			b_{1}  \\
			\vdots \\
			b_{n}
		\end{bmatrix}
	\right)                                         & \longmapsto
	\begin{bmatrix}
		a_{1}b_{1} & \cdots & a_{1}b_{n} \\
		\vdots     & \ddots & \vdots     \\
		a_{m}b_{1} & \cdots & a_{m}b_{n}
	\end{bmatrix}.
\end{align*}
\nocite{*}
\printbibliography[title={Referencias},heading=bibintoc]

\end{document}
