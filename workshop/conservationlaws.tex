\chapter{Sistema de leyes de conservación}

Presentamos un sistema de leyes de conservación en una dimensión
espacial, así como los ejemplos más destacados en la física de los
medios continuos.

\begin{definition}
	Sea $\Omega\subset\mathbb{R}^{p}$ un conjunto abierto.
	Un \textcolor{DarkBlue}{\bfseries sistema de leyes de conservación}
	es
	\begin{equation}\label{eq:systemofconservationlaw}
		\diffp{\symbf{u}}{t}+
		\diffp{}{x}
		\symbf{f}\left(\symbf{u}\right)=
		\symbf{0}.
	\end{equation}
	Donde $\Omega$ es el conjunto de estados,
	$\symbf{f}\in C^{1}\left(\Omega,\mathbb{R}^{p}\right)$ es la
	función flujo y $\symbf{u}\in C^{1}\left(\mathbb{R}\times\left[0,\infty\right[,\Omega\right)$.
	\begin{align*}
		\symbf{f}\colon\Omega
		                & \longrightarrow\mathbb{R}^{p} &
		\symbf{u}\colon\mathbb{R}\times\left[0,\infty\right[
		                & \longrightarrow\Omega           \\
		\begin{bmatrix}
			u_{1}  \\
			\vdots \\
			u_{p}
		\end{bmatrix}
		                & \longmapsto
		\begin{bmatrix}
			f_{1}  \\
			\vdots \\
			f_{p}
		\end{bmatrix}, &
		\left(x,t\right)
		                & \longmapsto
		\begin{bmatrix}
			u_{1}\left(x,t\right) \\
			\vdots                \\
			u_{p}\left(x,t\right)
		\end{bmatrix}=
		\begin{bmatrix}
			u_{1}  \\
			\vdots \\
			u_{p}
		\end{bmatrix}.
	\end{align*}
\end{definition}
Si $I=\left[x_{l},x_{r}\right]\subset\mathbb{R}$, entonces
de~\eqref{eq:systemofconservationlaw} se obtiene la ecuación de
balance que expresa que la variación del tiempo de la cantidad en el
medio es igual a las pérdidas a través de la interface.
\begin{equation*}
	\diff{}{t}
	\int^{x_{r}}_{x_{l}}\symbf{u}\dl x+
	\symbf{f}\left(\symbf{u}\right)\Big|^{x_{r}}_{x_{l}}=
	\symbf{0}.
\end{equation*}

\begin{definition}
	Un sistema de leyes de
	conservación~\eqref{eq:systemofconservationlaw} es
	\textcolor{DarkBlue}{\bfseries hiperbólico} si y solo si
	\begin{equation}
		\forall\symbf{u}\in\Omega\!:
		\forall\omega\in\mathbb{R}\setminus\left\{0\right\}\!:
		\exists\left\{\left(\lambda_{k},\symbf{r}_{k}\right)\right\}^{p}_{k=1}\subset
		\mathbb{R}\times\mathbb{R}^{p}
		\text{ tal que }
		\symbf{A}\left(\symbf{u},\omega\right)
		\symbf{r}_{k}=
		\lambda_{k}
		\symbf{r}_{k}.
	\end{equation}
	Donde
	\begin{math}
		\symbf{A}\left(\symbf{u},\omega\right)\coloneqq
		\omega
		{
			\begin{bmatrix}
				\diffp{f_{i}}{u_{k}}
				\left(\symbf{u}\right)
			\end{bmatrix}}_{\substack{1\leq i\leq p\\1\leq k\leq p}}
	\end{math}
	es un múltiplo de la matriz jacobiana de $\mathbf{f}$.
\end{definition}

\begin{definition}
	El \textcolor{DarkBlue}{\bfseries problema de Cauchy} asociado
	a~\eqref{eq:systemofconservationlaw} es
	\begin{equation}\label{eq:cauchysystemofconservationlaw}
		\begin{cases}
			\diffp{\symbf{u}}{t}+
			\diffp{}{x}
			\symbf{f}\left(\symbf{u}\right)=
			\symbf{0}               &
			\text{en }\mathbb{R}\times\left(0,\infty\right). \\
			\symbf{u}=\symbf{u}_{0} &
			\text{en }\mathbb{R}\times\left\{t=0\right\}.
		\end{cases}
	\end{equation}
\end{definition}
Sean los estados $\symbf{u}_{l},\symbf{u}_{r}\in\Omega$.
Un \textcolor{DarkBlue}{\bfseries problema de Riemann} es un problema
de Cauchy donde
\begin{equation*}
	\symbf{u}_{0}\left(x\right)=
	\begin{cases}
		\symbf{u}_{l}, & x<0. \\
		\symbf{u}_{r}, & x>0.
	\end{cases}
\end{equation*}

\begin{example}[La ecuación de Burgers]
	\begin{equation*}
		\diffp{u}{t}+
		u\diffp{u}{x}-
		\nu\diffp[2]{u}{x}=
		0.
	\end{equation*}
	\begin{equation*}
		\diffp{u}{t}+
		\diffp{}{x}
		\left(\frac{u^{2}}{2}\right)=
		0.
	\end{equation*}
\end{example}

\begin{example}[Ecuación de Buckley-Leverett]
	\begin{equation*}
		f\left(s\right)=
		\frac{\frac{\kappa_{\text{rel,water}}\left(s\right)}{\mu_{\text{water}}}}{
			\frac{\kappa_{\text{rel,water}}\left(s\right)}{\mu_{\text{water}}}+
			\frac{\kappa_{\text{rel,oil}}\left(s\right)}{\mu_{\text{oil}}}
		}.
	\end{equation*}
\end{example}

\begin{example}[$p$-sistema]
	\begin{equation*}
		\begin{cases}
			\diffp{v}{t}-\diffp{u}{x}=0               &
			\text{en }\mathbb{R}\times\left(0,\infty\right). \\
			\diffp{u}{t}+\diffp{}{x}p\left(v\right)=0 &
			\text{en }\mathbb{R}\times\left(0,\infty\right).
		\end{cases}
	\end{equation*}
\end{example}

\begin{example}[Ecuaciones de la dinámica de gases]
	\begin{equation*}
		\begin{dcases}
			\diffp{\rho}{t}+
			\sum_{j=1}^{3}
			\diffp{}{x_{j}}
			\left(\rho u_{j}\right)=0                   &
			\text{en }\mathbb{R}\times\left(0,\infty\right). \\
			\forall i\in\left\{1,2,3\right\}:
			\diffp{}{t}
			\left(\rho u_{i}\right)+
			\sum_{j=1}^{3}
			\diffp{}{x_{j}}
			\left(\rho u_{i}u_{j}+p\delta_{ij}\right)=0 &
			\text{en }\mathbb{R}\times\left(0,\infty\right). \\
			\diffp{}{t}
			\left(\rho e\right)+
			\sum_{j=1}^{3}
			\diffp{}{x_{j}}
			\left(\left(\rho e+p\right)u_{j}\right)=0   &
			\text{en }\mathbb{R}\times\left(0,\infty\right).
		\end{dcases}
	\end{equation*}
\end{example}

\section{Soluciones débiles}

\begin{definition}
	Sea
	\begin{math}
		\symbf{u}_{0}\in
		{\symbf{L}^{\infty}_{\text{loc}}\left(\mathbb{R}\right)}^{p}
	\end{math}.
	\begin{math}
		\symbf{u}\in
		{\symbf{L}^{\infty}_{\text{loc}}\left(\mathbb{R}\times\left[0,\infty\right[\right)}^{p}
	\end{math}
	es una \textcolor{DarkBlue}{\bfseries solución débil}
	de~\eqref{eq:cauchysystemofconservationlaw} si y solo si
	$\symbf{u}\left(x,t\right)\in\Omega$ casi en todas partes y
	\begin{equation*}
		\forall\symbf{\varphi}\in
		{\symbf{C}^{1}_{0}\left(\mathbb{R}\times\left[0,\infty\right[\right)}^{p}\!:
		\int_{0}^{\infty}
		\int_{\mathbb{-\infty}}^{\infty}
		\left[
			\symbf{u}\cdot
			\diffp{\symbf{\varphi}}{t}+
			\symbf{f}\left(\symbf{u}\right)\cdot
			\diffp{\symbf{\varphi}}{x}
			\right]
		\dl x
		\dl t+
		\int_{-\infty}^{\infty}
		\symbf{u}_{0}\left(x\right)\cdot
		\symbf{\varphi}\left(x,0\right)
		\dl x=
		0.
	\end{equation*}
\end{definition}

% \begin{align*}
%   -\int_{0}^{\infty}
%   \int_{\mathbb{-\infty}}^{\infty}
%   \left[
%     \diffp{\symbf{u}}{t}+
%     \diffp{}{x}\symbf{f}\left(\symbf{u}\right)
%   \right]\cdot\symbf{\varphi}
%   \dl x
%   \dl t.
% \end{align*}

\begin{definition}
	\begin{math}
		\symbf{u}\in
		C^{1}_{\text{piec}}
		\left(\mathbb{R}\times\left[0,\infty\right[,\Omega\right)
	\end{math}
	si y solo si
	cualquier curva de discontinuidad $\Sigma$ tiene una
	parametrización de la forma $\left(t,\xi\left(t\right)\right)$,
	donde $\xi\in C^{1}\left(I,\mathbb{R}\right)$.
	\begin{equation*}
		\symbf{u}_{\pm}\left(\xi\left(t\right),t\right)=
		\lim_{\substack{\varepsilon\to0\\\varepsilon>0}}
		\symbf{u}\left(\xi\left(t\right)\pm\varepsilon,t\right).
	\end{equation*}
\end{definition}

\begin{theorem}
	Si
	\begin{math}
		\symbf{u}\in
		C^{1}_{\text{piec}}
		\left(\mathbb{R}\times\left[0,\infty\right[,\Omega\right)
	\end{math},
	entonces
	\begin{enumerate}%[(i)]
		\item

		      $\symbf{u}$ es una solución clásica
		      de~\eqref{eq:systemofconservationlaw}.

		\item

		      $\symbf{u}$ satisface la condición de salto de
		      Rankine-Hugoniot
		      \begin{equation*}
			      \left(\symbf{u}_{+}-\symbf{u}_{-}\right)
			      n_{t}+
			      \left[
				      \symbf{f}\left(\symbf{u}_{+}\right)-
				      \symbf{f}\left(\symbf{u}_{-}\right)
				      \right]
			      n_{x}=
			      \symbf{0}.
		      \end{equation*}
		      Donde
		      $\left[\symbf{u}\right]=\symbf{u}_{+}-\symbf{u}_{-}$ y
		      $\left[\symbf{f}\left(\symbf{u}\right)\right]=\symbf{f}\left(\symbf{u}_{+}\right)-\symbf{f}\left(\symbf{u}_{-}\right)$.
	\end{enumerate}
\end{theorem}

\begin{example}[]
	Sea.
	\begin{equation*}
		\left\{
		\begin{aligned}
			\diffp{u}{t}+
			\diffp{}{x}
			\frac{u^{2}}{2} & =
			0.                  \\
			u_{0}\left(x\right)=
			\begin{cases}
				u_{l}, & x<0. \\
				u_{r}, & x>0.
			\end{cases}
		\end{aligned}
		\right.
	\end{equation*}
\end{example}

\begin{equation*}
	\diffp{}{t}
	U\left(\symbf{u}\right)+
	\diffp{}{x}
	F\left(\symbf{u}\right)=
	0.
\end{equation*}

\section{Soluciones de entropía}

\begin{definition}
	Sea $\Omega\subset\mathbb{R}^{p}$ un conjunto convexo.
	$U\in C^{1}\left(\Omega,\mathbb{R}\right)$ es una
	\textcolor{DarkBlue}{\bfseries entropía}
	de~\eqref{eq:systemofconservationlaw} si y solo si
	$\exists F\in C^{1}\left(\Omega,\mathbb{R}\right)$ tal que
	\begin{equation*}
		{\begin{bmatrix}
				\diffp{U}{u_{k}}
			\end{bmatrix}}_{1\leq k\leq p}
		{\begin{bmatrix}
				\diffp{f_{i}}{u_{k}}
			\end{bmatrix}}_{\substack{1\leq i\leq p\\1\leq k\leq p}}
		\left(\symbf{u}\right)=
		{\begin{bmatrix}
			\diffp{F}{u_{k}}
		\end{bmatrix}}_{1\leq k\leq p}.
	\end{equation*}
	% \begin{equation*}
	%   \diffp{}{t}
	%   U\left(\symbf{u}\right)+
	%   \sum_{j=1}^{d}
	%   \diffp{}{x_{j}}
	%   F_{j}\left(\symbf{u}\right)=0.
	% \end{equation*}
\end{definition}

\begin{example}[$p$-sistema]
	.
\end{example}

\section{Método de viscosidad}

Dado $\varepsilon>0$, asociamos
con~\eqref{eq:systemofconservationlaw} el sistema parabólico
\begin{equation}\label{eq:parabolicsystemofconservationlaw}
	\diffp{\symbf{u}_{\varepsilon}}{t}+
	\diffp{}{x}
	\symbf{f}
	\left(\symbf{u}_{\varepsilon}\right)-
	\varepsilon
	\Delta\symbf{u}_{\varepsilon}=
	\symbf{0},
\end{equation}
donde $-\varepsilon\Delta\symbf{u}_{\varepsilon}$ puede ser visto
como un término de viscosidad.

\begin{theorem}
	Sea
	\begin{math}
		\left\{\symbf{u}_{\varepsilon}\right\}_{\varepsilon}
	\end{math}
	una sucesión de soluciones suaves
	de~\eqref{eq:parabolicsystemofconservationlaw}.
	Si~\eqref{eq:systemofconservationlaw} admite una entropía
	$U$ con flujo entropía $F$, $\exists C>0$ tal que
	\begin{math}
		{\left\|\symbf{u}_{\varepsilon}\right\|}_{
			{\symbf{L}^{\infty}\left(\mathbb{R}\times\left[0,\infty\right[\right)}^{p}
		}\leq C
	\end{math}
	y $\lim\limits_{\varepsilon\to0}\symbf{u}_{\varepsilon}=\symbf{u}$
	casi en todas partes en $\mathbb{R}\times\left[0,\infty\right[$.
	Entonces, $\symbf{u}$ es una solución débil
	de~\eqref{eq:systemofconservationlaw} y
	\begin{equation*}
		\forall\varphi\in
		C^{\infty}_{0}\left(\mathbb{R}\times\left]0,\infty\right[\right),
		\varphi\geq 0:
		\int_{0}^{\infty}
		\int_{-\infty}^{\infty}
		\left[
			U\left(\symbf{u}\right)
			\diffp{\varphi}{t}+
			F\left(\symbf{u}\right)
			\diffp{\varphi}{x}
			\right]
		\dl x
		\dl t\geq 0.
	\end{equation*}
	% \begin{equation*}
	% 	\diffp{}{t}
	% 	U\left(\symbf{u}\right)+
	% 	\diffp{}{x}
	% 	F\left(\symbf{u}\right)\leq
	% 	0
	% \end{equation*}
\end{theorem}
