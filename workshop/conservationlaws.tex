\chapter{Sistema de leyes de conservación}

Presentamos un sistema de leyes de conservación en una dimensión
espacial, así como los ejemplos más destacados en la física de los
medios continuos.

\begin{definition}
	Sea $\Omega\subset\mathbb{R}^{p}$ un conjunto abierto.
	Un \textcolor{DarkBlue}{\bfseries sistema de leyes de conservación}
	\index{sistema de leyes de conservación} es
	\begin{equation}\label{eq:systemofconservationlaw}
		\diffp{\symbf{u}}{t}+
		\diffp{}{x}
		\symbf{f}\left(\symbf{u}\right)=
		\symbf{s}\left(\symbf{u},x\right).
	\end{equation}
	Donde $\Omega$ es el conjunto de estados,
	\begin{math}
		\symbf{f}\in\symbf{C}^{1}
		\left(\Omega,\mathbb{R}^{p}\right)
	\end{math}
	es la función flujo,
	\begin{math}
		\symbf{s}\in
		\symbf{C}^{1}
		\left(\Omega\times\mathbb{R},\mathbb{R}^{p}\right)
	\end{math}
	es la función fuente sin términos de derivadas de $\symbf{u}$ y
	\begin{math}
		\symbf{u}\in
		\symbf{C}^{1}
		\left(\mathbb{R}\times\left[0,\infty\right[,\Omega\right)
	\end{math} es la solución de~\eqref{eq:systemofconservationlaw}.
	\begin{align*}
		\symbf{f}\colon\Omega                                &
		\longrightarrow\mathbb{R}^{p}                        &
		\symbf{s}\colon\Omega\times\mathbb{R}                &
		\longrightarrow\mathbb{R}^{p}                        &
		\symbf{u}\colon\mathbb{R}\times\left[0,\infty\right[ &
		\longrightarrow\Omega                                  \\
		\begin{bmatrix}
			u_{1}  \\
			\vdots \\
			u_{p}
		\end{bmatrix}                                      &
		\longmapsto
		\begin{bmatrix}
			f_{1}  \\
			\vdots \\
			f_{p}
		\end{bmatrix},                                      &
		\left(\symbf{u},x\right)                             &
		\longmapsto
		\begin{bmatrix}
			s_{1}\left(\symbf{u},x\right) \\
			\vdots                        \\
			s_{p}\left(\symbf{u},x\right)
		\end{bmatrix},                     &
		\left(x,t\right)                                     &
		\longmapsto
		\begin{bmatrix}
			u_{1}\left(x,t\right) \\
			\vdots                \\
			u_{p}\left(x,t\right)
		\end{bmatrix}=
		\begin{bmatrix}
			u_{1}  \\
			\vdots \\
			u_{p}
		\end{bmatrix}.
	\end{align*}
\end{definition}
Si $I\subset\mathbb{R}$, entonces
de~\eqref{eq:systemofconservationlaw} se obtiene la ecuación de
balance que expresa que la variación en el tiempo de la cantidad
total en el medio es igual al flujo neto a través de la interface
más la contribución del término fuente.
\begin{equation*}
	\diff{}{t}
	\int_{I}\symbf{u}\dl x+
	{\symbf{f}\left(\symbf{u}\right)\Big|}_{\partial I}=
	\int_{I}\symbf{s}\left(\symbf{u},x\right)\dl x.
\end{equation*}

\begin{definition}
	Un sistema~\eqref{eq:systemofconservationlaw} es
	\textcolor{DarkBlue}{\bfseries hiperbólico}\index{hiperbólico} si y
	solo si
	\begin{equation}
		\forall\symbf{u}\in\Omega\!:
		\forall\omega\in\mathbb{R}\setminus\left\{0\right\}\!:
		\exists
		{\left\{
			\left(\lambda_{k},\symbf{r}_{k}\right)
			\right\}}^{p}_{k=1}\subset
		\mathbb{R}\times\mathbb{R}^{p}
		\text{ tal que }
		\symbf{A}\left(\symbf{u},\omega\right)
		\symbf{r}_{k}=
		\lambda_{k}
		\symbf{r}_{k}.
	\end{equation}
	Donde
	\begin{math}
		\symbf{A}\left(\symbf{u},\omega\right)\coloneqq
		\omega
		{
			\begin{bmatrix}
				\diffp{f_{i}}{u_{k}}
				\left(\symbf{u}\right)
			\end{bmatrix}}_{\substack{1\leq i\leq p\\1\leq k\leq p}}
	\end{math}
	es un múltiplo de la matriz jacobiana de $\mathbf{f}$.
\end{definition}

\begin{definition}
	Un \textcolor{DarkBlue}{\bfseries problema de Riemann}
	\index{problema de Riemann} es un problema de valor inicial
	asociado a~\eqref{eq:systemofconservationlaw}
	\begin{equation}\label{eq:cauchysystemofconservationlaw}
		\begin{cases}
			\diffp{\symbf{u}}{t}+
			\diffp{}{x}
			\symbf{f}\left(\symbf{u}\right)=
			\symbf{s}\left(\symbf{u},x\right) &
			\text{en }\mathbb{R}\times\left(0,\infty\right). \\
			\symbf{u}=\symbf{u}_{0}           &
			\text{en }\mathbb{R}\times\left\{t=0\right\}.
		\end{cases}
	\end{equation}
	Donde $\symbf{u}_{l},\symbf{u}_{r}\in\Omega$ son los estados y
	\begin{equation*}
		\symbf{u}_{0}\left(x\right)=
		\begin{cases}
			\symbf{u}_{l}, & x<0. \\
			\symbf{u}_{r}, & x>0.
		\end{cases}
	\end{equation*}
\end{definition}

\begin{definition}
	El problema de valor inicial y de frontera asociado
	a~\eqref{eq:systemofconservationlaw} es
	\begin{equation}
		\begin{dcases}
			\diffp{\symbf{u}}{t}+
			\diffp{}{x}\symbf{f}\left(\symbf{u}\right)=
			\symbf{s}\left(\symbf{u},x\right) &
			\text{ en }I\times\left(0,T\right].         \\
			\symbf{u}=\symbf{u}_{0}           &
			\text{c.t.p. en }I\times\left\{t=0\right\}. \\
			\symbf{u}=\symbf{g}               &
			\text{ en }\partial I\times\left[0,T\right].
		\end{dcases}
	\end{equation}
\end{definition}

\begin{example}[La ecuación de Bateman-Burgers no viscosa]\index{ecuación de Burgers}
	\begin{equation}
		\begin{dcases}
			\diffp{u}{t}+
			\frac{1}{2}\diffp{u^{2}}{x}=0 &
			\text{ en }I\times\left(0,T\right].         \\
			u=u_{0}                       &
			\text{c.t.p. en }I\times\left\{t=0\right\}. \\
			u=g                           &
			\text{ en }\partial I\times\left[0,T\right].
		\end{dcases}
	\end{equation}
	% \begin{equation*}
	% 	\diffp{u}{t}+
	% 	u\diffp{u}{x}-
	% 	\nu\diffp[2]{u}{x}=
	% 	0.
	% \end{equation*}
\end{example}

\begin{example}%[Ecuación de Buckley-Leverett]
	La ecuación clásica de
	Buckley-Leverett\index{ecuación de Buckley-Leverett}
	es un modelo simple para un flujo de fluido de dos fases en un medio
	poroso.
	Una aplicación es la recuperación secundaria mediante impulsión de
	agua en la simulación de yacimientos de petróleo.
	\begin{equation}
		\begin{dcases}
			\diffp{u}{t}+
			\diffp{}{x}
			f\left(u\right)=0 &
			\text{ en }I\times\left(0,T\right].         \\
			u=u_{0}                       &
			\text{c.t.p. en }I\times\left\{t=0\right\}. \\
			u=g                           &
			\text{ en }\partial I\times\left[0,T\right].
		\end{dcases}
	\end{equation}
	\begin{equation*}
		f\left(s\right)=
		\frac{\frac{\kappa_{\text{rel,water}}\left(s\right)}{\mu_{\text{water}}}}{
			\frac{\kappa_{\text{rel,water}}\left(s\right)}{\mu_{\text{water}}}+
			\frac{\kappa_{\text{rel,oil}}\left(s\right)}{\mu_{\text{oil}}}
		}.
	\end{equation*}
\end{example}

\begin{example}[$p$-sistema]\index{$p$-sistema}
	\begin{equation*}
		\begin{cases}
			\diffp{v}{t}-\diffp{u}{x}=0               &
			\text{en }\mathbb{R}\times\left(0,\infty\right). \\
			\diffp{u}{t}+\diffp{}{x}p\left(v\right)=0 &
			\text{en }\mathbb{R}\times\left(0,\infty\right).
		\end{cases}
	\end{equation*}
\end{example}


% Sistema de fluidos % de la dinámica de gases
\begin{example}[Ecuaciones de Euler]\index{ecuaciones de Euler}
	\begin{equation*}
		\begin{dcases}
			\diffp{\rho}{t}+
			\sum_{j=1}^{3}
			\diffp{}{x_{j}}
			\left(\rho u_{j}\right)=0                   &
			\text{en }\mathbb{R}\times\left(0,\infty\right). \\
			\forall i\in\left\{1,2,3\right\}:
			\diffp{}{t}
			\left(\rho u_{i}\right)+
			\sum_{j=1}^{3}
			\diffp{}{x_{j}}
			\left(\rho u_{i}u_{j}+p\delta_{ij}\right)=0 &
			\text{en }\mathbb{R}\times\left(0,\infty\right). \\
			\diffp{}{t}
			\left(\rho e\right)+
			\sum_{j=1}^{3}
			\diffp{}{x_{j}}
			\left(\left(\rho e+p\right)u_{j}\right)=0   &
			\text{en }\mathbb{R}\times\left(0,\infty\right).
		\end{dcases}
	\end{equation*}
\end{example}
