\chapter{Preliminares}

Para facilitar la comprensión de los capítulos posteriores y para
definir la notación y la terminología, este capítulo resume algunos
puntos preliminares que fundamentan el material posterior.
Ningún contenido de este capítulo es original.
Los lectores familiarizados con el tema pueden consultar los
encabezados de las secciones y omitir el material que les resulte
familiar, consultando el texto cuando necesiten aclarar alguna
terminología o notación específica.

Presentamos un sistema de leyes de conservación en una dimensión
espacial, así como los ejemplos más destacados en la física de los
medios continuos.

\section{Sistema de leyes de conservación con un término fuente}

\begin{definition}
	Sea $\Omega\subset\mathbb{R}^{p}$ un conjunto abierto.
	Un \textcolor{DarkBlue}{\bfseries sistema de leyes de conservación
		con un término fuente}\index{sistema de leyes de conservación} es
	\begin{equation}\label{eq:systemofconservationlaw}
		\diffp{\symbf{u}}{t}+
		\diffp*{\symbf{f}\left(\symbf{u}\right)}{x}=
		\symbf{s}\left(\symbf{u}\right)
		\text{ en }\mathbb{R}\times\left[0,\infty\right[.
	\end{equation}
	Donde $\Omega$ es el conjunto de estados,
	\begin{math}
		\symbf{f}\in\symbf{C}^{1}
		\left(\Omega,\mathbb{R}^{p}\right)
	\end{math}
	es la función flujo,
	\begin{math}
		\symbf{s}\in
		\symbf{C}
		\left(\Omega,\mathbb{R}^{p}\right)
	\end{math}
	es la función fuente sin términos de las derivadas de la
	solución de~\eqref{eq:systemofconservationlaw},
	\begin{math}
		\symbf{u}\in
		\symbf{C}^{1}
		\left(\mathbb{R}\times\left[0,\infty\right[,\Omega\right)
	\end{math}.
	\begin{align*}
		\symbf{f}\colon\Omega                                &
		\longrightarrow\mathbb{R}^{p}                        &
		\symbf{s}\colon\Omega                                &
		\longrightarrow\mathbb{R}^{p}                        &
		\symbf{u}\colon\mathbb{R}\times\left[0,\infty\right[ &
		\longrightarrow\Omega                                  \\
		\begin{bmatrix}
			u_{1}  \\
			\vdots \\
			u_{p}
		\end{bmatrix}                                      &
		\longmapsto
		\begin{bmatrix}
			f_{1}\left(\symbf{u}\right) \\
			\vdots                      \\
			f_{p}\left(\symbf{u}\right)
		\end{bmatrix},                       &
		\begin{bmatrix}
			u_{1}  \\
			\vdots \\
			u_{p}
		\end{bmatrix}                                      &
		\longmapsto
		\begin{bmatrix}
			s_{1}\left(\symbf{u}\right) \\
			\vdots                      \\
			s_{p}\left(\symbf{u}\right)
		\end{bmatrix},                       &
		\left(x,t\right)                                     &
		\longmapsto
		\begin{bmatrix}
			u_{1}\left(x,t\right) \\
			\vdots                \\
			u_{p}\left(x,t\right)
		\end{bmatrix}=
		\begin{bmatrix}
			u_{1}  \\
			\vdots \\
			u_{p}
		\end{bmatrix}.
	\end{align*}
\end{definition}
Si $I\subset\mathbb{R}$, entonces
de~\eqref{eq:systemofconservationlaw} se obtiene la ecuación de
balance que expresa que la variación en el tiempo de la cantidad
total en el medio es igual al flujo neto a través de la interface
más la contribución del término fuente.
\begin{equation*}
	\diff{}{t}
	\int_{I}
	\symbf{u}
	\left(x,t\right)
	\dl x+
	{\symbf{f}
	\left(\symbf{u}\left(x,t\right)\right)
	\Big|}_{\partial I}=
	\int_{I}
	\symbf{s}
	\left(\symbf{u}\left(x,t\right)\right)
	\dl x.
\end{equation*}
A continuación, presentamos las leyes de conservación más conocidas
cuando $p=1$.
\begin{example}[Ecuación de Bateman-Burgers~\cite{Bateman1915,Burgers1948}]\index{ecuación de Burgers}
	\begin{equation*}
		\diffp{u}{t}+
		\diffp*{\left(\frac{u^{2}}{2}-\nu\diffp{u}{x}\right)}{x}=
		0
		\text{ en }\mathbb{R}\times\left[0,\infty\right[.
	\end{equation*}
\end{example}

\begin{example}[Ecuación de Buckley-Leverett~\cite{Buckley1942}]\index{ecuación de Buckley-Leverett}
	Es un modelo simple para un flujo de fluido de dos fases en un
	medio poroso.
	Una aplicación es la recuperación secundaria mediante impulsión de
	agua en la simulación de yacimientos de petróleo.
	\begin{equation*}
		\diffp{u}{t}+
		\diffp*{\left[\frac{\frac{\kappa_{\text{rel,water}}\left(u\right)}{\mu_{\text{water}}}}{\frac{\kappa_{\text{rel,water}}\left(u\right)}{\mu_{\text{water}}}+\frac{\kappa_{\text{rel,oil}}\left(u\right)}{\mu_{\text{oil}}}}\right]}{x}=
		0
		\text{ en }\mathbb{R}\times\left[0,\infty\right[.
	\end{equation*}
\end{example}

A continuación, presentamos las leyes de conservación más conocidas
cuando $p\geq 2$.

\begin{example}[$q$-sistema~\cite{Young2002}]\index{$q$-sistema}
	Sea $q\in C^{1}\left(\mathbb{R},\mathbb{R}\right)$ tal que
	$\forall v\in\mathbb{R}$: $\diffp{}{x}p\left(v\right)<0$,
	$\lim\limits_{v\to0}q\left(v\right)=\infty$,
	$\lim\limits_{v\to\infty}q\left(v\right)=0$.
	\begin{equation*}
		\diffp*{
			\begin{bmatrix}
				v \\
				u
			\end{bmatrix}
		}{t}+
		\begin{bmatrix}
			0                          & -1 \\
			\diffp{}{x}q\left(v\right) & 0
		\end{bmatrix}
		\diffp*{
			\begin{bmatrix}
				v \\
				u
			\end{bmatrix}
		}{x}=
		\begin{bmatrix}
			0 \\
			0
		\end{bmatrix}
		\text{ en }\mathbb{R}\times\left[0,\infty\right).
	\end{equation*}
\end{example}
% Sistema de fluidos % de la dinámica de gases
\begin{example}[Ecuaciones de Euler~\cite{Euler1757,Frisch2008}]\index{ecuaciones de Euler}
	\begin{equation*}
		\begin{dcases}
			\diffp{\rho}{t}+
			\sum_{j=1}^{3}
			\diffp{}{x_{j}}
			\left(\rho u_{j}\right)=0                   &
			\text{en }\mathbb{R}\times\left(0,\infty\right). \\
			\forall i\in\left\{1,2,3\right\}:
			\diffp{}{t}
			\left(\rho u_{i}\right)+
			\sum_{j=1}^{3}
			\diffp{}{x_{j}}
			\left(\rho u_{i}u_{j}+p\delta_{ij}\right)=0 &
			\text{en }\mathbb{R}\times\left(0,\infty\right). \\
			\diffp{}{t}
			\left(\rho e\right)+
			\sum_{j=1}^{3}
			\diffp{}{x_{j}}
			\left(\left(\rho e+p\right)u_{j}\right)=0   &
			\text{en }\mathbb{R}\times\left(0,\infty\right).
		\end{dcases}
	\end{equation*}
\end{example}

\begin{definition}
	Un sistema~\eqref{eq:systemofconservationlaw} es
	\textcolor{DarkBlue}{\bfseries hiperbólico}\index{hiperbólico} si y
	solo si
	\begin{equation}
		\forall\symbf{u}\in\Omega\!:
		\forall\omega\in\mathbb{R}\setminus\left\{0\right\}\!:
		\exists
		{\left\{
			\left(\lambda_{k},\symbf{r}_{k}\right)
			\right\}}^{p}_{k=1}\subset
		\mathbb{R}\times\mathbb{R}^{p}
		\text{ tal que }
		\symbf{A}\left(\symbf{u},\omega\right)
		\symbf{r}_{k}=
		\lambda_{k}
		\symbf{r}_{k}.
	\end{equation}
	Donde
	\begin{math}
		\symbf{A}\left(\symbf{u},\omega\right)\coloneqq
		\omega
		{
			\begin{bmatrix}
				\diffp{f_{i}}{u_{k}}
				\left(\symbf{u}\right)
			\end{bmatrix}}_{\substack{1\leq i\leq p\\1\leq k\leq p}}
	\end{math}
	es un múltiplo de la matriz jacobiana de $\mathbf{f}$.
\end{definition}

\begin{definition}
	Un \textcolor{DarkBlue}{\bfseries problema de Riemann}
	\index{problema de Riemann} es un problema de valor inicial
	asociado a~\eqref{eq:systemofconservationlaw}
	\begin{equation}\label{eq:cauchysystemofconservationlaw}
		\begin{cases}
			\diffp{\symbf{u}}{t}+
			\diffp*{\symbf{f}\left(\symbf{u}\right)}{x}=
			\symbf{s}\left(\symbf{u}\right) &
			\text{en }\mathbb{R}\times\left(0,\infty\right). \\
			\symbf{u}=\symbf{u}_{0}         &
			\text{en }\mathbb{R}\times\left\{t=0\right\}.
		\end{cases}
	\end{equation}
	Donde $\symbf{u}_{l},\symbf{u}_{r}\in\Omega$ son los estados y
	\begin{equation*}
		\symbf{u}_{0}\left(x\right)=
		\begin{cases}
			\symbf{u}_{l}, & x<0. \\
			\symbf{u}_{r}, & x>0.
		\end{cases}
	\end{equation*}
\end{definition}

\begin{definition}
	El problema de valor inicial y de frontera asociado
	a~\eqref{eq:systemofconservationlaw} es
	\begin{equation}
		\begin{dcases}
			\diffp{\symbf{u}}{t}+
			\diffp*{\symbf{f}\left(\symbf{u}\right)}{x}=
			\symbf{s}\left(\symbf{u}\right) &
			\text{ en }I\times\left(0,T\right].         \\
			\symbf{u}=\symbf{u}_{0}         &
			\text{c.t.p. en }I\times\left\{t=0\right\}. \\
			\symbf{u}=\symbf{g}             &
			\text{ en }\partial I\times\left[0,T\right].
		\end{dcases}
	\end{equation}
\end{definition}

% TODO: Lema 4.1.7, pág 139 NUMERICAL SOLUTION OF HYPERBOLIC PARTIAL DIFFERENTIAL EQUATIONS JOHN A. TRANGENSTEIN
% \begin{equation*}
% 	\diffp{\symbf{u}}{t}+
% 	\symbf{A}
% 	\diffp{\symbf{u}}{x}=
% 	0.\qquad
% 	\symbf{u}\left(x,0\right)=
% 	\symbf{u}_{0}\left(x\right)
% \end{equation*}
% $\symbf{A}$ es diagonalizable
% \begin{equation*}
% 	\symbf{u}\left(x,t\right)=
% 	\sum_{j}Xe_{j}e^{T}_{j}X^{-1}\symbf{u}_{0}\left(x-\lambda_{j}t\right)
% \end{equation*}
