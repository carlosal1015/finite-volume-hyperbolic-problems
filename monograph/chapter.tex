\motto{
	Por tanto, estudiantes estudien matemáticas y no construyan sin
	fundamentos.
	\begin{flushright}\normalfont
		Leonardo da Vinci (1452-1519)
	\end{flushright}
}
\chapter{Motivación}
\label{intro}
% use \chaptermark{}
% to alter or adjust the chapter heading in the running head

\abstract{
	Este capítulo inicia nuestro estudio de las Leyes de Conservación y
	resume el estado del arte de los avances~\cite{Mishra2020} en este
	fascinante campo de las EDPs.
	El principal objetivo es ilustrar las dificultades tanto analíticas
	como numéricas al resolver un \emph{problema de Cauchy} con una
	condición inicial discontinua, soluciones de este tipo no son
	satisfechas en todo punto de su dominio en el sentido clásico, ya
	que las derivadas no son definidas en las discontinuidades.
	Para encontrar la definición correcta, necesitamos entender el
	significado de la forma diferencial de las ecuaciones desde el
	enfoque físico.
	Esto nos llevará al estudio de la forma integral de estas
	ecuaciones y será justificado en este capítulo.
	Además, presentamos diferentes modelos de leyes de conservación que
	nos resume su amplio rango de aplicaciones.
}

\section{Introducción}

Considere un dominio $\Omega\subset\mathbb{R}^{n}$ y una cantidad de
interés $\mathbf{U}$ definida para todos los $\mathbf{x}\in\Omega$.
La cantidad de interés podría ser la concentración de un químico o la
densidad de una población humana, la presión de un fluido o la
temperatura de una cuerda.

La evolución en tiempo de esta cantidad de interés $\mathbf{U}$
puede ser descrito mediante la observación:
\begin{quote}
	La tasa temporal de cambio de $\mathbf{U}$ en cualquier subdominio
	fijo $\omega\in\Omega$ es igual a la cantidad total de
	$\mathbf{U}$ producido o destruido dentro de $\omega$ y el flujo
	de $\mathbf{U}$ a través de la frontera $\partial\omega$.
\end{quote}
Esta observación se describe matemáticamente como
\begin{equation}\label{eq:integralbalance}
	\diff{}{t}
	\int_{\omega}
	\mathbf{U}\dl{\mathbf{x}}=
	-\int_{\partial\omega}
	\mathbf{F}\cdot\nu\dl{\sigma\left(\mathbf{x}\right)}+
	\int_{\omega}
	\mathbf{S}\dl{\mathbf{x}}.
\end{equation}
donde $\nu$ es la normal unitaria exterior,
$\dl{\sigma\left(\mathbf{x}\right)}$ es la medida de la superficie,
$\mathbf{F}$ es el flujo y $\mathbf{S}$ es la fuente.
Así,~\eqref{eq:integralbalance} es la ecuación integral para la
evolución de la cantidad total de $\mathbf{U}$ en $\omega$.
\begin{figure}[ht!]
	\sidecaption
	\includegraphics{conservationscheme}
	% If not, use
	%\picplace{5cm}{2cm} % Give the correct figure height and width in cm
	%
	\caption{Ley de conservación para una magnitud escalar.
		Adaptado de~\cite{Hirsch2007}.}
	\label{fig:1}       % Give a unique label
\end{figure}
Simplificamos~\eqref{eq:integralbalance} utilizando el teorema de la
divergencia de Gauß en la integral de superficie para obtener
\begin{equation}\label{eq:integralbalance2}
	\diff{}{t}
	\int_{\omega}
	\mathbf{U}\dl{\mathbf{x}}
	+\int_{\omega}
	\operatorname{div}
	\left(\mathbf{F}\right)
	\dl{\mathbf{x}}=
	\int_{\omega}
	\mathbf{S}\dl{\mathbf{x}}.
\end{equation}
Dado que~\eqref{eq:integralbalance2} se cumple para cualquier
subdominio $\omega\subset\Omega$,
\begin{equation}\label{eq:balancelaw}
	\forall\left(x,t\right)\in\Omega\times\mathbb{R}_{+}:
	\diffp{\mathbf{U}}{t}+
	\operatorname{\operatorname{div}
		\left(\mathbf{F}\right)}=
	\mathbf{S}.
\end{equation}
La ecuación~\eqref{eq:balancelaw} se llama \emph{ley de balance}.
Frequentemente, el único cambio en $\mathbf{U}$ proviene de los
flujos y la fuente se fija en cero.
\begin{equation}\label{eq:conservationlaw}
	\forall\left(x,t\right)\in
	\Omega\times\mathbb{R}_{+}:
	\diffp{\mathbf{U}}{t}+
	\operatorname{\operatorname{div}
		\left(\mathbf{F}\right)}=
	0.
\end{equation}
La ecuación~\eqref{eq:conservationlaw} se llama
\emph{ley de conservación}, ya que el único cambio en $\mathbf{U}$
procede de la cantidad que entra o sale del dominio de interés.

\section*{Ejemplos de leyes de conservación}

Algunos ejemplos de leyes de conservación son la ecuación de
transporte escalar, la ecuación de difusión, las ecuaciones de Euler,
la ecuación de Richards y la ecuación Buckley-Leverett.

\subsection*{Ecuación de transporte escalar}

Sea $\mathbf{U}=U$ la concentración de un contaminante en un río.
Suponga que el río fluye con un campo de velocidad
$\mathbf{a}\left(\mathbf{x},t\right)$ y conocemos el campo de
velocidad en todos los puntos del río.
El contaminante es transportado en la dirección de la velocidad, y
así el flujo es $\mathbf{F}=\mathbf{a}U$.
Dado que no hay producción ni destrucción del contaminante durante el
flujo, el término fuente en~\eqref{eq:balancelaw} es cero.
La ley de conservación~\eqref{eq:conservationlaw} resulta ser
\begin{equation}
	\diffp{U}{t}+
	\operatorname{div}
	\left(\mathbf{a}\left(\mathbf{x},t\right)U\right)=
	0.
\end{equation}

\subsection*{Ecuación de difusión}

Sea $\mathbf{U}=U$ la temperatura de un bloque metálico.
Suponga que el bloque se calienta por un extremo y se deja enfriar
después, sin aportar ninguna fuente de calor adicional.
El calor se propaga o difunde y la temperatura del bloque se
uniformiza al cabo de un tiempo.
La difusión del calor se rige por la ley de Fick
\begin{equation}\label{eq:ficklaw}
	\mathbf{F}\left(U\right)=
	-\mathbf{k}\nabla U.
\end{equation}
Aquí, $\mathbf{k}$ es el tensor de conductividad del medio.
Sustituyendo~\eqref{eq:ficklaw} en~\eqref{eq:conservationlaw}
resulta ser
\begin{equation}
	\diffp{U}{t}-
	\operatorname{div}
	\left(\mathbf{k}\nabla U\right)=
	0.
\end{equation}

\subsection*{Ecuaciones de Euler}

El aire consiste de un gran número de moléculas.
El movimiento de cada molécula puede seguirse individualmente y
da lugar a un gran número de EDOs.
El sistema EDO resultante es demasiado grande para ser
computacionalmente viable.
En su lugar, se utiliza una descripción macroscópica de un gas ideal
y se ignora los efectos de viscosidad y la conducción de calor.
Las variables de interés son la densidad $\rho$, el campo de
velocidad $\mathbf{u}$ y la presión del gas $p$.
Las leyes de conservación relevante son

\begin{description}
	\item[\bfseries Conservación de la masa]

	      La masa total de un gas es conservado, garantizado
	      por el Teorema de la circulación de Kelvin. % https://en.wikipedia.org/wiki/Kelvin%27s_circulation_theorem

	\item[\bfseries Conservación de la cantidad de movimiento]

	      Se sigue de la segunda ley de Newton que la tasa de cambio de
	      la cantidad de movimiento es igual a la fuerza aplicada.
	      En ausencia de fuerzas externas, la presión del gas es la
	      única fuerza que actúa sobre el gas.

	\item[\bfseries Conservación de la energía]

	      La suma de la energía cinética y de la energía interna
	      (potencial) del gas ideal es la energía total
	      \begin{math}
		      E=
		      \frac{p}{\gamma-1}+
		      \frac{1}{2}
		      \rho{\left|\mathbf{u}\right|}^{2}
	      \end{math},
	      donde la constante del gas $\gamma$ es $\frac{5}{3}$ si este
	      es monoatómico y $\frac{7}{5}$ si es diatómico.
\end{description}

Estas tres leyes de conservación forman un sistema EDP no lineal
llamado las ecuaciones de Euler para la dinámica de gases.
\begin{align}
	\diffp{\rho}{t}+
	\operatorname{div}
	\left(\rho\mathbf{u}\right)             & =
	0.                                          \\
	\diffp{\rho\mathbf{u}}{t}+
	\operatorname{div}
	\left(\rho\mathbf{u}\otimes\mathbf{u}\right)+
	\nabla p                                & =
	0.                                          \\
	\diffp{E}{t}+
	\operatorname{div}
	\left(\left(E+p\right)\mathbf{u}\right) & =
	0.
\end{align}

Para cualquier
\begin{math}
	a=
	\begin{bmatrix}
		a_{1} \\
		a_{2} \\
		a_{3}
	\end{bmatrix},
	b=
	\begin{bmatrix}
		b_{1} \\
		b_{2} \\
		b_{3}
	\end{bmatrix}\in\mathbb{R}^{3}
\end{math}
se define el producto tensorial como
\begin{equation*}
	a\otimes b\coloneqq
	\begin{bmatrix}
		a_{1}b_{1} & a_{1}b_{2} & a_{1}b_{3} \\
		a_{2}b_{1} & a_{2}b_{2} & a_{2}b_{3} \\
		a_{3}b_{1} & a_{3}b_{2} & a_{3}b_{3} \\
	\end{bmatrix}\in\mathbb{R}^{3\times 3}.
\end{equation*}

\subsection*{Ecuación de Richards}

Derivada la conservación de la masa, la ecuación de continuidad
unidimensional de la infiltración se da como~\cite{Tan2018}
\begin{equation}\label{eq:reynoldstheorem}
	\diffp{\theta}{t}+
	\operatorname{div}\left(\mathbf{q}\right)=
	0.
\end{equation}
Por la ley de Darcy, $q_{z}$ representa
\begin{equation}\label{eq:darcy}
	q_{z}=
	-K\diffp{H}{z}=
	-K\diffp{\left(h-z\right)}{z}=
	K\left(
	1-
	\diffp{h}{z}
	\right).
\end{equation}
Uniendo las ecuaciones~\eqref{eq:reynoldstheorem} y~\eqref{eq:darcy}
obtenemos la formulación mixta de Richards
\begin{equation*}
	\frac{\partial\theta}{\partial t}+
	\frac{\partial}{\partial z}
	\left[K\left(1-\frac{\partial h}{\partial z}\right)\right]=0.
\end{equation*}
donde
\begin{itemize}
	\item $\theta$ es el contenido de agua del suelo $\left(\unit[per-mode=symbol]{\cubic\metre\per\cubic\metre}\right)$.
	\item $z$ es la profundidad del suelo.
	\item $\mathbf{q}$ es la tasa de infiltración de agua $\left(\unit[per-mode=symbol]{\metre\per\second}\right)$.
	\item $K$ es la conductividad hidráulica.
	\item $H$ es el potencial hídrico total en el eje $z$.
	\item $h$ es la cabeza de presión\footnote{\url{https://en.wikipedia.org/wiki/Pressure_head}}.
\end{itemize}

\subsection*{Ecuación de Buckley-Leverett}

La ecuación de Buckley-Leverett definida por la ley de conservación
con
\begin{equation*}
	f\left(u\right)=
	\frac{\mu u^{2}}{\mu u^{2}+{\left(1-\mu\right)}^{2}}
\end{equation*}
provee un modelo simple para el flujo de dos fluidos inmiscibles en
un medio poroso y tiene aplicaciones en la simulación de yacimientos
petrolíferos.
La solución representa la saturación de agua en un yacimiento de
petróleo y $0<\mu<1$ representa la movilidad.

Para profundizar en los modelos presentados vea~\cite{Vázquez2015}.

\section{Antecedentes}
\label{sec:1}
Use the template \emph{chapter.tex} together with the

\section{Trabajos relacionados}
\section{Modelo de leyes de conservación}
\label{sec:2}
% Always give a unique label
% and use \ref{<label>} for cross-references
% and \cite{<label>} for bibliographic references
% use \sectionmark{}
% to alter or adjust the section heading in the running head

\eject

\begin{eqnarray}
	\left|\nabla U_{\alpha}^{\mu}(y)\right| &\le&\frac1{d-\alpha}\int
	\left|\nabla\frac1{|\xi-y|^{d-\alpha}}\right|\,d\mu(\xi) =
	\int \frac1{|\xi-y|^{d-\alpha+1}} \,d\mu(\xi)\qquad  \\
	&=&(d-\alpha+1) \int\limits_{d(y)}^\infty
	\label{eq:01}
\end{eqnarray}

\enlargethispage{24pt}

\subsection{Modelo de Burgers}
\label{subsec:2}
Instead of simply listing\index{cross-references} and citations\index{citations} as has already been described in Sect.~\ref{sec:2}.

\begin{quotation}
	Please do not use quotation marks when quoting texts! Simply use the \verb|quotation| environment -- it will automatically be rendered in the preferred layout.
\end{quotation}

\subsection{Buckley-Leverett}

\paragraph{Paragraph Heading} %
Instead of simply listing headings of different levels we recommend to let every heading be followed by at least a short passage of text. Furtheron please use the \LaTeX\ automatism for all your cross-references and citations as has already been described in Sect.~\ref{sec:2}.

\begin{enumerate}
	\item{Livelihood and survival mobility are oftentimes coutcomes of uneven socioeconomic development.}
\end{enumerate}


\subparagraph{Subparagraph Heading} In order to avoid simply listing headings of different levels we recommend to let every heading be followed by at least a short passage of text. Use the \LaTeX\ automatism for all your cross-references and citations as has already been described in Sect.~\ref{sec:2}, see also Fig.~\ref{fig:2}.

Please note that the first line of text that follows a heading is not indented, whereas the first lines of all subsequent paragraphs are.

\begin{itemize}
	\item{Livelihood and survival mobility are oftentimes coutcomes of uneven socioeconomic development, cf. Table~\ref{tab:1}.}
\end{itemize}

\begin{figure}[t]
	\sidecaption[t]
	% Use the relevant command for your figure-insertion program
	% to insert the figure file.
	% For example, with the option graphics use
	\includegraphics{figure}
	%
	% If not, use
	%\picplace{5cm}{2cm} % Give the correct figure height and width in cm
	%
	\caption{Please write your figure caption here}
	\label{fig:2}       % Give a unique label
\end{figure}

\runinhead{Run-in Heading Boldface Version} Use the \LaTeX\ automatism for all your cross-references and citations as has already been described in Sect.~\ref{sec:2}.

\subruninhead{Run-in Heading Boldface and Italic Version} Use the \LaTeX\ automatism for all your cross-refer\-ences and citations as has already been described in Sect.~\ref{sec:2}\index{paragraph}.

\subsubruninhead{Run-in Heading Displayed Version} Use the \LaTeX\ automatism for all your cross-refer\-ences and citations as has already been described in Sect.~\ref{sec:2}\index{paragraph}.
% Use the \index{} command to code your index words
%
% For tables use
%
\begin{table}[!t]
	\caption{Please write your table caption here}
	\label{tab:1}       % Give a unique label
	%
	% For LaTeX tables use
	%
	\begin{tabular}{p{2cm}p{2.4cm}p{2cm}p{4.9cm}}
		\hline\noalign{\smallskip}
		Classes     & Subclass & Length      & Action Mechanism                      \\
		\noalign{\smallskip}\svhline\noalign{\smallskip}
		Translation & mRNA$^a$ & 22 (19--25) & Translation repression, mRNA cleavage \\
		\noalign{\smallskip}\hline\noalign{\smallskip}
	\end{tabular}
	$^a$ Table foot note (with superscript)
\end{table}
%
\section{Contenido principal y organización}
\label{sec:3}
% Always give a unique label
% and use \ref{<label>} for cross-references
% and \cite{<label>} for bibliographic references
% use \sectionmark{}
% to alter or adjust the section heading in the running head
Instead of simply listing headings of different levels we recommend to let every heading be followed by at least a short passage of text. Furtheron please use the \LaTeX\ automatism for all your cross-references and citations as has already been described in Sect.~\ref{sec:2}.

If you want to list definitions or the like we recommend to use the Springer-enhanced \verb|description| environment -- it will automatically render Springer's preferred layout.

\begin{description}[Type 1]
	\item[Type 1]{That addresses central themes pertainng to migration, health, and disease. In Sect.~\ref{sec:1}, Wilson discusses the role of human migration in infectious disease distributions and patterns.}
	\item[Type 2]{That addresses central themes pertainng to migration, health, and disease. In Sect.~\ref{subsec:2}, Wilson discusses the role of human migration in infectious disease distributions and patterns.}
\end{description}

\begin{svgraybox}
	If you want to emphasize complete paragraphs of texts we recommend to use the newly defined Springer class option \verb|graybox| and the newly defined environment \verb|svgraybox|. This will produce a 15 percent screened box 'behind' your text.

	If you want to emphasize complete paragraphs of texts we recommend to use the newly defined Springer class option and environment \verb|svgraybox|. This will produce a 15 percent screened box 'behind' your text.
\end{svgraybox}

\begin{theorem}
	Theorem text goes here.
\end{theorem}

\begin{definition}
	Definition text goes here.
\end{definition}

\begin{proof}
	%\smartqed
	Proof text goes here.
	%\qed
\end{proof}

\paragraph{Paragraph Heading} %
Instead of simply listing headings of different levels we recommend to let every heading be followed by at least a short passage of text. Furtheron please use the \LaTeX\ automatism for all your cross-references and citations as has already been described in Sect.~\ref{sec:2}.

\begin{trailer}{Trailer Head}
	If you want to emphasize complete paragraphs of texts in a \verb|Trailer Head| we recommend to
	use  \begin{verbatim}\begin{trailer}{Trailer Head}
...
\end{trailer}\end{verbatim}
\end{trailer}
%
\begin{questype}{Questions}
	If you want to emphasize complete paragraphs of texts in an \verb|Questions| we recommend to
	use  \begin{verbatim}\begin{questype}{Questions}
...
\end{questype}\end{verbatim}
\end{questype}
%
%
\begin{important}{Important}
	If you want to emphasize complete paragraphs of texts in an \verb|Important| we recommend to
	use  \begin{verbatim}\begin{important}{Important}
...
\end{important}\end{verbatim}
\end{important}
%
\clearpage
\begin{warning}{Attention}
	If you want to emphasize complete paragraphs of texts in an \verb|Attention| we recommend to
	use  \begin{verbatim}\begin{warning}{Attention}
...
\end{warning}\end{verbatim}
\end{warning}

\begin{programcode}{Program Code}
	If you want to emphasize complete paragraphs of texts in a \verb|Program Code| we recommend to
	use

	\verb|\begin{programcode}{Program Code}|

	\verb|\begin{verbatim}...\end{verbatim}|

	\verb|\end{programcode}|

\end{programcode}
%
\begin{tips}{Tips}
	If you want to emphasize complete paragraphs of texts in a \verb|Tips| we recommend to
	use  \begin{verbatim}\begin{tips}{Tips}
...
\end{tips}\end{verbatim}
\end{tips}
%
%
\begin{overview}{Overview}
	If you want to emphasize complete paragraphs of texts in an \verb|Overview| we recommend to
	use  \begin{verbatim}\begin{overview}{Overview}
...
\end{overview}\end{verbatim}
\end{overview}
\clearpage
\begin{backgroundinformation}{Background Information}
	If you want to emphasize complete paragraphs of texts in a \verb|Background|
	\verb|Information| we recommend to
	use

	\verb|\begin{backgroundinformation}{Background Information}|

	\verb|...|

	\verb|\end{backgroundinformation}|
\end{backgroundinformation}
\begin{legaltext}{Legal Text}
	If you want to emphasize complete paragraphs of texts in a \verb|Legal Text| we recommend to
	use  \begin{verbatim}\begin{legaltext}{Legal Text}
...
\end{legaltext}\end{verbatim}
\end{legaltext}
%
\begin{acknowledgement}
	If you want to include acknowledgments of assistance and the like at the end of an individual chapter please use the \verb|acknowledgement| environment -- it will automatically render Springer's preferred layout.
\end{acknowledgement}

\chapter{Leyes de conservación hiperbólicas escalares}

\begin{equation}
	\diffp{U}{t}+
	\diffp{f\left(U\right)}{x}=
	0.
\end{equation}
donde $U$ es la función desconocida y $f$ es la función flujo.

\section*{Modelo de flujo de tráfico}

\begin{equation}
	\diffp{U}{t}+
	\diffp{V_{\text{max}}U\left(1-U\right)}{x}=
	0.
\end{equation}

\section*{Recuperación mejorada de petróleo}

\begin{equation}
	\diffp{S^{\text{oil}}}{t}+
	\diffp{
		\frac{q\left(S^{\text{oil}}\right)^{2}}{\left(S^{\text{oil}}\right)^{2}+\left(1-S^{\text{oil}}\right)^{2}}
	}{x}=
	0.
\end{equation}

\section*{Condición de salto de Rankine-Hugoniot}

\section*{Solución al problema de Riemann}

Una función
$U\in L^{\infty}\left(\mathbb{R}\times\mathbb{R}_{+}\right)$
es una solución entrópica


\section{Problema de Riemann}
\section{Solución débil}
\section{Función entrópica}
\section{Condición de Rankine-Hugoniot}
\section{Teorema de Lax-Wendroff}

\chapter{Método de Volúmenes Finitos}

\section*{Apéndice}
\addcontentsline{toc}{section}{Apéndice}
%
When placed at se \textit{do not} use the \verb|appendix| command when

\begin{equation}
	a \times b = c
\end{equation}
% % Problems or Exercises should be sorted chapterwise
% \section*{Problemas}
% \addcontentsline{toc}{section}{Problemas}
% %
% % Use the following environment.
% % Don't forget to label each problem;
% % the label is needed for the solutions' environment
% \begin{prob}
%     \label{prob1}
%     A given problem or Excercise is described here. The
%     problem is described here. The problem is described here.
% \end{prob}

% \begin{prob}
%     \label{prob2}
%     \textbf{Problem Heading}\\
%     (a) The first part of the problem is described here.
% \end{prob}

\nocite{*}
\printbibliography[title={Referencias},heading=bibintoc]
