%%%%%%%%%%%%%%%%%%%%%%foreword.tex%%%%%%%%%%%%%%%%%%%%%%%%%%%%%%%%%
% sample foreword
%
% Use this file as a template for your own input.
%
%%%%%%%%%%%%%%%%%%%%%%%% Springer %%%%%%%%%%%%%%%%%%%%%%%%%%

\foreword

El presente trabajo titulado
\emph{El Método de los Volúmenes Finitos: Fundamentos y Aplicaciones},
representa un esfuerzo riguroso y detallado por parte del estudiante
en el campo de la discretización numérica de las ecuaciones
diferenciales parciales (EDPs).
A lo largo de esta investigación, se aborda con claridad y
profundidad la formulación matemática del método de volúmenes finitos
(MVF), su implementación práctica y su aplicación a problemas
clásicos, como la ecuación de Poisson y la ecuación de transporte
unidimensional.
Uno de los aspectos más valiosos de esta tesis es la exposición
pedagógica de las aproximaciones de derivadas en las interfaces de
las celdas, respaldada por expansiones de Taylor y un manejo
cuidadoso de las condiciones de frontera.
Asimismo, la discusión sobre esquemas lineales para la ecuación de
transporte y la extensión a leyes de conservación bidimensionales
refleja una visión amplia y aplicada del tema.
Como supervisor, he sido testigo del compromiso y la dedicación del
estudiante en la elaboración de este trabajo.
Su capacidad para conectar teoría, análisis numérico y aplicaciones
prácticas es notable, y estoy convencido de que este documento será
un recurso valioso tanto para estudiantes como para investigadores
que incursionen en el estudio de métodos numéricos para EDPs.

Finalmente, invito al lector a apreciar no solo los resultados
técnicos aquí presentados, sino también el rigor con el que se ha
estructurado cada capítulo, equilibrando formalismo matemático con
intuición física.
Esta tesis no solo cumple con los estándares académicos esperados,
sino que también sienta las bases para futuras investigaciones en el
área\index{foreword}.

\vspace{\baselineskip}
\begin{flushright}\noindent
	Lima, Perú\hfill Prof. Fidel Jara Huanca\\
	julio 2025\hfill {\it\phantom{.}}
\end{flushright}


