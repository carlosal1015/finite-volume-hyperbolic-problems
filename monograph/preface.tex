\preface

Este trabajo nace de mi fascinación por los métodos numéricos y su
poder para transformar ecuaciones diferenciales complejas en problemas
abordables mediante algoritmos computacionales.
El método de volúmenes finitos (MVF) capturó especialmente mi interés
por su elegancia matemática y su capacidad para preservar leyes de
conservación fundamentales en física e ingeniería.
A lo largo de estas páginas, he buscado construir un puente entre la
teoría y la práctica.
Comenzando con los fundamentos del MVF para la ecuación de Poisson
unidimensional, el texto avanza hacia problemas más desafiantes, como
la ecuación de transporte y sistemas bidimensionales.
Cada capítulo refleja horas de estudio, pruebas numéricas y
discusiones valiosas con mi supervisor y colegas, quienes me ayudaron
a refinar mis ideas y corregir errores.

Este documento no pretende ser exhaustivo, sino más bien una guía
accesible para estudiantes que, como yo, inician su camino en el
análisis numérico.
He incluido deducciones detalladas de las aproximaciones clave
—como las derivadas en las interfaces de las celdas— porque hubiera
deseado encontrar explicaciones así de claras cuando comencé a
aprender estos temas.

Agradezco profundamente a mi supervisor, Fidel Jara Huanca, por su
paciencia y orientación; a mis compañeros de laboratorio, por sus
críticas constructivas; y a mi familia, por su apoyo incondicional
durante este proceso.
Invito al lector a abordar este texto con lápiz y papel en mano: los
conceptos se asimilan mejor cuando se recrean paso a paso.
Los errores que persistan son, por supuesto, de mi entera
responsabilidad\index{preface}.

\vspace{\baselineskip}
\begin{flushright}\noindent
	Lima, Perú\hfill Carlos Aznarán Laos
\end{flushright}


