% \section{Red de Aprendizaje}

% \begin{figure}[ht!]
%   \centering
%   \begin{plantuml}
%     @startuml

%     rectangle "Nociones Básicas de los errores y ecuaciones no lineales"
%     rectangle "Funciones de Variable Compleja"
%     rectangle "Aproximación"
%     rectangle "Resolución Numérica de EDO y EDP"
%     rectangle "Sistema de Ecuaciones Lineales y valores y vectores propios"
%     rectangle "Introducción a los métodos de Elementos Finitos"

%     "Aproximación" -u-> "Nociones Básicas de los errores y ecuaciones no lineales"
%     "Aproximación" -d-> "Resolución Numérica de EDO y EDP"
%     "Aproximación" -l-> "Sistema de Ecuaciones Lineales y valores y vectores propios"
%     "Aproximación" -d-> "Introducción a los métodos de Elementos Finitos"
%     "Resolución Numérica de EDO y EDP" -r-> "Introducción a los métodos de Elementos Finitos"
%     @enduml
%   \end{plantuml}
% \end{figure}

\section{Unidades de Aprendizaje}

\subsection*{Capítulo 1: Espacios vectoriales topológicos / 30 horas}

Espacios normados.
Espacios vectoriales.
Espacios topológicos.
Espacios vectoriales topológicos.
Propiedades de separación.
Aplicaciones lineales.
Espacios de dimensión finita.
Metrización.
Acotación y continuidad.
Aplicaciones lineales acotadas.
Seminormas, convexidad local.
Espacios cociente.

\subsection*{Capítulo 2: Completitud / 10 horas}

Categoría de Baire.
Equicontinuidad.
El Teorema de la aplicación abierta.
El teorema del gráfico cerrado.
Aplicaciones bilineales.

\subsection*{Capítulo 3: Convexidad / 26 horas}

Los teoremas de Hahn Banach.
Topología débil.
Topología débil$\ast$.
Conjuntos convexos compactos.
Puntos extremos.
El teorema de Krein-Millman.
El teorema de Millman.
Funciones holomórficas.

\subsection*{Capítulo 4: Dualidad en espacios de Banach / 18 horas}

Espacio dual de un espacio vectorial normado.
Dualidad.
El segundo dual de un espacio de Banach.
Dual de subespacios y espacios cociente.
El operador adjunto.
Operadores compactos.
Espectro y valor propio.

% \section{Programación semanal de los contenidos}
% \subsection*{Unidad Temática N°01: funciones}

% \textbf{Logro de la unidad}:

% Analiza, describe, caracteriza y diferencia las definiciones, teoremas en el ámbito de las funciones complejas y luego los aplica en el contexto de flujo y mapeos

% \textbf{N$^{\circ}$ de horas}: 06

% \begin{table}[ht!]
%   \centering
%   \begin{tabular}{|c|p{6cm}|p{6cm}|}
%     \hline
%     Semana & Contenidos
%            & Actividades de aprendizaje           \\
%     \hline
%     1
%            & Funciones complejas básicas.
%     Límite, continuidad y derivadas.
%     Mapeo Simple.
%     Teoremas de Cauchy - Riemann.
%            & Exposición de conceptos.             \\
%     \hline
%     1      &
%     Funciones analíticas.
%     Teoremas relacionados.
%     Función exponencial.
%     Potencial complejo de un Flujo.
%            & Exposición y ejemplos de aplicación.
%     Experiencia de Laboratorio 01.                \\
%     \hline
%   \end{tabular}
% \end{table}

% UNIDAD TEMÁTICA N°02:
% NOCIONES BÁSICAS DE LOS ERRORES Y ECUACIONES NO LINEALES
% Logro de la unidad:

% Analiza y aplica los conceptos de errores y resuelve las ecuaciones lineales teniendo en cuenta estas y utilizando adecuadamente en las aplicaciones que se presentan en el ámbito ingenieril.  .

% N° de horas: 18


% Semana	Contenidos	Actividades de aprenDIzaje
% 2
% Error y su clasificación. Error absoluto relativo. Propagación de error. Aplicación a la carrera.
% Proceso estable e inestable.
% Criterio para finalizar un proceso secuencial. EVA1

% - Exposición de conceptos.
% - Ejercicios de aplicación práctica.
% - Experiencia de Laboratorio 02
% EVA1

% 3

% Modelos que conducen a resolver una ecuación no lineal. Método de newton y aproximación sucesiva, secante Modificada y el Vago

% -Exposición y ejemplos de aplicación.
% Experiencia de Laboratorio 03.

% UNIDAD TEMÁTICA N°03: SISTEMA DE ECUACIONES LINEALES Y VALORES Y VECTORES PROPIOS

% Logro de la unidad:
% Analiza, y aplica los diferentes métodos de resolución numérica en el contexto de sistemas de ecuaciones lineales y valores propias; para luego desarrollar proyectos inherentes.
% N° de horas: 12

% Semana	Contenidos	Actividades de aprenDIzaje
% 4

% Métodos directos e iterativos en la resolución numérica de sistemas de ecuaciones lineales. Aplicaciones. EVA2
% -Exposición y ejemplos de aplicación.
% -Experiencia de Laboratorio 05.


% UNIDAD TEMÁTICA N°04: APROXIMACIÓN

% Logro de la unidad:
% Construye una función o ecuación que modela un fenómeno empleando información discreta confiable tanto para el ámbito de interpolación, cuadratura y cubicación para luego usar adecuadamente en las aplicaciones que se presentan en la carrera.
% N° de horas: 12

% Semana	Contenidos	Actividades de aprenDIzaje
% 5

% Interpolación polinomial.
% Ajuste por mínimos cuadrados
% Método del Trapecio caso abierto y cerrado.
% Métodos de Simpson caso abierto y cerrado.
% Cubicación en malla triangular.
% .

% -Exposición y ejemplos de aplicación.

% UNIDAD TEMÁTICA N°05: RESOLUCIÓN NUMÉRICA DE EDO Y EDP

% Logro de la unidad:
% Resuelve mediante y técnicas numéricas EDO y EDP; teniendo en cuenta la convergencia y estabilidad de los procesos iterativos para luego aplicarlos en los problemas que se presentan en la carrera.

% N° de horas: 12

% Semana	Contenidos	Actividades de aprenDIzaje

% 6
% Solución numérica de una EDO Métodos de paso simple: Euler, Runge-Kutta orden 2 y 4.
% Diferencia finita y las E.D. con condiciones de frontera. EVA3

% -Exposición y ejemplos de aplicación.

% 7

% Métodos para la resolución numérica de Ecuaciones diferenciales parciales: Parabólico
% Métodos para la resolución numérica de Ecuaciones diferenciales parciales: hiperbólico y elíptico.
% -Exposición y ejemplos de aplicación.
% Experiencia de Laboratorio 07.

% UNIDAD TEMÁTICA N°06: INTRODUCCIÓN A LOS MÉTODOS DE ELEMENTOS FINITOS
% Logro de la unidad:

% Analiza y recomienda los métodos de elementos finitos adecuadas para resolver problemas de aplicación que se presentan en la carrera.
% N° de horas: 12

% Semana	Contenidos	Actividades de aprenDIzaje
% 8	Método de elemento finito en una dimensión.  Aproximación polinómica por partes en una dimensión y dos dimensiones. EVA4
% -Exposición y ejemplos de aplicación.