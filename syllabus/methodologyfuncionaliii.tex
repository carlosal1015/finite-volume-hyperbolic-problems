\section{Metodología}
% \section{Técnicas didácticas}

% Metodología presencial de aprendizaje, en el cual el profesor deduce e induce las bases teóricas,
% complementada con aplicaciones preferentemente relacionadas a la especialidad respectiva.
% Tutoría académica permanente en forma semanal según horarios fuera de clase.

La asignatura se desarrolla en dos modalidades didácticas:

\begin{enumerate}[label=\arabic*., font=\bfseries, nosep]
      \item

            Clases teóricas: Se desarrollan mediante exposición del
            profesor cumpliendo el calendario establecido.
            En estas clases se estimula la participación activa del
            estudiante, mediante el aprendizaje colaborativo,
            solución de problemas, estudio de casos, búsqueda de
            información bibliográfica y por internet.

      \item

            Trabajo independiente del alumno:
            En esta parte el estudiante complementa su aprendizaje
            mediante la lectura de materiales compartidos por el
            profesor.
\end{enumerate}

% \section{Equipos y materiales}

% \begin{enumerate}[label=\arabic*., font=\bfseries, nosep]
%   \item Equipos e Instrumentos: Proyector multimedia. Computadora personal.

%   \item Materiales: Tizas. Plumones. Diapositivas del curso en el aula virtual. Lista de talleres. Ejercicios y problemas propuestos.
% \end{enumerate}