% \section{Red de Aprendizaje}

% \begin{figure}[ht!]
%   \centering
%   \begin{plantuml}
%     @startuml

%     rectangle "Nociones Básicas de los errores y ecuaciones no lineales"
%     rectangle "Funciones de Variable Compleja"
%     rectangle "Aproximación"
%     rectangle "Resolución Numérica de EDO y EDP"
%     rectangle "Sistema de Ecuaciones Lineales y valores y vectores propios"
%     rectangle "Introducción a los métodos de Elementos Finitos"

%     "Aproximación" -u-> "Nociones Básicas de los errores y ecuaciones no lineales"
%     "Aproximación" -d-> "Resolución Numérica de EDO y EDP"
%     "Aproximación" -l-> "Sistema de Ecuaciones Lineales y valores y vectores propios"
%     "Aproximación" -d-> "Introducción a los métodos de Elementos Finitos"
%     "Resolución Numérica de EDO y EDP" -r-> "Introducción a los métodos de Elementos Finitos"
%     @enduml
%   \end{plantuml}
% \end{figure}

\section{Unidades de Aprendizaje}

\subsection*{Capítulo 1: Generalidades de los sistemas de leyes de conservación hiperbólicas / 4 horas}

Conceptos básicos y ejemplos de interés medioambiental e industrial.
Leyes de conservación.
Tipos de soluciones: clásicas, débiles y entrópicas.
El problema de Riemann.
Aplicaciones.

\subsection*{Capítulo 2: Nociones sobre ecuaciones hiperbólicas / 8 horas}

La ecuación de advección lineal.
Problema de Valor inicial.
El problema de Riemann.
Sistemas lineales hiperbólicos.
Ecuaciones escalares no lineales.

\subsection*{Capítulo 3: Nociones en métodos numéricos / 8 horas}

Aproximación numérica de las ecuaciones hiperbólicas.
Aproximación de diferencias finitas de las EDPs.
Método de diferencias finitas bien conocidas.
Propiedades básicas de los métodos numéricos.
Formas de expresar un método numérico.
Monotonicidad, precisión y el teorema de Godunov.
Forma viscosa de un esquema.
Resultados Computacionales.
Problemas test, métodos y parámetros.

\subsection*{Capítulo 4: Método de volúmenes finitos unidimensional / 24 horas}

Problema elíptico unidimensional.
El método de volúmenes finitos para el problema de Dirichlet.
Convergencia y análisis de error para el problema de Dirichlet.
Estimación de error con regularidad $C^{2}$.
Ecuaciones elípticas generales unidimensionales.
Un problema elíptico semilineal.
Problema parabólico unidimensional.
Estimación de error para el caso lineal.
Convergencia en el caso no lineal.
Problema hiperbólico lineales unidimensionales.
Esquemas numéricos en el caso lineal.
El caso no lineal.

\subsection*{Capítulo 5: Sistemas de ecuaciones hiperbólicos  / 24 horas}

Sistemas hiperbólicos.
Método de Godunov.
Condiciones de estabilidad.
Aplicaciones.
Resolución de problemas hiperbólicos no lineales unidimensionales.
Esquemas conservativos.
Esquemas descentrados.
Teorema de Lax-Wendroff.
Método de Godunov.
Resolventes de Riemann aproximadas.
Técnicas de descomposición del flujo.
Esquemas conservativos para leyes de conservación generalizadas.
Esquemas monótonos y de variación total decreciente.
Esquemas consistentes con la condición de entropía.
Aplicaciones.
Resolución de problemas hiperbólicos bidimensionales.
Métodos de las direcciones alternadas.
Definición de volúmenes finitos en mallas no estructuradas.
Esquemas conservativos para leyes de conservación generalizadas.
Aplicaciones.

\subsection*{Capítulo 6: Aplicaciones / 16 horas}

Aproximaciones numéricas de modelos de Ecuaciones Diferenciales Parciales.
La ecuación de convección.
La ecuación de onda.
La ecuación de calor.
Ecuaciones diferenci no lineales.
Aplicación a la cinética química.
Solución de la ecuación de advección-difusión.
Dinámica de gases.
El problema de Riemann y soluciones discontinuas.
Ecuaciones de Euler de la dinámica de gases.

% \section{Programación semanal de los contenidos}
% \subsection*{Unidad Temática N°01: funciones}

% \textbf{Logro de la unidad}:

% Analiza, describe, caracteriza y diferencia las definiciones, teoremas en el ámbito de las funciones complejas y luego los aplica en el contexto de flujo y mapeos

% \textbf{N$^{\circ}$ de horas}: 06

% \begin{table}[ht!]
%   \centering
%   \begin{tabular}{|c|p{6cm}|p{6cm}|}
%     \hline
%     Semana & Contenidos
%            & Actividades de aprendizaje           \\
%     \hline
%     1
%            & Funciones complejas básicas.
%     Límite, continuidad y derivadas.
%     Mapeo Simple.
%     Teoremas de Cauchy - Riemann.
%            & Exposición de conceptos.             \\
%     \hline
%     1      &
%     Funciones analíticas.
%     Teoremas relacionados.
%     Función exponencial.
%     Potencial complejo de un Flujo.
%            & Exposición y ejemplos de aplicación.
%     Experiencia de Laboratorio 01.                \\
%     \hline
%   \end{tabular}
% \end{table}

% UNIDAD TEMÁTICA N°02:
% NOCIONES BÁSICAS DE LOS ERRORES Y ECUACIONES NO LINEALES
% Logro de la unidad:

% Analiza y aplica los conceptos de errores y resuelve las ecuaciones lineales teniendo en cuenta estas y utilizando adecuadamente en las aplicaciones que se presentan en el ámbito ingenieril.  .

% N° de horas: 18


% Semana	Contenidos	Actividades de aprenDIzaje
% 2
% Error y su clasificación. Error absoluto relativo. Propagación de error. Aplicación a la carrera.
% Proceso estable e inestable.
% Criterio para finalizar un proceso secuencial. EVA1

% - Exposición de conceptos.
% - Ejercicios de aplicación práctica.
% - Experiencia de Laboratorio 02
% EVA1

% 3

% Modelos que conducen a resolver una ecuación no lineal. Método de newton y aproximación sucesiva, secante Modificada y el Vago

% -Exposición y ejemplos de aplicación.
% Experiencia de Laboratorio 03.

% UNIDAD TEMÁTICA N°03: SISTEMA DE ECUACIONES LINEALES Y VALORES Y VECTORES PROPIOS

% Logro de la unidad:
% Analiza, y aplica los diferentes métodos de resolución numérica en el contexto de sistemas de ecuaciones lineales y valores propias; para luego desarrollar proyectos inherentes.
% N° de horas: 12

% Semana	Contenidos	Actividades de aprenDIzaje
% 4

% Métodos directos e iterativos en la resolución numérica de sistemas de ecuaciones lineales. Aplicaciones. EVA2
% -Exposición y ejemplos de aplicación.
% -Experiencia de Laboratorio 05.


% UNIDAD TEMÁTICA N°04: APROXIMACIÓN

% Logro de la unidad:
% Construye una función o ecuación que modela un fenómeno empleando información discreta confiable tanto para el ámbito de interpolación, cuadratura y cubicación para luego usar adecuadamente en las aplicaciones que se presentan en la carrera.
% N° de horas: 12

% Semana	Contenidos	Actividades de aprenDIzaje
% 5

% Interpolación polinomial.
% Ajuste por mínimos cuadrados
% Método del Trapecio caso abierto y cerrado.
% Métodos de Simpson caso abierto y cerrado.
% Cubicación en malla triangular.
% .

% -Exposición y ejemplos de aplicación.

% UNIDAD TEMÁTICA N°05: RESOLUCIÓN NUMÉRICA DE EDO Y EDP

% Logro de la unidad:
% Resuelve mediante y técnicas numéricas EDO y EDP; teniendo en cuenta la convergencia y estabilidad de los procesos iterativos para luego aplicarlos en los problemas que se presentan en la carrera.

% N° de horas: 12

% Semana	Contenidos	Actividades de aprenDIzaje

% 6
% Solución numérica de una EDO Métodos de paso simple: Euler, Runge-Kutta orden 2 y 4.
% Diferencia finita y las E.D. con condiciones de frontera. EVA3

% -Exposición y ejemplos de aplicación.

% 7

% Métodos para la resolución numérica de Ecuaciones diferenciales parciales: Parabólico
% Métodos para la resolución numérica de Ecuaciones diferenciales parciales: hiperbólico y elíptico.
% -Exposición y ejemplos de aplicación.
% Experiencia de Laboratorio 07.

% UNIDAD TEMÁTICA N°06: INTRODUCCIÓN A LOS MÉTODOS DE ELEMENTOS FINITOS
% Logro de la unidad:

% Analiza y recomienda los métodos de elementos finitos adecuadas para resolver problemas de aplicación que se presentan en la carrera.
% N° de horas: 12

% Semana	Contenidos	Actividades de aprenDIzaje
% 8	Método de elemento finito en una dimensión.  Aproximación polinómica por partes en una dimensión y dos dimensiones. EVA4
% -Exposición y ejemplos de aplicación.