\section{Sistema de Evaluación}

% \subsection*{Criterios}
\noindent

Sistema de evaluación: G

Cantidad de prácticas o trabajos calificados: seis $\left(06\right)$

El promedio final (\textbf{PF}) se calcula tal como se muestra a
continuación:
% El sistema de evaluación es permanente.
% Comprende evaluaciones de los conocimientos, habilidades y actitudes.

% Para evaluar los conocimientos se utilizan las evaluaciones escritas.
% Para evaluar las habilidades se utilizan adicionalmente a las anteriores las intervenciones orales, exposiciones y el trabajo de laboratorio. Para evaluar las actitudes, se utiliza la observación del alumno, su comportamiento, responsabilidad, respeto, iniciativa y relaciones con el profesor y alumnos.
% La redacción, orden y ortografía influyen en la calificación de las pruebas escritas.

% Los instrumentos de evaluación del curso son:
\begin{equation*}
  \textbf{PF} =
  \frac{
    \textbf{EP}+
    \textbf{EF}+
    \textbf{PP}
  }{3}.
\end{equation*}

\begin{table}[ht!]
  \begin{tabular}{lll}
    \textbf{EP}: & Examen Parcial        & (Peso 1) \\
    \textbf{EF}: & Examen Final          & (Peso 1) \\
    \textbf{PP}: & Promedio de Prácticas & (Peso 1) \\
  \end{tabular}
\end{table}

PP: Se obtiene del promedio aritmético de las cinco $\left(05\right)$
mejores notas de las prácticas o trabajos calificados.

% 10.REFERENCIAS BIBLIOGRÁFICAS Y OTRAS FUENTES

% AUTOR	TITULO	AÑO	LUGAR	EDITORIAL	Nº PÁG.

% Kreyszig, Erwin	Matemática Avanzadas para Ingeniería Vol II	1997	México

% Referencias Bibliográficas Complementaria

% AUTOR	TITULO	AÑO	LUGAR	EDITORIAL	Nº PÁG.
% Zill Dennos G	Ecuaciones Diferenciales con aplicaciones de modelado	2002	México	Thomson	438

% Referencias en la Web
% 1.www.unalmed.edu.co/~ifasmar