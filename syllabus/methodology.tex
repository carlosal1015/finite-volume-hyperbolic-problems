\section{Metodología}
% \section{Técnicas didácticas}

% Metodología presencial de aprendizaje, en el cual el profesor deduce e induce las bases teóricas,
% complementada con aplicaciones preferentemente relacionadas a la especialidad respectiva.
% Tutoría académica permanente en forma semanal según horarios fuera de clase.

La asignatura se desarrolla en dos modalidades didácticas:

\begin{enumerate}[label=\arabic*., font=\bfseries, nosep]
  \item

        Clases teóricas: Se desarrollan mediante exposición del profesor
        cumpliendo el calendario establecido.
        En estas clases se estimula la participación activa del
        estudiante, mediante el aprendizaje colaborativo, solución de
        problemas, estudio de casos, búsqueda de información
        bibliográfica y por Internet.

  \item

        Clases de laboratorio: Se utiliza el software Python para
        realizar simulaciones en todas las sesiones promoviendo la
        participación activa de los alumnos.
\end{enumerate}

Los equipos como computador y proyector multimedia y los materiales
como el texto, separatas, software y el aula virtual permitirán la
mejor comprensión de los temas tratados.

% \section{Equipos y materiales}

% \begin{enumerate}[label=\arabic*., font=\bfseries, nosep]
%   \item Equipos e Instrumentos: Proyector multimedia. Computadora personal.

%   \item Materiales: Tizas. Plumones. Diapositivas del curso en el aula virtual. Lista de talleres. Ejercicios y problemas propuestos.
% \end{enumerate}